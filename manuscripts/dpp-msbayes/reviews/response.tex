%&<latex>
\documentclass[12pt]{article}

\usepackage{parskip}
\documentclass[table]{beamer}
\usepackage{beamerthemesplit}
\usetheme{boxes}
\usecolortheme{seahorse}
% \useinnertheme{myboxes}
% \usepackage{amsmath}
% \usepackage[fleqn]{amsmath}
\usepackage{ifthen}
\usepackage{xspace}
\usepackage{multirow}
\usepackage{booktabs}
\usepackage{xcolor}
\usepackage[style=nature]{biblatex}
\bibliography{../manuscripts/bib/references}
\newrobustcmd*{\footlessfullcite}{\AtNextCite{\renewbibmacro{title}{}\renewbibmacro{in:}{}}\footfullcite}
\AtEveryBibitem{\clearfield{month}}
\AtEveryCitekey{\clearfield{month}}

% Make all footnotes smaller
\renewcommand{\footnotesize}{\scriptsize}

\definecolor{myGray}{gray}{0.9}
\colorlet{rowred}{red!30!white}

\setbeamertemplate{blocks}[rounded][shadow=true]

\setbeamercolor{defaultcolor}{bg=structure!30!normal text.bg,fg=black}
\setbeamercolor{block body}{bg=structure!30!normal text.bg,fg=black}
\setbeamercolor{block title}{bg=structure!50!normal text.bg,fg=black}

\newenvironment<>{varblock}[2][\textwidth]{%
  \setlength{\textwidth}{#1}
  \begin{actionenv}#3%
    \def\insertblocktitle{#2}%
    \par%
    \usebeamertemplate{block begin}}
  {\par%
    \usebeamertemplate{block end}%
  \end{actionenv}}

\newenvironment{displaybox}[1][\textwidth]
{
    \centerline\bgroup\hfill
    \begin{beamerboxesrounded}[lower=defaultcolor,shadow=true,width=#1]{}
}
{
    \end{beamerboxesrounded}\hfill\egroup
}

\newenvironment{onlinebox}[1][4cm]
{
    \newbox\mybox
    \newdimen\myboxht
    \setbox\mybox\hbox\bgroup%
        \begin{beamerboxesrounded}[lower=defaultcolor,shadow=true,width=#1]{}
    \centering
}
{
    \end{beamerboxesrounded}\egroup
    \myboxht\ht\mybox
    \raisebox{-0.25\myboxht}{\usebox\mybox}\hspace{2pt}
}

\newenvironment{mydescription}{
    \begin{description}
        \setlength{\leftskip}{-1.5cm}}
    {\end{description}}

\newenvironment{myitemize}{
    \begin{itemize}
        \setlength{\leftskip}{-.3cm}}
    {\end{itemize}}

% define formatting for footer
\newcommand{\myfootline}{%
    {\it
    \insertshorttitle
    \hspace*{\fill} 
    \insertshortauthor, \insertshortinstitute
    % \ifx\insertsubtitle\@empty\else, \insertshortsubtitle\fi
    \hspace*{\fill}
    \insertframenumber/\inserttotalframenumber}}

% set up footer
\setbeamertemplate{footline}{%
    \usebeamerfont{structure}
    \begin{beamercolorbox}[wd=\paperwidth,ht=2.25ex,dp=1ex]{frametitle}%
        \Tiny\hspace*{4mm}\myfootline\hspace{4mm}
    \end{beamercolorbox}}

% remove navigation bar
\beamertemplatenavigationsymbolsempty


\usepackage[numbers,sort&compress]{natbib}
% \newcommand{\change}[1]{{\color{blue} #1}\xspace}
\newcommand{\change}[1]{{\color{black} #1}\xspace}


\newcommand{\citationNeeded}{\textcolor{magenta}{\textbf{[CITATION NEEDED!]}}\xspace}
\newcommand{\tableNeeded}{\textcolor{magenta}{\textbf{[TABLE NEEDED!]}}\xspace}
\newcommand{\figureNeeded}{\textcolor{magenta}{\textbf{[FIGURE NEEDED!]}}\xspace}
\newcommand{\highLight}[1]{\textcolor{magenta}{\MakeUppercase{#1}}}

\newcommand{\editorialNote}[1]{\textcolor{red}{[\textit{#1}]}}
\newcommand{\ignore}[1]{}
\newcommand{\addTail}[1]{\textit{#1}.---}
\newcommand{\super}[1]{\ensuremath{^{\textrm{#1}}}}
\newcommand{\sub}[1]{\ensuremath{_{\textrm{#1}}}}
\newcommand{\dC}{\ensuremath{^\circ{\textrm{C}}}}

\providecommand{\e}[1]{\ensuremath{\times 10^{#1}}}

\newcommand{\mthnote}[2]{{\color{red} #2}}

\newcommand{\ifTwoArgs}[3]{\ifthenelse{\equal{#1}{}\or\equal{#2}{}}{}{#3}\xspace}
\newcommand{\ifArg}[2]{\ifthenelse{\equal{#1}{}}{}{#2}\xspace}

\newcommand{\allDatasets}{\ensuremath{\mathcal{\alignment{}{}}}\xspace}
\newcommand{\allParameterValues}{\ensuremath{\boldsymbol{\Theta}}\xspace}
\newcommand{\bayesfactor}[2]{\ensuremath{BF_{#1\protect\ifArg{#2}{,}#2}}}
\newcommand{\given}{\ensuremath{\,|\,}\xspace}
\newcommand{\msb}{\upshape\texttt{\MakeLowercase{ms\MakeUppercase{B}ayes}}\xspace}
\newcommand{\hky}{HKY85\xspace}
\newcommand{\uniformMin}[1]{\ensuremath{a_{#1}}\xspace}
\newcommand{\uniformMax}[1]{\ensuremath{b_{#1}}\xspace}
\newcommand{\locusRateHetShapeParameter}{\ensuremath{\alpha}\xspace}
\newcommand{\ancestralThetaVector}{\ensuremath{\boldsymbol{\theta_{A}}}\xspace}
\newcommand{\descendantThetaVector}[1]{\ensuremath{\boldsymbol{\theta_{D#1}}}\xspace}
\newcommand{\divtscaledvector}{\ensuremath{\mathbf{{\divtscaled{}{}}}}\xspace}
\newcommand{\divtvector}{\ensuremath{\boldsymbol{\divt{}}}\xspace}
\newcommand{\divtuniquevector}{\ensuremath{\mathbf{\divtunique{}}}\xspace}
\newcommand{\bottleTimeVector}{\ensuremath{\boldsymbol{\bottleTime{}}}\xspace}
\newcommand{\bottleTime}[1]{\ensuremath{\divt{B\ifArg{#1}{,}#1}}\xspace}
\newcommand{\bottleScalarVector}[1]{\ensuremath{\boldsymbol{\bottleScalar{#1}{}}}\xspace}
\newcommand{\bottleScalar}[2]{\ensuremath{\zeta_{D#1\protect\ifArg{#2}{,}#2}}\xspace}
\newcommand{\migrationRateVector}{\ensuremath{\mathbf{\migrationRate{}}}\xspace}
\newcommand{\geneTreeVector}{\ensuremath{\mathbf{\geneTree{}{}}}\xspace}
\newcommand{\alignmentVector}{\ensuremath{\mathbf{\alignment{}{}}}\xspace}
\newcommand{\alignment}[2]{\ensuremath{X_{#1\protect\ifTwoArgs{#1}{#2}{,}#2}}\xspace}
\newcommand{\geneTree}[2]{\ensuremath{G_{#1\protect\ifTwoArgs{#1}{#2}{,}#2}}\xspace}
\newcommand{\migrationRate}[1]{\ensuremath{m_{#1}}\xspace}
\newcommand{\recombinationRate}{\ensuremath{r}\xspace}
\newcommand{\ploidyScalar}[2]{\ensuremath{\rho_{#1\protect\ifTwoArgs{#1}{#2}{,}#2}}\xspace}
\newcommand{\ploidyScalarVector}{\ensuremath{\boldsymbol{\ploidyScalar{}{}}}\xspace}
\newcommand{\descendantRelativeThetaVector}[1]{\ensuremath{\boldsymbol{\eta_{D#1}}}\xspace}
\newcommand{\descendantRelativeTheta}[2]{\ensuremath{\eta_{D#1\protect\ifArg{#2}{,}#2}}\xspace}
\newcommand{\mutationRateScalarConstant}[2]{\ensuremath{\nu_{#1\protect\ifTwoArgs{#1}{#2}{,}#2}}\xspace}
\newcommand{\mutationRateScalarConstantVector}{\ensuremath{\boldsymbol{\mutationRateScalarConstant{}{}}}\xspace}
\newcommand{\locusMutationRateScalar}[1]{\ensuremath{\upsilon_{#1}}\xspace}
\newcommand{\locusMutationRateScalarVector}{\ensuremath{\boldsymbol{\upsilon}}\xspace}
\newcommand{\hkyModel}[2]{\ensuremath{\phi_{#1\protect\ifTwoArgs{#1}{#2}{,}#2}}\xspace}
\newcommand{\hkyModelVector}{\ensuremath{\boldsymbol{\hkyModel{}{}}}\xspace}
\newcommand{\mutationRate}{\ensuremath{\mu}\xspace}
\newcommand{\iid}{\textit{iid}\xspace}
\newcommand{\model}[1]{\ensuremath{\Theta}\xspace}
\newcommand{\npairs}[1]{\ensuremath{Y_{#1}}}
\newcommand{\nloci}[1]{\ensuremath{k_{#1}}\xspace}
\newcommand{\nlociTotal}{\ensuremath{K}\xspace}
\newcommand{\myTheta}[1]{\ensuremath{\theta_{#1}}}
\newcommand{\ancestralTheta}[1]{\ensuremath{\theta_{A\protect\ifArg{#1}{,}#1}}\xspace}
\newcommand{\descendantTheta}[2]{\ensuremath{\theta_{D#1\protect\ifArg{#2}{,}#2}}\xspace}
\newcommand{\meanDescendantTheta}[1]{\ensuremath{\descendantTheta{}{#1}}\xspace}
\newcommand{\nucdiv}[1]{\ensuremath{\pi_{#1}}}

\newcommand{\ssVector}[1]{\ensuremath{\mathbf{\alignmentSS{#1}{}}}\xspace}
\newcommand{\ssVectorObs}{\ensuremath{\ssVector{}^*}\xspace}
\newcommand{\ssSpace}{\ensuremath{\euclideanSpace{\ssVectorObs}}\xspace}
\newcommand{\ssVectorObsPLS}{\ensuremath{\ssVectorObs_{PLS}}\xspace}
\newcommand{\alignmentSS}[2]{\ensuremath{S_{#1\protect\ifTwoArgs{#1}{#2}{,}#2}}\xspace}
\newcommand{\alignmentSSObs}[2]{\ensuremath{\alignmentSS{#1}{#2}^*}\xspace}
\newcommand{\tol}{\ensuremath{\epsilon}\xspace}
\newcommand{\euclideanSpace}[1]{\ensuremath{B_{\tol}(#1)}\xspace}
\newcommand{\hpvector}[1]{\ensuremath{\Lambda_{#1}}}
\newcommand{\divtscaled}[2]{\ensuremath{t_{#1\protect\ifTwoArgs{#1}{#2}{,}#2}}}
\newcommand{\divt}[1]{\ensuremath{\tau_{#1}}}
\newcommand{\divtunique}[1]{\ensuremath{T_{#1}}}
\newcommand{\ssMatrix}{\ensuremath{\mathbb \alignmentSS{}{}}\xspace}
\newcommand{\ssMatrixRaw}[1]{\ensuremath{{\ssMatrix}_{stats#1}}\xspace}
\newcommand{\ssMatrixPLS}[1]{\ensuremath{{\ssMatrix}_{PLS#1}}\xspace}
\newcommand{\hpmatrix}[1]{\ensuremath{\mathcal{P}_{#1}}}
\newcommand{\meant}[2]{\ensuremath{E(\divt{#1})_{#2}}}
\newcommand{\meantestimate}{\ensuremath{\hat{E(\divt{})}}\xspace}
\newcommand{\vart}[2]{\ensuremath{Var(\divt{#1}{})_{#2}}}
\newcommand{\vmratio}[1]{\ensuremath{\Omega_{#1}}}
\newcommand{\numt}[1]{\ensuremath{\Psi_{#1}}}
\newcommand{\probnumt}[2]{\ensuremath{p(\numt{#1} = {#2})}}
\newcommand{\postprobnumt}[1]{\ensuremath{p(\numt{} = {#1}|\ssSpace)}}
\newcommand{\postprobnumtnot}[1]{\ensuremath{p(\numt{} \neq {#1}|\ssSpace)}}
\newcommand{\postprobomegasimult}{\ensuremath{p(\vmratio{} < 0.01 | \ssSpace)}\xspace}
\newcommand{\modelprior}[1]{\ensuremath{f(\model{})}}
\newcommand{\modelpost}[1]{\ensuremath{f(\model{}|\ssSpace)}}
\newcommand{\npriorsamples}{\ensuremath{n}\xspace}
\newcommand{\globalcoalunit}{\ensuremath{4\globalpopsize}\xspace}
\newcommand{\globalpopsize}{\ensuremath{N_C}\xspace}
\newcommand{\effectivePopSize}[1]{\ensuremath{N_e{#1}}\xspace}
\newcommand{\coalunit}{\ensuremath{4\effectivePopSize{}}\xspace}
\newcommand{\priorsample}[1]{\ensuremath{\hpmatrix{\modelprior{}}}}
\newcommand{\truncprior}[1]{\ensuremath{\hpmatrix{\tol}}\xspace}
\newcommand{\postsample}[1]{\ensuremath{\hpmatrix{\modelpost{}}}}
\newcommand{\abcllr}[1]{ABC\sub{LLR}}
\newcommand{\abcglm}[1]{ABC\sub{GLM}}
\newcommand{\integerPartition}[1]{\ensuremath{a({#1})}}
\newcommand{\uniqueModel}[2]{\ensuremath{M_{#1\protect\ifTwoArgs{#1}{#2}{,}#2}}}
\newcommand{\taxonLocusVector}[1]{\ensuremath{\{#1{1}{1},\ldots,#1{\npairs{}}{\nloci{\npairs{}}}\}}\xspace}
\newcommand{\taxonVector}[1]{\ensuremath{\{#1{1},\ldots,#1{\npairs{}}\}}\xspace}
\newcommand{\locusVector}[1]{\ensuremath{\{#1{1},\ldots,#1{\nlociTotal}\}}\xspace}

\newcommand{\simulationDescription}[2]{\change{Each plot represents #1
    simulation replicates using the same $#2$ samples from the prior}}
\newcommand{\simulationDistribution}{\ensuremath{\divt{} \sim U(0,
    \divt{max})}\xspace}
\newcommand{\estimateDescription}[2]{All estimates were obtained using #1 and #2}
\newcommand{\estimateDescriptionUncorrected}[1]{All estimates based on
    unadjusted posterior, \truncprior{}, obtained using #1}
\newcommand{\priorDescription}[4]{Prior settings were \priorSettings{#1}{#2}{#3}{#4}}
\newcommand{\priorSettings}[4]{$\divt{} \sim U(0, #1)$,
    $\meanDescendantTheta{} \sim U(#2, #3)$, and
    $\ancestralTheta{}{} \sim U(#2, #4)$}
\newcommand{\priorDescriptionBug}[4]{Prior settings were
    \priorSettingsBug{#1}{#2}{#3}{#4}}
\newcommand{\priorSettingsBug}[4]{$\divt{} \sim U(0, #1)$,
    $\meanDescendantTheta{} \sim U(#2, #3)$, and
    $\ancestralTheta{}{} \sim U(0.01, #4)$}
\newcommand{\simulationScheme}{simulations where \divt{} (in \globalcoalunit
    generations) for 22 population pairs is drawn from a series of uniform
    distributions, \simulationDistribution}
\newcommand{\captionPowerOmega}{Histograms of the estimated dispersion index
    of divergence times ($\hat{\vmratio{}}$) from \simulationScheme.
    The threshold for one divergence event \citep{Hickerson2006} is indicated
    by the dashed line, and the estimated probability of inferring one
    divergence event, $p(\hat{\vmratio{}}\le 0.01)$, is given for each
    \divt{max}}
\newcommand{\captionPowerPsiMode}{Histograms of the estimated number of
    divergence events ($\hat{\numt{}}$) from \simulationScheme.
    The estimated probability of inferring one divergence event,
    $p(\hat{\numt{}} = 1)$, is given for each \divt{max}}
\newcommand{\captionPowerPsi}{Histograms of the estimated posterior
    probability of one divergence event, \postprobnumt{1}, from
    \simulationScheme.
    The estimated probability of inferring one divergence event with a
    Bayes factor greater than 10 (dashed black line),
    $p(\bayesfactor{\numt{}=1}{\numt{} \ne 1} > 10)$, is given for each \divt{max}.
    The red line indicates $\postprobnumt{1} = 0.95$, and the estimated
    probability of inferring a posterior probability greater than 0.95 is given
    to the right of the line.}
\newcommand{\captionAccuracy}[1]{Accuracy and precision of #1 estimates from
    \simulationScheme.
    The proportion of estimates less than the true value ($p(\hat{#1}<#1)$) is
    given for each \divt{max}}
\newcommand{\samplingErrorTableNote}{An estimate of 1.0 for a posterior probability
    is an artifact of sampling error}


\newcommand{\refAccuracyALL}[1]{\labelcref{fig_acc_t_ss_llr_bug,fig_acc_t_ss_glm_bug,fig_acc_t_pls_llr_bug,fig_acc_t_pls_glm_bug,fig_acc_o_ss_llr_bug,fig_acc_o_ss_glm_bug,fig_acc_o_pls_llr_bug,fig_acc_o_pls_glm_bug}}
\newcommand{\refAccuracySS}[1]{\labelcref{fig_acc_t_ss_llr_bug,fig_acc_t_ss_glm_bug,fig_acc_o_ss_llr_bug,fig_acc_o_ss_glm_bug}}
\newcommand{\refAccuracySSfull}[1]{\labelcref{fig_acc_t_ssfull_llr_bug,fig_acc_t_ssfull_glm_bug,fig_acc_o_ssfull_llr_bug,fig_acc_o_ssfull_glm_bug}}
\newcommand{\refSSfull}[1]{\labelcref{fig_acc_t_ssfull_llr_bug,fig_acc_t_ssfull_glm_bug,fig_acc_o_ssfull_llr_bug,fig_acc_o_ssfull_glm_bug,fig_pow_o_ssfull_llr_bug,fig_pow_o_ssfull_glm_bug,fig_pow_psi_modes_ssfull_glm_bug}}
\newcommand{\refSS}[1]{\labelcref{fig_acc_t_ss_llr_bug,fig_acc_t_ss_glm_bug,fig_acc_o_ss_llr_bug,fig_acc_o_ss_glm_bug,fig_pow_o_ss_llr_bug,fig_pow_o_ss_glm_bug,fig_pow_psi_ss}}
\newcommand{\refAccuracyPLS}[1]{\labelcref{fig_acc_t_pls_llr_bug,fig_acc_t_pls_glm_bug,fig_acc_o_pls_llr_bug,fig_acc_o_pls_glm_bug}}
\newcommand{\refAccuracySScorrected}[1]{\labelcref{fig_acc_t_ss_llr_bug,fig_acc_t_ss_glm_bug,fig_acc_o_ss_llr_bug,fig_acc_o_ss_glm_bug}}
\newcommand{\refAccuracyPLScorrected}[1]{\labelcref{fig_acc_t_pls_llr_bug,fig_acc_t_pls_glm_bug,fig_acc_o_pls_llr_bug,fig_acc_o_pls_glm_bug}}
\newcommand{\refAccuracyUncorrected}[1]{\labelcref{fig_acc_t_ss_unc,fig_acc_t_pls_unc,fig_acc_o_ss_unc,fig_acc_o_pls_unc}}
\newcommand{\refAccuracyCorrected}[1]{\labelcref{fig_acc_t_ss_llr_bug,fig_acc_t_ss_glm_bug,fig_acc_t_pls_llr_bug,fig_acc_t_pls_glm_bug,fig_acc_o_ss_llr_bug,fig_acc_o_ss_glm_bug,fig_acc_o_pls_llr_bug,fig_acc_o_pls_glm_bug}}
\newcommand{\refAccuracyGLM}[1]{\labelcref{fig_acc_t_ss_glm_bug,fig_acc_t_pls_glm_bug,fig_acc_o_ss_glm_bug,fig_acc_o_pls_glm_bug}}
\newcommand{\refAccuracyLLR}[1]{\labelcref{fig_acc_t_ss_llr_bug,fig_acc_t_pls_llr_bug,fig_acc_o_ss_llr_bug,fig_acc_o_pls_llr_bug}}
\newcommand{\refAccuracyOmega}[1]{\labelcref{fig_acc_o_ss_llr_bug,fig_acc_o_ss_glm_bug,fig_acc_o_pls_llr_bug,fig_acc_o_pls_glm_bug}}
\newcommand{\refAccuracyOmegaUncorrected}[1]{\labelcref{fig_acc_o_ss_unc,fig_acc_o_pls_unc}}
\newcommand{\refAccuracyOmegaCorrected}[1]{\labelcref{fig_acc_o_ss_llr_bug,fig_acc_o_ss_glm_bug,fig_acc_o_pls_llr_bug,fig_acc_o_pls_glm_bug}}
\newcommand{\refAccuracyTime}[1]{\labelcref{fig_acc_t_ss_llr_bug,fig_acc_t_ss_glm_bug,fig_acc_t_pls_llr_bug,fig_acc_t_pls_glm_bug}}

\newcommand{\tn}{\tabularnewline}

\newcommand{\widthFigure}[5]{\begin{figure}[htbp]
\begin{center}
    \includegraphics[width=#1\textwidth]{#2}
    \captionsetup{#3}
    \caption{#4}
    \label{#5}
    \end{center}
    \end{figure}}

\newcommand{\heightFigure}[5]{\begin{figure}[htbp]
\begin{center}
    \includegraphics[height=#1]{#2}
    \captionsetup{#3}
    \caption{#4}
    \label{#5}
    \end{center}
    \end{figure}}

\newcommand{\mFigure}[3]{\widthFigure{1.0}{#1}{listformat=figList}{#2}{#3}\clearpage}
\newcommand{\siFigure}[3]{\widthFigure{1.0}{#1}{name=Figure S, labelformat=noSpace, listformat=sFigList}{#2}{#3}\clearpage}



\newcounter{commentCounter}
\newcommand{\revcomment}[1]{{\addtocounter{commentCounter}{1}}
    \medskip \hrule \noindent
\textbf{\arabic{section}.\arabic{commentCounter}}: {\sl #1}\par\xspace}
\newcommand{\response}[1]{{\addtolength{\leftskip}{0.25in} #1\par}\xspace}

\let\quoteOld\quote
\let\endquoteOld\endquote
% \renewenvironment{quote}{\sffamily\slshape\small\quoteOld}{\endquoteOld}
\renewenvironment{quote}{\sffamily\small\quoteOld}{\endquoteOld}

% \makeatletter
% \renewcommand{\section}{\@startsection{section}{1}{0mm}%
%     {-8pt}%
%     {4pt}%
% {\sffamily\large\itshape}}
% \makeatother

\begin{document}
% \raggedright
\setlength{\parindent}{0in}

We greatly appreciate the helpful comments of the editors and reviewers. In
this document, we provide responses to these comments, including descriptions
of the changes we have made.

\section{Editor's Comments}
\setcounter{commentCounter}{0}
\revcomment{
    Looking at Reviewer 2's list of issues, they appear to relate only to the
    presentation, and Reviewer 2 is not asking for more analysis.  These
    requests for revision are requests for comments, clarifications, and
    statements of caveats to be made by the author.  I think that if these
    requests are attended to then the ms will be improved.
}
\response{
    ADD!!
}

\revcomment{
    (Concerning Reviewer 2's comment 1.  I did not get the impression that the
    author was criticizing Bayesian foundations, but this Reviewer did, and
    that in itself shows that there is room for misunderstanding.  I suspect
    that the author may feel that the intention of the paragraph is clear
    enough, but perhaps the wording here can be adjusted in the light of this
    reviewer's comment to more clearly say that you are talking about options
    for prior choice, that msBayes has opted for uniform priors, and your
    comments on that.)
}
\response{
    ADD!!
}

\revcomment{
    I would like to ask that in replying to the requests for revision that a
    point-by-point be made to make it easy for us to see what you have done."
}
\response{
    ADD!!
}


\section{Reviewer 1's Comments}
\setcounter{commentCounter}{0}
\revcomment{
    In reference to ``Bayesian model-choice method'' in the second paragraph
    of page 2, Reviewer 1 requested a clearer description of the models being
    sampled by such a method.
}
\response{
    I have added the following sentence to this paragraph that provides a
    specific possibility for the model sample space and, thanks to Reviewer 1's
    suggestion, provides a smoother transition into the description of the \msb
    model in the following paragraph:
    \begin{quote}
        More specifically, the sample space of such a model-choice procedure
        could include all models ranging from a single divergence-time
        parameter (i.e., simultaneous divergence of all co-distributed taxa) to
        the fully generalized model in which each taxon diverged at a unique
        time.
    \end{quote}
}

\revcomment{
    Reviewer 1 felt the equating of ``biogeographically interesting models''
    with models of shared divergences was unnecessarily subjective.
}
\response{
    I agree. I have removed the use of the model descriptor ``biogeographically
    interesting'' from the manuscript, and have added the following to the
    Conclusions:
    \begin{quote}
        Compared to \msb, the new approach better estimates posterior
        uncertainty, which greatly reduces the chances of incorrectly
        estimating biogeographical scenarios of shared divergence events.  This
        is important, because models of shared divergence events are often of
        particular interest to researchers who employ these methods.
    \end{quote}
    Thanks to Reviewer 1's comment, this is a much clearer description of what
    I was intending to convey with ``biogeographically interesting.''
}

\revcomment{
    Many readers may be generally unaware of several important factors of how
    phylogeographers use msBayes. I'd suggest a review paragraph somewhere in
    the intro that describes your empirical paper and others such that the
    readers have a better concept of why these technical details matter.
}
\response{
    Good point. I have added the following paragraph to the introduction to
    familiarize readers with how \msb has been applied, and the general results
    of such applications.
    \begin{quote}
        \msb has been used to address biogeographical questions in variety of
        empirical systems. Some examples include (1) whether rising of the
        Isthmus of Panama caused co-divergence across species of echinoids
        co-distributed across the Pacific and Atlantic sides of the ishmus
        \cite{Hickerson2006}, (2) if an historical seaway across the Baja
        Peninsula caused co-divergence across species of squamates and mammals
        co-distributed both north and south of the putative seaway
        \cite{Leache2007}, (3) if species of gall-wasps and their associated
        parasitoids share divergences across putative glacial refugia
        \cite{Stone2012}, and (4) whether repeated coalescence and
        fragmentation of the oceanic Islands of the Philippines during by
        Pleistocene sea-level fluctuations caused diversification of vertebrate
        taxa distributed across the islands \cite{Oaks2012}.  Such applications
        of the method often result in strong posterior support for
        co-divergence among some of the taxa investigated (e.g.,
        \cite{Barber2010,Carnaval2009,Chan2011,Hickerson2006,Leache2007,Plouviez2009,Stone2012,Voje2009,Oaks2012}).
    \end{quote}
}

\revcomment{
    In reference to ``I provide the ability to control the richness of the model'' on Page 9,
    Reviewer 1 commented:

    While I appreciate how this flexibility is a selling point to someone who
    is very familiar with this model (such as the author), it might also be
    viewed as an increased difficulty for the average user. Is this to say that
    the 'richness' of the model has to be specified by the user, or can be
    specified? Please clarify.
}
\response{
    To clarify that the user is not required to control the richness of the
    model, I have changed the subsequent sentences to:
    \begin{quote}
        For the \myTheta{} parameters, by default, the model is fully
        generalized to allow each population pair to have three parameters:
        \ancestralTheta{}, \descendantTheta{1}{}, and \descendantTheta{2}{}.
        Furthermore, if an investigator prefers to reduce the number of
        parameters, any model of \myTheta{} parameters nested within this
        general model can also be specified, including the most restricted
        model where the ancestral and descendant populations of each pair share
        a single \myTheta{} parameter.
    \end{quote}
}

\revcomment{
    Reviewer 1 commented that I awkwardly started two consecutive sentences
    with ``However'' on Page 11.
}
\response{
    Thank you for catching this. I have removed ``However'' from the beginning
    of the first sentence, because it was unnecessary.
}

\revcomment{
    Reviewer 1 commented that I was underselling the results in section
    ``Validation analyses: Ordered divergence models'' on Page 16.
}
\response{
    I have expanded this section by including the following sentences:
    \begin{quote}
        ADD!!
    \end{quote}
}

\revcomment{
    Given these caveats, what (in your opinion) is the best usage of this
    approach to phylogeographic analysis?
}
\response{
    ADD!!!
}

\revcomment{
    Reviewer 1 requested that we elaborate about how Robert et al's
    \cite{Robert2011} critique of ABC model choice relates to \dppmsbayes.
}
\response{
    I have expanded:
    \begin{quote}
        I also urge caution when using dpp-msbayes due to the fact that ABC
        methods in general are known to be biased for model choice
        \cite{Robert2011}.
    \end{quote}
    to:
    \begin{quote}
        I also urge caution when using \dppmsbayes due to the lack of
        theoretical validation of Bayesian model choice when the full data are
        replaced by summary statistics that are insufficient for discriminating
        across models under comparision \cite{Robert2011}, which is certainly
        the case here.  Robert et al.\ \cite{Robert2011} demonstrated that ABC
        estimates of model posterior probabilities can be biased when such
        across-model insufficient statistics are used \cite{Robert2011}.
    \end{quote}
    I think this makes it more clear how the findings of \cite{Robert2011} are
    potentially relevant to the posterior probabilities estimated by
    \dppmsbayes and \msb.
}

\section{Reviewer 2's Comments}
\setcounter{commentCounter}{0}

\revcomment{
    The complaint that ``This causes divergence models with more divergence-time
    parameters to integrate over a much greater parameter space with low
    likelihood yet high prior density, resulting in small marginal likelihoods
    relative to models with fewer divergence-time parameters.'' (p.2) is
    essentially stressing the point with Occam's razor argument. Which I deem a
    rather positive feature of Bayesian model choice. A reflection on the
    determination of the prior distribution away from uniform priors thus
    sounds most timely, rather than a criticism of the Bayesian foundations.
}
\response{
    ADD!!
}

\revcomment{
    The current paper claims to have reached a satisfactory prior modelling
    with ``improved robustness, accuracy, and power'' (p.3). If I understand
    correctly, the changes are in replacing a uniform distribution with a Gamma
    or a Dirichlet prior.  Which means introducing a seriously large number of
    hyperparameters into the picture. Having a lot of flexibility in the prior
    also means a lot of variability in the resulting inference. This is a
    general feature of all Bayesian modellings and I am rather surprised by the
    conclusion that things look better by taking this step away from the
    uniform prior. Depending on the problem they could also look worse.
}
\response{
    ADD!!
}

\revcomment{
    The ABC part is rather standard, except for the rather strange feature that
    the divergence times are used to construct summary statistics (p.10). I
    find it strange because these times are not observed for the actual data.
    So I must be missing something, which should be spelled out more clearly.
    (And I object to the sentence ``I have introduced a new approximate-Bayesian
    model choice method.'' (p.21) since there is no new ABC technique, simply a
    different form of prior.)
}
\response{
    ADD!!
}

\revcomment{
    I do not understand much the part about (reshuffling) introducing bias as
    detailed on p.11: every approximate method gives a ``biased'' answer in the
    sense this answer is not the true and proper posterior distribution. Using
    a different (re-ordered) vector of statistics provides a different ABC
    outcome, hence a different approximate posterior, for which it seems truly
    impossible to check whether or not it increases the discrepancy from the
    true posterior, compared with the other version. I always find it annoying
    to see the word `bias' used in a vague meaning and esp. within a Bayesian
    setting. All Bayesian methods are biased.
}
\response{
    ADD!!
}

\revcomment{
    Quoting [43] as concluding that ABC model choice is biased is
    somewhatmisleading: the warning therein was that Bayes factors and
    posterior probabilities resulting from those ABC steps could be quite
    unrelated with those based on the whole dataset. That the same authors
    concluded that the proper choice of summary statistics leads to a
    consistent model choice shows ABC model choice is not necessarily
    ``biased''.
}
\response{
    ADD!!
}

\revcomment{
    I also fail to understand why the posterior probability of model i should
    be distributed as a uniform (``If the method is unbiased, the points should
    fall near the identity line'') when the data is from model i: this is not a
    p-value but a posterior probability and the posterior probability is not
    the frequentist coverage.
}
\response{
    ADD!!
}

\revcomment{
    My overall problem with the paper is that, all in all, this is a single
    Monte Carlo study about the performances of some classes of priors and, as
    such, it does not carry enough weight to validate an approach that remains
    highly subjective in the selection of its hyperparameters. I think this
    remains a controlled experiment with simulated data where the true
    parameters are know and the prior is ``true''. This obviously helps in
    getting better performances. In a different and less controlled experiment,
    a new experimenter may (should?) be at a loss as to which hyperparameters
    to chose.
}
\response{
    ADD!!
}

\revcomment{
    The conclusion of the paper that ``With improving numerical methods (...),
    advances in Monte Carlo techniques and increasing efficiency of likelihood
    calculations, analyzing rich comparative phylo-geographical models in a
    full-likelihood Bayesian framework is becoming computationally feasible.''
    (p.21) sounds overly optimistic and rather premature. As to this day, I do
    not know of any significant advance in computing the observed likelihood
    for the population genetics models ABC is currently handling. (The SMC
    algorithm of Bouchard-Côté, Sankaraman and Jordan 2012 does not apply to
    Kingman's coalescent, as far as I can judge.) This is certainly a goal
    worth pursuing and borrowing strength from multiple techniques cannot hurt,
    but it remains so far a lofty goal, still beyond our reach... I thus think
    the major message of the paper is to reinforce our earlier calls for
    caution when interpreting the output of an ABC model choice (p.20), or even
    of a regular Bayesian analysis, agreeing that we should aim at seeing ``a
    large amount of posterior uncertainty'' rather than posterior probability
    values close to 0 and 1.
}
\response{
    ADD!!
}

\bibliographystyle{../../bib/bmc-mathphys.bst}
\bibliography{../../bib/references}

\end{document}

