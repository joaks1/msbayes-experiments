\documentclass[letterpaper]{letter}
\usepackage{anysize}
\marginsize{1.25in}{1.25in}{0.5in}{0.5in}
\usepackage{setspace}
\usepackage{url}
\usepackage{verbatim}
\usepackage{parskip}
\usepackage{xspace}
\usepackage{hyperref}
\hypersetup{pdfborder={0 0 0}, colorlinks=true, urlcolor=blue}

\newcommand{\ignore}[1]{}
\newcommand{\super}[1]{\ensuremath{^{\textrm{#1}}}}
\newcommand{\sub}[1]{\ensuremath{_{\textrm{#1}}}}
\newcommand{\dC}{\ensuremath{^\circ{\textrm{C}}}}
\newcommand{\highLight}[1]{\textcolor{magenta}{\MakeUppercase{#1}}\xspace}

\makeatletter
\let\@texttop\relax
\makeatother

\newenvironment{myEnumerate}{
  \begin{enumerate}
    \setlength{\itemsep}{1pt}
    \setlength{\parskip}{0pt}
    \setlength{\parsep}{0pt}}
  {\end{enumerate}}

\newenvironment{myItemize}{
  \begin{itemize}
    \setlength{\itemsep}{1pt}
    \setlength{\parskip}{0pt}
    \setlength{\parsep}{0pt}}
  {\end{itemize}}

%%%%%%%%%%%%%%%%%%%%%%%%%%%%%%%%%%%%%%%%%%%%%%%%%%%%%%%%%%%%%%

\signature{Jamie Oaks \\ 
                 Postdoctoral Fellow \\ 
                 Department of Biology \\
                 University of Washington \\ 
                 Box 351800 \\ 
                 Seattle, WA 98195 \\ 
                 \href{mailto:joaks1@uw.edu}{\texttt{joaks1@uw.edu}} \\ 
                 \href{http://www.phyletica.com}{\texttt{www.phyletica.com}}}
\begin{document}
\begin{letter}{Journal/Society \\
                address \\
                here}
\opening{Dear Editor:}
% \raggedright
I am submitting a manuscript entitled ``An Improved Approximate-Bayesian
Model-choice Method for Estimating Shared Evolutionary History'' to be
considered for publication in \highLight{journal here}.
The field of research is comparative biogeography and Bayesian model choice.
All of the computer code for the newly implemented model I present in the
the manuscript is freely available in the open-source
software packages \texttt{dpp-msbayes} and \texttt{PyMsBayes}
(\href{https://github.com/joaks1/dpp-msbayes}{\url{https://github.com/joaks1/dpp-msbayes}}
and
\href{https://github.com/joaks1/PyMsBayes}{\url{https://github.com/joaks1/PyMsBayes}},
respectively).
Also, I performed the work described in the manuscript following the principles
of Open Notebook Science; using version-control software, I make progress in
all aspects of the work freely and publicly available in real-time at
\href{https://github.com/joaks1/msbayes-experiments}{\url{https://github.com/joaks1/msbayes-experiments}}.
All information necessary to reproduce my results is provided there.

In this paper, I introduce a novel Bayesian method for estimating the
probability of shared divergence histories across a set of taxa from
multi-locus DNA sequence data.
More specifically, given a set of species pairs, the method estimates
the probabilities of all possible models in which multiple sets of taxa
can diverge at the same time.
These models range from a single divergence-time parameter (i.e., simultaneous
divergence of all pairs of taxa) to the fully generalized model in which each
pair of taxa diverged at a unique time.
The current state-of-the-art method for this estimation problem is implemented
in the software package \texttt{msBayes}.
By introducing more flexible priors on the parameters the model and a
non-parametric Dirichlet-process prior over divergence models, the new method
has improved robustness, accuracy, and power for estimating shared evolutionary
history across taxa when compared to \texttt{msBayes}.
Also, simulations have shown that \texttt{msBayes} will often to incorrectly
support biogeographically interesting scenarios of shared divergences when
divergences were random over broad time periods.
My results show that the new method is \emph{much} less likely to make such
misleading and costly errors.

The new method provides an important tool for evolutionary biologists to
investigate how large-scale changes to the environment, including geological
and climatic events, have affected the evolutionary history of entire
communities of co-distributed species and their associated microbiota.
Given the dynamic nature of our planet, such biogeographical processes likely
play a significant role regulating biodiversification and community assembly.

Thank you for your consideration, and I look forward to your correspondence
regarding this manuscript.

\addtolength{\medskipamount}{-5pt}
\closing{Sincerely,}
\end{letter}

\end{document}
