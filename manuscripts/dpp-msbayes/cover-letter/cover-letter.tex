\documentclass[letterpaper,11pt]{letter}
\usepackage{anysize}
\marginsize{0.8in}{0.8in}{-0.2in}{0.3in}
\usepackage{setspace}
\usepackage{url}
\usepackage{verbatim}
\usepackage{parskip}
\usepackage{xspace}
\usepackage{hyperref}
\hypersetup{pdfborder={0 0 0}, colorlinks=true, urlcolor=blue}

\newcommand{\ignore}[1]{}
\newcommand{\super}[1]{\ensuremath{^{\textrm{#1}}}}
\newcommand{\sub}[1]{\ensuremath{_{\textrm{#1}}}}
\newcommand{\dC}{\ensuremath{^\circ{\textrm{C}}}}
\newcommand{\highLight}[1]{\textcolor{magenta}{\MakeUppercase{#1}}\xspace}

\makeatletter
\let\@texttop\relax
\makeatother

\newenvironment{myEnumerate}{
  \begin{enumerate}
    \setlength{\itemsep}{1pt}
    \setlength{\parskip}{0pt}
    \setlength{\parsep}{0pt}}
  {\end{enumerate}}

\newenvironment{myItemize}{
  \begin{itemize}
    \setlength{\itemsep}{1pt}
    \setlength{\parskip}{0pt}
    \setlength{\parsep}{0pt}}
  {\end{itemize}}

%%%%%%%%%%%%%%%%%%%%%%%%%%%%%%%%%%%%%%%%%%%%%%%%%%%%%%%%%%%%%%

\signature{Jamie Oaks, Postdoctoral Fellow \\ 
                 Department of Biology \\
                 University of Washington \\ 
                 % Box 351800 \\ 
                 Seattle, WA 98195, USA \\
                 \href{mailto:joaks1@uw.edu}{\texttt{joaks1@uw.edu}} \\ 
                 \href{http://www.phyletica.com}{\texttt{www.phyletica.com}}}
\begin{document}
\begin{letter}{\emph{BMC Evolutionary Biology} \\
                        BioMed Central \\
                        236 Gray's Inn Road \\
                        London WC1X 8HB \\
                        United Kingdom}
\opening{Dear Editor:}
% \raggedright
I am submitting a manuscript entitled ``An Improved Approximate-Bayesian
Model-choice Method for Estimating Shared Evolutionary History'' to be
considered for publication in \emph{BMC Evolutionary Biology}.
This is a methodological article, and the field of research is comparative
biogeography and Bayesian model choice.

In the paper, I introduce a novel Bayesian method for estimating the
probability of shared divergence histories across a set of taxa from
multi-locus DNA sequence data.
The new method provides an important tool for evolutionary biologists to
investigate how large-scale changes to the environment, including geological
and climatic events, have affected the evolutionary history of entire
communities of co-distributed species and their associated microbiota.
Given the dynamic nature of our planet, such biogeographical processes likely
play a significant role in regulating biodiversification and community assembly.

More specifically, given a set of species pairs, the method estimates
the probabilities of all possible models in which multiple sets of taxa
can diverge at the same time.
These models range from a single divergence event shared across all of the
taxa, to the fully generalized model in which each pair of taxa diverged at a
unique time.
The current state-of-the-art method for this estimation problem is implemented
in the popular software package \texttt{msBayes}, which has a broad community
of users.
I introduce a non-parametric Bayesian approach to this inference problem that
uses a Dirichlet-process prior over all possible divergence models.
Using extensive simulations and applications to empirical data, I demonstrate
that the new method has improved robustness, accuracy, and power for estimating
shared evolutionary history across taxa when compared to \texttt{msBayes}.
Also, previous simulation work has shown that \texttt{msBayes} often
incorrectly supports biogeographically compelling scenarios of simultaneous
divergence across taxa, even when divergences were random over broad time
periods.
Importantly, my results show that the new method is \emph{much} less likely to
make such misleading and costly errors.

All of the computer code for the new method presented in the the manuscript is
freely available in the open-source software packages \texttt{dpp-msbayes} and
\texttt{PyMsBayes}
(\href{https://github.com/joaks1/dpp-msbayes}{\url{https://github.com/joaks1/dpp-msbayes}}
and
\href{https://github.com/joaks1/PyMsBayes}{\url{https://github.com/joaks1/PyMsBayes}},
respectively).
Also, I performed the work described in the manuscript following the principles
of Open Notebook Science; using version-control software, I make progress in
all aspects of the work freely and publicly available in real-time at
\href{https://github.com/joaks1/msbayes-experiments}{\url{https://github.com/joaks1/msbayes-experiments}}.
All information necessary to reproduce my results is provided there.

Thank you for your consideration, and I look forward to your correspondence
regarding this manuscript.

\addtolength{\medskipamount}{-6pt}
\closing{Sincerely,}
\end{letter}

\end{document}
