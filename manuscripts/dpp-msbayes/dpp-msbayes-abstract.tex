\ifbmc{\parttitle{Background}}{}
To understand biological diversification, it is important to account for
large-scale processes that affect the evolutionary history of groups of
co-distributed populations of organisms.
Such events predict temporally clustered divergences times, a pattern
that can be estimated using genetic data from co-distributed species.
I introduce a new approximate-Bayesian method for comparative
phylogeographical model-choice that estimates the temporal distribution of
divergences across taxa from multi-locus DNA sequence data.
The model is an extension of that implemented in \msb.
\ifbmc{\parttitle{Results}}{}
By reparameterizing the model, introducing more flexible priors on
demographic and divergence-time parameters, and implementing a
non-parametric Dirichlet-process prior over divergence models, I improved
the robustness, accuracy, and power of the method for estimating shared
evolutionary history across taxa.
\ifbmc{\parttitle{Conclusions}}{}
The results demonstrate the improved performance of the new method is due
to (1) more appropriate priors on divergence-time and demographic
parameters that avoid prohibitively small marginal likelihoods for models
with more divergence events,
and (2) the Dirichlet-process providing a flexible prior on divergence
histories that does not strongly disfavor models with intermediate numbers
of divergence events.
% The results demonstrate that the tendency of the \msb model to spuriously
% support models of clustered divergences is caused by a combination of (1)
% uniform priors on nuisance parameters hindering the marginal likelihoods of
% models with more divergence-time parameters, and (2) a prior on divergence
% models that disfavors models with intermediate numbers of divergence-time
% parameters.
The new method yields more robust estimates of posterior uncertainty, and
thus greatly reduces the tendency to incorrectly estimate
\change{models of shared evolutionary history} with strong support.
% biogeographically interesting models with strong support.
