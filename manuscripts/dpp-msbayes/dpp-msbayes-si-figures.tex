%% Validation accuracy
\siFigure{../../validation/images/validation-accuracy-omega.pdf}{
    \validationAccuracyComparisonCaption{unadjusted}{\divTimeDispersion}
}{figValidationAccuracyOmega}

\siFigure{../../validation/images/validation-accuracy-omega-glm.pdf}{
    \validationAccuracyComparisonCaption{GLM-adjusted}{\divTimeDispersion}
}{figValidationAccuracyOmegaGlm}


\siFigure{../../validation/images/validation-accuracy-time.pdf}{
    \validationAccuracyComparisonCaption{unadjusted}{\divTimeMean}
}{figValidationAccuracyTime}

\siFigure{../../validation/images/validation-accuracy-time-glm.pdf}{
    \validationAccuracyComparisonCaption{GLM-adjusted}{\divTimeMean}
}{figValidationAccuracyTimeGlm}


\siFigure{../../validation/images/validation-accuracy-psi.pdf}{
    \validationAccuracyComparisonCaption{unadjusted}{\divTimeNum}
    \jitterComment
}{figValidationAccuracyPsi}

\siFigure{../../validation/images/validation-accuracy-psi-glm.pdf}{
    \validationAccuracyComparisonCaption{GLM-adjusted}{\divTimeNum}
    \jitterComment
}{figValidationAccuracyPsiGlm}



%% Validation model choice
\siFigure{../../validation/images/validation-model-choice-psi-glm.pdf}{
    \validationModelChoiceComparisonCaption{GLM-adjusted}{$\divTimeNum = 1$}
}{figValidationModelChoicePsiGlm}

\siFigure{../../validation/images/validation-model-choice-omega-glm.pdf}{
    \validationModelChoiceComparisonCaption{GLM-adjusted}{$\divTimeDispersion < 0.01$}
}{figValidationModelChoiceOmegaGlm}


%% Validation ordered models
\siFigure{../../validation/no-sort/results/pymsbayes-results/plots/prior-dpp_prior-dpp_accuracy.pdf}{
    \validationAccuracyCaption{\modelDPPOrdered}{\modelDPPOrdered}
}{figValidationAccuracyDPPOrderedDPPOrdered}

\siFigure{../../validation/no-sort/results/pymsbayes-results/plots/prior-dpp_prior-dpp_mc_behavior.pdf}{
    \validationModelChoiceCaption{\modelDPPOrdered}{\modelDPPOrdered}
}{figValidationModelChoiceDPPOrderedDPPOrdered}


%% Power accuracy results
\siSidewaysFigure{../../power-comparison/images/old-power-omega-accuracy-6.pdf}{
    \powerAccuracyComparisonCaption{\powerSeriesOld}
}{figPowerAccuracyOld}

\siSidewaysFigure{../../power-comparison/images/uniform-power-omega-accuracy-6.pdf}{
    \powerAccuracyComparisonCaption{\powerSeriesUniform}
}{figPowerAccuracyUniform}

\siSidewaysFigure{../../power-comparison/images/exp-power-omega-accuracy-6.pdf}{
    \powerAccuracyComparisonCaption{\powerSeriesExp}
}{figPowerAccuracyExp}


%% Power psi results
\siSidewaysFigure{../../power-comparison/images/old-power-psi-6.pdf}{
    \powerComment{\powerSeriesOld}
    \powerPsiComment{\powerSeriesOld}
    \timeConversionComment
}{figPowerPsiOld}

\siSidewaysFigure{../../power-comparison/images/uniform-power-psi-6.pdf}{
    \powerComment{\powerSeriesUniform}
    \powerPsiComment{\powerSeriesUniform}
    \timeConversionComment
}{figPowerPsiUniform}

\siSidewaysFigure{../../power-comparison/images/exp-power-psi-6.pdf}{
    \powerComment{\powerSeriesExp}
    \powerPsiComment{\powerSeriesExp}
    \timeConversionComment
}{figPowerPsiExp}


%% Power dispersion results
\siSidewaysFigure{../../power-comparison/images/old-power-omega-6.pdf}{
    \powerComment{\powerSeriesOld}
    \powerDispersionComment{\powerSeriesOld}
    \timeConversionComment
}{figPowerOmegaOld}

\siSidewaysFigure{../../power-comparison/images/uniform-power-omega-6.pdf}{
    \powerComment{\powerSeriesUniform}
    \powerDispersionComment{\powerSeriesUniform}
    \timeConversionComment
}{figPowerOmegaUniform}

\siSidewaysFigure{../../power-comparison/images/exp-power-omega-6.pdf}{
    \powerComment{\powerSeriesExp}
    \powerDispersionComment{\powerSeriesExp}
    \timeConversionComment
}{figPowerOmegaExp}


%% Power psi probability results
\siSidewaysFigure{../../power-comparison/images/old-power-psi-prob-6.pdf}{
    \powerSupportComment{\powerSeriesOld}
    \powerProbComment{$p(\divTimeNum = 1 | \ssSpace)$}{\powerSeriesOld}
    \timeConversionComment
}{figPowerPsiProbOld}

\siSidewaysFigure{../../power-comparison/images/uniform-power-psi-prob-6.pdf}{
    \powerSupportComment{\powerSeriesUniform}
    \powerProbComment{$p(\divTimeNum = 1 | \ssSpace)$}{\powerSeriesUniform}
    \timeConversionComment
}{figPowerPsiProbUniform}

\siSidewaysFigure{../../power-comparison/images/exp-power-psi-prob-6.pdf}{
    \powerSupportComment{\powerSeriesExp}
    \powerProbComment{$p(\divTimeNum = 1 | \ssSpace)$}{\powerSeriesExp}
    \timeConversionComment
}{figPowerPsiProbExp}


%% Power dispersion prob results
\siSidewaysFigure{../../power-comparison/images/old-power-omega-prob-6.pdf}{
    \powerSupportComment{\powerSeriesOld}
    \powerProbComment{$p(\divTimeDispersion < 0.01 | \ssSpace)$}{\powerSeriesOld}
    \timeConversionComment
}{figPowerOmegaProbOld}

\siSidewaysFigure{../../power-comparison/images/uniform-power-omega-prob-6.pdf}{
    \powerSupportComment{\powerSeriesUniform}
    \powerProbComment{$p(\divTimeDispersion < 0.01 | \ssSpace)$}{\powerSeriesUniform}
    \timeConversionComment
}{figPowerOmegaProbUniform}

\siSidewaysFigure{../../power-comparison/images/exp-power-omega-prob-6.pdf}{
    \powerSupportComment{\powerSeriesExp}
    \powerProbComment{$p(\divTimeDispersion < 0.01 | \ssSpace)$}{\powerSeriesExp}
    \timeConversionComment
}{figPowerOmegaProbExp}

% Empirical results
\siEightFigure{../../empirical-analyses/plots/philippines-dpp-div-models.pdf}{
    The divergence-model results when the 22 pairs of taxa from the Philippines
    are analyzed under the \empModelDPP model (Table~\ref{tabEmpiricalModels}).
    The 10 unordered divergence models with highest posterior probability
    ($p(\divTimeIndexVector \given \ssSpace)$) are shown, where the numbers
    indicate the inferred number of taxon pairs that diverged at each event.
    The times indicate the posterior median and 95\% highest posterior density
    (HPD) interval conditional on each divergence model.
    For each model, times are summarized across posterior samples by the number
    of taxon pairs associated with each divergence. For models in which there
    are multiple divergence events with the same number of taxon pairs, the
    events are sorted by time to summarize the divergence times in a consistant
    way.
}{figDivModelsDPP}

\siEightFigure{../../empirical-analyses/plots/philippines-dpp-inform-div-models.pdf}{
    The divergence-model results when the 22 pairs of taxa from the Philippines
    are analyzed under the \empModelDPPInform model
    (Table~\ref{tabEmpiricalModels}).
    The 10 unordered divergence models with highest posterior probability
    ($p(\divTimeIndexVector \given \ssSpace)$) are shown, where the numbers
    indicate the inferred number of taxon pairs that diverged at each event.
    The times indicate the posterior median and 95\% highest posterior density
    (HPD) interval conditional on each divergence model.
    For each model, times are summarized across posterior samples by the number
    of taxon pairs associated with each divergence. For models in which there
    are multiple divergence events with the same number of taxon pairs, the
    events are sorted by time to summarize the divergence times in a consistant
    way.
}{figDivModelsDPPInform}

\siEightFigure{../../empirical-analyses/plots/philippines-dpp-simple-div-models.pdf}{
    The divergence-model results when the 22 pairs of taxa from the Philippines
    are analyzed under the \empModelDPPSimple model
    (Table~\ref{tabEmpiricalModels}).
    The 10 unordered divergence models with highest posterior probability
    ($p(\divTimeIndexVector \given \ssSpace)$) are shown, where the numbers
    indicate the inferred number of taxon pairs that diverged at each event.
    The times indicate the posterior median and 95\% highest posterior density
    (HPD) interval conditional on each divergence model.
    For each model, times are summarized across posterior samples by the number
    of taxon pairs associated with each divergence. For models in which there
    are multiple divergence events with the same number of taxon pairs, the
    events are sorted by time to summarize the divergence times in a consistant
    way.
}{figDivModelsDPPSimple}

\siEightFigure{../../empirical-analyses/plots/philippines-uniform-div-models.pdf}{
    The divergence-model results when the 22 pairs of taxa from the Philippines
    are analyzed under the \empModelUniform model
    (Table~\ref{tabEmpiricalModels}).
    The 10 unordered divergence models with highest posterior probability
    ($p(\divTimeIndexVector \given \ssSpace)$) are shown, where the numbers
    indicate the inferred number of taxon pairs that diverged at each event.
    The times indicate the posterior median and 95\% highest posterior density
    (HPD) interval conditional on each divergence model.
    For each model, times are summarized across posterior samples by the number
    of taxon pairs associated with each divergence. For models in which there
    are multiple divergence events with the same number of taxon pairs, the
    events are sorted by time to summarize the divergence times in a consistant
    way.
}{figDivModelsUniform}

\siEightFigure{../../empirical-analyses/plots/philippines-old-div-models.pdf}{
    The divergence-model results when the 22 pairs of taxa from the Philippines
    are analyzed under the \empModelOld model
    (Table~\ref{tabEmpiricalModels}).
    The 10 unordered divergence models with highest posterior probability
    ($p(\divTimeIndexVector \given \ssSpace)$) are shown, where the numbers
    indicate the inferred number of taxon pairs that diverged at each event.
    The times indicate the posterior median and 95\% highest posterior density
    (HPD) interval conditional on each divergence model.
    For each model, times are summarized across posterior samples by the number
    of taxon pairs associated with each divergence. For models in which there
    are multiple divergence events with the same number of taxon pairs, the
    events are sorted by time to summarize the divergence times in a consistant
    way.
}{figDivModelsOld}

\siEightFigure{../../empirical-analyses/plots/negros-panay-dpp-unordered-div-models.pdf}{
    The divergence-model results when the 9 pairs of taxa from the Islands of
    Negros and Panay are analyzed under the \npModelDPP model sampling over
    unordered models of divergence.
    (Table~\ref{tabEmpiricalModels}).
    The 10 unordered divergence models with highest posterior probability
    ($p(\divTimeIndexVector \given \ssSpace)$) are shown, where the numbers
    indicate the inferred number of taxon pairs that diverged at each event.
    The times indicate the posterior median and 95\% highest posterior density
    (HPD) interval conditional on each divergence model.
    For each model, times are summarized across posterior samples by the number
    of taxon pairs associated with each divergence. For models in which there
    are multiple divergence events with the same number of taxon pairs, the
    events are sorted by time to summarize the divergence times in a consistant
    way.
}{figDivModelsNP}

\siEightFigure{../../empirical-analyses/plots/negros-panay-dpp-ordered-div-models.pdf}{
    The divergence-model results when the 9 pairs of taxa from the Islands of
    Negros and Panay are analyzed under the \npModelDPPOrdered model sampling
    over ordered models of divergence.
    (Table~\ref{tabEmpiricalModels}).
    The posterior sample of divergence models were summarized while ignoring
    the identity of the taxon pairs in order to compare the results of the
    \npModelDPP model.
    The 10 unordered divergence models with highest posterior probability
    ($p(\divTimeIndexVector \given \ssSpace)$) are shown, where the numbers
    indicate the inferred number of taxon pairs that diverged at each event.
    The times indicate the posterior median and 95\% highest posterior density
    (HPD) interval conditional on each divergence model.
    For each model, times are summarized across posterior samples by the number
    of taxon pairs associated with each divergence. For models in which there
    are multiple divergence events with the same number of taxon pairs, the
    events are sorted by time to summarize the divergence times in a consistant
    way.
}{figDivModelsNPOrdered}

\siFigure{../../empirical-analyses/plots/negros-panay-marginal-times.pdf}{
    The marginal divergence-time results when the 9 pairs of taxa from the
    Islands of Negros and Panay are analyzed under the \npModelDPPOrdered model
    that samples over ordered models of divergence
    (Table~\ref{tabEmpiricalModels}).
}{figMarginalTimes}
