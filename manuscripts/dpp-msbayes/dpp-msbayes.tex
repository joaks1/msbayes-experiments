%&<latex>
\documentclass[letterpaper,12pt]{article}

%%%%%%%%%%%%%%%%%%%%%%%%%%%%%%%%%%%%%%%%%%%%%%%%%%%%%%%%%%%%
%% preamble %%%%%%%%%%%%%%%%%%%%%%%%%%%%%%%%%%%%%%%%%%%%%%%%
\documentclass[table]{beamer}
\usepackage{beamerthemesplit}
\usetheme{boxes}
\usecolortheme{seahorse}
% \useinnertheme{myboxes}
% \usepackage{amsmath}
% \usepackage[fleqn]{amsmath}
\usepackage{ifthen}
\usepackage{xspace}
\usepackage{multirow}
\usepackage{booktabs}
\usepackage{xcolor}
\usepackage[style=nature]{biblatex}
\bibliography{../manuscripts/bib/references}
\newrobustcmd*{\footlessfullcite}{\AtNextCite{\renewbibmacro{title}{}\renewbibmacro{in:}{}}\footfullcite}
\AtEveryBibitem{\clearfield{month}}
\AtEveryCitekey{\clearfield{month}}

% Make all footnotes smaller
\renewcommand{\footnotesize}{\scriptsize}

\definecolor{myGray}{gray}{0.9}
\colorlet{rowred}{red!30!white}

\setbeamertemplate{blocks}[rounded][shadow=true]

\setbeamercolor{defaultcolor}{bg=structure!30!normal text.bg,fg=black}
\setbeamercolor{block body}{bg=structure!30!normal text.bg,fg=black}
\setbeamercolor{block title}{bg=structure!50!normal text.bg,fg=black}

\newenvironment<>{varblock}[2][\textwidth]{%
  \setlength{\textwidth}{#1}
  \begin{actionenv}#3%
    \def\insertblocktitle{#2}%
    \par%
    \usebeamertemplate{block begin}}
  {\par%
    \usebeamertemplate{block end}%
  \end{actionenv}}

\newenvironment{displaybox}[1][\textwidth]
{
    \centerline\bgroup\hfill
    \begin{beamerboxesrounded}[lower=defaultcolor,shadow=true,width=#1]{}
}
{
    \end{beamerboxesrounded}\hfill\egroup
}

\newenvironment{onlinebox}[1][4cm]
{
    \newbox\mybox
    \newdimen\myboxht
    \setbox\mybox\hbox\bgroup%
        \begin{beamerboxesrounded}[lower=defaultcolor,shadow=true,width=#1]{}
    \centering
}
{
    \end{beamerboxesrounded}\egroup
    \myboxht\ht\mybox
    \raisebox{-0.25\myboxht}{\usebox\mybox}\hspace{2pt}
}

\newenvironment{mydescription}{
    \begin{description}
        \setlength{\leftskip}{-1.5cm}}
    {\end{description}}

\newenvironment{myitemize}{
    \begin{itemize}
        \setlength{\leftskip}{-.3cm}}
    {\end{itemize}}

% define formatting for footer
\newcommand{\myfootline}{%
    {\it
    \insertshorttitle
    \hspace*{\fill} 
    \insertshortauthor, \insertshortinstitute
    % \ifx\insertsubtitle\@empty\else, \insertshortsubtitle\fi
    \hspace*{\fill}
    \insertframenumber/\inserttotalframenumber}}

% set up footer
\setbeamertemplate{footline}{%
    \usebeamerfont{structure}
    \begin{beamercolorbox}[wd=\paperwidth,ht=2.25ex,dp=1ex]{frametitle}%
        \Tiny\hspace*{4mm}\myfootline\hspace{4mm}
    \end{beamercolorbox}}

% remove navigation bar
\beamertemplatenavigationsymbolsempty


% \newcommand{\change}[1]{{\color{blue} #1}\xspace}
\newcommand{\change}[1]{{\color{black} #1}\xspace}


\newcommand{\citationNeeded}{\textcolor{magenta}{\textbf{[CITATION NEEDED!]}}\xspace}
\newcommand{\tableNeeded}{\textcolor{magenta}{\textbf{[TABLE NEEDED!]}}\xspace}
\newcommand{\figureNeeded}{\textcolor{magenta}{\textbf{[FIGURE NEEDED!]}}\xspace}
\newcommand{\highLight}[1]{\textcolor{magenta}{\MakeUppercase{#1}}}

\newcommand{\editorialNote}[1]{\textcolor{red}{[\textit{#1}]}}
\newcommand{\ignore}[1]{}
\newcommand{\addTail}[1]{\textit{#1}.---}
\newcommand{\super}[1]{\ensuremath{^{\textrm{#1}}}}
\newcommand{\sub}[1]{\ensuremath{_{\textrm{#1}}}}
\newcommand{\dC}{\ensuremath{^\circ{\textrm{C}}}}

\providecommand{\e}[1]{\ensuremath{\times 10^{#1}}}

\newcommand{\mthnote}[2]{{\color{red} #2}}

\newcommand{\ifTwoArgs}[3]{\ifthenelse{\equal{#1}{}\or\equal{#2}{}}{}{#3}\xspace}
\newcommand{\ifArg}[2]{\ifthenelse{\equal{#1}{}}{}{#2}\xspace}

\newcommand{\allDatasets}{\ensuremath{\mathcal{\alignment{}{}}}\xspace}
\newcommand{\allParameterValues}{\ensuremath{\boldsymbol{\Theta}}\xspace}
\newcommand{\bayesfactor}[2]{\ensuremath{BF_{#1\protect\ifArg{#2}{,}#2}}}
\newcommand{\given}{\ensuremath{\,|\,}\xspace}
\newcommand{\msb}{\upshape\texttt{\MakeLowercase{ms\MakeUppercase{B}ayes}}\xspace}
\newcommand{\hky}{HKY85\xspace}
\newcommand{\uniformMin}[1]{\ensuremath{a_{#1}}\xspace}
\newcommand{\uniformMax}[1]{\ensuremath{b_{#1}}\xspace}
\newcommand{\locusRateHetShapeParameter}{\ensuremath{\alpha}\xspace}
\newcommand{\ancestralThetaVector}{\ensuremath{\boldsymbol{\theta_{A}}}\xspace}
\newcommand{\descendantThetaVector}[1]{\ensuremath{\boldsymbol{\theta_{D#1}}}\xspace}
\newcommand{\divtscaledvector}{\ensuremath{\mathbf{{\divtscaled{}{}}}}\xspace}
\newcommand{\divtvector}{\ensuremath{\boldsymbol{\divt{}}}\xspace}
\newcommand{\divtuniquevector}{\ensuremath{\mathbf{\divtunique{}}}\xspace}
\newcommand{\bottleTimeVector}{\ensuremath{\boldsymbol{\bottleTime{}}}\xspace}
\newcommand{\bottleTime}[1]{\ensuremath{\divt{B\ifArg{#1}{,}#1}}\xspace}
\newcommand{\bottleScalarVector}[1]{\ensuremath{\boldsymbol{\bottleScalar{#1}{}}}\xspace}
\newcommand{\bottleScalar}[2]{\ensuremath{\zeta_{D#1\protect\ifArg{#2}{,}#2}}\xspace}
\newcommand{\migrationRateVector}{\ensuremath{\mathbf{\migrationRate{}}}\xspace}
\newcommand{\geneTreeVector}{\ensuremath{\mathbf{\geneTree{}{}}}\xspace}
\newcommand{\alignmentVector}{\ensuremath{\mathbf{\alignment{}{}}}\xspace}
\newcommand{\alignment}[2]{\ensuremath{X_{#1\protect\ifTwoArgs{#1}{#2}{,}#2}}\xspace}
\newcommand{\geneTree}[2]{\ensuremath{G_{#1\protect\ifTwoArgs{#1}{#2}{,}#2}}\xspace}
\newcommand{\migrationRate}[1]{\ensuremath{m_{#1}}\xspace}
\newcommand{\recombinationRate}{\ensuremath{r}\xspace}
\newcommand{\ploidyScalar}[2]{\ensuremath{\rho_{#1\protect\ifTwoArgs{#1}{#2}{,}#2}}\xspace}
\newcommand{\ploidyScalarVector}{\ensuremath{\boldsymbol{\ploidyScalar{}{}}}\xspace}
\newcommand{\descendantRelativeThetaVector}[1]{\ensuremath{\boldsymbol{\eta_{D#1}}}\xspace}
\newcommand{\descendantRelativeTheta}[2]{\ensuremath{\eta_{D#1\protect\ifArg{#2}{,}#2}}\xspace}
\newcommand{\mutationRateScalarConstant}[2]{\ensuremath{\nu_{#1\protect\ifTwoArgs{#1}{#2}{,}#2}}\xspace}
\newcommand{\mutationRateScalarConstantVector}{\ensuremath{\boldsymbol{\mutationRateScalarConstant{}{}}}\xspace}
\newcommand{\locusMutationRateScalar}[1]{\ensuremath{\upsilon_{#1}}\xspace}
\newcommand{\locusMutationRateScalarVector}{\ensuremath{\boldsymbol{\upsilon}}\xspace}
\newcommand{\hkyModel}[2]{\ensuremath{\phi_{#1\protect\ifTwoArgs{#1}{#2}{,}#2}}\xspace}
\newcommand{\hkyModelVector}{\ensuremath{\boldsymbol{\hkyModel{}{}}}\xspace}
\newcommand{\mutationRate}{\ensuremath{\mu}\xspace}
\newcommand{\iid}{\textit{iid}\xspace}
\newcommand{\model}[1]{\ensuremath{\Theta}\xspace}
\newcommand{\npairs}[1]{\ensuremath{Y_{#1}}}
\newcommand{\nloci}[1]{\ensuremath{k_{#1}}\xspace}
\newcommand{\nlociTotal}{\ensuremath{K}\xspace}
\newcommand{\myTheta}[1]{\ensuremath{\theta_{#1}}}
\newcommand{\ancestralTheta}[1]{\ensuremath{\theta_{A\protect\ifArg{#1}{,}#1}}\xspace}
\newcommand{\descendantTheta}[2]{\ensuremath{\theta_{D#1\protect\ifArg{#2}{,}#2}}\xspace}
\newcommand{\meanDescendantTheta}[1]{\ensuremath{\descendantTheta{}{#1}}\xspace}
\newcommand{\nucdiv}[1]{\ensuremath{\pi_{#1}}}

\newcommand{\ssVector}[1]{\ensuremath{\mathbf{\alignmentSS{#1}{}}}\xspace}
\newcommand{\ssVectorObs}{\ensuremath{\ssVector{}^*}\xspace}
\newcommand{\ssSpace}{\ensuremath{\euclideanSpace{\ssVectorObs}}\xspace}
\newcommand{\ssVectorObsPLS}{\ensuremath{\ssVectorObs_{PLS}}\xspace}
\newcommand{\alignmentSS}[2]{\ensuremath{S_{#1\protect\ifTwoArgs{#1}{#2}{,}#2}}\xspace}
\newcommand{\alignmentSSObs}[2]{\ensuremath{\alignmentSS{#1}{#2}^*}\xspace}
\newcommand{\tol}{\ensuremath{\epsilon}\xspace}
\newcommand{\euclideanSpace}[1]{\ensuremath{B_{\tol}(#1)}\xspace}
\newcommand{\hpvector}[1]{\ensuremath{\Lambda_{#1}}}
\newcommand{\divtscaled}[2]{\ensuremath{t_{#1\protect\ifTwoArgs{#1}{#2}{,}#2}}}
\newcommand{\divt}[1]{\ensuremath{\tau_{#1}}}
\newcommand{\divtunique}[1]{\ensuremath{T_{#1}}}
\newcommand{\ssMatrix}{\ensuremath{\mathbb \alignmentSS{}{}}\xspace}
\newcommand{\ssMatrixRaw}[1]{\ensuremath{{\ssMatrix}_{stats#1}}\xspace}
\newcommand{\ssMatrixPLS}[1]{\ensuremath{{\ssMatrix}_{PLS#1}}\xspace}
\newcommand{\hpmatrix}[1]{\ensuremath{\mathcal{P}_{#1}}}
\newcommand{\meant}[2]{\ensuremath{E(\divt{#1})_{#2}}}
\newcommand{\meantestimate}{\ensuremath{\hat{E(\divt{})}}\xspace}
\newcommand{\vart}[2]{\ensuremath{Var(\divt{#1}{})_{#2}}}
\newcommand{\vmratio}[1]{\ensuremath{\Omega_{#1}}}
\newcommand{\numt}[1]{\ensuremath{\Psi_{#1}}}
\newcommand{\probnumt}[2]{\ensuremath{p(\numt{#1} = {#2})}}
\newcommand{\postprobnumt}[1]{\ensuremath{p(\numt{} = {#1}|\ssSpace)}}
\newcommand{\postprobnumtnot}[1]{\ensuremath{p(\numt{} \neq {#1}|\ssSpace)}}
\newcommand{\postprobomegasimult}{\ensuremath{p(\vmratio{} < 0.01 | \ssSpace)}\xspace}
\newcommand{\modelprior}[1]{\ensuremath{f(\model{})}}
\newcommand{\modelpost}[1]{\ensuremath{f(\model{}|\ssSpace)}}
\newcommand{\npriorsamples}{\ensuremath{n}\xspace}
\newcommand{\globalcoalunit}{\ensuremath{4\globalpopsize}\xspace}
\newcommand{\globalpopsize}{\ensuremath{N_C}\xspace}
\newcommand{\effectivePopSize}[1]{\ensuremath{N_e{#1}}\xspace}
\newcommand{\coalunit}{\ensuremath{4\effectivePopSize{}}\xspace}
\newcommand{\priorsample}[1]{\ensuremath{\hpmatrix{\modelprior{}}}}
\newcommand{\truncprior}[1]{\ensuremath{\hpmatrix{\tol}}\xspace}
\newcommand{\postsample}[1]{\ensuremath{\hpmatrix{\modelpost{}}}}
\newcommand{\abcllr}[1]{ABC\sub{LLR}}
\newcommand{\abcglm}[1]{ABC\sub{GLM}}
\newcommand{\integerPartition}[1]{\ensuremath{a({#1})}}
\newcommand{\uniqueModel}[2]{\ensuremath{M_{#1\protect\ifTwoArgs{#1}{#2}{,}#2}}}
\newcommand{\taxonLocusVector}[1]{\ensuremath{\{#1{1}{1},\ldots,#1{\npairs{}}{\nloci{\npairs{}}}\}}\xspace}
\newcommand{\taxonVector}[1]{\ensuremath{\{#1{1},\ldots,#1{\npairs{}}\}}\xspace}
\newcommand{\locusVector}[1]{\ensuremath{\{#1{1},\ldots,#1{\nlociTotal}\}}\xspace}

\newcommand{\simulationDescription}[2]{\change{Each plot represents #1
    simulation replicates using the same $#2$ samples from the prior}}
\newcommand{\simulationDistribution}{\ensuremath{\divt{} \sim U(0,
    \divt{max})}\xspace}
\newcommand{\estimateDescription}[2]{All estimates were obtained using #1 and #2}
\newcommand{\estimateDescriptionUncorrected}[1]{All estimates based on
    unadjusted posterior, \truncprior{}, obtained using #1}
\newcommand{\priorDescription}[4]{Prior settings were \priorSettings{#1}{#2}{#3}{#4}}
\newcommand{\priorSettings}[4]{$\divt{} \sim U(0, #1)$,
    $\meanDescendantTheta{} \sim U(#2, #3)$, and
    $\ancestralTheta{}{} \sim U(#2, #4)$}
\newcommand{\priorDescriptionBug}[4]{Prior settings were
    \priorSettingsBug{#1}{#2}{#3}{#4}}
\newcommand{\priorSettingsBug}[4]{$\divt{} \sim U(0, #1)$,
    $\meanDescendantTheta{} \sim U(#2, #3)$, and
    $\ancestralTheta{}{} \sim U(0.01, #4)$}
\newcommand{\simulationScheme}{simulations where \divt{} (in \globalcoalunit
    generations) for 22 population pairs is drawn from a series of uniform
    distributions, \simulationDistribution}
\newcommand{\captionPowerOmega}{Histograms of the estimated dispersion index
    of divergence times ($\hat{\vmratio{}}$) from \simulationScheme.
    The threshold for one divergence event \citep{Hickerson2006} is indicated
    by the dashed line, and the estimated probability of inferring one
    divergence event, $p(\hat{\vmratio{}}\le 0.01)$, is given for each
    \divt{max}}
\newcommand{\captionPowerPsiMode}{Histograms of the estimated number of
    divergence events ($\hat{\numt{}}$) from \simulationScheme.
    The estimated probability of inferring one divergence event,
    $p(\hat{\numt{}} = 1)$, is given for each \divt{max}}
\newcommand{\captionPowerPsi}{Histograms of the estimated posterior
    probability of one divergence event, \postprobnumt{1}, from
    \simulationScheme.
    The estimated probability of inferring one divergence event with a
    Bayes factor greater than 10 (dashed black line),
    $p(\bayesfactor{\numt{}=1}{\numt{} \ne 1} > 10)$, is given for each \divt{max}.
    The red line indicates $\postprobnumt{1} = 0.95$, and the estimated
    probability of inferring a posterior probability greater than 0.95 is given
    to the right of the line.}
\newcommand{\captionAccuracy}[1]{Accuracy and precision of #1 estimates from
    \simulationScheme.
    The proportion of estimates less than the true value ($p(\hat{#1}<#1)$) is
    given for each \divt{max}}
\newcommand{\samplingErrorTableNote}{An estimate of 1.0 for a posterior probability
    is an artifact of sampling error}


\newcommand{\refAccuracyALL}[1]{\labelcref{fig_acc_t_ss_llr_bug,fig_acc_t_ss_glm_bug,fig_acc_t_pls_llr_bug,fig_acc_t_pls_glm_bug,fig_acc_o_ss_llr_bug,fig_acc_o_ss_glm_bug,fig_acc_o_pls_llr_bug,fig_acc_o_pls_glm_bug}}
\newcommand{\refAccuracySS}[1]{\labelcref{fig_acc_t_ss_llr_bug,fig_acc_t_ss_glm_bug,fig_acc_o_ss_llr_bug,fig_acc_o_ss_glm_bug}}
\newcommand{\refAccuracySSfull}[1]{\labelcref{fig_acc_t_ssfull_llr_bug,fig_acc_t_ssfull_glm_bug,fig_acc_o_ssfull_llr_bug,fig_acc_o_ssfull_glm_bug}}
\newcommand{\refSSfull}[1]{\labelcref{fig_acc_t_ssfull_llr_bug,fig_acc_t_ssfull_glm_bug,fig_acc_o_ssfull_llr_bug,fig_acc_o_ssfull_glm_bug,fig_pow_o_ssfull_llr_bug,fig_pow_o_ssfull_glm_bug,fig_pow_psi_modes_ssfull_glm_bug}}
\newcommand{\refSS}[1]{\labelcref{fig_acc_t_ss_llr_bug,fig_acc_t_ss_glm_bug,fig_acc_o_ss_llr_bug,fig_acc_o_ss_glm_bug,fig_pow_o_ss_llr_bug,fig_pow_o_ss_glm_bug,fig_pow_psi_ss}}
\newcommand{\refAccuracyPLS}[1]{\labelcref{fig_acc_t_pls_llr_bug,fig_acc_t_pls_glm_bug,fig_acc_o_pls_llr_bug,fig_acc_o_pls_glm_bug}}
\newcommand{\refAccuracySScorrected}[1]{\labelcref{fig_acc_t_ss_llr_bug,fig_acc_t_ss_glm_bug,fig_acc_o_ss_llr_bug,fig_acc_o_ss_glm_bug}}
\newcommand{\refAccuracyPLScorrected}[1]{\labelcref{fig_acc_t_pls_llr_bug,fig_acc_t_pls_glm_bug,fig_acc_o_pls_llr_bug,fig_acc_o_pls_glm_bug}}
\newcommand{\refAccuracyUncorrected}[1]{\labelcref{fig_acc_t_ss_unc,fig_acc_t_pls_unc,fig_acc_o_ss_unc,fig_acc_o_pls_unc}}
\newcommand{\refAccuracyCorrected}[1]{\labelcref{fig_acc_t_ss_llr_bug,fig_acc_t_ss_glm_bug,fig_acc_t_pls_llr_bug,fig_acc_t_pls_glm_bug,fig_acc_o_ss_llr_bug,fig_acc_o_ss_glm_bug,fig_acc_o_pls_llr_bug,fig_acc_o_pls_glm_bug}}
\newcommand{\refAccuracyGLM}[1]{\labelcref{fig_acc_t_ss_glm_bug,fig_acc_t_pls_glm_bug,fig_acc_o_ss_glm_bug,fig_acc_o_pls_glm_bug}}
\newcommand{\refAccuracyLLR}[1]{\labelcref{fig_acc_t_ss_llr_bug,fig_acc_t_pls_llr_bug,fig_acc_o_ss_llr_bug,fig_acc_o_pls_llr_bug}}
\newcommand{\refAccuracyOmega}[1]{\labelcref{fig_acc_o_ss_llr_bug,fig_acc_o_ss_glm_bug,fig_acc_o_pls_llr_bug,fig_acc_o_pls_glm_bug}}
\newcommand{\refAccuracyOmegaUncorrected}[1]{\labelcref{fig_acc_o_ss_unc,fig_acc_o_pls_unc}}
\newcommand{\refAccuracyOmegaCorrected}[1]{\labelcref{fig_acc_o_ss_llr_bug,fig_acc_o_ss_glm_bug,fig_acc_o_pls_llr_bug,fig_acc_o_pls_glm_bug}}
\newcommand{\refAccuracyTime}[1]{\labelcref{fig_acc_t_ss_llr_bug,fig_acc_t_ss_glm_bug,fig_acc_t_pls_llr_bug,fig_acc_t_pls_glm_bug}}

\newcommand{\tn}{\tabularnewline}

\newcommand{\widthFigure}[5]{\begin{figure}[htbp]
\begin{center}
    \includegraphics[width=#1\textwidth]{#2}
    \captionsetup{#3}
    \caption{#4}
    \label{#5}
    \end{center}
    \end{figure}}

\newcommand{\heightFigure}[5]{\begin{figure}[htbp]
\begin{center}
    \includegraphics[height=#1]{#2}
    \captionsetup{#3}
    \caption{#4}
    \label{#5}
    \end{center}
    \end{figure}}

\newcommand{\mFigure}[3]{\widthFigure{1.0}{#1}{listformat=figList}{#2}{#3}\clearpage}
\newcommand{\siFigure}[3]{\widthFigure{1.0}{#1}{name=Figure S, labelformat=noSpace, listformat=sFigList}{#2}{#3}\clearpage}


%%%%%%%%%%%%%%%%%%%%%%%%%%%%%%%%%%%%%%%%%%%%%%%%%%%%%%%%%%%%
%%%%%%%%%%%%%%%%%%%%%%%%%%%%%%%%%%%%%%%%%%%%%%%%%%%%%%%%%%%%

\begin{document}
\doublespacing
\raggedright
\setlength{\parindent}{0.5in}
\begin{linenumbers}

\begin{titlepage}
    \begin{flushleft}
        \sffamily

        \MakeUppercase{\large\bfseries Title}

        \vspace{12pt}
        \textbf{Running head:} \MakeUppercase{Running head}

        \vspace{12pt}
        Jamie R.\ Oaks$^{1,2}$ and Mark T.\ Holder$^{1}$

        \bigskip
        $^1$\emph{Department of Ecology and Evolutionary Biology,
            % Biodiversity Institute,
            University of Kansas,
            Lawrence, Kansas 66045}\\[.1in]
        $^2$\emph{Corresponding author} (\href{mailto:joaks1@gmail.com}{\tt
        joaks1@gmail.com})\\

    \end{flushleft}
\end{titlepage}

{\sffamily
    \noindent\textbf{ABSTRACT} \\
    \noindent Abstract here \ldots

    \vspace{12pt}
    \noindent\textbf{KEY WORDS: } 
}

\newpage
\noindent \ldots

\section*{Methods}

\subsection*{The model}
We follow much of the notation of \citet{Oaks2012}, but modify it to
accommodate the new model \ldots

We assume an investigator is interested in inferring the distribution
of divergence times among \npairs{} pairs of populations.
For each pair $i$, \popSampleSize{i}{} genome copies have been sampled, with
\popSampleSize{1}{i} copies sampled from population 1, and \popSampleSize{2}{i}
sampled from population 2.
From these genomes, let \nloci{i} be the number of DNA sequence loci collected
for population pair $i$, and \nlociTotal be the total number of unique loci
sampled across the \npairs{} pairs of populations.
We use \alignment{i}{j} to represent the multiple sequence alignment of
locus $j$ for population pair $i$.
$\alignmentVector = (\alignment{1}{1}, \ldots,
    \alignment{\npairs{}}{\nloci{\npairs{}}})$
is the full dataset,
i.e., a vector of sequence alignments for all pairs and loci.
We let \geneTree{i}{j} represent the gene tree upon which \alignment{i}{j}
evolved according to fixed \hky substitution model parameters \hkyModel{i}{j}.
The investigtor must specify the parameters of all
$\hkyModelVector = (\hkyModel{1}{1}, \ldots,
\hkyModel{\npairs{}}{\nloci{\npairs{}}})$
substitution models by which the alignments evolved along the
$\geneTreeVector = (\geneTree{1}{1}, \ldots,
\geneTree{\npairs{}}{\nloci{\npairs{}}})$
gene trees.
Furthermore, the investigator must specify a vector of fixed contants
$\ploidyScalarVector = (\ploidyScalar{1}{1}, \ldots,
\ploidyScalar{\npairs{}}{\nloci{\npairs{}}})$
that scale the population size parameters for known differences in ploidy among
loci and/or differences in generation times among population pairs.
Lastly, the investigator must also specify a vector of fixed constants
$\mutationRateScalarConstantVector = (\mutationRateScalarConstant{1}{1},
\ldots, \mutationRateScalarConstant{\npairs{}}{\nloci{\npairs{}}})$
that scale the population size parameters for known differences in
mutation rates among loci and/or among taxa.

\begin{linenomath}
With \alignmentVector, \hkyModelVector, \ploidyScalarVector, and
\mutationRateScalarConstantVector in hand, the joint posterior distribution
of our model is given by
\begin{equation}
    p(\geneTreeVector, \divTimeMapVector, \demographicParamVector, 
    \locusMutationRateScalarVector, \locusRateHetShapeParameter \given
    \alignmentVector, \hkyModelVector, \ploidyScalarVector,
    \mutationRateScalarConstantVector) =
    \frac{p(\alignmentVector \given \geneTreeVector, \hkyModelVector)
        p(\geneTreeVector \given \divTimeMapVector, \demographicParamVector,
        \locusMutationRateScalarVector, \ploidyScalarVector,
        \mutationRateScalarConstantVector)
        p(\locusMutationRateScalarVector \given \locusRateHetShapeParameter)
        p(\locusRateHetShapeParameter)
        p(\divTimeMapVector)
        p(\demographicParamVector)}{
        p(\alignmentVector)},
    \label{eq:fullModelCompact}
\end{equation}
where
$\divTimeMapVector = (\divTimeMap{1}, \ldots, \divTimeMap{\npairs{}})$
is a vector of population divergence times for each of \npairs{} pairs,
$\demographicParamVector = (\demographicParams{1}, \ldots,
\demographicParams{\npairs{}})$
is a vector of the demographic parameters for each of the \npairs{} population
pairs,
$\locusMutationRateScalarVector = (\locusMutationRateScalar{1}, \ldots
\locusMutationRateScalar{\nlociTotal})$
is a vector of locus-specific mutation-rate scaling parameters for each of the
\nlociTotal loci,
\locusRateHetShapeParameter is the shape parameter of a gamma-distributed
prior on \locusMutationRateScalar{}, and
$p(\alignmentVector)$ is the probability of the data (or the marginal
likelihood of the model).
\end{linenomath}

\begin{linenomath}
To avoid calculating the likelihood terms of Equation \ref{eq:fullModelCompact},
we distill each sequence alignment \alignment{}{} into a vector of insufficient
summary statistics \alignmentSS{}{}, thus replacing the full dataset
$\alignmentVector = (\alignment{1}{1}, \ldots,
    \alignment{\npairs{}}{\nloci{\npairs{}}})$
with vectors of summary statistics for each alignment
$\ssVectorObs = (\alignmentSSObs{1}{1}, \ldots,
    \alignmentSSObs{\npairs{}}{\nloci{\npairs{}}})$.
For each populationn pair, the means of the summary statistics are calculated
across the \nloci{} loci, reducing the vector to
$\ssVectorObs = (\alignmentSSObs{1}{}, \ldots,
    \alignmentSSObs{\npairs{}}{})$.
This allows us to estimate the approximate joint posterior distribution
{\small
\begin{equation}
    p(\geneTreeVector, \divTimeMapVector, \demographicParamVector, 
    \locusMutationRateScalarVector, \locusRateHetShapeParameter \given
    \ssSpace, \hkyModelVector, \ploidyScalarVector,
    \mutationRateScalarConstantVector) =
    \frac{p(\ssSpace \given \geneTreeVector, \hkyModelVector)
        p(\geneTreeVector \given \divTimeMapVector, \demographicParamVector,
        \locusMutationRateScalarVector, \ploidyScalarVector,
        \mutationRateScalarConstantVector)
        p(\locusMutationRateScalarVector \given \locusRateHetShapeParameter)
        p(\locusRateHetShapeParameter)
        p(\divTimeMapVector)
        p(\demographicParamVector)}{
        p(\ssSpace)},
    \label{eq:approxModelCompact}
\end{equation}
}
where \ssSpace is the multidimensional Euclidean space around the vector of
summary statistics, the radius of which is the tolerance \tol.
The approximation of this posterior is caused by the insufficiency of the
statistics and the \tol being greater than zero.
We describe the full model in detail before delving into the numerical
method of estimating the approximate model.
\end{linenomath}

\subsubsection*{Likelihood terms of Equation \ref{eq:fullModelCompact}}
\begin{linenomath}
The likelihood terms of Equation \ref{eq:fullModelCompact} can
be expanded out as a product over population pairs and loci
\begin{equation}
    p(\alignmentVector \given \geneTreeVector, \hkyModelVector)
    p(\geneTreeVector \given \divTimeMapVector, \demographicParamVector,
    \locusMutationRateScalarVector, \ploidyScalarVector,
    \mutationRateScalarConstantVector) = 
    \prod_{i=1}^{\npairs{}}
    \prod_{j=1}^{\nloci{i}}
    p(\alignment{i}{j} \given \geneTree{i}{j}, \hkyModel{i}{j})
    p(\geneTree{i}{j} \given \divTimeMap{i}, \demographicParams{i},
    \locusMutationRateScalar{j}, \ploidyScalar{i}{j},
    \mutationRateScalarConstant{i}{j}).
    \label{eq:modelLikelihoodExpanded}
\end{equation}
The first term,
$p(\alignment{i}{j} \given \geneTree{i}{j}, \hkyModel{i}{j})$,
is the probability of the sequence alignment of locus $j$ for population pair
$i$ given the gene tree and \hky substitution model paramters \citep[i.e., the
``Felsenstein likelhood'';][]{Felsenstein1981}.
The model allows for an intra-locus recombination rate \recombinationRate,
which, for simplicity, is assumed to be zero in Equation
\ref{eq:fullModelCompact}.
If \recombinationRate is non-zero, this term requires an additional product
over the columns (sites) of each sequence alignment to allow sites to have
different genealogies.
The second term,
p(\geneTree{i}{j} \given \divTimeMap{i}, \demographicParams{i},
\locusMutationRateScalar{j}, \ploidyScalar{i}{j},
\mutationRateScalarConstant{i}{j}),
is the probability of the gene tree under a multi-population coalescent model
(i.e., species tree) where the ancestral population of pair $i$ diverges
and gives rise to the two sampled descendant populations.
Each \demographicParams contains the following demographic parameters:
The mutation-scaled sizes of the ancestral, \ancestralTheta{}, and descendant
populations, \descendantTheta{1}{} and \descendantTheta{2}{};
the proportion of the first, \bottleScalar{1}{}, and second population,
\bottleScalar{2}{}, that persists during bottlenecks that begin immmediately
after divergence in forward-time;
the proportion of time between present and divergence when the bottlenecks
end for both populations, \bottleTime{};
and the symmetric migration rate between the descendant populations,
\migrationRate{}.
Thus, the probability of the $\popSampleSize{i}{}-1$ coalescence times (node
heights) of gene tree \geneTree{i}{j} is given by a multi-population
Kingman-coalescent model \citep{Kingman1982} where the ancestral population of
size
$\ancestralTheta{i}\ploidyScalar{i}{j}\mutationRateScalarConstant{i}{j}\locusMutationRateScalar{j}$
diverges at time \divTimeMap{i} into two descendant populations of constant
size
$\descendantTheta{1}{i}\ploidyScalar{i}{j}\mutationRateScalarConstant{i}{j}\locusMutationRateScalar{j}\bottleScalar{1}{i}$
and
$\descendantTheta{2}{i}\ploidyScalar{i}{j}\mutationRateScalarConstant{i}{j}\locusMutationRateScalar{j}\bottleScalar{2}{i}$,
which, after time $\divTimeMap{i}\bottleTime{i}$, grow exponentially to their
present size 
$\descendantTheta{1}{i}\ploidyScalar{i}{j}\mutationRateScalarConstant{i}{j}\locusMutationRateScalar{j}$
and
$\descendantTheta{2}{i}\ploidyScalar{i}{j}\mutationRateScalarConstant{i}{j}\locusMutationRateScalar{j}$,
respectively.
Following divergence, the descendant populations of pair $i$ exchange migrants at a symmetric rate of
\migrationRate{i}.
\end{linenomath}


\subsubsection*{Prior terms of Equation \ref{eq:fullModelCompact}}
\begin{linenomath}
The term $p(\locusRateHetShapeParameter)$ is the prior density
function for the shape-parameter for the gamma-distributed prior on
rate heterogeneity among loci.
This prior is $\locusRateHetShapeParameter \sim U(1, 20)$.
The prior probability of the vector of locus-specific mutation-rate scaling parameters
given \locusRateHetShapeParameter then expands out as a product over
the loci
\begin{equation}
    p(\locusMutationRateScalarVector \given \locusRateHetShapeParameter) =
    \prod_{j=1}^{\nlociTotal}
    p(\locusMutationRateScalar{j} \given \locusRateHetShapeParameter),
    \label{eq:locusRateHetPrior}
\end{equation}
where each \locusMutationRateScalar{} is independently and identicially
distributed (\iid) as
$\locusMutationRateScalar{} \sim Gamma(\locusRateHetShapeParameter,
1/\locusRateHetShapeParameter)$.
If the recombination rate \recombinationRate is allowed to be
non-zero, the prior term $p(\recombinationRate)$ would be added
to Equation \ref{eq:fullModelCompact}, and the prior would be
$\recombinationRate \sim Gamma(\gammaShape{\recombinationRate},
\gammaScale{\recombinationRate})$.
\end{linenomath}

\begin{linenomath}
The prior term for the demographic parameters, $p(\demographicParamVector)$,
expands out into its components and as a product over the \npairs{}
pairs of populations
\begin{equation}
    p(\demographicParamVector) =
    \prod_{i=1}^{\npairs{}}
    p(\ancestralTheta{i})
    p(\descendantTheta{1}{i})
    p(\descendantTheta{2}{i})
    p(\bottleScalar{1}{i})
    p(\bottleScalar{2}{i})
    p(\bottleTime{i})
    p(\migrationRate{i}).
    \label{eq:demographicPrior}
\end{equation}
The priors for the demographic parameters are:
$\ancestralTheta{} \sim Gamma(\gammaShape{\ancestralTheta{}},
\gammaScale{\ancestralTheta{}})$,
$\descendantTheta{1}{} \sim Gamma(\gammaShape{\descendantTheta{}{}},
\gammaScale{\descendantTheta{}{}})$,
$\descendantTheta{2}{} \sim Gamma(\gammaShape{\descendantTheta{}{}},
\gammaScale{\descendantTheta{}{}})$,
$\bottleScalar{1}{} \sim Beta(\betaA{\bottleScalar{}{}},
\betaB{\bottleScalar{}{}})$,
$\bottleScalar{2}{} \sim Beta(\betaA{\bottleScalar{}{}},
\betaB{\bottleScalar{}{}})$,
$\bottleTime{} \sim U(0, 1)$,
and
$\migrationRate{} \sim Gamma(\gammaShape{\migrationRate{}},
\gammaScale{\migrationRate{}})$.
\end{linenomath}

\subsubsection*{Priors on divergence models}
\begin{linenomath}
The prior term for the vector of divergence times for each of
the \npairs{} pairs of populations can be expanded as
\begin{equation}
    p(\divTimeMapVector) = p(\divTimeIndexVector)p(\divTimeVector \given \divTimeIndexVector),
    \label{eq:divModelPrior}
\end{equation}
where \divTimeVector is an ordered set of unique divergence time parameters
$\{\divTime{1}, \ldots, \divTime{\divTimeNum}\}$ whose length
\divTimeNum can range from 1 to \npairs{},
and \divTimeIndexVector is vector of indices of divergence time parameters 
$(\divTimeIndex{1}, \ldots, \divTimeIndex{\npairs{}})$
that map the divergence time in \divTimeVector to each of the \npairs{}
pairs of populations.
Thus, \divTimeMapVector is the result of applying the mapping function
\begin{equation}
    f(\divTimeVector, \divTimeIndexVector, i) = \divTime{\divTimeIndex{i}}
    \label{eq:divTimeMapFunction}
\end{equation}
to each population pair $i$, such that
$\divTimeMapVector = (\divTimeMap{1} = f(\divTimeVector, \divTimeIndexVector,
1), \ldots, \divTimeMap{\npairs{}} = f(\divTimeVector, \divTimeIndexVector,
\npairs{}))$.
\end{linenomath}

Biologically speaking, \divTimeVector contains the times of divergence events,
the length of which \divTimeNum is the number of divergence events shared across
the \npairs{} pairs of populations.
For example, if \divTimeVector contains a single divergence time parameter
\divTime{1}, all \npairs{} pairs of populations are constrained to diverge at
this time (i.e., \divTimeIndexVector would contain the index 1 repeated
\npairs{} times, and \divTimeMapVector would contain the value \divTime{1}
repeated \npairs{} times), whereas if it contains \npairs{} divergence time
parameters, the model is fully generalized to allow all of the pairs to diverge
at unique times.

\begin{linenomath}
Each \divTime{} within \divTimeVector is \iid as $\divTime{} \sim
Gamma(\gammaShape{\divTime{}}, \gammaScale{\divTime{}})$.
Thus, given the number of unique divergence time classes in
\divTimeIndexVector, this determines the probability of prior term
$p(\divTimeVector \given \divTimeIndexVector)$
The divergence time parameters are in coalescent units relative to the size of
a constant reference population, which we denote \myTheta{C}, that is equal to
the expectation of the prior on the size of the descendant populations
\begin{equation}
    \myTheta{C} = E(\descendantTheta{}{}).
    \label{eq:thetaC}
\end{equation}
Given the size of the descendant populations are \iid as
$\descendantTheta{}{} \sim Gamma(\gammaShape{\descendantTheta{}{}},
\gammaScale{\descendantTheta{}{}})$,
this becomes
\begin{equation}
    \myTheta{C} = \gammaShape{\descendantTheta{}{}}\gammaScale{\descendantTheta{}{}}.
    \label{eq:thetaCGamma}
\end{equation}
The \divTime parameters are then in units $\myTheta{C}/\mutationRate$
generations, which we denote as \globalcoalunit generations.
Thus, each \divTime within \divTimeVector is proportional to time and
can be converted to the number of generations of the reference population,
which we denote \divTime{G_C}, by assuming a mutation rate multiplying by
the effective size of the reference population
\begin{equation}
    \divTime{G_C} = \divTime \times \frac{
        \gammaShape{\descendantTheta{}{}}\gammaScale{\descendantTheta{}{}}}
        {\mutationRate}.
    \label{eq:divTimeGenerations}
\end{equation}
Thus, for each of the divergence times in \divTimeVector to be on the same
scale, and thus comparable, the relative mutation rates among the pairs of
populations are assumed to be fixed and known according the user-provided
values in \mutationRateScalarConstantVector.
\end{linenomath}

\begin{linenomath}
As described by \citet{Oaks2012}, to get the divergence times in units
proportional to the expected number of mutations, \divTimeScaled{}{}, we must
scale them by the realized population size for locus $j$ of population-pair $i$ 
\begin{equation}
    \divTimeScaled{i}{j} = \divTime{i} \times \frac{\myTheta{C}}{
        \descendantThetaMean{i} \ploidyScalar{i}{j}},
    \label{eq:divTimeScaled}
\end{equation}
where \descendantThetaMean{i} is the mean of \descendantTheta{1}{} and
\descendantTheta{2}{} for pair $i$.
This gives us the vector of scaled divergence times
$\divTimeScaledVector = (\divTimeScaled{1}{1}, \ldots,
\divTimeScaled{\npairs{}}{\nloci{\npairs{}}})$.
\end{linenomath}


\begin{linenomath}
As for the prior term $p(\divTimeMapVector)$,
the total sample space of ordered realizations of \divTimeMapVector is all the
possible partitions of \npairs{} elements into 1 to \npairs{} categories.
The total number of possible partitions is a sum of the Stirling numbers of
the second kind over all possible number of categories \divTimeNum
\begin{equation}
    B_{\npairs{}}=\sum_{\divTimeNum=1}^{\npairs{}} \left[
    \frac{1}{\divTimeNum!} \sum_{j=0}^{\divTimeNum-1} (-1)^{j}
    \binom{\divTimeNum}{j} (\divTimeNum-j)^{\npairs{}} \right],
    \label{eq:bell}
\end{equation}
which is the Bell Number \citep{Bell1934}.
The original \msb model only samples over the unordered realizations of
\divTimeIndexVector, such that the sample space is reduced to all the possible
integer partitions of \npairs{} \citep{Oaks2012,Huang2011,OeisPartitionNumber,
    OeisPartitionTriangle,Malenfant2011}.
We denote the set of all possible integer partitions of \npairs{} as
\integerPartitionSet{\npairs{}} and the length of that set as
\integerPartitionNum{\npairs{}}.
The advantages, disadvantages, and justification of ignoring the order
of \divTimeIndexVector is discussed in detail below.
\end{linenomath}

\begin{linenomath}
We implement two prior probability distributions over the space
of all possible discrete divergence models (\divTimeIndexVector).
The first simply gives all possible unordered partitions of \npairs{} elements
equal probability
\begin{equation}
    p(\divTimeIndexVector) = \frac{1}{\integerPartitionNum{\npairs{}}},
    \label{eq:divModelPriorUniform}
\end{equation}
i.e., a discrete uniform prior over all the integer partitions of \npairs{}
(discrete divergence models).
We denote this prior as
$\divTimeIndexVector \sim DU(\integerPartitionSet{\npairs{}})$.
\end{linenomath}

The second prior we implement is based on the Dirichlet-process, which is a
stochastic process that groups elements into an unknown number of discrete
parameter classes \citep{Ferguson1973,Antoniak1974}.
The process is controlled by the concentration parameter \concentrationParam,
which determines how clustered the process will be.
We use the Dirichlet process to place a prior over all possible ordered
partitions of \npairs{} elements, which we denote as $\divTimeIndexVector \sim
DP(\concentrationParam)$.
We use a hierarchical Bayesian approach and place a second-order prior
probability distribution (i.e., hyperprior) on \concentrationParam.  More
specifically, we use a gamma-distributed prior $\concentrationParam =
Gamma(\gammaShape{\concentrationParam}, \gammaScale{\concentrationParam})$,
where \gammaShape{\concentrationParam} and \gammaScale{\concentrationParam} are
specified by the user.

\begin{linenomath}
This provides a great deal of flexibility for specifying the prior uncertainty
regarding divergence models.
The \concentrationParam determines the prior probability that any two
pairs of populations $i$ and $j$ will be assigned to the same divergence time
parameter
\begin{equation}
    p(\divTimeIndex{i} = \divTimeIndex{j}) = \frac{1}{1 + \concentrationParam},
    \label{eq:dppPriorSameClass}
\end{equation}
and also the prior probability of the number of divergence time parameters
\begin{equation}
    p(\divTimeNum \given \concentrationParam, \npairs{}) = 
    \frac{\stirlingFirst{\npairs{}}{\divTimeNum} \concentrationParam^{\divTimeNum}}
    {\prod_{i=1}^{\npairs{}}(\concentrationParam + i - 1)},
    \label{eq:dppPriorNumClasses}
\end{equation}
where \stirlingFirst{\cdot}{\cdot} are the unsigned Stirling numbers of the
first kind.
Equations \ref{eq:dppPriorSameClass} and \ref{eq:dppPriorNumClasses} show that
smaller values of \concentrationParam will favor fewer divergence time
parameters, and thus more clustered models of divergence, whereas larger values
will favor more divergence time parameters, and thus less clustered models of
divergence.
\end{linenomath}

\subsection*{Differences between our model and the original \msb model}
\subsubsection*{The prior on divergence models}
One of the key differences between our model and that of \msb \citep{Huang2011}
is the prior distribution placed on divergence models.
As discussed in \citet{Oaks2012}, in \msb the only prior used for
\divTimeIndexVector is a combination of a discrete uniform prior over the
possible number of divergence events \divTimeNum from 1 to \npairs{} with a
multinomial distribution on the number of times each index of \divTimeVector
appears in \divTimeIndexVector, with the constraint that all \divTime
parameters are represented at least once (see Equation 2 of \citet{Oaks2012}).
\citet{Oaks2012} discuss how placing a uniform prior over the number of
divergence parameters (\divTimeNum here; \numt{} in \citet{Huang2011}) imposes
an odd U-shaped prior over discrete divergence models (\divTimeIndexVector; see
Figure 5B of \citet{Oaks2012}).
To avoid this, we place priors directly on the sample space of divergence
models, thus eliminating the parameter \numt{} from the model.
We introduce two priors on divergence models:
(1) a prior that is uniform over all unordered divergence models, and
(2) a Dirichlet-process prior on all ordered divergence models.
The latter provides the user with a great deal of flexibility in
expressing their prior beliefs about models of divergence.

\subsubsection*{The priors on nuisance parameters}
Following the recommendations of \citet{Oaks2012}, we have replaced the use of
inappropriate, continuous uniform distributions for priors on many of the
model's parameters (\divTime{}, \ancestralTheta{}, \descendantTheta{1}{},
\descendantTheta{2}{}, \bottleScalar{1}{}, \bottleScalar{2}{},
\recombinationRate, \migrationRate{}) with more flexible parametric
distributions from the exponential family.
We introduce gamma-distributed priors for rate parameters that have a sample
space of all positive real numbers (\divTime{}, \ancestralTheta{},
\descendantTheta{1}{}, \descendantTheta{2}{}, \recombinationRate,
\migrationRate{}), and beta-distributed priors for parameters that are
proportions bounded by zero and one (\bottleScalar{1}{} and
\bottleScalar{2}{}).
Not only do these priors have nice conjugacy properties, but they also provide
the user with much greater flexibility in expressing the prior uncertainty
regarding the parameters of the model.

Another distinction between our model and that of \msb, is the prior on the
sizes of the descendant populations of each pair.
As described by \citet{Oaks2012}, \msb uses the joint prior
\begin{equation}
    \descendantTheta{1}{}, \descendantTheta{2}{} \sim
    Dirichlet(1,1) \times 2 \times U(\uniformMin{\myTheta{}},
    \uniformMax{\descendantTheta{}{}}),
    \label{eq:jointThetaPrior}
\end{equation}
such that the user-specified uniform prior on descendant population
size is a prior on the \emph{mean} size of the two descendant
populations of each pair.
Under our model, the sizes of the descendant populations of each
pair are \iid as
$\descendantTheta{1}{} \sim Gamma(\gammaShape{\descendantTheta{}{}},
\gammaScale{\descendantTheta{}{}})$
and
$\descendantTheta{2}{} \sim Gamma(\gammaShape{\descendantTheta{}{}},
\gammaScale{\descendantTheta{}{}})$.
This relaxes the assumption that the sizes of the two descendant populations
are interdependent and negatively correlated.

\subsubsection*{Flexibility in parameterizing the model}
We also provide the user with the ability to control the richness of
the model.
For the \myTheta{} parameters, the model can be fully generalized to
allow each population pair to have three parameters:
\ancestralTheta{}, \descendantTheta{1}{}, and \descendantTheta{2}{}.
Furthermore, any model of \myTheta{} parameters nested within this
general model can also be specified, including the most restricted model
where the ancestral and descendant populations of each pair share
a single \myTheta{} parameter.
We also allow the user to eliminate the parameters associated with
the post-divergence bottlenecks in the descendant populations of
each pair
(\bottleTime{}, \bottleScalar{1}{}, and \bottleScalar{2}{}),
which constrains the descendant populations to be of
constant size from present back to the divergence event.
This allows the user to specify a model that has as many as
seven parameters per population pair
(\ancestralTheta{}, \descendantTheta{1}{}, \descendantTheta{2}{},
\bottleTime{}, \bottleScalar{1}{}, \bottleScalar{2}{}, and
\migrationRate{})
or as few as
one parameter per pair
(\myTheta{}),
in addition to the $\popSampleSize{i}{} - 1$ coalescence times.

\subsubsection*{Time scale}
As described above, in our model divergence times are in units of
$\myTheta{C}/\mutationRate$ generations, where \myTheta{C} is the expectation
of the prior on descendant-population size.
As described by \citet{Oaks2012}, in \msb, \myTheta{C} is half of the upper
limit of the continuous uniform prior on the mean of the descendant-population
sizes.

\subsection*{ABC estimation of the posterior}
To estimate the approximate posterior of Equation \ref{eq:approxModelCompact},
we use an ABC rejection algorithm.
The first step of this algorithm entails collecting a random sample of
parameters values from the joint prior and their associated summary
statistics.
Each sample is generated by
(1) drawing values of all the model's parameters, which we denote \hpvector{}, from their respective
prior distributions;
(2) rescaling the divergence times
$\divTimeMapVector = (\divTimeMap{1}, \ldots, \divTimeMap{\npairs{}})$
from units proportional to time to units proportional to the expected number
of mutations via Equation \ref{eq:divTimeScaled} to get 
$\divTimeScaledVector = (\divTimeScaled{1}{1}, \ldots, \divTimeScaled{\npairs{}}{\nloci{\npairs{}}})$;
(3) simulating gene trees $\geneTreeVector = (\geneTree{1}{1}, \ldots, 
\geneTree{\npairs{}}{\nloci{\npairs{}}})$
for each locus of each population pair by drawing coalescent times from
a multi-population Kingman-coalescent model given the demographic parameters;
(4) simulating sequences alignments 
$\alignmentVector = (\alignment{1}{1}, \ldots, \alignment{\npairs{}}{\nloci{\npairs{}}})$
along the gene trees under the
\hky substitution parameters
$\hkyModelVector = (\hkyModel{1}{1}, \ldots, \hkyModel{\npairs{}}{\nloci{\npairs{}}})$
that have the same number of sequences and sequence lengths as the observed
dataset;
(5) calculating population genetic summary statistics
$\ssVector{} = (\alignmentSS{1}{1}, \ldots, \alignmentSS{\npairs{}}{\nloci{\npairs{}}})$
from the simulated sequence alignments (these are the same statistics calculated from the observed
alignments, \ssVectorObs{});
and (6) reducing the summary statistics to the means across loci for each
population pair to get
$\ssVector{} = (\alignmentSS{1}{}, \ldots, \alignmentSS{\npairs{}}{})$, which is
the same summary statstic vector estimated from the observed data \ssVectorObs.
After repeating this procedure \numPriorSamples times, we obtain a random sample
of parameters
from the model prior and there associated summary statistics
.


This has the advantage of reducing the sample space of the number of
divergence models, but has two disadvantages.
First, additional information about the data is lost.
By throwing away the identity of the \npairs{} pairs of populations, all
information associating each pair with its amount of data (e.g., the number gene copies
collected from each of the populations [\popSampleSize{1}{} and
\popSampleSize{2}{}], the number of loci, and the length of the loci),
information content in the data (i.e., the amount of sequence variation), as
well as its taxon- and locus-specific parameters (\hkyModel{}{},
\mutationRateScalarConstant{}{}, \ploidyScalar{}{}, and
\locusMutationRateScalar{}) is lost.
The original descriptions of the \msb model claim that the independence
between the sample sizes \popSampleSize{1}{}, \popSampleSize{2}{} and
the summary statistic $\pi_b$ calculated from the alignment for each
population pair allows exchangeability.
This is incorrect.
The other summary statistics, and the alignments they are based on are
not independent of the sample sizes.
The alignments and summary statitics are not identically distributed because of
taxon- and locus-specific parameters \hkyModel{}{},
\mutationRateScalarConstant{}{}, and \ploidyScalar{}{}
Second, the identity of the pairs of populations is lost.

\section*{Acknowledgments}
We thank the National Science Foundation for supporting this work (DEB
1011423).
J.\ Oaks was also supported by the University of Kansas (KU) Office of Graduate
Studies, Society of Systematic Biologists, Sigma Xi Scientific Research
Society, KU Department of Ecology and Evolutionary Biology, and the KU
Biodiversity Institute.

\bibliography{../bib/references}

%% LIST OF FIGURES %%%%%%%%%%%%%%%%%%%%%%%%%%
\newpage
\singlespacing

\renewcommand\listfigurename{Figure Captions}
\cftsetindents{fig}{0cm}{2.2cm}
\renewcommand\cftdotsep{\cftnodots}
\setlength\cftbeforefigskip{10pt}
\cftpagenumbersoff{fig}
\listoffigures


\end{linenumbers}

%% TABLES %%%%%%%%%%%%%%%%%%%%%%%%%%%%%%%%
\newpage
\singlespacing


\clearpage

%% FIGURES %%%%%%%%%%%%%%%%%%%%%%%%%%%%%%%%
\newpage

%:FIGURE-accuracy omega SS GLM
\begin{figure}[hbtp]
    \captionsetup{listformat=figList}
    \centering
    \includegraphics[width=1.0\textwidth]{../../validation/results/dpp/pymsbayes-results/plots/prior-dpp_prior-dpp_mc_behavior.pdf}
    \caption{
        Model-choice behavior \ldots
    }
    \label{fig_dpp_dpp_val_mc}
\end{figure}

\clearpage


%%%%%%%%%%%%%%%%%%%%%%%%%%%%%%%%%%%%%%%%%%%%%%%%%%%%%%%%%%%%%%%%%%
%% SUPPORTING INFO %%%%%%%%%%%%%%%%%%%%%%%%%%%%%%%%%%%%%%%%%%%%%%%
\setcounter{figure}{0}
\setcounter{table}{0}
\setcounter{page}{1}

\singlespacing

\renewcommand{\refname}{\noindent\MakeUppercase{\LARGE\sffamily\upshape supporting information}}

% PUT MAIN TEXT CITATION HERE
% \begin{thebibliography}{1}
% \providecommand{\natexlab}[1]{#1}
% \providecommand{\url}[1]{\texttt{#1}}
% \providecommand{\urlprefix}{URL }

% \bibitem

% \end{thebibliography}


%% SUPPL TABLES %%%%%%%%%%%%%%%%%%%%%%%%%%%%
%\setcounter{table}{0}


\end{document}

