\section{Introduction}

Understanding the processes that generate biodiversity and regulate
community assembly is a major goal of evolutionary biology.
Large-scale changes to the environment, including geological and climatic
events, can affect the evolutionary history of entire communities of
co-distributed species and their associated microbiota.
For example, by partitioning communities, such an event can isolate groups of
populations and cause a temporal cluster of speciation events across
co-distributed taxa.
Given the dynamic nature of our planet, such biogeographical processes likely
play a significant role in determining diversification rates and patterns.
At recent timescales, temporal clusters of diversification caused by
biogeographical events can leave a signature in the genetic variation within
and among the affected lineages.
Thus, methods for accurately estimating models of shared evolutionary events
across co-distributed taxa from genetic data are important for better
understanding how regional and global biogeographical processes affect
biodiversity.

This inference problem is challenging due to the stochastic nature by which
mutations occur in populations and how they are inherited over generations
\citep{Hudson1990,WakeleyCoalescent}.
Thus, a method for estimating historical patterns of divergences across taxa
should explicitly model the stochastic mutational and ancestral processes that
generate and filter the genetic variation we observe in present-day genetic
data.
An appealing approach would be a comparative, Bayesian model-choice method for
inferring the probability of competing divergence histories while integrating
over uncertainty in mutational and ancestral processes via models of nucleotide
substitution and lineage coalescence.

The software package \msb implements such an approach in an
approximate-Bayesian model-choice framework \citep{Hickerson2006,Huang2011}.
The method models temporally clustered divergences across taxa caused by a
biogeographical event (or a ``divergence event'') as a single, instantaneous
occurrence.
In other words, a divergence event causes a set of taxa to share the same
moment of divergence along a continuous time scale (i.e., simultaneous
divergence).
Given aligned sequence data for \npairs{} pairs of populations, \msb 
estimates the number of divergence events shared among the pairs, the timing of
the events, and the assignment of pairs to the events, while integrating out
uncertainty in demographic parameters and the genealogical histories of the
sequences.
% samples over all the possible ways of assigning the divergences between the
% populations of each pair to 1--\npairs{} divergence events, while estimating
% the timing of the events and integrating out uncertainty in demographic
% parameters and the genealogical history of the sequences.
Thus, the method samples over all possible divergence models of differing
dimensionality (i.e., all the possible partitions of \npairs{} pairs to
$1,2,\ldots,\npairs{}$ divergence events), and, in so doing, estimates
the posterior probability of each model.

For priors on divergence-time and demographic parameters, \msb uses continuous
uniform probability distributions.
This causes divergence models with more divergence-time parameters to integrate
over a \emph{much} greater parameter space with low likelihood yet high prior
density, resulting in small marginal likelihoods relative to models with fewer
divergence-time parameters \citep{Jeffreys1939,Lindley1957}.
Given that the marginal likelihood of a model weighted by its prior is what
determines its posterior probability, this can cause support for models with
fewer divergence events \citep{Oaks2012}.
Furthermore, \msb uses a discrete uniform prior over the number of divergence
events $1,2,\ldots,\npairs{}$.
Because there are many more possible assignments of populations pairs to
intermediate numbers of divergence events, this imposes a prior on divergence
models that puts most of the prior mass on models with either very few or very
many divergence-time parameters (see Figure 5 of \citet{Oaks2012}; for brevity
I will refer to this prior as ``U-shaped'').
Given that models with many divergence events can have small marginal
likelihoods due to the uniform priors on divergence-time parameters, the
U-shaped prior will effectively create a strong prior preference for models
with very few divergence events.
Recently, \citet{Oaks2012} found via simulation that \msb will often strongly
support biogeographically interesting models with a small number of divergence
events shared among the taxa, even when divergences were random over broad
timescales.
They suggested this behavior was due to the combination of uniform priors on
parameters and the U-shaped prior on divergence models.

% Recently, however, \citet{Oaks2012} found that \msb can be strongly biased
% toward spuriously estimating biogeographically interesting models of highly
% clustered divergence times across taxa.
% The results of their simulation-based assessment of the method suggest that a
% combination of broad, uniformly distributed priors on nuisance parameters and a
% ``U-shaped'' prior on divergence models are causing the bias.
% The model implemented in \msb uses a discrete uniform prior over the number of
% possible divergence parameters, which \citet{Oaks2012} show imposes a prior on
% divergence models that puts most of the prior mass on models with either very
% few or very many divergence times shared across the taxa (see Figure 5 of
% \citet{Oaks2012}; for brevity we will refer to this prior as ``U-shaped'').
% Furthermore, uniform priors used for most of the model's parameters
% result in small marginal likelihoods of parameter-rich models, i.e., models
% with many divergence-time parameters \citep{Jeffreys1935, Lindley1957}.
% Thus, together, the priors used in \msb will often create a bias toward
% models with few divergence-time parameters \citep{Oaks2012}.

Here, I modify the \msb model to test whether alternative parameterizations and
priors will improve the behavior of the approximate-Bayesian model-choice
approach.
I implement a Dirichlet-process prior (DPP) over all possible models of
divergence, and use gamma and beta probability distributions in place of
uniform priors on many of the model's parameters.
Using simulations, I show that the new implementation has improved robustness,
accuracy, and power compared to the original model.
Also, my results confirm that the improved performance of the new model
is due to a combination of
(1) more flexible priors on divergence-time and demographic parameters that
avoid placing high prior density in improbable regions of parameter space, and
(2) a diffuse Dirichlet-process prior that does not strongly disfavor
divergence models with intermediate numbers of divergence events.
% Also, our results confirm that both the uniform priors on demographic and
% divergence-time parameters and the U-shaped prior on divergence models
% contribute to the tendency of \msb.
After reanalyzing sequence data from 22 pairs of taxa from the Philippines
\citep{Oaks2012} under the new model, I find a large amount of posterior
uncertainty in the number of divergence events shared among the taxa; a result
in contrast with the original \msb model and congruent with
intuition given the richness of the model and the relatively small amount of
information in the data.


\section{Methods}

\subsection{The model}
In this section, I describe the model, which is a modification of the model
implemented in \msb \citep{Huang2011,Oaks2012}.
The code implementing the new model is freely available in the open-source
software package \dppmsbayes
(\href{https://github.com/joaks1/dpp-msbayes}{\url{https://github.com/joaks1/dpp-msbayes}}).
To perform the analyses described below, I used the freely avaliable,
open-source software package \pymsbayes
(\href{https://github.com/joaks1/PyMsBayes}{\url{https://github.com/joaks1/PyMsBayes}}),
which is a Python API that provides a multi-processing interface to \msb and
\dppmsbayes.
I performed the work described below following the principles of
Open Notebook Science.
Using version-control software, I make progress in all aspects of the work
freely and publicly available in real-time at
\href{https://github.com/joaks1/msbayes-experiments}{\url{https://github.com/joaks1/msbayes-experiments}}.
All information necessary to reproduce my results is provided there.
I follow much of the notation of \citet{Oaks2012}, but modify it to
aid in my description of the new model.
A summary of my notation can be found in Table \ref{tabNotation}.

I assume an investigator is interested in inferring the distribution
of divergence times among \npairs{} pairs of populations.
For each pair $i$, \popSampleSize{i}{} genome copies have been sampled, with
\popSampleSize{1}{i} copies sampled from population 1, and \popSampleSize{2}{i}
sampled from population 2.
From these genomes, let \nloci{i} be the number of DNA sequence loci collected
for population pair $i$, and \nlociTotal be the total number of unique loci
sampled across the \npairs{} pairs of populations.
I use \alignment{i}{j} to represent the multiple sequence alignment of
locus $j$ for population pair $i$.
$\alignmentVector = (\alignment{1}{1}, \ldots,
    \alignment{\npairs{}}{\nloci{\npairs{}}})$
is the full dataset,
i.e., a vector of sequence alignments for all pairs and loci.
I let \geneTree{i}{j} represent the gene tree upon which \alignment{i}{j}
evolved according to fixed \hky substitution model parameters \hkyModel{i}{j}.
The investigator must specify the parameters of all
$\hkyModelVector = (\hkyModel{1}{1}, \ldots,
\hkyModel{\npairs{}}{\nloci{\npairs{}}})$
substitution models by which the alignments evolved along the
$\geneTreeVector = (\geneTree{1}{1}, \ldots,
\geneTree{\npairs{}}{\nloci{\npairs{}}})$
gene trees.
Furthermore, the investigator must specify a vector of fixed constants
$\ploidyScalarVector = (\ploidyScalar{1}{1}, \ldots,
\ploidyScalar{\npairs{}}{\nloci{\npairs{}}})$
that scale the population-size parameters for known differences in ploidy among
loci and/or differences in generation times among population pairs.  Lastly,
the investigator must also specify a vector of fixed constants
$\mutationRateScalarConstantVector = (\mutationRateScalarConstant{1}{1},
\ldots, \mutationRateScalarConstant{\npairs{}}{\nloci{\npairs{}}})$
that scale the population-size parameters for known differences in
mutation rates among loci and/or among taxa.

\begin{linenomath}
With \alignmentVector, \hkyModelVector, \ploidyScalarVector, and
\mutationRateScalarConstantVector in hand, the joint posterior distribution
of the model is given by Bayes' rule as
\begin{equation}
    p(\geneTreeVector, \divTimeMapVector, \demographicParamVector, 
    \locusMutationRateScalarVector, \locusRateHetShapeParameter \given
    \alignmentVector, \hkyModelVector, \ploidyScalarVector,
    \mutationRateScalarConstantVector) =
    \frac{p(\alignmentVector \given \geneTreeVector, \divTimeMapVector,
        \demographicParamVector, \locusMutationRateScalarVector,
        \locusRateHetShapeParameter, \hkyModelVector, \ploidyScalarVector,
        \mutationRateScalarConstantVector)
        p(\geneTreeVector, \divTimeMapVector, \demographicParamVector,
        \locusMutationRateScalarVector, \locusRateHetShapeParameter \given
        \hkyModelVector, \ploidyScalarVector,
        \mutationRateScalarConstantVector)}{
        p(\alignmentVector \given \hkyModelVector, \ploidyScalarVector,
        \mutationRateScalarConstantVector)},
    \label{eq:fullModelCompactJoint}
\end{equation}
which can be expanded using the chain rule of probability into components that
are assumed to be independent to get
\begin{equation}
    p(\geneTreeVector, \divTimeMapVector, \demographicParamVector, 
    \locusMutationRateScalarVector, \locusRateHetShapeParameter \given
    \alignmentVector, \hkyModelVector, \ploidyScalarVector,
    \mutationRateScalarConstantVector) =
    \frac{p(\alignmentVector \given \geneTreeVector, \hkyModelVector)
        p(\geneTreeVector \given \divTimeMapVector, \demographicParamVector,
        \locusMutationRateScalarVector, \ploidyScalarVector,
        \mutationRateScalarConstantVector)
        p(\locusMutationRateScalarVector \given \locusRateHetShapeParameter)
        p(\locusRateHetShapeParameter)
        p(\divTimeMapVector)
        p(\demographicParamVector)}{
        p(\alignmentVector \given \hkyModelVector, \ploidyScalarVector,
        \mutationRateScalarConstantVector)},
    \label{eq:fullModelCompact}
\end{equation}
where
$\divTimeMapVector = (\divTimeMap{1}, \ldots, \divTimeMap{\npairs{}})$
is a vector of population divergence times for each of the \npairs{} pairs of
populations,
$\demographicParamVector = (\demographicParams{1}, \ldots,
\demographicParams{\npairs{}})$
is a vector of the demographic parameters for each of the \npairs{} population
pairs,
$\locusMutationRateScalarVector = (\locusMutationRateScalar{1}, \ldots
\locusMutationRateScalar{\nlociTotal})$
is a vector of locus-specific mutation-rate scaling parameters for each of the
\nlociTotal loci,
\locusRateHetShapeParameter is the shape parameter of a gamma-distributed
prior on \locusMutationRateScalar{}, and
$p(\alignmentVector \given \hkyModelVector, \ploidyScalarVector,
\mutationRateScalarConstantVector)$
is the probability of the data (or the marginal likelihood of the model) given
the fixed constants provided by the investigator.
\end{linenomath}

\begin{linenomath}
To avoid calculating the likelihood terms of Equation \ref{eq:fullModelCompact},
I distill each sequence alignment \alignment{}{} into a vector of insufficient
summary statistics \alignmentSS{}{}, thus replacing the full dataset
$\alignmentVector = (\alignment{1}{1}, \ldots,
    \alignment{\npairs{}}{\nloci{\npairs{}}})$
with vectors of summary statistics for each alignment
$\ssVectorObs = (\alignmentSSObs{1}{1}, \ldots,
    \alignmentSSObs{\npairs{}}{\nloci{\npairs{}}})$.
For each population pair, the means of the summary statistics are calculated
across the \nloci{} loci, reducing the vector to
$\ssVectorObs = (\alignmentSSObs{1}{}, \ldots,
    \alignmentSSObs{\npairs{}}{})$.
This allows us to estimate the approximate joint posterior distribution
{\small
\begin{equation}
    p(\geneTreeVector, \divTimeMapVector, \demographicParamVector, 
    \locusMutationRateScalarVector, \locusRateHetShapeParameter \given
    \ssSpace, \hkyModelVector, \ploidyScalarVector,
    \mutationRateScalarConstantVector) =
    \frac{p(\ssSpace \given \geneTreeVector, \hkyModelVector)
        p(\geneTreeVector \given \divTimeMapVector, \demographicParamVector,
        \locusMutationRateScalarVector, \ploidyScalarVector,
        \mutationRateScalarConstantVector)
        p(\locusMutationRateScalarVector \given \locusRateHetShapeParameter)
        p(\locusRateHetShapeParameter)
        p(\divTimeMapVector)
        p(\demographicParamVector)}{
        p(\ssSpace \given \hkyModelVector, \ploidyScalarVector,
        \mutationRateScalarConstantVector)},
    \label{eq:approxModelCompact}
\end{equation}
}
where \ssSpace is the multidimensional Euclidean space around the vector of
summary statistics, the radius of which is the tolerance \tol.
The sources of approximation are the insufficiency of the statistics and the
\tol being greater than zero.
I describe the full model in detail before delving into the numerical
method of estimating the approximate model.
\end{linenomath}

\subsubsection{Likelihood terms of Equation \ref{eq:fullModelCompact}}
\begin{linenomath}
The likelihood terms of Equation \ref{eq:fullModelCompact} can
be expanded out as a product over population pairs and loci
\begin{equation}
    p(\alignmentVector \given \geneTreeVector, \hkyModelVector)
    p(\geneTreeVector \given \divTimeMapVector, \demographicParamVector,
    \locusMutationRateScalarVector, \ploidyScalarVector,
    \mutationRateScalarConstantVector) = 
    \prod_{i=1}^{\npairs{}}
    \prod_{j=1}^{\nloci{i}}
    p(\alignment{i}{j} \given \geneTree{i}{j}, \hkyModel{i}{j})
    p(\geneTree{i}{j} \given \divTimeMap{i}, \demographicParams{i},
    \locusMutationRateScalar{j}, \ploidyScalar{i}{j},
    \mutationRateScalarConstant{i}{j}).
    \label{eq:modelLikelihoodExpanded}
\end{equation}
The first term,
$p(\alignment{i}{j} \given \geneTree{i}{j}, \hkyModel{i}{j})$,
is the probability of the sequence alignment of locus $j$ for population pair
$i$ given the gene tree and \hky substitution model parameters \citep[i.e., the
``Felsenstein likelihood'';][]{Felsenstein1981}.
The model allows for an intra-locus recombination rate \recombinationRate,
which, for simplicity, is assumed to be zero in Equation
\ref{eq:fullModelCompact}.
If \recombinationRate is non-zero, this term requires an additional product
over the columns (sites) of each sequence alignment to allow sites to have
different genealogies.
The second term,
p(\geneTree{i}{j} \given \divTimeMap{i}, \demographicParams{i},
\locusMutationRateScalar{j}, \ploidyScalar{i}{j},
\mutationRateScalarConstant{i}{j}),
is the probability of the gene tree under a multi-population coalescent model
(i.e., species tree) where the ancestral population of pair $i$ diverges
and gives rise to the two sampled descendant populations.
Each \demographicParams{} contains the following demographic parameters: The
mutation-scaled effective sizes ($\myTheta{} = 4N\mutationRate$) of the
ancestral, \ancestralTheta{}, and descendant populations, \descendantTheta{1}{}
and \descendantTheta{2}{};
the proportion of the first, \bottleScalar{1}{}, and second population,
\bottleScalar{2}{}, that persist during bottlenecks that begin immediately
after divergence in forward-time;
the proportion of time between present and divergence when the bottlenecks
end for both populations, \bottleTime{};
and the symmetric migration rate between the descendant populations,
\migrationRate{}.
Thus, the probability of the $\popSampleSize{i}{}-1$ coalescence times (node
heights) of gene tree \geneTree{i}{j} is given by a multi-population
Kingman-coalescent model \citep{Kingman1982} where the ancestral population of
size
$\ancestralTheta{i}\ploidyScalar{i}{j}\mutationRateScalarConstant{i}{j}\locusMutationRateScalar{j}$
diverges at time \divTimeMap{i} into two descendant populations of constant
size
$\descendantTheta{1}{i}\ploidyScalar{i}{j}\mutationRateScalarConstant{i}{j}\locusMutationRateScalar{j}\bottleScalar{1}{i}$
and
$\descendantTheta{2}{i}\ploidyScalar{i}{j}\mutationRateScalarConstant{i}{j}\locusMutationRateScalar{j}\bottleScalar{2}{i}$,
which, after time $\divTimeMap{i}\bottleTime{i}$, grow exponentially to their
present size 
$\descendantTheta{1}{i}\ploidyScalar{i}{j}\mutationRateScalarConstant{i}{j}\locusMutationRateScalar{j}$
and
$\descendantTheta{2}{i}\ploidyScalar{i}{j}\mutationRateScalarConstant{i}{j}\locusMutationRateScalar{j}$,
respectively.
Following divergence, the descendant populations of pair $i$ exchange migrants at a symmetric rate of
\migrationRate{i}.
\end{linenomath}


\subsubsection{Prior terms of Equation \ref{eq:fullModelCompact}}
\begin{linenomath}
The term $p(\locusRateHetShapeParameter)$ is the prior density
function for the shape-parameter of the gamma-distributed prior on
rate heterogeneity among loci.
This prior is $\locusRateHetShapeParameter \sim U(1, 20)$.
The prior probability of the vector of locus-specific mutation-rate scaling parameters
given \locusRateHetShapeParameter then expands out as a product over
the loci
\begin{equation}
    p(\locusMutationRateScalarVector \given \locusRateHetShapeParameter) =
    \prod_{j=1}^{\nlociTotal}
    p(\locusMutationRateScalar{j} \given \locusRateHetShapeParameter),
    \label{eq:locusRateHetPrior}
\end{equation}
where each \locusMutationRateScalar{} is independently and identically
distributed (\iid) as
$\locusMutationRateScalar{} \sim Gamma(\locusRateHetShapeParameter,
1/\locusRateHetShapeParameter)$.
If the recombination rate \recombinationRate is allowed to be
non-zero, the prior term $p(\recombinationRate)$ would be added
to Equation \ref{eq:fullModelCompact}, and the prior would be
$\recombinationRate \sim Gamma(\gammaShape{\recombinationRate},
\gammaScale{\recombinationRate})$.
\end{linenomath}

\begin{linenomath}
The prior term for the demographic parameters, $p(\demographicParamVector)$,
expands out into its components and as a product over the \npairs{}
pairs of populations
\begin{equation}
    p(\demographicParamVector) =
    \prod_{i=1}^{\npairs{}}
    p(\ancestralTheta{i})
    p(\descendantTheta{1}{i})
    p(\descendantTheta{2}{i})
    p(\bottleScalar{1}{i})
    p(\bottleScalar{2}{i})
    p(\bottleTime{i})
    p(\migrationRate{i}).
    \label{eq:demographicPrior}
\end{equation}
The priors for the demographic parameters are:
$\ancestralTheta{} \sim Gamma(\gammaShape{\ancestralTheta{}},
\gammaScale{\ancestralTheta{}})$,
$\descendantTheta{1}{} \sim Gamma(\gammaShape{\descendantTheta{}{}},
\gammaScale{\descendantTheta{}{}})$,
$\descendantTheta{2}{} \sim Gamma(\gammaShape{\descendantTheta{}{}},
\gammaScale{\descendantTheta{}{}})$,
$\bottleScalar{1}{} \sim Beta(\betaA{\bottleScalar{}{}},
\betaB{\bottleScalar{}{}})$,
$\bottleScalar{2}{} \sim Beta(\betaA{\bottleScalar{}{}},
\betaB{\bottleScalar{}{}})$,
$\bottleTime{} \sim U(0, 1)$,
and
$\migrationRate{} \sim Gamma(\gammaShape{\migrationRate{}},
\gammaScale{\migrationRate{}})$.
\end{linenomath}

\subsubsection{Priors on divergence models}
\begin{linenomath}
The prior term for the vector of divergence times for each of
the \npairs{} pairs of populations can be expanded as
\begin{equation}
    p(\divTimeMapVector) = p(\divTimeIndexVector)p(\divTimeVector \given \divTimeIndexVector),
    \label{eq:divModelPrior}
\end{equation}
where \divTimeVector is an ordered set of unique divergence-time parameters
$\{\divTime{1}, \ldots, \divTime{\divTimeNum}\}$ whose length
\divTimeNum can range from 1 to \npairs{},
and \divTimeIndexVector is the vector of indices of divergence-time parameters 
$(\divTimeIndex{1}, \ldots, \divTimeIndex{\npairs{}})$
that map the divergence times in \divTimeVector to each of the \npairs{}
pairs of populations.
Thus, \divTimeMapVector is the result of applying the mapping function
\begin{equation}
    f(\divTimeVector, \divTimeIndexVector, i) = \divTime{\divTimeIndex{i}}
    \label{eq:divTimeMapFunction}
\end{equation}
to each population pair $i$, such that
$\divTimeMapVector = (\divTimeMap{1} = f(\divTimeVector, \divTimeIndexVector,
1), \ldots, \divTimeMap{\npairs{}} = f(\divTimeVector, \divTimeIndexVector,
\npairs{}))$.
\end{linenomath}

Biologically speaking, \divTimeVector contains the times of divergence events,
the length of which \divTimeNum is the number of divergence events shared across
the \npairs{} pairs of populations.
For example, if \divTimeVector contains a single divergence time parameter
\divTime{1}, all \npairs{} pairs of populations are constrained to diverge at
this time (i.e., \divTimeIndexVector would contain the index 1 repeated
\npairs{} times, and \divTimeMapVector would contain the value \divTime{1}
repeated \npairs{} times), whereas if it contains \npairs{} divergence time
parameters, the model is fully generalized to allow all of the pairs to diverge
at unique times.

\begin{linenomath}
Each \divTime{} within \divTimeVector is \iid as $\divTime{} \sim
Gamma(\gammaShape{\divTime{}}, \gammaScale{\divTime{}})$.
Thus, given the number of unique divergence-time classes in
\divTimeIndexVector, this determines the probability of prior term
$p(\divTimeVector \given \divTimeIndexVector)$.
The divergence times are in coalescent units relative to the size of
a constant reference population, which I denote \myTheta{C}, that is equal to
the expectation of the prior on the size of the descendant populations
\begin{equation}
    \myTheta{C} = \mathbb{E}(\dThetas),
    \label{eq:thetaC}
\end{equation}
where \popIndex is the descendant-population index which can be either 1 or 2.
Given the size of the descendant populations are \iid as
$\dThetas \sim Gamma(\gammaShape{\descendantTheta{}{}},
\gammaScale{\descendantTheta{}{}})$,
this becomes
\begin{equation}
    \myTheta{C} = \gammaShape{\descendantTheta{}{}}\gammaScale{\descendantTheta{}{}}.
    \label{eq:thetaCGamma}
\end{equation}
The \divTime{} parameters are in units of $\myTheta{C}/\mutationRate$
generations, which I denote as \globalcoalunit generations.
Thus, each \divTime{} within \divTimeVector is proportional to time and can be
converted to the number of generations of the reference population, which I
denote \divTime{G_C}, by assuming a mutation rate and multiplying by the
effective size of the reference population
\begin{equation}
    \divTime{G_C} = \divTime \times \frac{\myTheta{C}}{\mutationRate} = \divTime
    \times \frac{
    \gammaShape{\descendantTheta{}{}}\gammaScale{\descendantTheta{}{}}}
    {\mutationRate}.
    \label{eq:divTimeGenerations}
\end{equation}
Thus, for each of the divergence times in \divTimeVector to be on the same
scale, the relative mutation rates among the pairs of populations are assumed
to be fixed and known according to the user-provided values in
\mutationRateScalarConstantVector.
\end{linenomath}

\begin{linenomath}
As described by \citet{Oaks2012}, to get the divergence times in units
proportional to the expected number of mutations, I must
scale them by the realized population size for locus $j$ of population-pair $i$ 
\begin{equation}
    \divTimeScaled{i}{j} = \divTimeMap{i} \times \frac{\myTheta{C}}{
        \descendantThetaMean{i} \ploidyScalar{i}{j}},
    \label{eq:divTimeScaled}
\end{equation}
where \descendantThetaMean{i} is the mean of \descendantTheta{1}{} and
\descendantTheta{2}{} for pair $i$.
This gives us the vector of scaled divergence times
$\divTimeScaledVector = (\divTimeScaled{1}{1}, \ldots,
\divTimeScaled{\npairs{}}{\nloci{\npairs{}}})$.
\end{linenomath}


\begin{linenomath}
As for the prior term $p(\divTimeIndexVector)$, the total sample space of
ordered realizations of \divTimeIndexVector is all the possible partitions of
\npairs{} elements into 1 to \npairs{} categories.
The total number of possible partitions is a sum of the Stirling numbers of
the second kind over all possible numbers of categories \divTimeNum
\begin{equation}
    B_{\npairs{}}=\sum_{\divTimeNum=1}^{\npairs{}} \left[
    \frac{1}{\divTimeNum!} \sum_{j=0}^{\divTimeNum-1} (-1)^{j}
    \binom{\divTimeNum}{j} (\divTimeNum-j)^{\npairs{}} \right],
    \label{eq:bell}
\end{equation}
which is the Bell Number \citep{Bell1934}.
The original \msb model only samples over the unordered realizations of
\divTimeIndexVector, such that the sample space is reduced to all the possible
integer partitions of \npairs{} \citep{Oaks2012,Huang2011,OeisPartitionNumber,
    OeisPartitionTriangle,Malenfant2011}.
I denote the set of all possible integer partitions of \npairs{} as
\integerPartitionSet{\npairs{}} and the length of that set as
\integerPartitionNum{\npairs{}}.
The advantages, disadvantages, and justification of ignoring the order
of \divTimeIndexVector is discussed in detail below.
\end{linenomath}

\begin{linenomath}
I implement two prior probability distributions over the space
of all possible discrete divergence models (\divTimeIndexVector).
The first simply gives all possible unordered partitions of \npairs{} elements
equal probability
\begin{equation}
    p(\divTimeIndexVector) = \frac{1}{\integerPartitionNum{\npairs{}}},
    \label{eq:divModelPriorUniform}
\end{equation}
i.e., a discrete uniform prior over all the integer partitions of \npairs{}
(discrete divergence models).
I denote this prior as
$\divTimeIndexVector \sim \priorUniform$.
\end{linenomath}

The second prior is based on the Dirichlet-process, which is a
stochastic process that groups elements into an unknown number of discrete
parameter classes \citep{Ferguson1973,Antoniak1974}.
The process is controlled by the concentration parameter \concentrationParam,
which determines how clustered the process will be.
I use the Dirichlet process to place a prior over all possible ordered
partitions of \npairs{} elements, which I denote as $\divTimeIndexVector \sim
\priorDPP{}$.
I take a hierarchical approach and use a prior probability
distribution (i.e., hyperprior) for \concentrationParam.
More specifically, I use a gamma-distributed prior $\concentrationParam \sim
Gamma(\gammaShape{\concentrationParam}, \gammaScale{\concentrationParam})$,
where \gammaShape{\concentrationParam} and \gammaScale{\concentrationParam} are
specified by the investigator.

\begin{linenomath}
This provides a great deal of flexibility for specifying the prior uncertainty
regarding divergence models.
The \concentrationParam determines the prior probability that any two
pairs of populations $i$ and $j$ will be assigned to the same divergence time
parameter
\begin{equation}
    p(\divTimeIndex{i} = \divTimeIndex{j}) = \frac{1}{1 + \concentrationParam},
    \label{eq:dppPriorSameClass}
\end{equation}
and also the prior probability of the number of divergence-time parameters
\begin{equation}
    p(\divTimeNum \boldsymbol{\mid} \concentrationParam, \npairs{}) = 
    \frac{\stirlingFirst{\npairs{}}{\divTimeNum} \concentrationParam^{\divTimeNum}}
    {\prod_{i=1}^{\npairs{}}(\concentrationParam + i - 1)},
    \label{eq:dppPriorNumClasses}
\end{equation}
where \stirlingFirst{\cdot}{\cdot} are the unsigned Stirling numbers of the
first kind.
Equations \ref{eq:dppPriorSameClass} and \ref{eq:dppPriorNumClasses} show that
smaller values of \concentrationParam will favor fewer divergence-time
parameters, and thus more clustered models of divergence, whereas larger values
will favor more divergence-time parameters, and thus less clustered models of
divergence.
\end{linenomath}

\subsection{Differences between my model and the original \texttt{msBayes} model}
\subsubsection{The prior on divergence models}
One of the key differences between my model and that of \msb \citep{Huang2011}
is the prior distribution on divergence models.
As discussed in \citet{Oaks2012}, in \msb the prior used for
\divTimeIndexVector is a combination of a discrete uniform prior over the
possible number of divergence events \divTimeNum from 1 to \npairs{} with a
multinomial distribution on the number of times each index of \divTimeVector
appears in \divTimeIndexVector, with the constraint that all \divTime{}
parameters are represented at least once (see Equation 2 of \citet{Oaks2012}).
I denote this prior used by \msb as $\divTimeIndexVector \sim \priorOld$.
\citet{Oaks2012} discuss how placing a uniform prior over the number of
divergence parameters (denoted \divTimeNum here, and as \numt{} in
\citet{Huang2011}) imposes an U-shaped prior over discrete divergence
models (\divTimeIndexVector; see
Figure 5B of \citet{Oaks2012}).
To avoid this, I place priors directly on the sample space of divergence
models, thus eliminating the parameter \numt{} from the model.
I introduce two priors on divergence models:
(1) a prior that is uniform over all unordered divergence models, and
(2) a Dirichlet-process prior on all ordered divergence models.
The latter provides an investigator with a great deal of flexibility in
expressing their prior beliefs about models of divergence.

\subsubsection{Estimating ordered divergence models}
As mentioned above, \msb samples over unordered divergence models
(i.e., unordered partitions of the \npairs{} pairs of populations).
That is, the identity of each population pair, and all the information
associated with it, is discarded.
In my implementation, inference can be done on either unordered or ordered
models of divergence.
This is discussed in more detail in the description of the ABC implementation
below.

\subsubsection{The priors on nuisance parameters}
I have replaced the use of continuous uniform distributions for priors on many
of the model's parameters (\divTime{}, \ancestralTheta{},
\descendantTheta{1}{}, \descendantTheta{2}{}, \bottleScalar{1}{},
\bottleScalar{2}{}, \recombinationRate, \migrationRate{}) with more flexible
parametric distributions from the exponential family.
I introduce gamma-distributed priors for rate parameters that have a sample
space of all positive real numbers (\divTime{}, \ancestralTheta{},
\descendantTheta{1}{}, \descendantTheta{2}{}, \recombinationRate,
\migrationRate{}), and beta-distributed priors for parameters that are
proportions bounded by zero and one (\bottleScalar{1}{} and
\bottleScalar{2}{}).
These priors provide an investigator with much greater flexibility in
expressing prior uncertainty regarding the parameters of the model.

In addition, I have modified the prior on the sizes of the descendant
populations of each pair.
As described by \citet{Oaks2012}, \msb uses the joint prior
\begin{equation}
    \descendantTheta{1}{}, \descendantTheta{2}{} \sim
    Beta(1,1) \times 2 \times U(\uniformMin{\myTheta{}},
    \uniformMax{\descendantTheta{}{}}),
    \label{eq:jointThetaPrior}
\end{equation}
such that the user-specified uniform prior on descendant population
size is a prior on the \emph{mean} size of the two descendant
populations of each pair.
Under my model, the sizes of the descendant populations of each
pair are \iid as
$\descendantTheta{1}{} \sim Gamma(\gammaShape{\descendantTheta{}{}},
\gammaScale{\descendantTheta{}{}})$
and
$\descendantTheta{2}{} \sim Gamma(\gammaShape{\descendantTheta{}{}},
\gammaScale{\descendantTheta{}{}})$.
This relaxes the assumption that the sizes of the two descendant populations
are interdependent and negatively correlated.

\subsubsection{Flexibility in parameterizing the model}
In my implementation, I provide the ability to control the richness of the
model.
For the \myTheta{} parameters, the model can be fully generalized to
allow each population pair to have three parameters:
\ancestralTheta{}, \descendantTheta{1}{}, and \descendantTheta{2}{}.
Furthermore, any model of \myTheta{} parameters nested within this
general model can also be specified, including the most restricted model
where the ancestral and descendant populations of each pair share
a single \myTheta{} parameter.

I also provide the option of eliminating the parameters associated with the
post-divergence bottlenecks in the descendant populations of each pair
(\bottleTime{}, \bottleScalar{1}{}, and \bottleScalar{2}{}),
which constrains the descendant populations to be of
constant size from present back to the divergence event.
Also, rather than eliminate the bottleneck parameters,
I allow \bottleScalar{1}{} and \bottleScalar{2}{} to be constrained to be
equal, which removes one free parameter from the model for each of the
population pairs.

Overall, my implementation allows an investigator to specify a model that has
as many as seven parameters per population pair
(\ancestralTheta{}, \descendantTheta{1}{}, \descendantTheta{2}{},
\bottleTime{}, \bottleScalar{1}{}, \bottleScalar{2}{}, and
\migrationRate{})
or as few as one parameter per pair
(\myTheta{}),
in addition to the $\popSampleSize{i}{} - 1$ coalescence-time parameters.

\subsubsection{Time scale}
As described above, in my model divergence times are in units of
$\myTheta{C}/\mutationRate$ generations, where \myTheta{C} is the expectation
of the prior on descendant-population size.
As described by \citet{Oaks2012}, in \msb, \myTheta{C} is half of the upper
limit of the continuous uniform prior on the mean of the descendant population
sizes.

\subsection{ABC estimation of the posterior of the model}
\subsubsection{Sampling from the prior}
To estimate the approximate posterior of Equation \ref{eq:approxModelCompact},
I use an ABC rejection algorithm.
The first step of this algorithm entails collecting a random sample of
parameter values from the joint prior and their associated summary
statistics.
Each sample is generated by
(1) drawing values of all the model's parameters, which I denote \hpvector{},
from their respective prior distributions;
(2) rescaling the divergence times
$\divTimeMapVector = (\divTimeMap{1}, \ldots, \divTimeMap{\npairs{}})$
from units proportional to time to units proportional to the expected number
of mutations via Equation \ref{eq:divTimeScaled} to get 
$\divTimeScaledVector = (\divTimeScaled{1}{1}, \ldots,
\divTimeScaled{\npairs{}}{\nloci{\npairs{}}})$;
(3) simulating gene trees $\geneTreeVector = (\geneTree{1}{1}, \ldots, 
\geneTree{\npairs{}}{\nloci{\npairs{}}})$
for each locus of each population pair by drawing coalescent times from
a multi-population Kingman-coalescent model given the demographic parameters;
(4) simulating sequence alignments 
$\alignmentVector = (\alignment{1}{1}, \ldots, \alignment{\npairs{}}{\nloci{\npairs{}}})$
along the gene trees under the
\hky substitution parameters
$\hkyModelVector = (\hkyModel{1}{1}, \ldots, \hkyModel{\npairs{}}{\nloci{\npairs{}}})$
that have the same number of sequences and sequence lengths as the observed
dataset;
(5) calculating population genetic summary statistics
$\ssVector{} = (\alignmentSS{1}{1}, \ldots, \alignmentSS{\npairs{}}{\nloci{\npairs{}}})$
from the simulated sequence alignments;
% (these are the same statistics
% calculated from the observed alignments, \ssVectorObs{});
and (6) reducing the summary statistics to the means across loci for each
population pair to get
$\ssVector{} = (\alignmentSS{1}{}, \ldots, \alignmentSS{\npairs{}}{})$, which is
the same summary statistic vector estimated from the observed data \ssVectorObs.
After repeating this procedure \numPriorSamples times, I obtain a random
sample of parameter vectors
$\paramSampleMatrix = (\paramSampleVector{1}, \ldots, \paramSampleVector{\numPriorSamples})$
from the model prior and their associated vectors of summary statistics
$\ssMatrix = (\ssVector{1}, \ldots, \ssVector{\numPriorSamples})$.

For all of the analyses below, I use four summary statistics for each of
the pairs of populations:
$\pi$ \citep{Tajima1983}, $\theta_W$ \citep{Watterson1975}, $\pi_{net}$
\citep{Takahata1985}, and $SD(\pi-\theta_W)$ \citep{Tajima1989}.
Furthermore, in addition to model parameters, each sample \hpvector{}
also contains four statistics that summarize \divTimeMapVector:
the mean (\divTimeMean), variance (\divTimeVar), dispersion index
($\divTimeDispersion = \divTimeVar/\divTimeMean$), and the number
of divergence time parameters (\divTimeNum).
These have been denoted as \meant{}{}, \vart{}{}, \vmratio{}, and \numt{},
respectively \citep{Hickerson2006,Huang2011,Oaks2012}.
I use the new notation to avoid confusion that the values represent prior or
posterior expectations, and to indicate that \vmratio{} is not a parameter of
the model and \numt{} is no longer a parameter in the new implementation.

\subsubsection{Ordering of taxon-specific summary statistics}
As alluded to in the model description, \msb does not maintain the order of the
taxon-specific summary statistics \alignmentSS{}{} within each \ssVector{}.
Rather, the summary statistics are re-ordered by descending values of average
pairwise differences between the descendant populations
\citep[$\pi_b$;][]{NeiLi1979,Huang2011}.
This has the advantage of reducing the sample space of the number of discrete
divergence models \divTimeIndexVector, but there are at least two disadvantages.
First, additional information in the data is lost.
By discarding the identity of the \npairs{} pairs of populations, all
pair-specific information about the amount of data (e.g., the number gene
copies collected from each of the populations [\popSampleSize{1}{} and
\popSampleSize{2}{}], the number of loci, and the length of the loci), and the
taxon- and locus-specific parameters (\hkyModel{}{},
\mutationRateScalarConstant{}{}, \ploidyScalar{}{}, and
\locusMutationRateScalar{}) is lost.
Second, the results are more difficult to interpret, because they can be no
longer be directly associated to the taxa under study.

The original descriptions of the \msb model claim that this re-ordering is
justified by the fact that of $\pi_b$ (and other summary statistics) are
unrelated to the sample sizes \popSampleSize{1}{} and \popSampleSize{2}{} of
each pair and are thus exchangeable \citep{Hickerson2006,Huang2011}.
This is actually incorrect for two reasons.
First, the expectation of $\pi_b$ is not independent of samples sizes.
If there are more than one coalescent events in the ancestor, which is expected
to be common on phylogeographic timescales, more samples will increase the
probability of capturing these deeper events, and thus affect the average
pairwise differences between the descendant populations.
Also, other statistics that estimate gross diversity (e.g., $\pi$ and
$\theta_W$) are clearly not independent of sample size.
Second, for variables to be exchangeable, they do not need to be independent,
but their marginal distributions must be the same (i.e., they must be
identically distributed).
The simulated alignments and their summary statistics are not identically
distributed because of differing sample sizes \emph{and} taxon- and
locus-specific parameters \hkyModel{}{}, \mutationRateScalarConstant{}{}, and
\ploidyScalar{}{}.

Thus, the theoretical basis for this reshuffling is questionable.
It may be justifiable as an additional coarsening of the data,
by removing the information associated with the identity of the
population pairs.
However, I do not provide a formal proof that this re-ordering of the pairs
does not introduce bias.
One can imagine situations in which the sampling intensity (i.e., the
number of gene copies, loci, and locus length) is highly skewed across
pairs and/or there is large heterogeneity in mutational processes
(e.g., \hkyModel{}{}, \mutationRateScalarConstant{}{}) among the pairs.
In such cases, it seems possible that discarding this information could
cause bias.
Furthermore, given that (1) part of the motivation for re-ordering by $\pi_b$
is to minimize Euclidean distances between simulated datasets in which the true
model has a single divergence event, and (2) the reordering increases the
model's tendency to infer a single divergence event \citep{Huang2011}, it seems
possible that this approach could be biasing the method \citep{Oaks2012}.

To maintain compatibility and comparability with \msb I maintain
the re-ordering of taxon-specific summary statistics by $\pi_b$.
However, I also allow the order to be maintained, and ordered
divergence models to be estimated.

\subsubsection{Obtaining an approximate posterior from the prior samples}
I use a rejection algorithm to retain an approximate posterior sample of
\paramSampleVector{} from the prior sample
$\paramSampleMatrix = (\paramSampleVector{1}, \ldots, \paramSampleVector{\numPriorSamples})$.
First, the observed summary statistics \ssVectorObs, and the summary statistics
of the prior samples 
$\ssMatrix = (\ssVector{1}, \ldots, \ssVector{\numPriorSamples})$,
are standardized using the means and standard deviations of the statistics from
the prior sample (i.e., the prior mean is subtracted from each statistic, and the
difference is divided by the prior standard deviation).
After all statistics are standardized, the Euclidean distance between
\ssVectorObs and each \ssVector{} within \ssMatrix is calculated.
The samples that fall within a range of tolerance \tol around \ssVectorObs
are retained.
The range of tolerance is determined by specifying the desired number of
posterior samples to be retained.
Post-hoc adjustment of the posterior sample can also be performed with a number
of regression techniques \citep{Beaumont2002,Blum2009,Leuenberger2010}.
For my analyses, I use the general linear model (GLM) regression adjustment
\citep{Leuenberger2010} as implemented in \abctoolbox
\citep[v1.1;][]{ABCtoolbox}, which \citet{Oaks2012} showed performed very
similarly to weighted local-linear regression and multinomial logistic
regression adjustments \citep{Beaumont2002} for \msb posteriors.


\subsection{Assessing model-choice behavior and robustness}
Following the simulation-based approach of \citet{Oaks2012}, I characterize
the behavior of several models under the ideal conditions where the
data are generated from parameters drawn from the same prior distributions used
for analysis (i.e., the prior is correct).
I selected four model priors for these analyses (Table~\ref{tabPriors}):
(1) A model to represent the original \msb model, \modelOld, with
priors
$\divTimeIndexVector \sim \priorOld$,
$\divTime{} \sim U(0,10)$,
$\ancestralTheta{} \sim U(0, 0.05)$,
and
$\descendantThetaMean{} \sim U(0, 0.05)$;
(2) a model with the \msb prior on unordered divergence models, but with
exponential priors on nuisance parameters,
\modelUshaped, with priors
$\divTimeIndexVector \sim \priorOld$,
$\divTime{} \sim Exp(mean=2.887)$,
$\ancestralTheta{} \sim Exp(mean=0.025)$,
and
$\dThetas \sim Exp(mean=0.025)$.
(3) a model with uniform prior probability over unordered divergence models,
\modelUniform, with priors
$\divTimeIndexVector \sim \priorUniform$,
$\divTime{} \sim Exp(mean=2.887)$,
$\ancestralTheta{} \sim Exp(mean=0.025)$,
and
$\dThetas \sim Exp(mean=0.025)$;
and
(4) a Dirichlet-process prior model, \modelDPP, with priors
$\divTimeIndexVector \sim \priorDPP{\sim Gamma(2,2)}$,
$\divTime{} \sim Exp(mean=2.887)$,
$\ancestralTheta{} \sim Exp(mean=0.025)$,
and
$\dThetas \sim Exp(mean=0.025)$.
I selected the exponential prior on divergence time used in models \modelDPP,
\modelUniform, and \modelUshaped to have the same variance as the uniform prior
in model \modelOld.
I selected the exponential prior on population size used in models \modelDPP,
\modelUniform, and \modelUniform to have the same mean as the uniform prior in
model \modelOld, so that all four models have the same \myTheta{C} and thus the
same units of time.
All of the models were the same in other respects, with three free \myTheta{}
parameters for each population pair, two uniformly distributed ($beta(1,1)$)
\bottleScalar{}{} parameters per pair, no migration, no recombination,
and unordered divergence models.
For all of my simulations, I used a simulated data structure of eight
population pairs, with a single 1000 base-pair locus sampled from 10
individuals from each population.

For each of the four models, I simulated $1\e6$ samples from the prior and
50,000 datasets, also drawn from the prior.
I then analyzed each of the simulated datasets, retaining a
posterior of 1000 samples from the respective prior.
A GLM-regression adjusted posterior was also estimated from each of the
posterior samples \citep{Leuenberger2010}.
To assess the robustness of each of the four models, I also analyzed the
datasets simulated under the other three models.
Overall, for each model, I produced 200,000 posterior estimates,
50,000 from the datasets simulated under that model,
and 150,000 estimated from the datasets simulated under the
other three models.

For each set of 50,000 simulated datasets, I used the posterior estimates
to assess the model-choice behavior of each model.
I did this by assigning the 50,000 estimates of the posterior probability
of one divergence event to 20 bins of width 0.05, and plotted
the estimated posterior probability of each bin against the proportion of
replicates in that bin with a true value consistent with one divergence
event \citep{Huelsenbeck2004,Oaks2012}.
I did this using two criteria for the one divergence model:
(1) the number of divergence time parameters ($\divTimeNum = 1$) and
(2) the dispersion index of divergence times \citep[$\divTimeDispersion <
0.01$][]{Hickerson2006,Huang2011}.
I used the one-divergence model to assess model-choice behavior, because
it is often of biogeographic interest and is easily comparable among
the three different priors used on divergence models.

In addition to the four models above, I also assessed the behavior of a model
with ordered divergence models (i.e., the order of the taxon-specific summary
statistic vectors were maintained for the observed and simulated datasets).
For this, I used a model with identical priors as \modelDPP, but that samples
over ordered divergence models.
I denote this model as \modelDPPOrdered.
I simulated $1\e6$ prior samples and 50,000 datasets, and
analyzed them as above.
I was not able to analyze the simulated datasets of the other models under
the ordered model, because the identity of the population pairs is not
contained in the simulations of the other models.

\subsection{Assessing power}
I evaluated the power of the same four models (Table~\ref{tabPriors}) to
detect random variation in divergence times using methods similar to
\citet{Oaks2012}.
The only difference from the prior models used in the validation analyses above
is the prior on the concentration parameter \concentrationParam for the
\modelDPP model.
Rather than the prior
$\divTimeIndexVector \sim \priorDPP{\sim Gamma(2,2)}$,
I use the prior
$\divTimeIndexVector \sim \priorDPP{\sim Gamma(1.5,18.1)}$
over the discrete divergence models for the model \modelDPP.
The reason for this difference is because for all of the power simulations, I
used a simulated data structure identical to that of the empirical dataset of
Philippine vertebrates analyzed by \citet{Oaks2012}, which consists of 22 pairs
of populations.
For each of the four models, I generated $2\e6$ samples from the prior.
I then simulated datasets from three series of models
(Table~\ref{tabPowerModels}).
One series of models, which I denote \powerSeriesOld, is identically
distributed as \modelOld except that the divergence times for each of the 22
pairs of populations are randomly drawn from a series of uniform distributions,
$U(0, \divt{max})$, where \divt{max} was set to: 0.2, 0.4, 0.6, 0.8, 1.0, and
2.0, in \globalcoalunit generations.
A second series of models, \powerSeriesUniform, is identically distributed as
\modelUniform and \modelDPP except that the 22 divergence times are randomly
drawn from the same series of uniform priors as above.
The third series of models, \powerSeriesExp, is also identically distributed as
\modelUniform and \modelDPP except that the 22 divergence times are randomly
drawn from a series of of exponential distributions:
$Exp(mean=0.058)$, $Exp(mean=0.115)$, $Exp(mean=0.173)$, $Exp(mean=0.231)$,
$Exp(mean=0.289)$, and $Exp(mean=0.577)$.
These exponential distributions have the same variance as their uniform
counterparts in the first two series of models.

For each of the six models in each of the three series of models, I simulated
1000 datasets (18,000 datasets in total).
I then analyzed each simulated dataset under all four prior models
(Table~\ref{tabPriors}), producing 72,000 posterior estimates, each with 1000
samples.
I also estimated a GLM-regression adjusted posterior from each of the
posterior samples \citep{Leuenberger2010}.

\subsection{An empirical application}
I also assessed the behavior of the newly implemented models when applied to
the empirical dataset of \citet{Oaks2012} comprised of sequence data from 22
pairs of taxa from the Philippine Islands (Drayad DOI: 10.5061/dryad.5s07m).
I analyzed these data under five different models, which are detailed in
Table~\ref{tabEmpiricalModels}.
All of these models except one (\empModelDPPSimple) have six free demographic
parameters per pair of taxa (
\ancestralTheta{},
\descendantTheta{1}{},
\descendantTheta{2}{},
\bottleTime{},
\bottleScalar{1}{}, and
\bottleScalar{2}{}),
in addition to the $\popSampleSize{i}{} - 1$ coalescent times.
Three of these models use a Dirichlet-process prior on divergence models:
\empModelDPP, \empModelDPPInform, and \empModelDPPSimple.
The \empModelDPP model is meant to represent the prior that \citet{Oaks2012}
would have selected to represent their prior uncertainty about the parameters
of the model if provided the more flexible distributions that are now
implemented.
To assess prior sensitivity, the \empModelDPPInform model uses a more
informative exponentially distributed prior on divergence times, but otherwise
is identical to \empModelDPP.
To assess sensitivity to parameterization, I also applied the simplest
possible model under the new implementation (\empModelDPPSimple) with only a
single demographic parameter (\myTheta{}) per taxon pair, in addition to the
$\popSampleSize{i}{} - 1$ coalescent times.
I also applied the original \msb model (\empModelOld) with priors
selected to make it comparable to the \empModelDPP model;
the uniform prior on divergence times was selected to have the same variance as
the exponential prior of the \empModelDPP, and the prior on population
size was selected to have the same mean so that the models are on the same
timescale.
I also applied a model with a uniform distribution over divergence models
(\empModelUniform).
For each of these models, I simulated 2\e{7} samples from the prior, and
retained an approximate posterior of the 10,000 samples with the smallest
Euclidean distance from the summary statistics calculated from the empirical
sequence alignments.

To compare models that sample over ordered versus unordered models of
divergence, I also analyzed the data from the subset of nine-taxon pairs that
are sampled from the Islands of Negros and Panay in the Philippines.
The model I used for these analyses had a Dirichlet-process prior over
divergence models and two demographic parameters (\ancestralTheta{} and
\descendantTheta{}{}) for each pair of taxa, in addition to the
$\popSampleSize{i}{} - 1$ coalescent times (see Table~\ref{tabEmpiricalModels}
for details).
One of the models, which I denote \npModelDPPOrdered, maintained the identity
of the taxon pairs and sampled over ordered models of divergence, while the
other (\npModelDPP) re-sorted the summary statistics of the pairs by $\pi_b$,
losing the identity of the taxa and thus sampled over unordered models of
divergence.
For both analyses, I simulated 5\e{7} samples from the prior and retained an
approximate posterior of 10,000 samples.

\section{Results}
\subsection{Validation analyses: Estimation accuracy}
In terms of estimating the variance of divergence times (\divTimeDispersion),
the models with exponentially distributed priors (\modelUshaped, \modelUniform,
and \modelDPP) perform similarly when applied to datasets generated under all
four of the models of Table \ref{tabPriors}
(Figure~S\ref{figValidationAccuracyOmega}).
The \modelOld model performs similarly to these models when applied to its
own datasets, but is sensitive to model violations and is more biased when
applied to data generated under the other three models
(Figure~S\ref{figValidationAccuracyOmega}).
Results are similar for the GLM-adjusted estimates of \divTimeDispersion,
albeit the regression adjustment tends to improve estimates of this continuous
statistic for all the models
(Figure~S\ref{figValidationAccuracyPsiGlm}).

The same general pattern is seen for estimates of \divTimeMean, with all four
models performing similarly when applied to the data generated under the
\modelOld model, the models with exponentially distributed priors performing
similarly when applied to data generated under the other three models, and the
\modelOld model is sensitive to model violations and is more biased whenever
applied to data generated under other models
(Figure~S\ref{figValidationAccuracyTime}).
Also, the regression adjustment tends to slightly improve estimates of
this continuous statistic for all of the models
(Figure~S\ref{figValidationAccuracyTimeGlm}).

In terms of estimating the number of divergence events (\divTimeNum), the
\modelDPP model has the lowest root mean square error (RMSE) when applied
to data generated under most of the models of Table \ref{tabPriors}
(Figure~S\ref{figValidationAccuracyPsi}).
The \modelOld model performs slightly better when applied to its own data,
but is the worst performer when applied to data generated under other models
(Figure~S\ref{figValidationAccuracyPsi}).
There is a trend of $\modelDPP > \modelUniform > \modelUshaped > \modelOld$
in terms of estimation accuracy as measured by RMSE when the models are applied
to data generated under most of the models
(Figure~S\ref{figValidationAccuracyPsi}).
Unlike for the continuous statistics, regression adjustment of this discrete
statistic tends to increase estimation bias; all of the models tend to
underestimate \divTimeDispersion after the GLM-adjustment
(Figure~S\ref{figValidationAccuracyPsiGlm}).

\subsection{Validation analyses: Model-choice accuracy}
The \msb model, and my modification of it, is a model-choice method
with the primary purpose of estimating the probabilities of models
of divergence across taxa.
Thus, it is critical to assess the method's ability to accurately
estimate the posterior probabilities of divergence models.
Consistent with the findings of \citet{Oaks2012}, my results demonstrate that
the unadjusted estimates of divergence-model posterior probabilities are
generally more accurate than regression-adjusted estimates
(compare the plots along the upper-left to lower-right diagonal for
Figure~\ref{figValidationModelChoicePsi} versus
S\ref{figValidationModelChoicePsiGlm} and
Figure~\ref{figValidationModelChoiceOmega} versus
S\ref{figValidationModelChoiceOmegaGlm}).
Regression adjustment results in biased estimates of the posterior probability
of the one-divergence model when all model assumptions are satisfied, which is
well illustrated in Figure S\ref{figValidationModelChoiceOmegaGlm}.
As a result, I will focus my discussion of the results on the unadjusted
estimates.

I find that all four models accurately estimate the posterior probability of
the one-divergence model when applied to their own datasets (i.e., when the
prior is correct; see diagonal of Figures~\ref{figValidationModelChoicePsi} \&
\ref{figValidationModelChoiceOmega}).
The \modelUniform and \modelDPP show robustness to prior violations and
perform well when applied to data generated under other models
(Figures~\ref{figValidationModelChoicePsi} \&
\ref{figValidationModelChoiceOmega}).
However, both are less accurate and tend to underestimate the probability of the
one-divergence model when applied to the data generated under \modelUshaped
(Figures~\ref{figValidationModelChoicePsi} \&
\ref{figValidationModelChoiceOmega}).
In contrast, the \modelOld model is biased toward overestimating the posterior
probability of the one-divergence model when applied to data generated under
the other three models
(Figures~\ref{figValidationModelChoicePsi} \&
\ref{figValidationModelChoiceOmega}).
This bias is particularly strong whenever divergence models are not
distributed under its U-shaped prior
(Figure~\ref{figValidationModelChoicePsi}C--D).
The other model with the U-shaped prior on divergence models, but exponential
priors on parameters (\modelUshaped), performs similarly to the \modelOld model
in that it performs well when applied to its own data, but overestimates the
probability of the one-divergence model when applied to data generated by the
other models
(Figures~\ref{figValidationModelChoicePsi} \&
\ref{figValidationModelChoiceOmega}).
However, the bias is stronger in the \modelOld model than \modelUshaped.

Overall, the results suggest that the \modelDPP and \modelUniform models are
relatively robust in terms of model-choice accuracy, and when model violations
do cause them to be biased, they tend to under-estimate the probability of the
biogeographically interesting model of a single divergence event.
In contrast, the \modelOld model is very sensitive to model violations, and
strongly over-estimates the one-divergence model whenever the model is
misspecified.
Furthermore, the results suggest that using exponentially distributed priors on
nuisance parameters rather than uniform priors helps the \modelUshaped model
perform better than \modelOld, but it is still hindered by the U-shaped prior
on divergence models and tends to over-estimate the probability of the
one-divergence model whenever there are violations of the model.


\subsection{Validation analyses: Ordered divergence models}
The results show that the method performs similarily when sampling over ordered
models of divergence
(Figures S\ref{figValidationAccuracyDPPOrderedDPPOrdered} \&
S\ref{figValidationModelChoiceDPPOrderedDPPOrdered}).
This suggests that the method is not adversely affected by the large increase
in the number of possible discrete models (from 22 unordered to 4140 ordered
models) when there are eight pairs of populations.

\subsection{Power analyses: Estimation accuracy}
All of the models I evaluated (Table \ref{tabPriors}) struggle to estimate the
variance of divergence times \divTimeDispersion regardless of which of the
three series of models (Table \ref{tabPowerModels}) the data were generated
under
(Figures
S\labelcref{figPowerAccuracyOld,figPowerAccuracyUniform,figPowerAccuracyExp}).
The models with the U-shaped prior on divergence models (\modelOld and
\modelUshaped) tend to underestimate the variance in divergence times 
(Plots A--L of Figures
S\labelcref{figPowerAccuracyOld,figPowerAccuracyUniform,figPowerAccuracyExp}).
whereas the models with Uniform or Dirichlet-process priors over divergence
models tend to overestimate variance in divergence times
(Plots M--X of Figures
S\labelcref{figPowerAccuracyOld,figPowerAccuracyUniform,figPowerAccuracyExp}).

When the divergence times of the 22 population pairs are randomly drawn from a
series of exponential priors (\powerSeriesExp), the \modelDPP model is the
best estimator of \divTimeDispersion, followed by \modelUniform
(Figure~S\ref{figPowerAccuracyExp}).
The \modelOld model is strongly biased toward underestimating
\divTimeDispersion, estimating values of zero for most of the replicates across
all the data models of \powerSeriesExp (Figure~S\ref{figPowerAccuracyExp}).
The results of the \modelUshaped model are intermediate between those of
\modelOld and the new models \modelDPP and \modelUniform
(Figure~S\ref{figPowerAccuracyExp}).

Similarly, when the true divergence times are randomly drawn from a series of
uniform priors (\powerSeriesUniform), the \modelDPP and \modelUniform models
tend to over-estimate the variance in divergence times, whereas the \modelOld
model underestimates \divTimeDispersion, estimating values of zero for most
replicates across all the data models of \powerSeriesUniform
(Figure~S\ref{figPowerAccuracyUniform}).
Again, the performance of the \modelUshaped model is intermediate between the
\modelOld and \modelDPP/\modelUniform models (Figure~S\ref{figPowerAccuracyUniform}).
The results are very similar when the four models are applied to the data
simulated under the \powerSeriesOld series of models
(Figure~S\ref{figPowerAccuracyOld}).

\subsection{Power analyses: Model choice}
The modifications of the \msb model decrease the method's bias toward clustered
divergences when applied to data generated under random divergence times
(Figure~\ref{figPowerPsiOld4} \& 
S\labelcref{figPowerPsiOld,figPowerPsiUniform,figPowerPsiExp}).
The \modelOld model performs the worst of the four models across
all three series of data-generating models, inferring a single divergence event across
most of the 18,000 simulations
(Figure~\ref{figPowerPsiOld4}A--D \& plots A--F of Figures
S\labelcref{figPowerPsiOld,figPowerPsiUniform,figPowerPsiExp}).
Importantly, the \modelOld tends to strongly support these estimates
of one divergence across most of the simulations
(Figure~\ref{figPowerPsiProbOld4}A--D \& plots A--F of Figures
S\labelcref{figPowerPsiProbOld,figPowerPsiProbUniform,figPowerPsiProbExp}).
The \modelDPP model also prefers the one-divergence model 
when divergences are random over narrow windows of time, but
performs much better when divergences are random over a timescale
of 1--2 coalescent units
(Figure~\ref{figPowerPsiOld4}M--P \& plots S--X of Figures
S\labelcref{figPowerPsiOld,figPowerPsiUniform,figPowerPsiExp}).
However, even when \modelDPP infers the one-divergence model over narrow
timescales, the posterior probability support is always low
(Figure~\ref{figPowerPsiProbOld4}M--P \& plots S--X of Figures
S\labelcref{figPowerPsiProbOld,figPowerPsiProbUniform,figPowerPsiProbExp}).
% Even when the simulated data are identically distributed as the \modelOld model
% (except for the divergence times) the \modelOld is still strongly biased
% towards a single event (Figure \ref{figPowerPsiOldOld}), whereas the \modelDPP
% model is the least likely to infer a single event when applied to these data
% (Figure \ref{figPowerPsiOldDPP}).
The \modelUniform model never infers the one-divergence model in any of the
simulation replicates but still tends to infer relatively few (4--6) divergence
events when divergences are random over longer periods 
(Figure~\ref{figPowerPsiOld4} I--L \& plots M--R of Figures
S\labelcref{figPowerPsiOld,figPowerPsiUniform,figPowerPsiExp}).
Using exponential priors on divergence-time and \myTheta{} parameters does
increase the power of the \modelUshaped model compared to \modelOld across all
three series of data models, but the U-shaped prior still prevents the model
from performing as well as the \modelDPP and \modelUniform models 
(Figure~\ref{figPowerPsiOld4} \& 
S\labelcref{figPowerPsiOld,figPowerPsiUniform,figPowerPsiExp}).

The improved power of the new models is even more pronounced when looking at
estimates of the variance of divergence times (\divTimeDispersion) across the
simulations
(Figure~\ref{figPowerOmegaOld4} \& 
S\labelcref{figPowerOmegaOld,figPowerOmegaUniform,figPowerOmegaExp}).
The performance among the models is so different, that the histograms of
\divTimeDispersion estimates cannot be plotted along a shared x-axis.
The \modelDPP and \modelUniform models perform similarly across all three
series of data models, inferring values of \divTimeDispersion consistent with
one divergence event ($\divTimeDispersion < 0.01)$ in almost none of the
replicates across all the simulations.
In contrast, the \modelOld model infers values consistent with a single
divergence event in most of the replicates across all the simulations.
Using exponential priors on divergence-time and \myTheta{} parameters greatly
increases the power of the \modelUshaped model to detect variation in
divergence times relative to \modelOld, but it still has less power than the
models with Dirichlet-process or uniform priors across divergence models 
(Figure~\ref{figPowerOmegaOld4} \& 
S\labelcref{figPowerOmegaOld,figPowerOmegaUniform,figPowerOmegaExp}).
An estimated value of \divTimeDispersion less than 0.01 is an arbitrary
threshold for ``simultaneous'' divergence used throughout the \msb
literature.
Albeit arbitrary, \cite{Oaks2012} did show via simulation that the true value
of \divTimeDispersion will almost always be greater than 0.01 when divergences
are random over periods of 0.1 \globalcoalunit generations or more (see
Figure~S4 of \citet{Oaks2012}).

As mentioned above, I also see increased power of the new models when
looking at the estimated posterior probability of the one-divergence model
across the power analyses 
(Figure~\ref{figPowerPsiProbOld4} \& 
S\labelcref{figPowerPsiProbOld,figPowerPsiUniform,figPowerPsiExp}).
The \modelDPP and \modelUniform models estimate low posterior probability of
$\divTimeNum = 1$ across all of the simulations.
This is in contrast to the \modelOld model, which infers high posterior probabilities of a single
divergence for most replicates across all simulations
(Figure~\ref{figPowerPsiProbOld4} \& 
S\labelcref{figPowerPsiProbOld,figPowerPsiUniform,figPowerPsiExp}).
The exponential priors on divergence-time and \myTheta{} parameters (model
\modelUshaped) result in lower estimates of the probability of one divergence
when compared to \modelOld, but higher estimates when compared to \modelUniform
and \modelDPP
(Figure~\ref{figPowerPsiProbOld4} \& 
S\labelcref{figPowerPsiProbOld,figPowerPsiUniform,figPowerPsiExp}).
The \modelDPP and \modelUniform models do frequently support the one-divergence
model according to a Bayes factor criterion of greater than 10, however still
less frequently than the \modelOld model.
This result is not surprising given the extremely small prior probability of
the one-divergence model under the \modelDPP and \modelUniform models (I.e.,
very few posterior samples of the one-divergence model will result in a
large Bayes factor under these models).
However, the small posterior probability of the one-divergence model estimated
under \modelDPP and \modelUniform should prevent an investigator from
overinterpreting the Bayes factor as strong support for clustered divergences.

Lastly, when I look at the estimated posterior probability of
\divTimeDispersion being consistent with one shared divergence
($p(\divTimeDispersion < 0.01 | \ssSpace)$), I see the same pattern of model
behavior, with \modelDPP and \modelUniform inferring low probabilities across
all simulations, \modelOld inferring high probabilities, and \modelUshaped
inferring intermediate values
(Figure~\ref{figPowerOmegaProbOld4} \& 
S\labelcref{figPowerOmegaProbOld,figPowerOmegaUniform,figPowerOmegaExp}).

\subsection{Empirical results}
As expected based on the results of \citet{Oaks2012}, when the Philippines data
are analyzed under the \empModelOld model, there is strong support for very few
divergence events shared among all 22 pairs of taxa with a maximum \emph{a
posteriori} (MAP) estimate of one-shared divergence
(Figure~\ref{figPhilippines}A).
When these data are analyzed under models allowed under the new implementation,
there is much less support for highly clustered models and much greater
uncertainty regarding the number of divergence events shared among the taxa,
especially under the DPP models (Figure~\ref{figPhilippines}B--E).
Figure \ref{figPhilippines} also shows the prior distribution across the number
of divergence events (\divTimeNum) for each of the model, as well as the
average prior probability of an unordered and ordered model of divergence
(\divTimeIndexVector) across \divTimeNum for each model.
Estimates under the new models tend to be similar to the prior, which is
expected under such a parameter-rich model when there is limited information
from the data (four summary statistics from a single locus for each pair of
taxa).

The disparity between the results of the \empModelOld model and the new models
is even more pronounced when I look at the 10 divergence models
(\divTimeIndexVector) estimated to have the highest probability under each of the
models
(Figures~S\labelcref{figDivModelsDPP,figDivModelsDPPInform,figDivModelsDPPSimple,figDivModelsUniform,figDivModelsOld}).
Again, the new models estimate more divergences, a large amount of posterior
uncertainty, and an order of magnitude smaller probability for their respective
MAP-divergence model when compared to the \empModelOld model
(Figures~S\labelcref{figDivModelsDPP,figDivModelsDPPInform,figDivModelsDPPSimple,figDivModelsUniform,figDivModelsOld}).

Figure \ref{figNegrosPanay} shows the estimated posterior probability
distribution over the number of divergence events when the data from the
nine-taxon pairs from the Islands of Negros and Panay are analyzed under the DPP
models that sample over unordered (\npModelDPP) and ordered
(\npModelDPPOrdered) models of divergence.
The results are similar under both models and, again, yield a large amount of
uncertainty about the number of divergence events that is similar to the prior
uncertainty.
Both models estimate a similar set of 10 unordered divergence models with the
highest posterior probability
(Figures~S\labelcref{figDivModelsNP,figDivModelsNPOrdered}).

The main advantage of the \npModelDPPOrdered over the \npModelDPP is that the
identity of the taxa is maintained, and thus a fully marginalized estimate of
divergence times across the taxa can be obtained
(Figure~S\ref{figMarginalTimes}).
Furthermore, the probability of co-divergence among any set of taxa can be
estimated from the posterior sample.

\section{Discussion}
My results demonstrate that using alternative priors on parameters and
divergence models improved the behavior of the \msb model.
In my implementation, model-choice estimation is more accurate and shows
greater robustness to model violations
(Figure~\ref{figValidationModelChoicePsi} \&
\ref{figValidationModelChoiceOmega}).
The original model is very sensitive to model violations and, when present,
strongly over-estimates the probability of one-divergence event shared
across all taxa
(Figure~\ref{figValidationModelChoicePsi} \&
\ref{figValidationModelChoiceOmega}).
When more appropriate priors are used for divergence-time and demographic
parameters, and either a Dirichlet-process or uniform prior applied across
divergence models, the model is less sensitive to violations, and, when they do
cause bias, the method is conservative and tends to underestimate the
probability of biogeographically interesting models of temporally clustered
divergences (Figure~\ref{figValidationModelChoicePsi} \&
\ref{figValidationModelChoiceOmega}).

The modifications also improve the method's power to detect random
variation in divergence times, reducing the tendency to estimate clustered
divergences
(Figures~\labelcref{figPowerPsiOld4,figPowerPsiProbOld4,figPowerOmegaOld4,figPowerOmegaProbOld4}).
My results are similar to those of \citet{Oaks2012} in that I find
\msb will often infer strong support for clustered divergences when
divergences are random over quite broad timescales
(Figures~\labelcref{figPowerPsiOld4,figPowerPsiProbOld4,figPowerOmegaOld4,figPowerOmegaProbOld4}).
My results expand on this by showing that this behavior is consistent
across a range of conditions underlying the data.
The new method \dppmsbayes has greater power to detect random temporal variation in
divergences, is less prone to spurious inference of clustered divergence
models, and much less likely to incorrectly infer such models with strong
support
(Figures~\labelcref{figPowerPsiOld4,figPowerPsiProbOld4,figPowerOmegaOld4,figPowerOmegaProbOld4}).

By evaluating a model intermediate between the old and new implementation
(\modelUshaped), I was able to determine the relative effects of my 
modifications to the model.
Across all of the analyses, the results show that using better priors on
divergence-time and demographic parameters alone does improve the performance
of the method.
The magnitude of the bias toward inferring support for the one-divergence
model when there are model violations is reduced when the exponential
priors are used in place of the uniform priors
(Figure~\ref{figValidationModelChoicePsi} \&
\ref{figValidationModelChoiceOmega}).
Furthermore, using exponential priors improves the method's power to detect
temporally random divergences
(Figures~\labelcref{figPowerPsiOld4,figPowerPsiProbOld4,figPowerOmegaOld4,figPowerOmegaProbOld4}).
Throughout the analyses, the intermediate model (\modelUshaped) performs better
than the \msb model, but not as well as the models with alternative priors on
divergence models.
This suggests, as predicted by \citet{Oaks2012}, that the tendency of \msb to
erroneously support models of temporally clustered divergences is caused by a
combination of
(1) small marginal likelihoods of models with more \divTime{} parameters due to
uniform priors on divergence-time and demographic parameters and
(2) the U-shaped prior on divergence models giving low prior density to models
with intermediate numbers of divergence parameters.
The former essentially rules out models with many \divTime{} parameters, which
causes the latter to act like an "L-shaped" prior with a spike of
prior density on the one-divergence model.
Given the parameter richness of the model and the relatively small amount of
information in the summary statistics, it is not surprising that the
combination of these two factors can create a strong tendency to infer
clustered models of divergence.

While the modifications improve the behavior of the model, I urge caution when
using the method and interpreting its results.
The method attempts to approximate the posterior of a very parameter-rich model
using relatively little information from the data.
For example, when applied to the dataset of 22 taxon pairs from the Philippines
\citep{Oaks2012} the model has as many as 604--625 free parameters
(depending on \divTimeNum), and samples over 1002 unordered divergence models.
Even under the simplest possible model allowed under the new implementation,
the model still has 471--492 free parameters.
Furthermore, the stochastic coalescent and mutational processes being modeled
predict a large amount of variation in possible datasets even when the
parameter values are known.
The richness and stochastic nature of the model makes for a difficult inference
problem, especially when using a small number of summary statistics calculated
from the sequence alignments of each taxon pair.
The population-genetic summary statistics used by the method contain little
information about many of the free parameters in the model.
Thus, I expect the improved method will still be sensitive to priors, and the
power, while improved, may still be low.
While I see much less prior sensitivity under the new model compared to those
observed by \citet{Oaks2012}, I still see an effect when comparing the results
of the empirical data analyzed under a diffuse (\empModelDPP) and informative
(\empModelDPPInform) divergence-time prior (Figure~\ref{figPhilippines} C
versus D).
The fact that the posterior shifts toward the prior under the informative prior
suggests that the shift away from the prior toward fewer divergence events
under the diffuse prior might still be caused by small marginal likelihoods
of models with more divergence-time parameters (Figure~\ref{figPhilippines}).

Nonetheless, it is reassuring to see a large amount of posterior uncertainty
when the new implementation is applied to the empirical datasets
(Figure~\labelcref{figPhilippines,figNegrosPanay}).
Applications of the \msb model often result in strong posterior support for
estimated scenarios
\citep{Barber2010,Bell2012,Carnaval2009,Chan2011,Hickerson2006,Leache2007,Plouviez2009,Stone2012,Voje2009,Oaks2012},
as I found here (Figure~\ref{figPhilippines}).
Given the richness of the model, the variance of the processes being modeled,
and the relatively small amount of information in the summary statistics
calculated from the sequence data, I do not expect to find strong posterior
support for any scenario.
Based on results of the empirical and power analyses
(Figures~\labelcref{figPowerPsiProbOld4,figPowerOmegaProbOld4,figPhilippines,figNegrosPanay}),
the new implementation more accurately reflects posterior uncertainty and
avoids spurious support for biogeographic scenarios.

I also urge caution when using \dppmsbayes due to the fact that ABC methods
in general are known to be biased for model choice \citep{Robert2011}.
As recommended by \citet{Oaks2012}, any results from the method should be
accompanied by assessments of prior sensitivity and power.
I encourage investigators to view this method as a means of exploring their
data rather than a rigorous means of evaluating hypotheses.

Given the difficulty of this estimation problem, I anticipate that
full-likelihood methods that can leverage all of the information present in the
sequence data will become increasingly important for robustly estimating shared
evolutionary history across taxa \citep{JeetDiss}.
With improving numerical methods for sampling over models of differing
dimensionality \citep{Green1995,Lemey2009}, advances in Monte Carlo techniques
\citep{Jordan2012}, and increasing efficiency of likelihood calculations
\citep{Ayres2012}, analyzing rich comparative phylogeograpical models in a
full-likelihood Bayesian framework is becoming computationally practical,
especially when considering that simulating millions of random datasets from
the prior under the simple ABC rejection approach is inefficient and
computationally nontrivial.

