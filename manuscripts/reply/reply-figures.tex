\mFigure{../../response-redux/hickerson-et-al-posterior/hickerson-posterior-1k/mean_by_dispersion.pdf}{
    The joint posterior of the mean (\meant{}{}) and dispersion index ($\vmratio{} = 
    \vart{}{}/\meant{}{}$) of divergence times for 22 vertebrate taxon pairs as
    estimated by \citet{Hickerson2013} (see Figure 2B of \citet{Hickerson2013}).
    The posterior samples are color-coded to indicate the erroneous mixture of
    timescales in the analysis of \citet{Hickerson2013};
    grey = $0.05/\mutationRate$ generations and
    black = $0.02/\mutationRate$ generations.
}{figJointPosteriorHickerson}

% \mFigure{../images/coin_flip.pdf}{
%     A plot of three beta probability density functions that represent a prior
%     (black; $beta(0.5, 0.5)$), true posterior (blue; $beta(4.5, 1.5)$), and
%     empirical Bayes density (red; $beta(8.5, 2.5)$) for a dataset of five
%     Bernoulli trials, four of which are successes.
% }{figCoinFlip}

\mFigure{../../model-choice/priors-for-pc-plot/pc-plots.pdf}{
    The graphical checks recommended by \citet{Hickerson2013} for three prior
    models: (A) $M_1$ ($\divt{} \sim U(0, 0.1)$), (B) $M_{1A}$ ($\divt{} \sim
    U(0, 0.01)$), and (C) $M_{1B}$ ($\divt{} \sim U(0, 0.001)$).
    The plots project the summary statistics from 1000 random samples from each
    model onto the first two orthogonal axes of a principle component analysis,
    with the blue dot representing the observed summary statistics from the 22
    population pairs of Philippine vertebrates.
}{figPCA}

%% Exclusion simulation results
\mFigure{../../model-choice/results/m1-1-sim/pymsbayes-results/num_tau_excluded.pdf}{
    The propensity of the model-averaging approach of \citet{Hickerson2013} to
    exclude the truth.
    The plots illustrate the number of true \divt{} parameters excluded from analyses of
    simulated datasets where \divt{} for 22 pairs of populations is drawn from an
    exponential distribution, $\divt{} \sim Exp(2)$.
    The plots represent (A) unadjusted and (B) GLM-adjusted estimates from 1000
    simulation replicates analyzed using $5\e6$ samples from the prior.
    The proportion of simulation replicates in which at least one true
    parameter value is excluded from the model preferred by a Bayes factor
    ($p(\divt{} \notin \hat{M})$) is also given.
}{figExclusionSimTau}

\mFigure{../../model-choice/results/m1-1-sim/pymsbayes-results/prob_of_exclusion.pdf}{
    The propensity of the model-averaging approach of \citet{Hickerson2013} to
    exclude the truth.
    The plots illustrate the estimated probability of excluding at least one
    true \divt{} value from analyses of simulated datasets where \divt{} for 22
    pairs of populations is drawn from an exponential distribution, $\divt{}
    \sim Exp(2)$.
    The plots represent (A) unadjusted and (B) GLM-adjusted estimates from 1000
    simulation replicates analyzed using $5\e6$ samples from the prior.
    The proportion of simulation replicates in which there is strong support
    for at least one true parameter value being excluded from the model
    ($p(BF_{\divt{} \notin M, \divt{} \in M} > 10)$) is also given.
}{figExclusionSimProb}

%% Power results
\mFigure{../../model-choice/results/power-1/pymsbayes-results/plots/power_accuracy_omega_median.pdf}{
    The accuracy of the model-averaging approach of \citet{Hickerson2013} to
    estimate the dispersion index of divergence times (\vmratio{}) from
    analyses of simulated datasets where \divt{} for 22 pairs of populations is
    drawn from a series of uniform distributions, $\divt{} \sim U(0,
    \divt{max})$.
    The proportion of estimates less than the true value of,
    $p(\hat{\vmratio{}} < \vmratio{})$, is given for each \divt{max}.
    Each plot represents unadjusted median estimates from 500 simulation
    replicates analyzed using $5\e6$ samples from the prior.
}{figPowerAccOmegaMedian}

\mFigure{../../model-choice/results/power-1/pymsbayes-results/plots/power_accuracy_omega_mode_glm.pdf}{
    The accuracy of the model-averaging approach of \citet{Hickerson2013} to
    estimate the dispersion index of divergence times (\vmratio{}) from
    analyses of simulated datasets where \divt{} for 22 pairs of populations is
    drawn from a series of uniform distributions, $\divt{} \sim U(0,
    \divt{max})$.
    The proportion of estimates less than the true value of,
    $p(\hat{\vmratio{}} < \vmratio{})$, is given for each \divt{max}.
    Each plot represents GLM-regression-adjusted mode estimates from 500
    simulation replicates analyzed using $5\e6$ samples from the prior.
}{figPowerAccOmegaModeGLM}

\mFigure{../../model-choice/results/power-1/pymsbayes-results/plots/power_psi_mode.pdf}{
    The bias of the model-averaging approach of \citet{Hickerson2013} to infer
    clustered divergences in our simulation-based power analyses.
    The plots illustrate the estimated number of divergence events ($\hat{\numt{}}$)
    from analyses of simulated datasets where \divt{} for 22 pairs of populations is drawn from a series of
    uniform distributions, $\divt{} \sim U(0, \divt{max})$.
    The estimated probability of the method inferring one divergence event, $p(\hat{\numt{}} = 1)$,
    is given for each \divt{max}.
    Each plot represents 500 simulation replicates analyzed using $5\e6$
    samples from the prior.
}{figPowerPsiMode}

\mFigure{../../model-choice/results/power-1/pymsbayes-results/plots/power_omega_prob.pdf}{
    The bias of the model-averaging approach of \citet{Hickerson2013} to
    support one divergence event in our simulation-based power analyses.
    The plots illustrate histograms of the estimated posterior probability that
    the dispersion index of divergence times is less than 0.01 ($p(\vmratio{} <
    0.01 | \ssSpace)$) from analyses of simulated datasets where \divt{} for 22
    pairs of populations is drawn from a series of uniform distributions,
    $\divt{} \sim U(0, \divt{max})$.
    The proportion of simulation replicates that strongly support one
    divergence event, $p(BF_{\vmratio{} < 0.01, \vmratio{} \geq 0.01} > 10)$,
    is given for each \divt{max}.
    Each plot represents 500 simulation replicates analyzed using $5\e6$
    samples from the prior.
}{figPowerOmegaProb}

\mFigure{../../model-choice/results/power-1/pymsbayes-results/plots/power_num_excluded.pdf}{
    The propensity of the model-averaging approach of \citet{Hickerson2013} to exclude
    the truth in our simulation-based power analyses.
    The plots illustrate the number of true \divt{} parameters excluded from analyses of
    simulated datasets where \divt{} for 22 pairs of populations is drawn from a series of
    uniform distributions, $\divt{} \sim U(0, \divt{max})$. The proportion
    of simulation replicates in which at least one true parameter value is excluded from
    the model preferred by a Bayes factor ($p(\divt{} \notin \hat{M})$) is
    given for each \divt{max}.
    Each plot represents 500 simulation replicates analyzed using $5\e6$
    samples from the prior.
}{figPowerNumExcluded}

\mFigure{../../model-choice/results/power-1/pymsbayes-results/plots/power_prob_exclusion.pdf}{
    The propensity of the model-averaging approach of \citet{Hickerson2013} to
    exclude the truth in our simulatoin-based power analyses.  The plots
    illustrate the estimated posterior probability of excluding at least one
    true \divt{} value from analyses of simulated datasets where \divt{} for 22
    pairs of populations is drawn from a series of uniform distributions,
    $\divt{} \sim U(0, \divt{max})$. The proportion of simulation replicates in
    which there is strong support for at least one true parameter value being
    excluded from the model ($p(BF_{\divt{} \notin M, \divt{} \in M} > 10)$) is
    given for each \divt{max}.
    Each plot represents 500 simulation replicates analyzed using $5\e6$
    samples from the prior.
}{figPowerProbExclusion}

%% Validation results
\mFigure{../../model-choice/results/validation/results/pymsbayes-results/plots/mc_behavior.pdf}{
    An assessment of the approximate Bayesian model-averageing approach of
    \citet{Hickerson2013} under the ideal conditions when the prior model
    is correct (i.e., the pseudo-replicate datasets are simulated from
    parameters drawn from the same prior distributions used in the
    analysis).
    The plots show the relationship between the estimated posterior and true
    probability of (A \& C) $\numt{}=1$ and (B \& D) $\vmratio{} < 0.01$, based
    on 50,000 simulations.
    The results summarize the (A \& B) unadjusted and (C \& D) GLM-adjusted
    posterior estimate from each simulation replicate.
    The prior settings for all replicates included five prior models with
    $\meanDescendantTheta{} \sim U(0.0001, 0.1)$ and $\ancestralTheta{} \sim
    U(0.0001, 0.05)$ for all five models, and
    $M_1: \divt{} \sim U(0, 0.1)$,
    $M_2: \divt{} \sim U(0, 1)$,
    $M_3: \divt{} \sim U(0, 5)$,
    $M_4: \divt{} \sim U(0, 10)$, and
    $M_5: \divt{} \sim U(0, 20)$.
    The number of samples from the prior was $2.5\e6$.
    The simulated data structure was 8 population pairs, with a single 1000
    bp locus sampled from 10 individuals from each population.  The 50,000
    estimates of the posterior probability of one divergence event were
    assigned to 20 bins of width 0.05.
    The estimated posterior probability of each bin is plotted against the
    proportion of replicates in that bin with a true value consistent with
    one divergence event (i.e., $\numt{}=1$ or $\vmratio{} < 0.01)$.
}{figValidationMCBehavior}

\mFigure{../../response-redux/results/sampling-error/pymsbayes-results/omega_over_sampling.pdf}{
    Traces of the estimated lower and upper limits of the 95\% highest posterior density (HPD)
    interval of \vmratio{} (the dispersion index of divergence times) as 100 million prior
    samples are accumulated. Each pair of points is based on 1000 posterior samples retained
    from the prior. Both (A) unadjusted and (b) GLM-regression-adjusted estimates are shown.
    Prior settings were $\divt{} \sim U(0,10)$, $\meanDescendantTheta{} \sim U(0.0005, 0.04)$,
    and $\ancestralTheta{} \sim U(0.0005, 0.02)$.
}{figSamplingError}

%% Model-choice tau prior
\mFigure{../../model-choice/tau-prior/tau_prior.pdf}{
    The prior distribution on divergence times imposed by the model-averaging prior
    comprised of five models with different uniform priors on \divt{}:
    $M_1$ ($\divt{} \sim U(0, 0.1)$), $M_2$ ($\divt{} \sim U(0, 1)$), $M_3$
    ($\divt{} \sim U(0, 5)$), $M_4$ ($\divt{} \sim U(0, 10)$), $M_5$ ($\divt{}
    \sim U(0, 20)$).
}{figMCTauPrior}

%:FIGURE-saturation plot
\mFigure{../../saturation/saturation-plot.pdf}{
    The summary statistics $\pi$ \citep{Tajima1983} and $\pi_{net}$
    \citep{Takahata1985} as a function of divergence time between populations.
    Each plot represents 1100 pairs of parameter draws and summary statistics
    calculated from the simulated data.
    Prior settings for the simulations were $\divt{} \sim U(0, 20)$,
    $\meanDescendantTheta{} \sim U(0.0005, 0.04)$, and $\ancestralTheta{} \sim
    U(0.0005, 0.02)$.
}{figSaturationPlot}

% \mFigure{../../response-redux/results/hickerson/pymsbayes-results/pymsbayes-output/d1/m12345678-combined/mean_by_dispersion.pdf}{
%     Results of reanalysis.
% }{figJointPosterior}

