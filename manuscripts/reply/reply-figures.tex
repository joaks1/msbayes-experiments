\mFigure{../images/coin_flip.pdf}{
    A plot of three beta probability density functions that represent a prior
    (black; $beta(10, 10)$), posterior (blue; $beta(13, 17)$), and empirical
    Bayes density (red; $beta(16, 24)$) for a dataset of 10 coin flips, three
    of which are successes.
}{figCoinFlip}

%% Exclusion simulation results
\mFigure{../../model-choice/results/m1-1-sim/pymsbayes-results/num_tau_excluded.pdf}{
    Histograms of the number of true divergence times excluded from the model
    preferred by the empirically informed model-averaging approach of
    \citet{Hickerson2013} when applied to simulated datasets in which divergence
    times of 22 pairs of populations are drawn from an exponential
    distribution, $\divt{} \sim Exp(2)$.
    The plots represent (A) unadjusted and (B) GLM-adjusted estimates from 1000
    simulation replicates analyzed using $5\e6$ samples from the prior.
    The proportion of simulation replicates in which at least one true
    parameter value is excluded from the preferred model ($p(\divt{} \notin
    \hat{M})$) is also given.
}{figExclusionSimTau}

\mFigure{../../model-choice/results/m1-1-sim/pymsbayes-results/prob_of_exclusion.pdf}{
    Histograms of the support (estimated posterior probabilities) for excluding
    at least one true divergence time when the empirically informed
    model-averaging approach of \citet{Hickerson2013} is applied to simulated
    datasets in which divergence times of 22 pairs of populations are drawn
    from an exponential distribution, $\divt{} \sim Exp(2)$.
    The plots represent (A) unadjusted and (B) GLM-adjusted estimates from 1000
    simulation replicates analyzed using $5\e6$ samples from the prior.
    The proportion of simulation replicates in which there is strong support
    for at least one true parameter value being excluded from the model
    ($p(BF_{\divt{} \notin M, \divt{} \in M} > 10)$) is also given.
}{figExclusionSimProb}

%% Power results

\mFigure{../../model-choice/results/power-1/pymsbayes-results/plots/power-4.pdf}{
    The tendency of the empirically informed model-averaging approach of
    \citet{Hickerson2013} to (A--D) infer clustered divergences and (E--H)
    support the extreme model of one divergence when applied to simulated
    datasets in which the divergence times of 22 pairs of populations are
    randomly drawn from the uniform distributions $\divt{} \sim U(0,
    \divt{max})$ indicated at the top of each column of plots (divergence-time
    distributions are given in units of millions of generations ago (MGA)
    assuming a per-site rate of 1\e{-8} mutations per generation).
    Four of the six \divt{max} we simulated are provided; please see
    Figure~S\ref{figPower6} for a summary of all of the results.
}{figPower4}

\mFigure{../../model-choice/results/power-1/pymsbayes-results/plots/power-exclusion-4.pdf}{
    Histograms of the (A--D) number of true divergence-time parameters excluded
    from the preferred model and the (E--H) posterior probability of excluding
    at least one divergence-time parameter when the empirically informed
    model-averaging approach of \citet{Hickerson2013} is applied to simlated
    datasets in which divergence times of 22 pairs of populations are randomly
    drawn from the uniform distributions $\divt{} \sim U(0, \divt{max})$
    indicated at the top of each column of plots (divergence-time distributions
    are given in units of millions of generations ago (MGA) assuming a per-site
    rate of 1\e{-8} mutations per generation).
    Four of the six \divt{max} we simulated are provided; please see
    Figure~S\ref{figPowerExclusion6} for a summary of all of the results.
}{figPowerExclusion4}


%% Sampling error
\widthFigure{0.9}{../../response-redux/results/sampling-error/pymsbayes-results/omega_over_sampling.pdf}{listformat=figList}{
    Traces of the estimated lower and upper limits of the 95\% highest posterior density (HPD)
    interval of \vmratio{} (the dispersion index of divergence times) as 100 million prior
    samples are accumulated. Each pair of points is based on 1000 posterior samples retained
    from the prior. Both (A) unadjusted and (B) GLM-regression-adjusted estimates are shown.
    The data analyzed were the 22 pairs of Philippine taxa from \citet{Oaks2012}.
    Prior settings were $\divt{} \sim U(0,10)$, $\meanDescendantTheta{} \sim U(0.0005, 0.04)$,
    and $\ancestralTheta{} \sim U(0.0005, 0.02)$.
}{figSamplingError}
\clearpage

%% Model-choice tau prior
\mFigure{../../model-choice/tau-prior/tau_prior.pdf}{
    The prior distribution on divergence times imposed by the model-averaging prior
    comprised of five models with different uniform priors on \divt{}:
    $M_1$ ($\divt{} \sim U(0, 0.1)$), $M_2$ ($\divt{} \sim U(0, 1)$), $M_3$
    ($\divt{} \sim U(0, 5)$), $M_4$ ($\divt{} \sim U(0, 10)$), $M_5$ ($\divt{}
    \sim U(0, 20)$).
}{figMCTauPrior}

