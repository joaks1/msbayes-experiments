%&<latex>
\documentclass[letterpaper,12pt]{article}

%%%%%%%%%%%%%%%%%%%%%%%%%%%%%%%%%%%%%%%%%%%%%%%%%%%%%%%%%%%%
%% preamble %%%%%%%%%%%%%%%%%%%%%%%%%%%%%%%%%%%%%%%%%%%%%%%%
\pdfpagewidth = 8.5in
\pdfpageheight = 11.0in
\usepackage[left=1in,right=1in,top=1in,bottom=1in]{geometry}

\pagestyle{plain}
\pagenumbering{arabic}
\usepackage{setspace}
\usepackage[usenames]{color}
\usepackage[fleqn]{amsmath}
\usepackage{amssymb}
\usepackage{graphicx}
\usepackage{url}
\usepackage{verbatim}
\usepackage{appendix}
\usepackage{indentfirst}
\usepackage{booktabs}
\usepackage{multirow}
\usepackage[table, x11names]{xcolor}
\usepackage{ragged2e}
\usepackage{upgreek}
\usepackage{lscape}
\usepackage{longtable}
\usepackage[flushleft, referable]{threeparttablex}
\usepackage{rotating}
\usepackage[T1]{fontenc}
\usepackage[titles]{tocloft}
\usepackage{xspace}
\usepackage{ifthen}
\usepackage{cancel}
\usepackage{rotating}
\usepackage{array}
\usepackage{tabulary}
\usepackage{authblk}

\usepackage{hyperref}
\hypersetup{pdfborder={0 0 0}, colorlinks=true, urlcolor=black, linkcolor=black, citecolor=black}
\usepackage[capitalize]{cleveref}
\newcommand{\crefrangeconjunction}{--}

\usepackage[right, mathlines]{lineno}
\setlength\linenumbersep{1cm}
\def\linenumberfont{\normalfont\scriptsize\sffamily}

% \usepackage[format=plain, labelsep=period, justification=raggedright, singlelinecheck=true, skip=2pt, font=sf]{caption}
\usepackage{caption}
\DeclareCaptionLabelFormat{noSpace}{{#1}{#2}}
\DeclareCaptionListFormat{figList}{Figure {#2}.}
\DeclareCaptionListFormat{sFigList}{Figure S{#2}.}
\usepackage{subfig}

%\DeclareMathSizes{12}{12}{7}{5}

% \usepackage[round]{natbib}

% \makeatletter
%   \renewcommand{\section}{\@startsection{section}{1}{0mm}%
%     {-12pt}%
%     {12pt}%
%     {\sffamily\LARGE\itshape}}
% \makeatother

% \makeatletter
%   \renewcommand{\subsection}{\@startsection{subsection}{1}{0mm}%
%   {-10pt}%
%   {4pt}%
%   {\sffamily\large\bfseries\MakeUppercase}}
% \makeatother

% \makeatletter
%   \renewcommand{\subsubsection}{\@startsection{subsubsection}{1}{0mm}%
%   {-10pt}%
%   {10pt}%
%   {\sffamily\large\itshape}}
% \makeatother

% make list of figures ragged right
% \makeatletter
%   \renewcommand{\@tocrmarg}{0cm plus1fil}
% \makeatother

\setlength\linenumbersep{1cm}

% \newcommand{\change}[1]{{\color{blue} #1}\xspace}
\newcommand{\change}[1]{{\color{black} #1}\xspace}


\newcommand{\citationNeeded}{\textcolor{magenta}{\textbf{[CITATION NEEDED!]}}\xspace}
\newcommand{\tableNeeded}{\textcolor{magenta}{\textbf{[TABLE NEEDED!]}}\xspace}
\newcommand{\figureNeeded}{\textcolor{magenta}{\textbf{[FIGURE NEEDED!]}}\xspace}
\newcommand{\highLight}[1]{\textcolor{magenta}{\MakeUppercase{#1}}}

\newcommand{\editorialNote}[1]{\textcolor{red}{[\textit{#1}]}}
\newcommand{\ignore}[1]{}
\newcommand{\addTail}[1]{\textit{#1}.---}
\newcommand{\super}[1]{\ensuremath{^{\textrm{#1}}}}
\newcommand{\sub}[1]{\ensuremath{_{\textrm{#1}}}}
\newcommand{\dC}{\ensuremath{^\circ{\textrm{C}}}}

\providecommand{\e}[1]{\ensuremath{\times 10^{#1}}}

\newcommand{\mthnote}[2]{{\color{red} #2}\xspace}
\newcommand{\cwlnote}[2]{{\color{orange} #2}\xspace}

\newcommand{\ifTwoArgs}[3]{\ifthenelse{\equal{#1}{}\or\equal{#2}{}}{}{#3}\xspace}
\newcommand{\ifArg}[2]{\ifthenelse{\equal{#1}{}}{}{#2}\xspace}

%% New notation for divergence times
\newcommand{\divTime}[1]{\ensuremath{\tau_{#1}}\xspace}
\newcommand{\divTimeVector}{\ensuremath{\boldsymbol{\divTime{}}}\xspace}
\newcommand{\divTimeIndex}[1]{\ensuremath{t_{#1}}\xspace}
\newcommand{\divTimeIndexVector}{\ensuremath{\mathbf{\divTimeIndex{}}}\xspace}
\newcommand{\divTimeMap}[1]{\ensuremath{T_{#1}}\xspace}
\newcommand{\divTimeMapVector}{\ensuremath{\mathbf{\divTimeMap{}}}\xspace}
\newcommand{\divTimeScaled}[2]{\ensuremath{\mathcal{T}_{#1\protect\ifTwoArgs{#1}{#2}{,}#2}}\xspace}
\newcommand{\divTimeScaledVector}{\ensuremath{\mathbf{\divTimeScaled{}{}}}\xspace}
\newcommand{\divTimeMean}{\ensuremath{\bar{\divTimeMap{}}}\xspace}
\newcommand{\divTimeVar}{\ensuremath{s^{2}_{\divTimeMap{}}}\xspace}
\newcommand{\divTimeDispersion}{\ensuremath{D_{\divTimeMap{}}}\xspace}
\newcommand{\divTimeNum}{\ensuremath{\lvert \divTimeVector \rvert}\xspace}
\newcommand{\demographicParams}[1]{\ensuremath{\Theta_{#1}}\xspace}
\newcommand{\demographicParamVector}{\ensuremath{\mathbf{\demographicParams{}}}\xspace}
\newcommand{\popSampleSize}[2]{\ensuremath{n_{#1\protect\ifTwoArgs{#1}{#2}{,}#2}}}
\newcommand{\gammaShape}[1]{\ensuremath{a_{#1}}\xspace}
\newcommand{\gammaScale}[1]{\ensuremath{b_{#1}}\xspace}
\newcommand{\betaA}[1]{\ensuremath{a_{#1}}\xspace}
\newcommand{\betaB}[1]{\ensuremath{b_{#1}}\xspace}
\newcommand{\integerPartitionSet}[1]{\ensuremath{a({#1})}\xspace}
\newcommand{\integerPartitionNum}[1]{\ensuremath{\lvert \integerPartitionSet{#1} \rvert}\xspace}
\newcommand{\concentrationParam}{\ensuremath{\chi}\xspace}
\newcommand{\stirlingFirst}[2]{\ensuremath{c(#1, #2)}\xspace}
\newcommand{\descendantThetaMean}[1]{\ensuremath{\bar{\theta}_{D\protect\ifArg{#1}{,}#1}}\xspace}
\newcommand{\numPriorSamples}{\ensuremath{\mathbf{n}}\xspace}
\newcommand{\paramSampleVector}[1]{\ensuremath{\Lambda_{#1}}\xspace}
\newcommand{\paramSampleMatrix}{\ensuremath{\boldsymbol{\paramSampleVector{}}}\xspace}
\newcommand{\modelDPP}{\ensuremath{M_{DPP}}\xspace}
\newcommand{\modelDPPOrdered}{\ensuremath{M^{\circ}_{DPP}}\xspace}
\newcommand{\modelUniform}{\ensuremath{M_{Uniform}}\xspace}
\newcommand{\modelUshaped}{\ensuremath{M_{Ushaped}}\xspace}
\newcommand{\modelOld}{\ensuremath{M_{msBayes}}\xspace}
\newcommand{\priorDPP}[1]{\ensuremath{DP(\concentrationParam #1)}\xspace}
\newcommand{\priorUniform}{\ensuremath{DU\{\integerPartitionSet{\npairs{}}\}}\xspace}
\newcommand{\priorOld}{\ensuremath{DU\{1, \ldots, \npairs{}\}}\xspace}
\newcommand{\powerSeriesOld}{\ensuremath{\mathcal{M}_{msBayes}}\xspace}
\newcommand{\powerSeriesUniform}{\ensuremath{\mathcal{M}_{Uniform}}\xspace}
\newcommand{\powerSeriesExp}{\ensuremath{\mathcal{M}_{Exp}}\xspace}

\newcommand{\allDatasets}{\ensuremath{\mathcal{\alignment{}{}}}\xspace}
\newcommand{\allParameterValues}{\ensuremath{\boldsymbol{\Theta}}\xspace}
\newcommand{\bayesfactor}[2]{\ensuremath{BF_{#1\protect\ifArg{#2}{,}#2}}}
\newcommand{\given}{\ensuremath{\,|\,}\xspace}
\newcommand{\msb}{\upshape\texttt{\MakeLowercase{ms\MakeUppercase{B}ayes}}\xspace}
\newcommand{\abctoolbox}{\upshape\texttt{ABCtoolbox}\xspace}
\newcommand{\dppmsbayes}{\upshape\texttt{dpp-msbayes}\xspace}
\newcommand{\pymsbayes}{\upshape\texttt{PyMsBayes}\xspace}
\newcommand{\hky}{HKY85\xspace}
\newcommand{\uniformMin}[1]{\ensuremath{a_{#1}}\xspace}
\newcommand{\uniformMax}[1]{\ensuremath{b_{#1}}\xspace}
\newcommand{\locusRateHetShapeParameter}{\ensuremath{\alpha}\xspace}
\newcommand{\ancestralThetaVector}{\ensuremath{\boldsymbol{\theta_{A}}}\xspace}
\newcommand{\descendantThetaVector}[1]{\ensuremath{\boldsymbol{\theta_{D#1}}}\xspace}
\newcommand{\divtscaledvector}{\ensuremath{\mathbf{{\divtscaled{}{}}}}\xspace}
\newcommand{\divtvector}{\ensuremath{\boldsymbol{\divt{}}}\xspace}
\newcommand{\divtuniquevector}{\ensuremath{\mathbf{\divtunique{}}}\xspace}
\newcommand{\bottleTimeVector}{\ensuremath{\boldsymbol{\bottleTime{}}}\xspace}
\newcommand{\bottleTime}[1]{\ensuremath{\divt{B\ifArg{#1}{,}#1}}\xspace}
\newcommand{\bottleScalarVector}[1]{\ensuremath{\boldsymbol{\bottleScalar{#1}{}}}\xspace}
\newcommand{\bottleScalar}[2]{\ensuremath{\zeta_{D#1\protect\ifArg{#2}{,}#2}}\xspace}
\newcommand{\migrationRateVector}{\ensuremath{\mathbf{\migrationRate{}}}\xspace}
\newcommand{\geneTreeVector}{\ensuremath{\mathbf{\geneTree{}{}}}\xspace}
\newcommand{\alignmentVector}{\ensuremath{\mathbf{\alignment{}{}}}\xspace}
\newcommand{\alignment}[2]{\ensuremath{X_{#1\protect\ifTwoArgs{#1}{#2}{,}#2}}\xspace}
\newcommand{\geneTree}[2]{\ensuremath{G_{#1\protect\ifTwoArgs{#1}{#2}{,}#2}}\xspace}
\newcommand{\migrationRate}[1]{\ensuremath{m_{#1}}\xspace}
\newcommand{\recombinationRate}{\ensuremath{r}\xspace}
\newcommand{\ploidyScalar}[2]{\ensuremath{\rho_{#1\protect\ifTwoArgs{#1}{#2}{,}#2}}\xspace}
\newcommand{\ploidyScalarVector}{\ensuremath{\boldsymbol{\ploidyScalar{}{}}}\xspace}
\newcommand{\descendantRelativeThetaVector}[1]{\ensuremath{\boldsymbol{\eta_{D#1}}}\xspace}
\newcommand{\descendantRelativeTheta}[2]{\ensuremath{\eta_{D#1\protect\ifArg{#2}{,}#2}}\xspace}
\newcommand{\mutationRateScalarConstant}[2]{\ensuremath{\nu_{#1\protect\ifTwoArgs{#1}{#2}{,}#2}}\xspace}
\newcommand{\mutationRateScalarConstantVector}{\ensuremath{\boldsymbol{\mutationRateScalarConstant{}{}}}\xspace}
\newcommand{\locusMutationRateScalar}[1]{\ensuremath{\upsilon_{#1}}\xspace}
\newcommand{\locusMutationRateScalarVector}{\ensuremath{\boldsymbol{\upsilon}}\xspace}
\newcommand{\hkyModel}[2]{\ensuremath{\phi_{#1\protect\ifTwoArgs{#1}{#2}{,}#2}}\xspace}
\newcommand{\hkyModelVector}{\ensuremath{\boldsymbol{\hkyModel{}{}}}\xspace}
\newcommand{\mutationRate}{\ensuremath{\mu}\xspace}
\newcommand{\iid}{\textit{iid}\xspace}
\newcommand{\model}[1]{\ensuremath{\Theta}\xspace}
\newcommand{\npairs}[1]{\ensuremath{Y_{#1}}}
\newcommand{\nloci}[1]{\ensuremath{k_{#1}}\xspace}
\newcommand{\nlociTotal}{\ensuremath{K}\xspace}
\newcommand{\myTheta}[1]{\ensuremath{\theta_{#1}}}
\newcommand{\ancestralTheta}[1]{\ensuremath{\theta_{A\protect\ifArg{#1}{,}#1}}\xspace}
\newcommand{\descendantTheta}[2]{\ensuremath{\theta_{D#1\protect\ifArg{#2}{,}#2}}\xspace}
\newcommand{\meanDescendantTheta}[1]{\ensuremath{\descendantTheta{}{#1}}\xspace}
\newcommand{\nucdiv}[1]{\ensuremath{\pi_{#1}}}

\newcommand{\ssVector}[1]{\ensuremath{\mathbf{\alignmentSS{#1}{}}}\xspace}
\newcommand{\ssVectorObs}{\ensuremath{\ssVector{}^*}\xspace}
\newcommand{\ssSpace}{\ensuremath{\euclideanSpace{\ssVectorObs}}\xspace}
\newcommand{\ssVectorObsPLS}{\ensuremath{\ssVectorObs_{PLS}}\xspace}
\newcommand{\alignmentSS}[2]{\ensuremath{S_{#1\protect\ifTwoArgs{#1}{#2}{,}#2}}\xspace}
\newcommand{\alignmentSSObs}[2]{\ensuremath{\alignmentSS{#1}{#2}^*}\xspace}
\newcommand{\tol}{\ensuremath{\epsilon}\xspace}
\newcommand{\euclideanSpace}[1]{\ensuremath{B_{\tol}(#1)}\xspace}
\newcommand{\hpvector}[1]{\ensuremath{\Lambda_{#1}}}
\newcommand{\divtscaled}[2]{\ensuremath{t_{#1\protect\ifTwoArgs{#1}{#2}{,}#2}}}
\newcommand{\divt}[1]{\ensuremath{\tau_{#1}}}
\newcommand{\divtunique}[1]{\ensuremath{T_{#1}}}
\newcommand{\ssMatrix}{\ensuremath{\mathbb \alignmentSS{}{}}\xspace}
\newcommand{\ssMatrixRaw}[1]{\ensuremath{{\ssMatrix}_{stats#1}}\xspace}
\newcommand{\ssMatrixPLS}[1]{\ensuremath{{\ssMatrix}_{PLS#1}}\xspace}
\newcommand{\hpmatrix}[1]{\ensuremath{\mathcal{P}_{#1}}}
\newcommand{\meant}[2]{\ensuremath{E(\divt{#1})_{#2}}}
\newcommand{\meantestimate}{\ensuremath{\hat{E(\divt{})}}\xspace}
\newcommand{\vart}[2]{\ensuremath{Var(\divt{#1}{})_{#2}}}
\newcommand{\vmratio}[1]{\ensuremath{\Omega_{#1}}}
\newcommand{\numt}[1]{\ensuremath{\Psi_{#1}}}
\newcommand{\probnumt}[2]{\ensuremath{p(\numt{#1} = {#2})}}
\newcommand{\postprobnumt}[1]{\ensuremath{p(\numt{} = {#1}|\ssSpace)}}
\newcommand{\postprobnumtnot}[1]{\ensuremath{p(\numt{} \neq {#1}|\ssSpace)}}
\newcommand{\postprobomegasimult}{\ensuremath{p(\vmratio{} < 0.01 | \ssSpace)}\xspace}
\newcommand{\modelprior}[1]{\ensuremath{f(\model{})}}
\newcommand{\modelpost}[1]{\ensuremath{f(\model{}|\ssSpace)}}
\newcommand{\npriorsamples}{\ensuremath{n}\xspace}
\newcommand{\globalcoalunit}{\ensuremath{4\globalpopsize}\xspace}
\newcommand{\globalpopsize}{\ensuremath{N_C}\xspace}
\newcommand{\effectivePopSize}[1]{\ensuremath{N_e{#1}}\xspace}
\newcommand{\coalunit}{\ensuremath{4\effectivePopSize{}}\xspace}
\newcommand{\priorsample}[1]{\ensuremath{\hpmatrix{\modelprior{}}}}
\newcommand{\truncprior}[1]{\ensuremath{\hpmatrix{\tol}}\xspace}
\newcommand{\postsample}[1]{\ensuremath{\hpmatrix{\modelpost{}}}}
\newcommand{\abcllr}[1]{ABC\sub{LLR}}
\newcommand{\abcglm}[1]{ABC\sub{GLM}}
\newcommand{\integerPartition}[1]{\ensuremath{a({#1})}}
\newcommand{\uniqueModel}[2]{\ensuremath{M_{#1\protect\ifTwoArgs{#1}{#2}{,}#2}}}
\newcommand{\taxonLocusVector}[1]{\ensuremath{\{#1{1}{1},\ldots,#1{\npairs{}}{\nloci{\npairs{}}}\}}\xspace}
\newcommand{\taxonVector}[1]{\ensuremath{\{#1{1},\ldots,#1{\npairs{}}\}}\xspace}
\newcommand{\locusVector}[1]{\ensuremath{\{#1{1},\ldots,#1{\nlociTotal}\}}\xspace}

\newcommand{\validationAccuracyCaption}[2]{Estimation accuracy for model
    #2 when analyzing data generated under #1.
    A random sample of 5000 posterior estimates (from 50,000) are plotted,
    including both (A, B, \& C) unadjusted and (D, E, \&
    F) GLM-regression-adjusted estimates.
    Normal random variates ($N(0, 0.005)$) have been added to the estimates and
    true values of \divTimeNum (A \& D) to reduce overlap of plot symbols.
    The root mean square error (RMSE) calculated from the 5000 estimates is
    provided.}
\newcommand{\validationModelChoiceCaption}[2]{Model-choice accuracy for model
    #2 when analyzing data generated under #1.
    The estimated posterior probability of a single divergence event, based on
    (A \& C) $\divTimeNum = 1$ and (B \& D) $\divTimeDispersion < 0.01$, from
    50,000 posterior estimates are assigned to bins of width 0.05 and plotted
    against the proportion of replicates in each bin where the truth is
    $\divTimeNum = 1$ or $\divTimeDispersion < 0.01$.
    Results based on the (A \& B) unadjusted and (C \& D) GLM-adjusted
    posterior estimates are shown.}
\newcommand{\powerAccuracyCaption}[2]{Estimation accuracy for model
    #2 when analyzing data generated under the series of models #1.
    The true versus estimated value of the dispersion index of divergence
    times (\divTimeDispersion) is plotted for 1000 datasets simulated
    under each of the #1 models, and the proportion of estimates less
    than the truth, $p(\hat{\divTimeDispersion} < \divTimeDispersion)$,
    is shown for each data model.}
\newcommand{\powerPsiCaption}[2]{
    The power of model #2 to detect random variation in divergence times as
    simulated under the series of models #1.
    The plots illustrate the estimated number of divergence events
    ($\hat{\divTimeNum}$) from analyses of 1000 datasets simulated under each
    of the #1 models, with the the estimated probability of the model inferring
    one divergence event, $p(\hat{\divTimeNum} = 1)$, given for each data
    model.}
\newcommand{\powerDispersionCaption}[2]{
    The power of model #2 to detect random variation in divergence times as
    simulated under the series of models #1.
    The plots illustrate the estimated dispersion index of divergence times
    ($\hat{\divTimeDispersion}$) from analyses of 1000 datasets simulated under
    each of the #1 models, with the the estimated probability of the model
    inferring one divergence event, $p(\hat{\divTimeDispersion} < 0.01)$, given
    for each data model.}
\newcommand{\powerPsiProbCaption}[2]{
    The tendency of model #2 to support one divergence event when there is
    random variation in divergence times as simulated under the series of
    models #1.
    The plots illustrate histograms of the estimated posterior probability of
    the one divergence model, $p(\divTimeNum = 1 | \ssSpace)$, from analyses of
    1000 datasets simulated under each of the #1 models, with the the estimated
    probability of the model strongly supporting one divergence event,
    $p(BF_{\divTimeNum = 1, \divTimeNum \neq 1} > 10)$, given for each data
    model.}
\newcommand{\powerDispersionProbCaption}[2]{
    The tendency of model #2 to support one divergence event when there is
    random variation in divergence times as simulated under the series of
    models #1.
    The plots illustrate histograms of the estimated posterior probability of
    the one divergence model, $p(\divTimeDispersion < 0.01 | \ssSpace)$, from
    analyses of 1000 datasets simulated under each of the #1 models, with the
    the estimated probability of the model strongly supporting one divergence
    event, $p(BF_{\divTimeDispersion < 0.01, \divTimeDispersion \geq 0.01} >
    10)$, given for each data model.}
\newcommand{\simulationDescription}[2]{\change{Each plot represents #1
    simulation replicates using the same $#2$ samples from the prior}}
\newcommand{\simulationDistribution}{\ensuremath{\divt{} \sim U(0,
    \divt{max})}\xspace}
\newcommand{\estimateDescription}[2]{All estimates were obtained using #1 and #2}
\newcommand{\estimateDescriptionUncorrected}[1]{All estimates based on
    unadjusted posterior, \truncprior{}, obtained using #1}
\newcommand{\priorDescription}[4]{Prior settings were \priorSettings{#1}{#2}{#3}{#4}}
\newcommand{\priorSettings}[4]{$\divt{} \sim U(0, #1)$,
    $\meanDescendantTheta{} \sim U(#2, #3)$, and
    $\ancestralTheta{}{} \sim U(#2, #4)$}
\newcommand{\priorDescriptionBug}[4]{Prior settings were
    \priorSettingsBug{#1}{#2}{#3}{#4}}
\newcommand{\priorSettingsBug}[4]{$\divt{} \sim U(0, #1)$,
    $\meanDescendantTheta{} \sim U(#2, #3)$, and
    $\ancestralTheta{}{} \sim U(0.01, #4)$}
\newcommand{\simulationScheme}{simulations where \divt{} (in \globalcoalunit
    generations) for 22 population pairs is drawn from a series of uniform
    distributions, \simulationDistribution}
\newcommand{\captionPowerOmega}{Histograms of the estimated dispersion index
    of divergence times ($\hat{\vmratio{}}$) from \simulationScheme.
    The threshold for one divergence event \citep{Hickerson2006} is indicated
    by the dashed line, and the estimated probability of inferring one
    divergence event, $p(\hat{\vmratio{}}\le 0.01)$, is given for each
    \divt{max}}
\newcommand{\captionPowerPsiMode}{Histograms of the estimated number of
    divergence events ($\hat{\numt{}}$) from \simulationScheme.
    The estimated probability of inferring one divergence event,
    $p(\hat{\numt{}} = 1)$, is given for each \divt{max}}
\newcommand{\captionPowerPsi}{Histograms of the estimated posterior
    probability of one divergence event, \postprobnumt{1}, from
    \simulationScheme.
    The estimated probability of inferring one divergence event with a
    Bayes factor greater than 10 (dashed black line),
    $p(\bayesfactor{\numt{}=1}{\numt{} \ne 1} > 10)$, is given for each \divt{max}.
    The red line indicates $\postprobnumt{1} = 0.95$, and the estimated
    probability of inferring a posterior probability greater than 0.95 is given
    to the right of the line.}
\newcommand{\captionAccuracy}[1]{Accuracy and precision of #1 estimates from
    \simulationScheme.
    The proportion of estimates less than the true value ($p(\hat{#1}<#1)$) is
    given for each \divt{max}}
\newcommand{\samplingErrorTableNote}{An estimate of 1.0 for a posterior probability
    is an artifact of sampling error}


\newcommand{\refAccuracyALL}[1]{\labelcref{fig_acc_t_ss_llr_bug,fig_acc_t_ss_glm_bug,fig_acc_t_pls_llr_bug,fig_acc_t_pls_glm_bug,fig_acc_o_ss_llr_bug,fig_acc_o_ss_glm_bug,fig_acc_o_pls_llr_bug,fig_acc_o_pls_glm_bug}}
\newcommand{\refAccuracySS}[1]{\labelcref{fig_acc_t_ss_llr_bug,fig_acc_t_ss_glm_bug,fig_acc_o_ss_llr_bug,fig_acc_o_ss_glm_bug}}
\newcommand{\refAccuracySSfull}[1]{\labelcref{fig_acc_t_ssfull_llr_bug,fig_acc_t_ssfull_glm_bug,fig_acc_o_ssfull_llr_bug,fig_acc_o_ssfull_glm_bug}}
\newcommand{\refSSfull}[1]{\labelcref{fig_acc_t_ssfull_llr_bug,fig_acc_t_ssfull_glm_bug,fig_acc_o_ssfull_llr_bug,fig_acc_o_ssfull_glm_bug,fig_pow_o_ssfull_llr_bug,fig_pow_o_ssfull_glm_bug,fig_pow_psi_modes_ssfull_glm_bug}}
\newcommand{\refSS}[1]{\labelcref{fig_acc_t_ss_llr_bug,fig_acc_t_ss_glm_bug,fig_acc_o_ss_llr_bug,fig_acc_o_ss_glm_bug,fig_pow_o_ss_llr_bug,fig_pow_o_ss_glm_bug,fig_pow_psi_ss}}
\newcommand{\refAccuracyPLS}[1]{\labelcref{fig_acc_t_pls_llr_bug,fig_acc_t_pls_glm_bug,fig_acc_o_pls_llr_bug,fig_acc_o_pls_glm_bug}}
\newcommand{\refAccuracySScorrected}[1]{\labelcref{fig_acc_t_ss_llr_bug,fig_acc_t_ss_glm_bug,fig_acc_o_ss_llr_bug,fig_acc_o_ss_glm_bug}}
\newcommand{\refAccuracyPLScorrected}[1]{\labelcref{fig_acc_t_pls_llr_bug,fig_acc_t_pls_glm_bug,fig_acc_o_pls_llr_bug,fig_acc_o_pls_glm_bug}}
\newcommand{\refAccuracyUncorrected}[1]{\labelcref{fig_acc_t_ss_unc,fig_acc_t_pls_unc,fig_acc_o_ss_unc,fig_acc_o_pls_unc}}
\newcommand{\refAccuracyCorrected}[1]{\labelcref{fig_acc_t_ss_llr_bug,fig_acc_t_ss_glm_bug,fig_acc_t_pls_llr_bug,fig_acc_t_pls_glm_bug,fig_acc_o_ss_llr_bug,fig_acc_o_ss_glm_bug,fig_acc_o_pls_llr_bug,fig_acc_o_pls_glm_bug}}
\newcommand{\refAccuracyGLM}[1]{\labelcref{fig_acc_t_ss_glm_bug,fig_acc_t_pls_glm_bug,fig_acc_o_ss_glm_bug,fig_acc_o_pls_glm_bug}}
\newcommand{\refAccuracyLLR}[1]{\labelcref{fig_acc_t_ss_llr_bug,fig_acc_t_pls_llr_bug,fig_acc_o_ss_llr_bug,fig_acc_o_pls_llr_bug}}
\newcommand{\refAccuracyOmega}[1]{\labelcref{fig_acc_o_ss_llr_bug,fig_acc_o_ss_glm_bug,fig_acc_o_pls_llr_bug,fig_acc_o_pls_glm_bug}}
\newcommand{\refAccuracyOmegaUncorrected}[1]{\labelcref{fig_acc_o_ss_unc,fig_acc_o_pls_unc}}
\newcommand{\refAccuracyOmegaCorrected}[1]{\labelcref{fig_acc_o_ss_llr_bug,fig_acc_o_ss_glm_bug,fig_acc_o_pls_llr_bug,fig_acc_o_pls_glm_bug}}
\newcommand{\refAccuracyTime}[1]{\labelcref{fig_acc_t_ss_llr_bug,fig_acc_t_ss_glm_bug,fig_acc_t_pls_llr_bug,fig_acc_t_pls_glm_bug}}

\newcommand{\tn}{\tabularnewline}

\newcommand{\widthFigure}[5]{\begin{figure}[htbp]
\begin{center}
    \includegraphics[width=#1\textwidth]{#2}
    \captionsetup{#3}
    \caption{#4}
    \label{#5}
    \end{center}
    \end{figure}}

\newcommand{\heightFigure}[5]{\begin{figure}[htbp]
\begin{center}
    \includegraphics[height=#1]{#2}
    \captionsetup{#3}
    \caption{#4}
    \label{#5}
    \end{center}
    \end{figure}}

\newcommand{\mFigure}[3]{\widthFigure{1.0}{#1}{listformat=figList}{#2}{#3}\clearpage}
\newcommand{\siFigure}[3]{\widthFigure{1.0}{#1}{name=Figure S, labelformat=noSpace, listformat=sFigList}{#2}{#3}\clearpage}



\newcommand{\msTitle}{Why you should not fix a biased model-choice method by
adding an additional dimension of model choice: A reply to Hickerson et al.
\xspace}
%%%%%%%%%%%%%%%%%%%%%%%%%%%%%%%%%%%%%%%%%%%%%%%%%%%%%%%%%%%%
%%%%%%%%%%%%%%%%%%%%%%%%%%%%%%%%%%%%%%%%%%%%%%%%%%%%%%%%%%%%

\begin{document}
\doublespacing
\raggedright
\setlength{\parindent}{0.5in}
\begin{linenumbers}

\begin{titlepage}
    \begin{flushleft}
        \sffamily

        \MakeUppercase{\large\bfseries \msTitle}

        \vspace{12pt}
        \textbf{Running head:} \MakeUppercase{Approximate Bayesian model
        choice}

        \vspace{12pt}
        Jamie R.\ Oaks$^{1,2}$, Jeet Sukumaran$^{3}$, Jacob A.\
        Esselstyn$^{4}$, Charles W.\ Linkem$^{5}$, Cameron D.\
        Siler$^{6}$, Rafe M.\ Brown$^{1}$ and Mark T.\ Holder$^{1}$

        \bigskip
        $^1$\emph{Department of Ecology and Evolutionary Biology,
            Biodiversity Institute,
            University of Kansas,
            Lawrence, Kansas 66045}\\[.1in]
        $^3$\emph{Department of Biology,
            Duke University,
            Durham, North Carolina 27708} \\[.1in]
        $^4$\emph{Museum of Natural Science,
            Louisiana State University,
            119 Foster Hall,
            Baton Rouge, Louisiana 70803}\\[.1in]
        $^5$\emph{Department of Biology,
            University of Washington,
            Seattle, Washington 98195}\\[.1in]
        $^6$\emph{Sam Noble Museum,
            Department of Biology,
            University of Oklahoma,
            Norman, Oklahoma 73072}\\[.1in]
        $^2$\emph{Corresponding author} (\href{mailto:joaks1@gmail.com}{\tt
        joaks1@gmail.com})\\

    \end{flushleft}
\end{titlepage}

{\sffamily
    \noindent\textbf{ABSTRACT} \\
    \noindent Recently, \citet{Oaks2012} presented a simulation-based
    assessment of the approximate Bayesian phylogeographical model choice
    method implemented in \msb, and found the method to lack robustness and
    power to detect temporal variation in divergences that are randomly
    distributed over relatively broad periods of evolutionary history.
    They conclude that a likely mechanism behind the bias toward models with
    few divergence events is the small marginal likelihoods of models with more
    divergence time parameters due to the integration over vast parameter space
    with low likeihood and high prior density that is imposed by broad uniform
    priors.
    \citet{Hickerson2013} have responded to this work, questioning the priors
    used by \citet{Oaks2012} and presenting an approximate Bayesian
    model-averaging approach that uses narrow, empirically informed uniform
    priors.
    \citet{Hickerson2013} use this method to re-analyze the empirical dataset
    of \citet{Oaks2012} comprised of 22 pairs of vertebrate populations from
    the Philippines, and conclude the model-averaging approach circumvents the
    issues revealed by \citet{Oaks2012} and yields robust inference of
    divergence models.
    Furthermore, they argue that the biases revealed by \citet{Oaks2012} are
    inherent to the inefficient rejection algorithm implemented in \msb, rather
    than a result of low marginal likelihoods of parameter-rich models.
    Here we discuss an error in the methodology of \citet{Hickerson2013} that
    renders some of their results invalid and the remainder difficult to
    interpret.
    Furthermore, we demonstrate that the approach of \citet{Hickerson2013} is
    biased toward models with less parameter space, which manifests in a strong
    tendency to sample predominantly from models that exclude the true values
    of the model's parameters.
    Our results suggest that their model-averaged results are dominated by
    models that likely exclude at least some of the true divergence times
    across the 22 pairs of Philippine vertebrates, which adds to the
    difficulty in interpreting their results.
    We also show the approach of \citet{Hickerson2013} is still biased toward
    inferring simultaneous divergences even when the true divergences are
    randomly distributed over millions of generations.
    Our results, coupled with those of \citet{Oaks2012}, strongly suggest that
    low marginal likelihoods of complex models is a primary cause of the bias
    of \msb toward models with less parameter space.
    We discuss ways to mitigate this bias without resorting to adding an
    additional dimension of model choice to the model or using questionable
    empirically informed uniform priors.

    \vspace{12pt}
    \noindent\textbf{KEY WORDS: Approximate Bayesian computation; Bayesian
        model choice; empirical Bayes} 

}

\newpage

%CWL:
% \highLight{The goal of this paper needs to be to highlight the major
%     differences of opinion on how ABC can be used in comparative phylogeography
%     and how these changes to msbayes can be made that will improve the
%     inference in the future. We should avoid nit-picking the Hickerson paper
%     and try to stay positive.}
\section{Introduction}
Biogeographers frequently seek to explain population and species
differentiation on geographical phenomena.
Establishing that a set of population splitting events occurred
at the same time can be a potentially persuasive argument that a set of taxa
were affected by the same geographic events.
The approximate-Bayesian method, \msb, allows biogeographers to estimate the
probabilities of models in which multiple sets of taxa diverge at the same
time \citep{Hickerson2006,Huang2011}.

Recently, \citet{Oaks2012} used this model-choice framework to study 22 pairs
of vertebrate lineages distributed across the Philippines; they also studied
the behavior of the \msb approach using computer simulations.
They found the method is very sensitive to prior assumptions and often
supports shared divergences across taxa that diverged randomly over broad time
periods, to which \citet{Hickerson2013} responded.
\citet{Oaks2012} and \citet{Hickerson2013} agree on the fundamental
methodological point about the model selection performed in \msb:
\begin{itemize}
   \item The use of vague priors on the divergence-time parameters can lead to
       spurious support for models with few divergence events shared across
       taxa. Thus, the primary inference enabled by the approach is very
       sensitive to the priors on divergence times.
\end{itemize}
However, the two papers suggest alternative mechanisms by which the priors on
divergence times cause this behavior:
\begin{enumerate}[label=\textsl{\textbf{Hypothesis \arabic*)}},ref=\arabic*,align=left]
    \item \textsl{\textbf{Numerical approximation error}} \citep{Hickerson2013}:
        Under broad uniform priors, the rejection algorithm implemented in \msb
        is unable to adequately sample the space of the models within
        reasonable computational time, which leads to bias toward models with
        fewer divergence-time parameters because they are better sampled.
        \label{hypError}
    \item \textsl{\textbf{Strongly weighted marginal likelihoods}}
        \citep{Oaks2012}:
        The uniform priors on divergence times lead to very small marginal
        likelihoods (and thus posterior probabilities) of models with many
        divergence-time parameters.  The likelihood of these models is
        ``averaged'' over a much greater parameter space in which there is a
        large amount of prior weight and small probability of producing the
        data.  \citep{Jeffreys1939,Lindley1957}. \label{hypML}
\end{enumerate}
In Hypothesis~\ref{hypError}, the problem is numerical approximation error due to
insufficient computation;
given data from randomly diverged taxa, the exact posterior supports a model
with many divergence-time parameters, but we are unable to accurately
approximate this posterior.
In Hypothesis~\ref{hypML}, the problem is more fundamental; given data from
randomly diverged taxa, the exact posterior supports a model with a small
number of simultaneous divergences.
I.e., given the assumption that divergence times are broadly and uniformly
distributed, the exact posterior from Bayes' rule leads us to the wrong
conclusion about evolutionary history.
Such posterior support for simultaneous divergence, even if ``correct'' from
the perspective of Bayesian model choice, does not provide the biogeographical
insights that a researcher who employs \msb seeks to gain.

While these phenomena are not mutually exclusive, it is important to
distinguish between them in order to determine how to improve our ability to
estimate shared divergence histories.
If Hypothesis~\ref{hypError} is correct, then the model is sound and we need to
increase our computational effort or improve our Monte Carlo integration
procedures.
For example, Markov chain or sequential Monte Carlo algorithms might sample the
posterior more efficiently than the simple Monte Carlo rejection sampler
implemented in \msb.
Rather than alter the sampling algorithm, \citet{Hickerson2013} tried using
narrow, empirically informed uniform priors in the hope that with less
parameter space to sample, the rejection algorithm would produce better
estimates of the posterior.
Here, we discuss theoretical considerations for using empirically informed
priors for Bayesian model choice and evaluate the approach of
\citet{Hickerson2013} as a potential solution to the biases of \msb revealed by
\citet{Oaks2012}.
In their analyses, \citet{Hickerson2013} made an error by mixing different
units of time, which invalidates the results presented in their response (see
Supporting Information for details).
We correct this error, but still find their approach will often support
(1) clustered divergence models when divergences are random, and
(2) models that exclude from consideration the true values of the
parameters.

If Hypothesis~\ref{hypML} is correct, we need to correct the model, because no
amount of computation will help; even if we could calculate the exact
posterior, we would still reach the wrong interpretation about evolutionary
history.
Accordingly, \citet{Oaks2014dpp} has introduced a method that uses more
flexible probability distributions to accommodate prior uncertainty in
divergence times without overly inhibiting the marginal likelihoods of models
with more divergence-time parameters.
This greatly increases the method's robustness and power to detect temporal
variation in divergences \citep{Oaks2014dpp}.
This is not surprising given the rich statistical literature showing that
marginal likelihoods are very sensitive to the priors used in Bayesian model
selection
\citep[e.g.,][]{Jeffreys1939,Lindley1957}.

We also explore the distinct predictions made by Hypotheses 1 and 2.
We show the behavior of \msb matches the predictions of Hypothesis~\ref{hypML},
but not Hypothesis~\ref{hypError}.
This strongly suggests the method tends to support models of shared divergences
not because of insufficient computation, but rather due to the larger marginal
likelihoods of these models under the prior assumption of uniformly distributed
divergence times.



\section{The potential implications of empirical Bayesian model choice}
% \citet{Oaks2012} found \msb spuriously supports models of temporally
% clustered divergences even when using a prior on divergence times that was
% informed by the data being analyzed.
% The main argument of \citet{Hickerson2013} is that the priors used by
% \citet{Oaks2012} were not informative enough.
\citet{Hickerson2013} suggest a very narrow, highly informed uniform prior on
divergence times is necessary to avoid the method's preference for models with
few divergence-time parameters.
Such an empirical Bayesian approach to model selection raises some theoretical
and practical concerns, some of which were discussed by \citet{Oaks2012} (see
the last paragraph of ``Assessing prior sensitivity of \msb'' in
\citet{Oaks2012}); we expand on this here.

\subsection{Theoretical implications of empirical priors for Bayesian model
choice}
\begin{linenomath}
Bayesian inference is a method of inductive learning in which Bayes' rule is
used to update our beliefs about a model $M$ as new information becomes
available.
If we let \allParameterValues represent the set of all possible parameter
values for model $M$, we can define a prior distribution for all $\theta \in
\allParameterValues$ such that $p(\theta \given M)$ describes our belief that
any given \myTheta{} is the true value of the parameter.
If we let \allDatasets represent all possible datasets then we can 
define a sampling model for all $\theta \in
\allParameterValues$ and $\alignment{}{} \in \allDatasets$ such that
$p(\alignment{}{} | \theta, M)$ measures our belief that any dataset \alignment{}{}
will be generated by any state \myTheta{} of model $M$.
After collecting a new dataset \alignment{i}{}, we can use Bayes' rule to
calculate the posterior distribution
\begin{equation}
    p(\myTheta{} \given \alignment{i}{}, M) = \frac{p(\alignment{i}{} \given
    \myTheta{}, M)p(\myTheta{} \given M)}{p(\alignment{i}{} \given M)},
    \label{eq:bayesrule}
\end{equation}
as a measure of our beliefs after seeing the new information, where
\begin{equation}
    p(\alignment{i}{} \given M) = \int_{\myTheta{}} p(\alignment{i}{} \given
    \myTheta{}, M)p(\myTheta{} \given M) d\myTheta{}
    \label{eq:marginallikelhiood}
\end{equation}
is the marginal likelihood of the model.
\end{linenomath}

This is an elegant method of updating our beliefs as data are accumulated.
However, this all hinges on the fact that the prior ($p(\myTheta{} \given M)$)
is defined for all possible parameter values independently of the new data
being analyzed.
Any other datasets or external information can safely be used to inform our
beliefs about $p(\myTheta{} \given M)$.
However, if the same data are used to both inform the prior and calculate the
posterior, the prior becomes conditional on the data, and Bayes' rule breaks
down.

Thus, empirical Bayes methods have an uncertain theoretical basis and
do not yield a valid posterior distribution from Bayes' rule \citep[e.g.,
empirical Bayesian estimates of the posterior are often too narrow, off-center,
and incorrectly shaped;][]{Morris1983,Laird1987,Carlin1990,Efron2013}.
This is not to say that empirical Bayesian approaches are not useful.
Empirical Bayes is a well-studied branch of Bayesian statistics that has given
rise to many methods for obtaining parameter estimates that often
exhibit favorable frequentist properties.
\citep{Morris1983,Laird1987,Laird1989, Carlin1990,Hwang2009}.

\begin{linenomath}
Whereas empirical Bayesian approaches can provide powerful methods for
parameter estimation, a theoretical justification for empirical Bayesian
approaches to model choice is questionable.
In Bayesian model choice, the primary goal is not to estimate parameters, but
to estimate the probabilities of candidate models.
In a simple example where two candidate models, $M_1$ and $M_2$, are being
compared, the goal is to estimate the posterior probabilities of these two
models.
Again, we can use Bayes' rule to calculate this as
\begin{equation}
    p(M_1 \given \alignment{i}{}) = \frac{p(\alignment{i}{} \given
    M_1)p(M_1)}{p(\alignment{i}{} \given M_1)p(M_1) + p(\alignment{i}{} \given
    M_2)p(M_2)}.
    \label{eq:bayesmodelchoice}
\end{equation}
By comparing Equations \ref{eq:bayesrule} and \ref{eq:bayesmodelchoice}, we
see fundamental differences between Bayesian parameter estimation and
model choice.
\end{linenomath}

In Equation \ref{eq:bayesrule}, we see that the posterior density of any state
$\myTheta{}$ of the model, is the prior density updated by the probability of
the data given $\myTheta{}$ (the likelihood of $\myTheta{}$).
The marginal likelihood of the model only appears as a normalizing constant in
the denominator.
Thus, as long as the prior distribution contains the values of $\myTheta{}$
under which the data are probable and the data are strongly informative
relative to the prior, the values of the parameters that maximize the posterior
distribution will be relatively robust to prior choice, even if the posterior
is technically incorrect due to using the data to inform the priors.
However, if we look at Equation~\ref{eq:bayesmodelchoice}, we see that in
Bayesian model choice it is now the \emph{marginal} likelihood of a model that
updates the prior to yield the model's posterior probability.
The integral over the entire parameter space of the likelihood weighted by the
prior density is no longer a normalizing constant, rather it is how the data
inform the posterior probability of the model.
Because the prior probability distributions placed on the model's parameters
have a strong affect on the integrated, or ``average'', likelihood of a model,
Bayesian model choice tends to be much more sensitive to priors than parameter
estimation \citep{Jeffreys1939,Lindley1957}.
Another important difference of Bayesian model choice illustrated by
Equation~\ref{eq:bayesmodelchoice} is that the value of interest, the posterior
probability of a model, is not a function of \myTheta{} because the parameters
are integrated out of the marginal likelihoods of the candidate models.
Thus, unlike parameter estimates, the estimated posterior probability of a
model is a single value (rather than a distribution) lacking a measure of
posterior uncertainty.

The justification for an empirical Bayesian approach to parameter estimation is
that giving the data more weight relative to the prior (i.e., using the data
twice) will often shift the peak of the estimated distribution nearer to the
true value(s) of the model's parameter(s).
However, there is no such justification for model selection, because unlike
model parameters, the posterior probabilities of candidate models often have no
clear true values.
Model posterior probabilities are inherently measures of our belief in the
models after our prior beliefs are updated by the data being analyzed.
This complicates the meaning of model posterior probabilities when Bayes' rule
is violated by informing priors with the same data to be analyzed.
By using the data twice, we fail to account for prior uncertainty and mislead
our posterior beliefs in the models being compared; we will be over confident
in some models and under confident in others.

Nonetheless, empirical Bayesian model choice does perform well for some
problems.
Particularly, in cases where large aggregate data sets are used for many
parallel model-choice problems, pooling information to inform
priors can lead to favorable group-wise frequentist coverage across tests
\citep{Efron2008}.
However, this is far removed from the single model-choice problem of \msb.
In the Supporting Information we use a simple example to help highlight the
distinctions between Bayesian parameter estimation and model choice.

\subsection{Practical concerns about empirically informed uniform priors for
    Bayesian model choice}
In addition to the theoretical concerns discussed above, there are practical
problems with using narrow, empirically informed, uniform priors.
The results of Hickerson et al.'s (\citeyear{Hickerson2013}) reanalysis of the
Philippine dataset strongly favored models with the narrowest, empirically
informed prior on divergence times, and thus their model-averaged posterior
estimates are dominated by models $M_1$ and $M_2$ (see Table 1 of
\citet{Hickerson2013}).
This is concerning, because the narrowest \divt{} prior used by
\citet{Hickerson2013} ($\divt{} \sim U(0,0.1)$) likely excludes the true
divergence times for at least some of the Philippine taxa.
\citet{Hickerson2013} set this prior to match the 95\% highest posterior
density (HPD) interval for the mean divergence time estimated under one of the
priors used by \citet{Oaks2012} (see Tables 2 and 3 of \citet{Oaks2012}).
Given this interval estimate is for the \emph{mean} divergence time across all
22 taxa, it may be inappropriate to set this as the limit on the prior, because
some of the taxon pairs are expected to have diverged at times older than the
upper limit.
Furthermore, this prior is \emph{excluded} from the 95\% HPD interval estimates
of the mean divergence time under the other two priors explored by
\citet{Oaks2012} (under these priors the 95\% HPD is approximately 0.3--0.6;
see Table~6 of \citet{Oaks2012}).

The strong preference for the narrowest prior on divergence times suggests the
approach of \citet{Hickerson2013} is biased toward models with less parameter
space and, as a consequence, will estimate model-averaged posteriors dominated
by models that exclude true values of the parameters.
We explored this possibility in two ways.
First, we re-analyzed the Philippines dataset using the model-averaging
approach of \citet{Hickerson2013}, but set one of the prior models with a
uniform prior on divergence times that is unrealistically narrow and almost
certainly excludes most, if not all, of the true divergence times of the 22
taxon pairs.
If small likelihoods of large models cause the method to prefer models with
less parameter space (Hypothesis~\ref{hypML}), we expect \msb will
preferentially sample from this erroneous model yielding a posterior that is
misleading (i.e., the model-averaged posterior will be dominated by a model
that excludes the truth).
Second, we generated simulated datasets for which the divergence times are
drawn from an exponential distribution and applied the approach of
\citet{Hickerson2013} to each of them to see how often the method excludes the
truth.

\subsubsection{Re-analyses of the Philippines dataset using empirical Bayesian
model averaging}

For our re-analyses of the Philippines dataset we followed the model-averaging
approach of \citet{Hickerson2013}, but with a reduced set of prior models to
avoid their error of mixing units of time (see SI for details).
We used five prior models, all of which had priors on population sizes of
$\meanDescendantTheta{} \sim U(0.0001, 0.1)$ and $\ancestralTheta{} \sim
U(0.0001, 0.05)$.
Following \citet{Hickerson2013}, each of these models had the following
priors on divergence times:
$M_1$, $\divt{} \sim U(0, 0.1)$;
$M_2$, $\divt{} \sim U(0, 1)$;
$M_3$, $\divt{} \sim U(0, 5)$;
$M_4$, $\divt{} \sim U(0, 10)$; and
$M_5$, $\divt{} \sim U(0, 20)$.
We simulated $1\e6$ random samples from each of the models for a total of
$5\e6$ prior samples.
For each model, we retained the 10,000 samples with the smallest Euclidean
distance from the observed summary statistics after standardizing the
statistics using the prior means and standard deviations of the given model.
From the remaining 50,000 samples, we then retained the 10,000 samples with the
smallest Euclidean distance from the observed summary statistics, this time
standardizing the statistics using the prior means and standard deviations
across all five models.
We then repeated this analysis twice, replacing the $M_1$ model with
$M_{1A}$ and $M_{1B}$, which differ only by having priors on divergence
times of $\divt{} \sim U(0, 0.01)$ and $\divt{} \sim U(0, 0.001)$,
respectively.
While we suspect the prior of $\divt{} \sim U(0, 0.1)$ used by
\citet{Hickerson2013} likely excludes the true divergence times of at least
some of the 22 taxa, we are nearly certain that these narrower priors exclude
most, if not all, of the divergence times of the Philippine taxa.

Our results show that the model-averaging approach of \citet{Hickerson2013}
strongly prefers the prior model with the narrowest distribution on divergence
times across all three of our analyses, even when this model excludes the true
divergence times of the Philippine taxa
(Table~\ref{tabModelChoiceEmpirical}).
Given that the same number of simuations were sampled from each prior model,
this behavior is not clearly predicted by insufficient computation
(Hypothesis~\ref{hypError}), but is a straightforward prediction of
Hypothesis~\ref{hypML}.

\citet{Hickerson2013} vetted the priors used in their model-averaging
approach via ``graphical checks,'' in which the summary statistics from 1000
random samples of each prior model are plotted along the first two orthogonal
axes of a principle component analysis (see Figure 1 of \citet{Hickerson2013}).
To determine if such prior-predictive analyses would indicate the $M_{1A}$ and
$M_{1B}$ models are problematic, we performed these graphical checks on our
prior models.
Unfortunately, these prior-predictive checks provide no warning that these
priors are too narrow (Figure~S\ref{figPCA}).
Rather, the graphs suggest these invalid priors are ``better fit''
(Figure~S\ref{figPCA}A--C) than the valid priors used by \citet{Oaks2012}
(Figure~S\ref{figPCA}D--F).


\subsubsection{Simulation-based assessment of Hickerson et al.'s
    (\citeyear{Hickerson2013}) model averaging over empirical priors}

To better quantify the propensity of Hickerson et al.'s
(\citeyear{Hickerson2013}) approach to exclude the truth, we simulated 1000
datasets in which the divergence times for the 22-population pairs are drawn
randomly from an exponential distribution with a mean of 0.5 ($\divt{} \sim
Exp(2)$).
All other parameters were identically distributed as the $M_1$--$M_5$ models
(Table~\ref{tabModelChoiceEmpirical}).
We then repeated the model-averaging  analysis described above, retaining 1000
posterior samples for each of the 1000 simulated datasets.
For each simulation replicate, we estimated the Bayes factor in favor
of excluding the truth as the ratio of the posterior to prior odds of
excluding the true value of at least one parameter.
Whenever the Bayes factor preferred a model excluding the truth, we counted the
number of the 22 true divergence times that were excluded by the preferred
model.

Our results show that the model-averaging approach of \citet{Hickerson2013}
favors a model that excludes the true values of parameters in 97\% of the
replicates (90\% with GLM-regression adjustment), excluding up to 21 of the 22
true divergence times (Figure~\ref{figExclusionSimTau}).
Importantly, the posterior probability of excluding at least one true parameter
value is very high in most replicates
(Figure~\ref{figExclusionSimProb}).
Using a Bayes factor of greater than 10 as a criterion for strong support, 66\%
of the replicates (87\% with GLM-regression adjustment) strongly support the
exclusion of true values (Figure~\ref{figExclusionSimProb}).

The results of the above empirical and simulation analyses clearly demonstrate
the risk of using narrow, empirically guided uniform priors in a Bayesian
model-averaging framework.
The consequence of this approach is obtaining a model-averaged posterior
estimate that is heavily weighted toward models that exclude true
values of the parameters.
This is not a general critique of Bayesian model averaging.
Rather, model averaging can provide an elegant way of incorporating
model uncertainty in Bayesian inference.
However, as predicted by Hypothesis~\ref{hypML}, when averaging over models
with narrow and broad uniform priors on a parameter that is not expected to
have a uniformly distributed likelihood density, the posterior can be dominated
by models that exclude from consideration the true values of parameters due to
their larger marginal likelihoods (these models integrate over less space with
high prior weight and low likelihood).

When using uniformly distributed priors, the alternative to capturing prior
uncertainty is to risk excluding the true values one seeks to estimate.
Fortunately, more flexible continuous distributions that are better suited as
priors for the positive real-valued parameters of the \msb model have been
shown to greatly reduce spurious support for clustered divergence models while
allowing prior uncertainty to be accommodated
\citep{Oaks2014dpp}.


\section{Assessing the power of the model-averaging approach of
    \citet{Hickerson2013}}
While our results above clearly demonstrate the risks inherent to the empirical
Bayesian model-choice approach used by \citet{Hickerson2013}, one could justify
such risk if the approach does indeed increase power to detect temporal
variation in divergences.
We assess this possibility using simulations.
Following \citet{Oaks2012}, we simulated 1000 datasets with \divt{} for each of
the 22 population pairs randomly drawn from a uniform distribution, $U(0,
\divt{max})$, where \divt{max} was set to: 0.2, 0.4, 0.6, 0.8, 1.0, and 2.0, in
\globalcoalunit generations.
All other parameters were identically distributed as the prior models.
As above, we generated $5\e6$ samples from prior models $M_1$--$M_5$
(Table~\ref{tabModelChoiceEmpirical}).
For each of the 6000 simulated datasets, we approximated the posterior
by retaining 1000 samples from the prior.

Our results demonstrate that the approach of \citet{Hickerson2013} consistently
infers highly clustered divergences across all the \divt{max} we simulated
(Figure~\ref{figPower4}A--D \& S\ref{figPower6}A--F).
The approach often strongly supports (Bayes factor of greater than 10) the
extreme case of one divergence event across all our simulation conditions
(Figure~\ref{figPower4}E--H \& S\ref{figPower6}G--L).
The method also struggles to estimate the variance of divergence times
(\vmratio{}), whether evaluating the unadjusted
(Figure~S\ref{figPowerAccuracy}A--F) or GLM-adjusted
(Figure~S\ref{figPowerAccuracy}G--L) posterior estimates.
Overall, the empirical Bayesian model-averaging approach leads to erroneous
support for highly clustered divergences when populations diverged randomly
over the last $8\globalpopsize$ generations.
For loci with per-site rates of mutation on the order of $1\e{-8}$ and
$1\e{-9}$ per generation, this translates to 10 million and 100 million
generations, respectively.

Also, the results of our power analyses further demonstrate the propensity of
Hickerson et al.'s (\citeyear{Hickerson2013}) approach to exclude true
parameter values.
Across all but one of the \divt{max} we simulated, the method favors a model
that excludes the truth in a large proportion of replicates, and across many of
the \divt{max} the preferred model will exclude a large proportion of the true
divergence times (Figure~\ref{figPowerExclusion4}A--D \&
S\ref{figPowerExclusion6}A--F).
Importantly, the posterior probability of excluding at least one true
divergence value is also quite high across many of the \divt{max}
(Figure~\ref{figPowerExclusion4}E--H \& S\ref{figPowerExclusion6}G--L).


\section{The importance of power analyses to guide applications of \texttt{msBayes}}
\citet{Hickerson2013} presented a power analysis of \msb under a narrow uniform
divergence-time prior of 0--1 coalescent units ago.
They found that under these prior conditions \msb can, assuming a
per-site rate of $1.92\e{-8}$ mutations per generation, detect multiple
divergence events among 18 taxa when the true divergences were random over
hundreds of thousands of generations or more.
It is important that investigators perform such simulations to determine the
method's power for their dataset, and decide if \msb has sufficient temporal
resolution to address their hypotheses; in the case of the Philippines dataset,
it did not.
When doing so, it is important to consider what prior conditions are relevant
to the empirical system.
It is rare for there to be enough \emph{a priori} information to be certain
that all taxa diverged within the last 4$\globalpopsize$ generations (i.e.,
0--1 coalescent units).
Also, it seems unlikely that when such prior information is available that
being able to detect more than one divergence event in the face of 18 random
divergences over hundreds of thousands of generations will provide much insight
into the evolutionary history of the taxa.

Inferring more than one divergence time shared across all taxa does not confirm
the method is working well when analyzing data generated under random temporal
variation in divergences (e.g., an inference of two divergence events could be
biogeographically interesting yet spurious).
Thus, it is important that investigators not limit their assessment of the
method's power to only differentiating inferences of one event or more (i.e.
$\numt{} = 1$ versus $\numt{} > 1$).
Rather, looking at the distribution of estimates, as in Figure~\ref{figPower4}
and \citet{Oaks2012}, provides much more information about the behavior of the
method.


%%
\section{The causes of support for models of co-divergence}
To determine how best to improve the behavior of \msb, it is important to
determine the mechanism by which broad uniform priors cause support for
clustered models of divergence.
It is well established that vague priors can be problematic in Bayesian model
selection.
Models that integrate over more parameter space characterized by low
probability of producing the data and relatively high prior density will have
smaller marginal likelihoods \citep{Jeffreys1939,Lindley1957}.
Given the uniformly distributed priors on divergence times employed in \msb,
the likelihood of models with more divergence parameters will be ``averaged''
over much greater parameter space, all with equal prior weight, and much of it
with small likelihood (Hypothesis~\ref{hypML}).
In light of this fundamental statistical issue, it is not surprising that the
method tends to support simple models.

However, \citet{Hickerson2013} conclude that the bias is caused by numerical
approximation error due to insufficient computation
(Hypothesis~\ref{hypError}).
They argue the widest of the three priors used by \citet{Oaks2012} would
infrequently produce random samples with many independent population divergence
times as recent as the estimated gene divergence times presented in
\citet{Oaks2012}.
However, this argument assumes the gene divergence times presented in
\citet{Oaks2012} were estimated without error.
These estimates were intended to provide only a rough comparison of the
gene divergence times across the 22 taxa.
These analyses assumed an arbitrary strict per-site rate of $2\e{-8}$ mutations
per generation for all taxa, and are, of course, subject to estimation error.
Furthermore, the branch-length units of the gene trees are in millions of
years, whereas the divergence-time prior of \msb is in generations, thus
\citet{Hickerson2013} make the implicit assumption that all 22 Philippine taxa
have a generation time of one year.
The argument of \citet{Hickerson2013} that these estimated gene divergence
times should be used to set the upper bound of the uniform prior on divergence
times for the \msb analysis of the same sequence data seems questionable,
especially given our findings presented above regarding the behavior of \msb
when empirically informed uniform priors are employed.

Even if we assume (1) the arbitrary strict clock is correct, (2) gene
divergence times were estimated without error, and (3) all 22 taxa have
one-year generation times, Hickerson et al.'s
(\citeyear{Hickerson2013}) argument only applies to one of the \emph{three}
priors used by \citet{Oaks2012}.
Under these assumptions, the narrowest prior on divergence times used by
\citet{Oaks2012} ($U(0, 5)$) closely mirrors the range of estimates of
gene divergence times (0--5 million years ago).
Applying Hickerson et al.'s (\citeyear{Hickerson2013}) sampling-probability
argument demonstrates this prior is densely populated with samples with large
numbers of divergence parameters with values younger than the estimated gene
divergence estimates.
Thus, if insufficient prior sampling (Hypothesis~\ref{hypError}) is to blame
for the bias, it should be much reduced under the narrow prior on \divt{}.
However, the magnitude of the bias is very similar across all three priors
explored by \citet{Oaks2012}.
\citet{Hickerson2013} point out a case where the narrowest prior performs
slightly better (panel L of Figures S32, S37, and S38 of \citet{Oaks2012}).
However, it is important to note that these results suffered from a bug in
\msb, and after \citet{Oaks2012} corrected the bug, there are many cases where
the narrowest prior performs slightly worse (see panels D--J of Figures 3 and
S12).

To disentangle whether Hypothesis 1 or 2 is the primary cause the method's
erroneous support for simple models, we must look at the different predictions
made by these two phenomena.
For example, numerical error due to insufficient prior sampling
(Hypothesis~\ref{hypError}) should create large variance among posterior
estimates and cause analyses to be highly sensitive to the number of samples
drawn from the prior.
Furthermore, if insufficient prior sampling is \emph{biasing} estimates toward
models with less parameter space, as suggested by \citet{Hickerson2013}, we
expect to see support for these models decrease as sampling from the prior
increases.
\citet{Oaks2012} did not see such sensitivity when they compared prior sample
sizes of $2\e6$, $5\e6$, and $10^7$.

To explore this prediction further, we repeat the analysis of the Philippines
dataset under the intermediate prior used by \citet{Oaks2012} ($\divt{} \sim
U(0, 10)$, $\meanDescendantTheta{} \sim (0.0005, 0.04)$, $\ancestralTheta{}
\sim (0.0005, 0.02)$), using a very large prior sample size of $10^8$.
When we look at the trace of the estimates of the dispersion index of
divergence times (\vmratio{}) as the prior samples accumulate
(Figure~S\ref{figSamplingError}) we do not see the trend predicted by
Hypothesis~\ref{hypError}.
While approximation error is always present for any numerical analysis, it does
not appear to be playing a large role in the biases revealed by the results of
\citet{Oaks2012} or presented above.

A straightforward prediction if strongly weighted marginal likelihoods are
causing the preference for simple models (Hypothesis~\ref{hypML}) is that the
bias should disappear as the model generating the data converges to the prior.
\citet{Oaks2012} tested this prediction by performing 100,000 simulations to
assess the model-choice behavior of \msb when the prior model is correct.
The results confirm the prediction of Hypothesis~\ref{hypML}:
\msb estimates the probability of the one-divergence model quite well (or even
\emph{under}estimates it) when the prior is correct (see Figure 4 of
\citet{Oaks2012}).
We confirmed this same behavior for the model-averaging approach used by
\citet{Hickerson2013} (see SI text and Figure~S\ref{figValidationMCBehavior}).
These results are not clearly predicted if insufficient computation was causing
numerical error.
Even when the prior is correct, due to the discrete uniform prior on the number
of divergence events (\numt{}) implemented in \msb, models with larger numbers
of divergence-time parameters (and thus greater parameter space) will still be
far less densely sampled than those with fewer divergence events
\citep{Oaks2012}.
Thus, the results of the simulations of \citet{Oaks2012} are more consistent
with the fundamental sensitivity of marginal likelihoods to priors
(Hypothesis~\ref{hypML}).

This is further demonstrated by the results presented herein that show the
model-averaging approach of \citet{Hickerson2013} prefers models with narrower
\divt{} priors (Table~\ref{tabModelChoiceEmpirical} and
Figs.~\labelcref{figExclusionSimTau,figExclusionSimProb,figPowerExclusion4})
and fewer \divt{} parameters (Figure~\ref{figPower4}).
In all of these analyses, the same number of random samples were drawn from
each of the prior models.
Thus, while approximation error will always be present in any numerical
approximation method, insufficient prior sampling is an unlikely explanation
for the erroneous support for models with less parameter space.
While analyses that sample each model proportional to their parameter space
could be explored, it is clear that the marginal likelihoods under broad
uniform priors on divergence times will be greater for models with fewer
divergence-time dimensions.




\section{Improving inference of shared divergences}
In theory, the model-averaging approach of \citet{Hickerson2013} is appealing.
It leverages a great strength of Bayesian statistical procedures, namely the
ability to obtain marginalized estimates that incorporate uncertainty in
nuisance parameters.
However, when sampling over models with narrow-empirical and diffuse uniform
priors for a parameter that is expected to have a very non-uniform likelihood
density, models that exclude the true values of the parameters we aim to
estimate will often have the largest marginal likelihoods.

The recommendations of \citet{Oaks2012} for mitigating the lack of robustness
of \msb are similar to those of \citet{Hickerson2013}, but avoid the need for
imposing an additional dimension of model choice and using priors that often
exclude the truth.
\citet{Oaks2012} suggest that uniform priors may not be ideal for many
parameters of the \msb model, and recommend the use of probability
distributions from the exponential family.
If we look at the prior distribution on divergence times imposed by
the model-averaging approach of \citet{Hickerson2013} we see it is a mixture of
overlapping uniforms with lower limits of zero (Figure~S\ref{figMCTauPrior}).
This looks very much like an exponential distribution, except that in any state
of the model, all the divergence times are restricted to the hard
bounds of one of the uniform distributions.
Thus, it seems more appropriate to simply place a gamma prior (the exponential
being a special case) on divergence times.
This would capture the prior uncertainty that \citet{Hickerson2013} are
suggesting for divergence times (Figure~S\ref{figMCTauPrior}) while avoiding
costly model-averaging and the constraint that all divergence times must fall
within the hard bounds of the current model state.
It also would allow an investigator to place the majority of the prior density
in regions of parameter space they believe, \emph{a priori}, are most
plausible, but still capture uncertainty in the tails of distributions with low
density.
Indeed, \citet{Oaks2014dpp} has shown that the use of gamma distributions in
place of uniform priors improves the power of the method to detect temporal
variation in divergences and reduces erroneous support for clustered
divergences.


\section{Conclusions}
We demonstrate how the approximate Bayesian model-choice method implemented in
\msb can spuriously support models with less parameter space.
This is caused by the use of uniform priors on divergence times.
Uniform distributions necessitate the use of priors that place high density in
unlikely regions of parameter space, less the risk of excluding the true
divergence times \emph{a priori}.
These broad uniform priors reduce the marginal likelihoods of models with more
divergence-time parameters.
We show that the empirical Bayesian model-averaging approach of
\citet{Hickerson2013} does not mitigate this bias, but rather causes it to
manifest by sampling predominantly from models that often exclude the true
values of the divergence times the method is trying to estimate.
Our results show that it is difficult to choose an uniformly distributed prior
on divergence times that is broad enough to confidently contain the true values
of parameters while being narrow enough to avoid strongly weighted and
misleading posterior support for models with less parameter space.
More generally, it is important to carefully choose prior assumptions about
parameters in Bayesian model selection, because they can strongly influence the
\emph{true} posterior probabilities of the models we seek to estimate.

The common inference of temporally clustered historical events
\citep{Barber2010, Bell2012, Carnaval2009, Chan2011, Chan2014, Daza2010,
    Hickerson2006, Huang2011, Lawson2010, Leache2007, Plouviez2009, Stone2012,
    Voje2009},
when not accompanied with the necessary analyses to assess the robustness and
temporal resolution of such results, should be treated with caution, because
\msb has been shown to erroneously infer clustered events over a range of prior
conditions.
Fortunately, \citet{Oaks2014dpp} has shown that alternative probability
distributions allow prior uncertainty to be accommodated while avoiding
excessive prior density in regions of low likelihood, which greatly improves
inference of shared divergence histories.

The work presented herein follows the principles of Open Notebook Science.
All aspects of the work were recorded in real-time via version-control software
and are publicly available at
\href{https://github.com/joaks1/msbayes-experiments}{\url{https://github.com/joaks1/msbayes-experiments}}.
All information necessary to reproduce our results is provided there.



\section*{Acknowledgments}
We thank the National Science Foundation for supporting this work (DEB
1011423).
J.\ Oaks was also supported by the University of Kansas (KU) Office of Graduate
Studies, Society of Systematic Biologists, Sigma Xi Scientific Research
Society, KU Department of Ecology and Evolutionary Biology, and the KU
Biodiversity Institute.

\bibliography{../bib/references}
% \bibliography{references}

%% LIST OF FIGURES %%%%%%%%%%%%%%%%%%%%%%%%%%
\newpage
\singlespacing

\renewcommand\listfigurename{Figure Captions}
\cftsetindents{fig}{0cm}{2.2cm}
\renewcommand\cftdotsep{\cftnodots}
\setlength\cftbeforefigskip{10pt}
\cftpagenumbersoff{fig}
\listoffigures


\end{linenumbers}

%% TABLES %%%%%%%%%%%%%%%%%%%%%%%%%%%%%%%%
\newpage
\singlespacing

%:TABLE-re-analysis
% \begin{table}[htbp]
%     \sffamily
%     \footnotesize
%     \addtolength{\tabcolsep}{-0.08cm}
%     \rowcolors{2}{}{myGray}
%     %\captionsetup{font=footnotesize}
%     \caption{Results of the model-averaging approach of \citet{Hickerson2013}
%         applied to the Philippines dataset of \citet{Oaks2012} using eight
%         prior models (see \citet{Hickerson2013} Table 1).  The results of (1)
%         \citet{Hickerson2013}, (2) our full re-analysis, and of (3) repeating
%         the final rejection step of \citet{Hickerson2013} using the prior means and
%         standard deviations to normalize the summary statistics, are all compared.}
%     \centering
%     \begin{tabular}{ l l l l }
%         \toprule
%          & Hickerson et al. (2013) & Re-analysis & blah \\
%         $\hat{\numt{}}$ & 2 & 1 & 1 \\
%         $p(\numt{} = 1 | \ssSpace)$ & 0.062 & 0.206 & 0.146 \\
%         $BF_{\numt{} = 1, \numt{} \neq 1}$ 1.39 & 5.45 & 3.59 \\
%         $p(\vmratio{} < 0.01 | \ssSpace)$ & 0.298 & 0.416 & 0.361 \\
%         $BF_{\vmratio{} < 0.01, \vmratio{} \geq 0.01}$ 6.65 & 11.16 & 8.85 \\
%         $\hat{M}$ & 1 & 2 & 2 \\
%         $p(M_1 | \ssSpace)$ & 0.513 & 0.358 & 0.376 \\
%         $p(M_2 | \ssSpace)$ & 0.286 & 0.369 & 0.389 \\
%         $p(M_3 | \ssSpace)$ & 0.036 & 0.086 & 0.089 \\
%         $p(M_4 | \ssSpace)$ & 0.002 & 0.012 & 0.011 \\
%         $p(M_5 | \ssSpace)$ & 0.142 & 0.129 & 0.107 \\
%         $p(M_6 | \ssSpace)$ & 0.020 & 0.043 & 0.024 \\
%         $p(M_7 | \ssSpace)$ & 0.001 & 0.003 & 0.004 \\
%         $p(M_8 | \ssSpace)$ & 0.000 & 0.000 & 0.001 \\
%         \bottomrule
%     \end{tabular}
%     \label{tabModelChoiceEmpirical}
% \end{table}

%:TABLE-model-choice-empirical
\begin{table}[htbp]
    \sffamily
    % \footnotesize
    \addtolength{\tabcolsep}{-0.08cm}
    \rowcolors{2}{}{myGray}
    %\captionsetup{font=footnotesize}
    \caption{Results of the model-averaging approach of \citet{Hickerson2013}
        applied to the Philippines dataset of \citet{Oaks2012} using three sets
        of prior models. All models used priors on population size of
        $\meanDescendantTheta{} \sim U(0.0001,0.1)$ and $\ancestralTheta{} \sim
        U(0.0001, 0.05)$, and differ only in their prior on divergence time
        (\divt{}) parameters.  Each set of five models differ only in the
        divergence time prior used for the model with the narrowest prior:
        $M_1$ ($\divt{} \sim U(0, 0.1)$), $M_{1A}$ ($\divt{} \sim U(0, 0.01)$),
        or $M_{1B}$ ($\divt{} \sim U(0, 0.001)$). The approximate posterior
        probability of each model ($p(M_i \given \ssSpace)$) is given for each
        of the three analyses.  The posterior estimates are based on 10,000
        samples retained from $1\e6$ prior samples
    from each model.}
    \centering
    \begin{tabular}{ l l l l l }
        \toprule
        & & \multicolumn{3}{c}{$p(M_i \given \ssSpace)$} \\
        \cmidrule(){3-5}
        Model & \divt{} prior & $M_{*}=M_1$ & $M_{*}=M_{1A}$ & $M_{*}=M_{1B}$ \\
        \midrule
        $M_*$ & --        & 0.899 & 0.821 & 0.673 \\
        $M_2$ & $U(0,1)$  & 0.079 & 0.136 & 0.251 \\
        $M_3$ & $U(0,5)$  & 0.013 & 0.026 & 0.044 \\
        $M_4$ & $U(0,10)$ & 0.006 & 0.012 & 0.022 \\
        $M_5$ & $U(0,20)$ & 0.003 & 0.005 & 0.010 \\
        \bottomrule
    \end{tabular}
    \label{tabModelChoiceEmpirical}
\end{table}


\clearpage

%% FIGURES %%%%%%%%%%%%%%%%%%%%%%%%%%%%%%%%
\newpage

\mFigure{../images/coin_flip.pdf}{
    A plot of three beta probability density functions that represent a prior
    (black; $beta(10, 10)$), true posterior (blue; $beta(13, 17)$), and
    empirical Bayes density (red; $beta(16, 24)$) for a dataset of 10 Bernoulli
    trials, three of which are successes.
}{figCoinFlip}

\mFigure{../../model-choice/priors-for-pc-plot/pc-plots.pdf}{
    The graphical checks recommended by \citet{Hickerson2013} for three prior
    models: (A) $M_1$ ($\divt{} \sim U(0, 0.1)$), (B) $M_{1A}$ ($\divt{} \sim
    U(0, 0.01)$), and (C) $M_{1B}$ ($\divt{} \sim U(0, 0.001)$).
    The plots project the summary statistics from 1000 random samples from each
    model onto the first two orthogonal axes of a principle component analysis,
    with the blue dot representing the observed summary statistics from the 22
    population pairs of Philippine vertebrates.
}{figPCA}

%% Exclusion simulation results
\mFigure{../../model-choice/results/m1-1-sim/pymsbayes-results/num_tau_excluded.pdf}{
    The propensity of the model-averaging approach of \citet{Hickerson2013} to
    exclude the truth.
    The plots illustrate the number of true \divt{} parameters excluded from analyses of
    simulated datasets where \divt{} for 22 pairs of populations is drawn from an
    exponential distribution, $\divt{} \sim Exp(2)$.
    The plots represent (A) unadjusted and (B) GLM-adjusted estimates from 1000
    simulation replicates analyzed using $5\e6$ samples from the prior.
    The proportion of simulation replicates in which at least one true
    parameter value is excluded from the model preferred by a Bayes factor
    ($p(\divt{} \notin \hat{M})$) is also given.
}{figExclusionSimTau}

\mFigure{../../model-choice/results/m1-1-sim/pymsbayes-results/prob_of_exclusion.pdf}{
    The propensity of the model-averaging approach of \citet{Hickerson2013} to
    exclude the truth.
    The plots illustrate the estimated probability of excluding at least one
    true \divt{} value from analyses of simulated datasets where \divt{} for 22
    pairs of populations is drawn from an exponential distribution, $\divt{}
    \sim Exp(2)$.
    The plots represent (A) unadjusted and (B) GLM-adjusted estimates from 1000
    simulation replicates analyzed using $5\e6$ samples from the prior.
    The proportion of simulation replicates in which there is strong support
    for at least one true parameter value being excluded from the model
    ($p(BF_{\divt{} \notin M, \divt{} \in M} > 10)$) is also given.
}{figExclusionSimProb}

%% Power results
\mFigure{../../model-choice/results/power-1/pymsbayes-results/plots/power_accuracy_omega_median.pdf}{
    The accuracy of the model-averaging approach of \citet{Hickerson2013} to
    estimate the dispersion index of divergence times (\vmratio{}) from
    analyses of simulated datasets where \divt{} for 22 pairs of populations is
    drawn from a series of uniform distributions, $\divt{} \sim U(0,
    \divt{max})$.
    The proportion of estimates less than the true value of,
    $p(\hat{\vmratio{}} < \vmratio{})$, is given for each \divt{max}.
    Each plot represents unadjusted median estimates from 500 simulation
    replicates analyzed using $5\e6$ samples from the prior.
}{figPowerAccOmegaMedian}

\mFigure{../../model-choice/results/power-1/pymsbayes-results/plots/power_accuracy_omega_mode_glm.pdf}{
    The accuracy of the model-averaging approach of \citet{Hickerson2013} to
    estimate the dispersion index of divergence times (\vmratio{}) from
    analyses of simulated datasets where \divt{} for 22 pairs of populations is
    drawn from a series of uniform distributions, $\divt{} \sim U(0,
    \divt{max})$.
    The proportion of estimates less than the true value of,
    $p(\hat{\vmratio{}} < \vmratio{})$, is given for each \divt{max}.
    Each plot represents GLM-regression-adjusted mode estimates from 500
    simulation replicates analyzed using $5\e6$ samples from the prior.
}{figPowerAccOmegaModeGLM}

\mFigure{../../model-choice/results/power-1/pymsbayes-results/plots/power_psi_mode.pdf}{
    The bias of the model-averaging approach of \citet{Hickerson2013} to infer
    clustered divergences in our simulation-based power analyses.
    The plots illustrate the estimated number of divergence events ($\hat{\numt{}}$)
    from analyses of simulated datasets where \divt{} for 22 pairs of populations is drawn from a series of
    uniform distributions, $\divt{} \sim U(0, \divt{max})$.
    The estimated probability of the method inferring one divergence event, $p(\hat{\numt{}} = 1)$,
    is given for each \divt{max}.
    Each plot represents 500 simulation replicates analyzed using $5\e6$
    samples from the prior.
}{figPowerPsiMode}

\mFigure{../../model-choice/results/power-1/pymsbayes-results/plots/power_omega_prob.pdf}{
    The bias of the model-averaging approach of \citet{Hickerson2013} to
    support one divergence event in our simulation-based power analyses.
    The plots illustrate histograms of the estimated posterior probability that
    the dispersion index of divergence times is less than 0.01 ($p(\vmratio{} <
    0.01 | \ssSpace)$) from analyses of simulated datasets where \divt{} for 22
    pairs of populations is drawn from a series of uniform distributions,
    $\divt{} \sim U(0, \divt{max})$.
    The proportion of simulation replicates that strongly support one
    divergence event, $p(BF_{\vmratio{} < 0.01, \vmratio{} \geq 0.01} > 10)$,
    is given for each \divt{max}.
    Each plot represents 500 simulation replicates analyzed using $5\e6$
    samples from the prior.
}{figPowerOmegaProb}

\mFigure{../../model-choice/results/power-1/pymsbayes-results/plots/power_num_excluded.pdf}{
    The propensity of the model-averaging approach of \citet{Hickerson2013} to exclude
    the truth in our simulation-based power analyses.
    The plots illustrate the number of true \divt{} parameters excluded from analyses of
    simulated datasets where \divt{} for 22 pairs of populations is drawn from a series of
    uniform distributions, $\divt{} \sim U(0, \divt{max})$. The proportion
    of simulation replicates in which at least one true parameter value is excluded from
    the model preferred by a Bayes factor ($p(\divt{} \notin \hat{M})$) is
    given for each \divt{max}.
    Each plot represents 500 simulation replicates analyzed using $5\e6$
    samples from the prior.
}{figPowerNumExcluded}

\mFigure{../../model-choice/results/power-1/pymsbayes-results/plots/power_prob_exclusion.pdf}{
    The propensity of the model-averaging approach of \citet{Hickerson2013} to
    exclude the truth in our simulatoin-based power analyses.  The plots
    illustrate the estimated posterior probability of excluding at least one
    true \divt{} value from analyses of simulated datasets where \divt{} for 22
    pairs of populations is drawn from a series of uniform distributions,
    $\divt{} \sim U(0, \divt{max})$. The proportion of simulation replicates in
    which there is strong support for at least one true parameter value being
    excluded from the model ($p(BF_{\divt{} \notin M, \divt{} \in M} > 10)$) is
    given for each \divt{max}.
    Each plot represents 500 simulation replicates analyzed using $5\e6$
    samples from the prior.
}{figPowerProbExclusion}

%% Validation results
\mFigure{../../model-choice/results/validation/results/pymsbayes-results/plots/mc_behavior.pdf}{
    An assessment of the approximate Bayesian model-averageing approach of
    \citet{Hickerson2013} under the ideal conditions when the prior model
    is correct (i.e., the pseudo-replicate datasets are simulated from
    parameters drawn from the same prior distributions used in the
    analysis).
    The plots show the relationship between the estimated posterior and true
    probability of (A \& C) $\numt{}=1$ and (B \& D) $\vmratio{} < 0.01$, based
    on 50,000 simulations.
    The results summarize the (A \& B) unadjusted and (C \& D) GLM-adjusted
    posterior estimate from each simulation replicate.
    The prior settings for all replicates included five prior models with
    $\meanDescendantTheta{} \sim U(0.0001, 0.1)$ and $\ancestralTheta{} \sim
    U(0.0001, 0.05)$ for all five models, and
    $M_1: \divt{} \sim U(0, 0.1)$,
    $M_2: \divt{} \sim U(0, 1)$,
    $M_3: \divt{} \sim U(0, 5)$,
    $M_4: \divt{} \sim U(0, 10)$, and
    $M_5: \divt{} \sim U(0, 20)$.
    The number of samples from the prior was $2.5\e6$.
    The simulated data structure was 8 population pairs, with a single 1000
    bp locus sampled from 10 individuals from each population.  The 50,000
    estimates of the posterior probability of one divergence event were
    assigned to 20 bins of width 0.05.
    The estimated posterior probability of each bin is plotted against the
    proportion of replicates in that bin with a true value consistent with
    one divergence event (i.e., $\numt{}=1$ or $\vmratio{} < 0.01)$.
}{figValidationMCBehavior}

\mFigure{../../response-redux/results/sampling-error/pymsbayes-results/omega_over_sampling.pdf}{
    Traces of the estimated lower and upper limits of the 95\% highest posterior density (HPD)
    interval of \vmratio{} (the dispersion index of divergence times) as 100 million prior
    samples are accumulated. Each pair of points is based on 1000 posterior samples retained
    from the prior. Both (A) unadjusted and (b) GLM-regression-adjusted estimates are shown.
    Prior settings were $\divt{} \sim U(0,10)$, $\meanDescendantTheta{} \sim U(0.0005, 0.04)$,
    and $\ancestralTheta{} \sim U(0.0005, 0.02)$.
}{figSamplingError}

%% Model-choice tau prior
\mFigure{../../model-choice/tau-prior/tau_prior.pdf}{
    The prior distribution on divergence times imposed by the model-averaging prior
    comprised of five models with different uniform priors on \divt{}:
    $M_1$ ($\divt{} \sim U(0, 0.1)$), $M_2$ ($\divt{} \sim U(0, 1)$), $M_3$
    ($\divt{} \sim U(0, 5)$), $M_4$ ($\divt{} \sim U(0, 10)$), $M_5$ ($\divt{}
    \sim U(0, 20)$).
}{figMCTauPrior}

%:FIGURE-saturation plot
\mFigure{../../saturation/saturation-plot.pdf}{
    The summary statistics $\pi$ \citep{Tajima1983} and $\pi_{net}$
    \citep{Takahata1985} as a function of divergence time between populations.
    Each plot represents 1100 pairs of parameter draws and summary statistics
    calculated from the simulated data.
    Prior settings for the simulations were $\divt{} \sim U(0, 20)$,
    $\meanDescendantTheta{} \sim U(0.0005, 0.04)$, and $\ancestralTheta{} \sim
    U(0.0005, 0.02)$.
}{figSaturationPlot}

% \mFigure{../../response-redux/results/hickerson/pymsbayes-results/pymsbayes-output/d1/m12345678-combined/mean_by_dispersion.pdf}{
%     Results of reanalysis.
% }{figJointPosterior}



%%%%%%%%%%%%%%%%%%%%%%%%%%%%%%%%%%%%%%%%%%%%%%%%%%%%%%%%%%%%%%%%%%
%% SUPPORTING INFO %%%%%%%%%%%%%%%%%%%%%%%%%%%%%%%%%%%%%%%%%%%%%%%
\setcounter{figure}{0}
\setcounter{table}{0}
\setcounter{page}{1}

\singlespacing

% PUT MAIN TEXT CITATION HERE
{
\renewcommand{\refname}{\noindent\MakeUppercase{\LARGE\sffamily\upshape supporting information}}
\begin{thebibliography}{1}
\providecommand{\natexlab}[1]{#1}
\providecommand{\url}[1]{\texttt{#1}}
\providecommand{\urlprefix}{URL }

\bibitem[{Oaks et~al.(2014)Oaks, Sukumaran, Esselstyn, Linkem, Siler, Brown,
  and Holder}]{Oaks2013reply}
Oaks, J.~R., J.~Sukumaran, J.~A.\ Esselstyn, C.~W.\ Linkem, C.~D.\ Siler,
    R.~M.\ Brown, and M.~T.\ Holder,
  % 2013.
\newblock \msTitle
% \newblock Evolution 67:991--1010.

\end{thebibliography}
}

\doublespacing
\section*{An error in Hickerson et al.'s re-analysis of the Philippines data}
\citet{Hickerson2013} re-analyzed the dataset of \citet{Oaks2012} using a
model-averaging approach, where they placed a discrete uniform prior over eight
different prior models (see Table 1 of \citet{Hickerson2013}).
However, there was an error in their methodology; their model mixes different
units of time.

Each of the eight prior models used in the re-analysis by \citet{Hickerson2013}
has one of two priors on the mean size of the descendant populations of each
taxon pair:
$\meanDescendantTheta{} \sim U(0.0001, 0.1)$ or
$\meanDescendantTheta{} \sim U(0.0005, 0.04)$.
As described in \citet{Oaks2012}, the divergence-time parameters of the model
implemented in \msb are scaled relative to a constant reference population
size, \myTheta{C}.
This reference population size is defined in terms of the upper limit of the
uniform prior on the mean size of the descendant populations,
\meanDescendantTheta{}, such that for the prior $\meanDescendantTheta{} \sim
U(\uniformMin{\meanDescendantTheta{}},\uniformMax{\meanDescendantTheta{}})$,
$\myTheta{C} = \uniformMax{\meanDescendantTheta{}}/2$.
Thus, the model used by \citet{Hickerson2013} mixes two different units of
time.
In other words, some of their prior and posterior samples are in units of
$0.05/\mutationRate$ generations, whereas others are in units of
$0.02/\mutationRate$ generations.

The fact that their posterior samples are in different units makes the results
of \citet{Hickerson2013} difficult to interpret, and renders their
regression-adjusted results invalid.
A fundamental assumption of regression is that all of the values of the
response variable are in the same units.
Thus, the results in sections ``Using ABC Model Comparison to Weight
Alternative Priors for the Philippine Vertebrate Data'' and ``Improved Sampling
Efficiency by Prior Weighting Supports Asynchronous and Recent Divergence for
the Philippines Vertebrate Data'' and presented in Figure 2 of
\citet{Hickerson2013} should be disregarded.
The error is easily illustrated by re-plotting their results with the different
time units indicated (Figure~\ref{figJointPosteriorHickerson}).

%CWL: This seems to specific, not sure about including
\section*{Validation analyses}
% \highLight{Not sure where to put this section}.
% \highLight{This justifies presenting the unadjusted results in this paper.}
% \highLight{Is it worth including just for this purpose?}
Following \citet{Oaks2012}, we characterize the model-choice behavior of the
model-averaging approach of \citet{Hickerson2013} under the ideal conditions
where the prior is correct (i.e., the data are generated from parameters drawn
from the same prior distributions used in the analysis).
We used the same prior models as above ($M_1$--$M_5$;
Table~\ref{tabModelChoiceEmpirical}), and simulated 50,000 datasets under this
prior (10,000 from each model).
We used a simulated data structure of eight population pairs, with a single
1000 base-pair locus sampled from 10 individuals from each population.
We then analyzed each of these replicate datasets using the same prior with 2.5
million samples (500,000 from each of the five prior models), retaining 1000
posterior samples.
Our results are very similar to \citet{Oaks2012}, but we note that they
are not directly comparable as our simulations contained eight population
pairs rather than 10 (Figure~\ref{figValidationMCBehavior}).
We find that the approach of \citet{Hickerson2013} estimates the posterior
probability of divergence models reasonably well when all assumptions of the
method are met (i.e., the prior is correct) and the unadjusted posterior
estimates are used.
Similar to \cite{Oaks2012}, we find that the regression-adjusted estimates of
the model probabilities are biased.

\section*{Additional clarifications from \citet{Hickerson2013}}

\subsection*{Saturation of summary statistics}
\citet{Hickerson2013} claim the priors used by \citet{Oaks2012} ``cause much of
the explored parameter space to be beyond the threshold of saturation in most
mtDNA genes.'' To explore this possibility, we simulated datasets under prior
settings that match two of the three priors used by \citet{Oaks2012}:
$\meanDescendantTheta{} \sim U(0.0005, 0.04)$ and $\ancestralTheta{} \sim
U(0.0005, 0.02)$.
Under this prior, we randomly sample divergence-time parameters from a uniform
distribution of $U(0, 20)$ coalescent units, simulate datasets, and plot the
\divt{} values against the summary statistics calculated from the resulting
datasets (Figure~\ref{figSaturationPlot}).
Clearly, the priors used by \citet{Oaks2012} with upper limits on \divt{} of five
and 10 coalescent units suffered little to no effect from saturation.
Even at divergence times of 20 coalescent units, there is still signal in the
summary statistics used by \msb (Figure~\ref{figSaturationPlot}).
Thus, the assertion of \citet{Hickerson2013} does not apply to at least
two of the priors used by \citet{Oaks2012} and, as a result, does not
explain the bias they found.

\subsection*{Graphical prior comparisons}
\citet{Hickerson2013} advocate the use of what they call graphical checks of
prior models.
This prior-predictive approach entails generating a small number (1000) of
random samples from the prior and plotting the resulting summary statistics in
comparison to the observed statistics to see if they coincide (see Figure 1 of
\citet{Hickerson2013}).
% As we show above, this strategy can be misleading, because the resulting plots
% of this approach have little correlation with the appropriateness of priors.
Given the richness of the \msb model ($\approx 600$ parameters for the Philippine
dataset analyzed by \citet{Hickerson2013}), we do not expect that 1000
\emph{random} draws from the vast prior parameter space will yield data and
summary statistics consistent with the observed data.
In fact, when such random draws are tightly clustered around the observed
statistics, this can be an indication that the prior is over-fit, as we show in
the main text (Table~\ref{tabModelChoiceEmpirical} and Figure~S\ref{figPCA}).
Thus, using such plots to select priors should be avoided, and the use of
posterior-predictive analyses would be much more informative about the overall
fit of models.

\subsection*{Differing utilities of \numt{} and \vmratio{} in \msb}
The primary component of the \msb model is the vector of divergence
times for each of the taxon pairs,
$\divtvector = \{\divt{1}, \ldots, \divt{Y}\}$
\citep{Oaks2012}.
\citet{Hickerson2013} argue that the dispersion index of this vector,
\vmratio{}, is a better model-choice estimator than the number of 
divergence-time parameters within the vector,
\numt{}.
They present a plot of \numt{} against \vmratio{} (Fig.~S1 of
\citet{Hickerson2013}), which is essentially a plot of sample size versus
variance.
This plot shows, not surprisingly, that \vmratio{} has very little information
about the number of divergences among taxa.
Nonetheless, \citet{Hickerson2013} conclude \vmratio{} is more informative and
biogeographically relevant than \numt{}.
% We struggle to follow this logic.
However, all of the information about the temporal distribution of divergences
is contained within the divergence-time vector that \vmratio{} is summarizing.
Clearly, the number of divergence time parameters within the vector and their
values is more informative than its variance (i.e., the dispersion index is not
a sufficient statistic for \divtvector).
\citet{Hickerson2013} also argue that ``\msb can estimate \vmratio{} much
better than \numt{}.''
However, \citet{Oaks2012} demonstrate that even when all assumptions of the
model are met, \vmratio{} is a poor model-choice estimator (see plots B, D \& F
of Figure 4 in \citet{Oaks2012}), whereas \numt{} performs better.

Importantly, \vmratio{} is limited to estimating the probability of only a
single model (the one-divergence model), and thus its utility for model-choice
is very limited.
I.e., it can only be informative about the probability of whether there is one
divergence shared among the taxa ($\vmratio{} = 0.0$) or there is greater than
one divergence ($\vmratio{} > 0.0$).
As a result, not only is its model-choice utility limited, but it is also
very difficult to estimate.
\vmratio{} can range from zero to infinity, and the point density that it is
at its lower limit of zero will always be zero.
Thus, an arbitrary threshold (0.01 is used throughout the \msb literature) must
be chosen to make the probability of ``simultaneous'' divergence estimable.
Even with this arbitrary threshold, it is still not surprising to see that it
is numerically difficult to obtain reliable estimates of the probability that
\vmratio{} is ``near'' its lower limit of zero.
It is easier, less subjective, and more interpretable to estimate the
probability of the discrete parameter of the model, \numt{}, is at its lower
limit of one.
Thus, it is not surprising that \citet{Oaks2012} find that \numt{} is a better
estimator of model probability than \vmratio{}.



\end{document}

