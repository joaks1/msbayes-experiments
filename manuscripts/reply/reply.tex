%&<latex>
\documentclass[letterpaper,12pt]{article}

%%%%%%%%%%%%%%%%%%%%%%%%%%%%%%%%%%%%%%%%%%%%%%%%%%%%%%%%%%%%
%% preamble %%%%%%%%%%%%%%%%%%%%%%%%%%%%%%%%%%%%%%%%%%%%%%%%
\documentclass[table]{beamer}
\usepackage{beamerthemesplit}
\usetheme{boxes}
\usecolortheme{seahorse}
% \useinnertheme{myboxes}
% \usepackage{amsmath}
% \usepackage[fleqn]{amsmath}
\usepackage{ifthen}
\usepackage{xspace}
\usepackage{multirow}
\usepackage{booktabs}
\usepackage{xcolor}
\usepackage[style=nature]{biblatex}
\bibliography{../manuscripts/bib/references}
\newrobustcmd*{\footlessfullcite}{\AtNextCite{\renewbibmacro{title}{}\renewbibmacro{in:}{}}\footfullcite}
\AtEveryBibitem{\clearfield{month}}
\AtEveryCitekey{\clearfield{month}}

% Make all footnotes smaller
\renewcommand{\footnotesize}{\scriptsize}

\definecolor{myGray}{gray}{0.9}
\colorlet{rowred}{red!30!white}

\setbeamertemplate{blocks}[rounded][shadow=true]

\setbeamercolor{defaultcolor}{bg=structure!30!normal text.bg,fg=black}
\setbeamercolor{block body}{bg=structure!30!normal text.bg,fg=black}
\setbeamercolor{block title}{bg=structure!50!normal text.bg,fg=black}

\newenvironment<>{varblock}[2][\textwidth]{%
  \setlength{\textwidth}{#1}
  \begin{actionenv}#3%
    \def\insertblocktitle{#2}%
    \par%
    \usebeamertemplate{block begin}}
  {\par%
    \usebeamertemplate{block end}%
  \end{actionenv}}

\newenvironment{displaybox}[1][\textwidth]
{
    \centerline\bgroup\hfill
    \begin{beamerboxesrounded}[lower=defaultcolor,shadow=true,width=#1]{}
}
{
    \end{beamerboxesrounded}\hfill\egroup
}

\newenvironment{onlinebox}[1][4cm]
{
    \newbox\mybox
    \newdimen\myboxht
    \setbox\mybox\hbox\bgroup%
        \begin{beamerboxesrounded}[lower=defaultcolor,shadow=true,width=#1]{}
    \centering
}
{
    \end{beamerboxesrounded}\egroup
    \myboxht\ht\mybox
    \raisebox{-0.25\myboxht}{\usebox\mybox}\hspace{2pt}
}

\newenvironment{mydescription}{
    \begin{description}
        \setlength{\leftskip}{-1.5cm}}
    {\end{description}}

\newenvironment{myitemize}{
    \begin{itemize}
        \setlength{\leftskip}{-.3cm}}
    {\end{itemize}}

% define formatting for footer
\newcommand{\myfootline}{%
    {\it
    \insertshorttitle
    \hspace*{\fill} 
    \insertshortauthor, \insertshortinstitute
    % \ifx\insertsubtitle\@empty\else, \insertshortsubtitle\fi
    \hspace*{\fill}
    \insertframenumber/\inserttotalframenumber}}

% set up footer
\setbeamertemplate{footline}{%
    \usebeamerfont{structure}
    \begin{beamercolorbox}[wd=\paperwidth,ht=2.25ex,dp=1ex]{frametitle}%
        \Tiny\hspace*{4mm}\myfootline\hspace{4mm}
    \end{beamercolorbox}}

% remove navigation bar
\beamertemplatenavigationsymbolsempty


% \newcommand{\change}[1]{{\color{blue} #1}\xspace}
\newcommand{\change}[1]{{\color{black} #1}\xspace}


\newcommand{\citationNeeded}{\textcolor{magenta}{\textbf{[CITATION NEEDED!]}}\xspace}
\newcommand{\tableNeeded}{\textcolor{magenta}{\textbf{[TABLE NEEDED!]}}\xspace}
\newcommand{\figureNeeded}{\textcolor{magenta}{\textbf{[FIGURE NEEDED!]}}\xspace}
\newcommand{\highLight}[1]{\textcolor{magenta}{\MakeUppercase{#1}}}

\newcommand{\editorialNote}[1]{\textcolor{red}{[\textit{#1}]}}
\newcommand{\ignore}[1]{}
\newcommand{\addTail}[1]{\textit{#1}.---}
\newcommand{\super}[1]{\ensuremath{^{\textrm{#1}}}}
\newcommand{\sub}[1]{\ensuremath{_{\textrm{#1}}}}
\newcommand{\dC}{\ensuremath{^\circ{\textrm{C}}}}

\providecommand{\e}[1]{\ensuremath{\times 10^{#1}}}

\newcommand{\mthnote}[2]{{\color{red} #2}}

\newcommand{\ifTwoArgs}[3]{\ifthenelse{\equal{#1}{}\or\equal{#2}{}}{}{#3}\xspace}
\newcommand{\ifArg}[2]{\ifthenelse{\equal{#1}{}}{}{#2}\xspace}

\newcommand{\allDatasets}{\ensuremath{\mathcal{\alignment{}{}}}\xspace}
\newcommand{\allParameterValues}{\ensuremath{\boldsymbol{\Theta}}\xspace}
\newcommand{\bayesfactor}[2]{\ensuremath{BF_{#1\protect\ifArg{#2}{,}#2}}}
\newcommand{\given}{\ensuremath{\,|\,}\xspace}
\newcommand{\msb}{\upshape\texttt{\MakeLowercase{ms\MakeUppercase{B}ayes}}\xspace}
\newcommand{\hky}{HKY85\xspace}
\newcommand{\uniformMin}[1]{\ensuremath{a_{#1}}\xspace}
\newcommand{\uniformMax}[1]{\ensuremath{b_{#1}}\xspace}
\newcommand{\locusRateHetShapeParameter}{\ensuremath{\alpha}\xspace}
\newcommand{\ancestralThetaVector}{\ensuremath{\boldsymbol{\theta_{A}}}\xspace}
\newcommand{\descendantThetaVector}[1]{\ensuremath{\boldsymbol{\theta_{D#1}}}\xspace}
\newcommand{\divtscaledvector}{\ensuremath{\mathbf{{\divtscaled{}{}}}}\xspace}
\newcommand{\divtvector}{\ensuremath{\boldsymbol{\divt{}}}\xspace}
\newcommand{\divtuniquevector}{\ensuremath{\mathbf{\divtunique{}}}\xspace}
\newcommand{\bottleTimeVector}{\ensuremath{\boldsymbol{\bottleTime{}}}\xspace}
\newcommand{\bottleTime}[1]{\ensuremath{\divt{B\ifArg{#1}{,}#1}}\xspace}
\newcommand{\bottleScalarVector}[1]{\ensuremath{\boldsymbol{\bottleScalar{#1}{}}}\xspace}
\newcommand{\bottleScalar}[2]{\ensuremath{\zeta_{D#1\protect\ifArg{#2}{,}#2}}\xspace}
\newcommand{\migrationRateVector}{\ensuremath{\mathbf{\migrationRate{}}}\xspace}
\newcommand{\geneTreeVector}{\ensuremath{\mathbf{\geneTree{}{}}}\xspace}
\newcommand{\alignmentVector}{\ensuremath{\mathbf{\alignment{}{}}}\xspace}
\newcommand{\alignment}[2]{\ensuremath{X_{#1\protect\ifTwoArgs{#1}{#2}{,}#2}}\xspace}
\newcommand{\geneTree}[2]{\ensuremath{G_{#1\protect\ifTwoArgs{#1}{#2}{,}#2}}\xspace}
\newcommand{\migrationRate}[1]{\ensuremath{m_{#1}}\xspace}
\newcommand{\recombinationRate}{\ensuremath{r}\xspace}
\newcommand{\ploidyScalar}[2]{\ensuremath{\rho_{#1\protect\ifTwoArgs{#1}{#2}{,}#2}}\xspace}
\newcommand{\ploidyScalarVector}{\ensuremath{\boldsymbol{\ploidyScalar{}{}}}\xspace}
\newcommand{\descendantRelativeThetaVector}[1]{\ensuremath{\boldsymbol{\eta_{D#1}}}\xspace}
\newcommand{\descendantRelativeTheta}[2]{\ensuremath{\eta_{D#1\protect\ifArg{#2}{,}#2}}\xspace}
\newcommand{\mutationRateScalarConstant}[2]{\ensuremath{\nu_{#1\protect\ifTwoArgs{#1}{#2}{,}#2}}\xspace}
\newcommand{\mutationRateScalarConstantVector}{\ensuremath{\boldsymbol{\mutationRateScalarConstant{}{}}}\xspace}
\newcommand{\locusMutationRateScalar}[1]{\ensuremath{\upsilon_{#1}}\xspace}
\newcommand{\locusMutationRateScalarVector}{\ensuremath{\boldsymbol{\upsilon}}\xspace}
\newcommand{\hkyModel}[2]{\ensuremath{\phi_{#1\protect\ifTwoArgs{#1}{#2}{,}#2}}\xspace}
\newcommand{\hkyModelVector}{\ensuremath{\boldsymbol{\hkyModel{}{}}}\xspace}
\newcommand{\mutationRate}{\ensuremath{\mu}\xspace}
\newcommand{\iid}{\textit{iid}\xspace}
\newcommand{\model}[1]{\ensuremath{\Theta}\xspace}
\newcommand{\npairs}[1]{\ensuremath{Y_{#1}}}
\newcommand{\nloci}[1]{\ensuremath{k_{#1}}\xspace}
\newcommand{\nlociTotal}{\ensuremath{K}\xspace}
\newcommand{\myTheta}[1]{\ensuremath{\theta_{#1}}}
\newcommand{\ancestralTheta}[1]{\ensuremath{\theta_{A\protect\ifArg{#1}{,}#1}}\xspace}
\newcommand{\descendantTheta}[2]{\ensuremath{\theta_{D#1\protect\ifArg{#2}{,}#2}}\xspace}
\newcommand{\meanDescendantTheta}[1]{\ensuremath{\descendantTheta{}{#1}}\xspace}
\newcommand{\nucdiv}[1]{\ensuremath{\pi_{#1}}}

\newcommand{\ssVector}[1]{\ensuremath{\mathbf{\alignmentSS{#1}{}}}\xspace}
\newcommand{\ssVectorObs}{\ensuremath{\ssVector{}^*}\xspace}
\newcommand{\ssSpace}{\ensuremath{\euclideanSpace{\ssVectorObs}}\xspace}
\newcommand{\ssVectorObsPLS}{\ensuremath{\ssVectorObs_{PLS}}\xspace}
\newcommand{\alignmentSS}[2]{\ensuremath{S_{#1\protect\ifTwoArgs{#1}{#2}{,}#2}}\xspace}
\newcommand{\alignmentSSObs}[2]{\ensuremath{\alignmentSS{#1}{#2}^*}\xspace}
\newcommand{\tol}{\ensuremath{\epsilon}\xspace}
\newcommand{\euclideanSpace}[1]{\ensuremath{B_{\tol}(#1)}\xspace}
\newcommand{\hpvector}[1]{\ensuremath{\Lambda_{#1}}}
\newcommand{\divtscaled}[2]{\ensuremath{t_{#1\protect\ifTwoArgs{#1}{#2}{,}#2}}}
\newcommand{\divt}[1]{\ensuremath{\tau_{#1}}}
\newcommand{\divtunique}[1]{\ensuremath{T_{#1}}}
\newcommand{\ssMatrix}{\ensuremath{\mathbb \alignmentSS{}{}}\xspace}
\newcommand{\ssMatrixRaw}[1]{\ensuremath{{\ssMatrix}_{stats#1}}\xspace}
\newcommand{\ssMatrixPLS}[1]{\ensuremath{{\ssMatrix}_{PLS#1}}\xspace}
\newcommand{\hpmatrix}[1]{\ensuremath{\mathcal{P}_{#1}}}
\newcommand{\meant}[2]{\ensuremath{E(\divt{#1})_{#2}}}
\newcommand{\meantestimate}{\ensuremath{\hat{E(\divt{})}}\xspace}
\newcommand{\vart}[2]{\ensuremath{Var(\divt{#1}{})_{#2}}}
\newcommand{\vmratio}[1]{\ensuremath{\Omega_{#1}}}
\newcommand{\numt}[1]{\ensuremath{\Psi_{#1}}}
\newcommand{\probnumt}[2]{\ensuremath{p(\numt{#1} = {#2})}}
\newcommand{\postprobnumt}[1]{\ensuremath{p(\numt{} = {#1}|\ssSpace)}}
\newcommand{\postprobnumtnot}[1]{\ensuremath{p(\numt{} \neq {#1}|\ssSpace)}}
\newcommand{\postprobomegasimult}{\ensuremath{p(\vmratio{} < 0.01 | \ssSpace)}\xspace}
\newcommand{\modelprior}[1]{\ensuremath{f(\model{})}}
\newcommand{\modelpost}[1]{\ensuremath{f(\model{}|\ssSpace)}}
\newcommand{\npriorsamples}{\ensuremath{n}\xspace}
\newcommand{\globalcoalunit}{\ensuremath{4\globalpopsize}\xspace}
\newcommand{\globalpopsize}{\ensuremath{N_C}\xspace}
\newcommand{\effectivePopSize}[1]{\ensuremath{N_e{#1}}\xspace}
\newcommand{\coalunit}{\ensuremath{4\effectivePopSize{}}\xspace}
\newcommand{\priorsample}[1]{\ensuremath{\hpmatrix{\modelprior{}}}}
\newcommand{\truncprior}[1]{\ensuremath{\hpmatrix{\tol}}\xspace}
\newcommand{\postsample}[1]{\ensuremath{\hpmatrix{\modelpost{}}}}
\newcommand{\abcllr}[1]{ABC\sub{LLR}}
\newcommand{\abcglm}[1]{ABC\sub{GLM}}
\newcommand{\integerPartition}[1]{\ensuremath{a({#1})}}
\newcommand{\uniqueModel}[2]{\ensuremath{M_{#1\protect\ifTwoArgs{#1}{#2}{,}#2}}}
\newcommand{\taxonLocusVector}[1]{\ensuremath{\{#1{1}{1},\ldots,#1{\npairs{}}{\nloci{\npairs{}}}\}}\xspace}
\newcommand{\taxonVector}[1]{\ensuremath{\{#1{1},\ldots,#1{\npairs{}}\}}\xspace}
\newcommand{\locusVector}[1]{\ensuremath{\{#1{1},\ldots,#1{\nlociTotal}\}}\xspace}

\newcommand{\simulationDescription}[2]{\change{Each plot represents #1
    simulation replicates using the same $#2$ samples from the prior}}
\newcommand{\simulationDistribution}{\ensuremath{\divt{} \sim U(0,
    \divt{max})}\xspace}
\newcommand{\estimateDescription}[2]{All estimates were obtained using #1 and #2}
\newcommand{\estimateDescriptionUncorrected}[1]{All estimates based on
    unadjusted posterior, \truncprior{}, obtained using #1}
\newcommand{\priorDescription}[4]{Prior settings were \priorSettings{#1}{#2}{#3}{#4}}
\newcommand{\priorSettings}[4]{$\divt{} \sim U(0, #1)$,
    $\meanDescendantTheta{} \sim U(#2, #3)$, and
    $\ancestralTheta{}{} \sim U(#2, #4)$}
\newcommand{\priorDescriptionBug}[4]{Prior settings were
    \priorSettingsBug{#1}{#2}{#3}{#4}}
\newcommand{\priorSettingsBug}[4]{$\divt{} \sim U(0, #1)$,
    $\meanDescendantTheta{} \sim U(#2, #3)$, and
    $\ancestralTheta{}{} \sim U(0.01, #4)$}
\newcommand{\simulationScheme}{simulations where \divt{} (in \globalcoalunit
    generations) for 22 population pairs is drawn from a series of uniform
    distributions, \simulationDistribution}
\newcommand{\captionPowerOmega}{Histograms of the estimated dispersion index
    of divergence times ($\hat{\vmratio{}}$) from \simulationScheme.
    The threshold for one divergence event \citep{Hickerson2006} is indicated
    by the dashed line, and the estimated probability of inferring one
    divergence event, $p(\hat{\vmratio{}}\le 0.01)$, is given for each
    \divt{max}}
\newcommand{\captionPowerPsiMode}{Histograms of the estimated number of
    divergence events ($\hat{\numt{}}$) from \simulationScheme.
    The estimated probability of inferring one divergence event,
    $p(\hat{\numt{}} = 1)$, is given for each \divt{max}}
\newcommand{\captionPowerPsi}{Histograms of the estimated posterior
    probability of one divergence event, \postprobnumt{1}, from
    \simulationScheme.
    The estimated probability of inferring one divergence event with a
    Bayes factor greater than 10 (dashed black line),
    $p(\bayesfactor{\numt{}=1}{\numt{} \ne 1} > 10)$, is given for each \divt{max}.
    The red line indicates $\postprobnumt{1} = 0.95$, and the estimated
    probability of inferring a posterior probability greater than 0.95 is given
    to the right of the line.}
\newcommand{\captionAccuracy}[1]{Accuracy and precision of #1 estimates from
    \simulationScheme.
    The proportion of estimates less than the true value ($p(\hat{#1}<#1)$) is
    given for each \divt{max}}
\newcommand{\samplingErrorTableNote}{An estimate of 1.0 for a posterior probability
    is an artifact of sampling error}


\newcommand{\refAccuracyALL}[1]{\labelcref{fig_acc_t_ss_llr_bug,fig_acc_t_ss_glm_bug,fig_acc_t_pls_llr_bug,fig_acc_t_pls_glm_bug,fig_acc_o_ss_llr_bug,fig_acc_o_ss_glm_bug,fig_acc_o_pls_llr_bug,fig_acc_o_pls_glm_bug}}
\newcommand{\refAccuracySS}[1]{\labelcref{fig_acc_t_ss_llr_bug,fig_acc_t_ss_glm_bug,fig_acc_o_ss_llr_bug,fig_acc_o_ss_glm_bug}}
\newcommand{\refAccuracySSfull}[1]{\labelcref{fig_acc_t_ssfull_llr_bug,fig_acc_t_ssfull_glm_bug,fig_acc_o_ssfull_llr_bug,fig_acc_o_ssfull_glm_bug}}
\newcommand{\refSSfull}[1]{\labelcref{fig_acc_t_ssfull_llr_bug,fig_acc_t_ssfull_glm_bug,fig_acc_o_ssfull_llr_bug,fig_acc_o_ssfull_glm_bug,fig_pow_o_ssfull_llr_bug,fig_pow_o_ssfull_glm_bug,fig_pow_psi_modes_ssfull_glm_bug}}
\newcommand{\refSS}[1]{\labelcref{fig_acc_t_ss_llr_bug,fig_acc_t_ss_glm_bug,fig_acc_o_ss_llr_bug,fig_acc_o_ss_glm_bug,fig_pow_o_ss_llr_bug,fig_pow_o_ss_glm_bug,fig_pow_psi_ss}}
\newcommand{\refAccuracyPLS}[1]{\labelcref{fig_acc_t_pls_llr_bug,fig_acc_t_pls_glm_bug,fig_acc_o_pls_llr_bug,fig_acc_o_pls_glm_bug}}
\newcommand{\refAccuracySScorrected}[1]{\labelcref{fig_acc_t_ss_llr_bug,fig_acc_t_ss_glm_bug,fig_acc_o_ss_llr_bug,fig_acc_o_ss_glm_bug}}
\newcommand{\refAccuracyPLScorrected}[1]{\labelcref{fig_acc_t_pls_llr_bug,fig_acc_t_pls_glm_bug,fig_acc_o_pls_llr_bug,fig_acc_o_pls_glm_bug}}
\newcommand{\refAccuracyUncorrected}[1]{\labelcref{fig_acc_t_ss_unc,fig_acc_t_pls_unc,fig_acc_o_ss_unc,fig_acc_o_pls_unc}}
\newcommand{\refAccuracyCorrected}[1]{\labelcref{fig_acc_t_ss_llr_bug,fig_acc_t_ss_glm_bug,fig_acc_t_pls_llr_bug,fig_acc_t_pls_glm_bug,fig_acc_o_ss_llr_bug,fig_acc_o_ss_glm_bug,fig_acc_o_pls_llr_bug,fig_acc_o_pls_glm_bug}}
\newcommand{\refAccuracyGLM}[1]{\labelcref{fig_acc_t_ss_glm_bug,fig_acc_t_pls_glm_bug,fig_acc_o_ss_glm_bug,fig_acc_o_pls_glm_bug}}
\newcommand{\refAccuracyLLR}[1]{\labelcref{fig_acc_t_ss_llr_bug,fig_acc_t_pls_llr_bug,fig_acc_o_ss_llr_bug,fig_acc_o_pls_llr_bug}}
\newcommand{\refAccuracyOmega}[1]{\labelcref{fig_acc_o_ss_llr_bug,fig_acc_o_ss_glm_bug,fig_acc_o_pls_llr_bug,fig_acc_o_pls_glm_bug}}
\newcommand{\refAccuracyOmegaUncorrected}[1]{\labelcref{fig_acc_o_ss_unc,fig_acc_o_pls_unc}}
\newcommand{\refAccuracyOmegaCorrected}[1]{\labelcref{fig_acc_o_ss_llr_bug,fig_acc_o_ss_glm_bug,fig_acc_o_pls_llr_bug,fig_acc_o_pls_glm_bug}}
\newcommand{\refAccuracyTime}[1]{\labelcref{fig_acc_t_ss_llr_bug,fig_acc_t_ss_glm_bug,fig_acc_t_pls_llr_bug,fig_acc_t_pls_glm_bug}}

\newcommand{\tn}{\tabularnewline}

\newcommand{\widthFigure}[5]{\begin{figure}[htbp]
\begin{center}
    \includegraphics[width=#1\textwidth]{#2}
    \captionsetup{#3}
    \caption{#4}
    \label{#5}
    \end{center}
    \end{figure}}

\newcommand{\heightFigure}[5]{\begin{figure}[htbp]
\begin{center}
    \includegraphics[height=#1]{#2}
    \captionsetup{#3}
    \caption{#4}
    \label{#5}
    \end{center}
    \end{figure}}

\newcommand{\mFigure}[3]{\widthFigure{1.0}{#1}{listformat=figList}{#2}{#3}\clearpage}
\newcommand{\siFigure}[3]{\widthFigure{1.0}{#1}{name=Figure S, labelformat=noSpace, listformat=sFigList}{#2}{#3}\clearpage}


%%%%%%%%%%%%%%%%%%%%%%%%%%%%%%%%%%%%%%%%%%%%%%%%%%%%%%%%%%%%
%%%%%%%%%%%%%%%%%%%%%%%%%%%%%%%%%%%%%%%%%%%%%%%%%%%%%%%%%%%%

\begin{document}
\doublespacing
\raggedright
\setlength{\parindent}{0.5in}
\begin{linenumbers}

\begin{titlepage}
    \begin{flushleft}
        \sffamily

        \MakeUppercase{\large\bfseries Why you should not fix a biased
        model-choice method by adding an additional dimension of model choice:
        A reply to Hickerson et al.}

        \vspace{12pt}
        \textbf{Running head:} \MakeUppercase{Approximate Bayesian model
        choice}

        \vspace{12pt}
        Jamie R. Oaks$^{1,2,5}$, Jeet Sukumaran$^{1,2}$, Jacob A.
        Esselstyn$^{3}$, Charles W. Linkem$^{1,2}$, Cameron D.
        Siler$^{4}$, Mark T. Holder$^{2}$ and Rafe M. Brown$^{1,2}$

        \bigskip
        $^1$\emph{Biodiversity Institute\\
            $^2$Department of Ecology and Evolutionary Biology\\ 
            University of Kansas\\
            %1345 Jayhawk Blvd\\
            Lawrence, KS 66045\\
            USA}\\[.1in]
        $^3$\emph{Biology Department\\
            McMaster University\\
            % Life Sciences Building Rm 328\\
            Hamilton, Ontario L8S4K1\\
            Canada}\\[.1in]
        $^4$\emph{Department of Biology\\
            University of South Dakota\\
            Vermillion, SD 57069\\
            USA}\\[.1in]
        $^5$\emph{Corresponding author} (\href{mailto:joaks1@ku.edu}{\tt
        joaks1@ku.edu})\\

    \end{flushleft}
\end{titlepage}

{\sffamily
    \noindent\textbf{ABSTRACT} \\
    \noindent Abstract here \ldots

    \vspace{12pt}
    \noindent\textbf{KEY WORDS: } 
}

\newpage
\noindent Recently, this journal has served as a venue for discussion of the
potential pitfalls of approximate Bayesian methods of comparative
phylogeographical model choice.
\citet{Oaks2012} published their findings that the method \msb can often be
biased towards inferring models of temporally clustered divergence times
among taxon pairs.
\citet{Hickerson2013} has published a response to this paper where they
present a model-averaging approach that they conclude circumvents the poor
behavior of the method revealved by \citet{Oaks2012}.
Both of these papers are largely in agreement.
In fact, \citet{Hickerson2013} reiterate a lot of the discussion of
\citet{Oaks2012} regarding the impact of broad uniform priors on Bayesian model
choice.
The main differences in their perspectives are centered around
(1) the means by which broad uniform priors cause the poor behavior of \msb,
and
(2) how to potentially ameliorate this issue.

\citet{Oaks2012} conclude the primary mechanism by which broad priors cause the
poor behavior of the method is likely the low marginal likelihoods of
parameter-rich models integrated over vast parameter space with low probability
of producing the data, yet relatively high prior density \citep[this is often
referred to as Lindley's paradox;][]{Lindley1957}.
Note, this suggests the bias is extrinsic to \msb, and the numerical
approximation machinery of the method could be sound.
\citet{Hickerson2013} take a more pessimistic view of the bias, and suggest it
is intrinsic to \msb, i.e., the method's rejection algorithm is
inefficient and will be increasingly biased as the overall space of the model
increases, either as a function of the number of taxon pairs or the width of
the uniform priors on nuisance parameters.
They support their position by giving a probabilistic argument that focuses on
only one of the three prior models used by \citet{Oaks2012}.
We show that this argument is based on dubious assumptions, and does not
explain the bias of the method found in several analyses of \citet{Oaks2012}.
We reiterate several of the findings of \citet{Oaks2012} as well as present
results of additional analyses, which strongly suggest that Lindley's paradox
is playing a larger role in the poor behavior of the method.

\citet{Oaks2012} suggest that more cogent prior probability distributions on
divergence models and nuisance parameters could mitigate the effect of
Lindley's paradox.
They also layout a set of simulation-based procedures for determining power,
accuracy, and robustness of the method given a dataset, and recommend any
application of \msb should be accompanied by such procedures, especially
if any biological conclusions are going to drawn from the results.
\citet{Hickerson2013} present an approximate Bayesian model-averaging approach
for accommodating uncertainty in selecting among empirically guided priors, and
champion the method as a means of avoiding the pitfalls raised by
\citet{Oaks2012}.
Unfortunately, we show that fundamental errors in this approach render some of
their results invalid and leave the remaining results difficult to interpret.
Furthermore, we follow the advice of \citet{Oaks2012} and present
simulation-based assessments of Hickerson et al.'s \citeyear{Hickerson2013}
approach.
Our results demonstrate that the method is biased and dangerous.
The approach merely provides another means by which the model can ``escape''
large parameter space, and the bias towards models with smaller space still
remains.
This bias can be dangerous and often excludes the true parameter space due to
the use of uniform priors.
Furthermore we discuss the potential theoretical and practical problems of
empirical Bayesian model choice.

\section*{Their power analysis}
\citet{Hickerson2013} present a power analysis in which they demonstrate that
\msb has the power to infer multiple divergence events when divergence times
are random over hundreds of thousands of generations (rather than millions
as demonstrated by \citet{Oaks2012}.
It is not surprising to find that within certain parameter space the method
has increased power to detect temporal variation in divergences over narrower
time windows than millions of generations.
However, what is important to consider is whether those conditions are relevant
to real world applications of the method.
\citet{Oaks2012} explore the behavior of the method under three divergence time
priors as narrow as 0--5 coalescent units, which is quite narrow considering
that this expresses the prior belief that all 22 taxon pairs diverged within
this window.
\citet{Hickerson2013} limit their power analysis to a single prior of 0--1
coalescent unit.
It seems that being 100\% certain a priori that all taxa under comparison
diverged within the last coalescent unit is not generally applicable to most
empirical systems.
Furthermore, even when such extensive prior information is available to only be
able to detect multiple divergences on the scale of hundreds of thousands of
generations does not seem very ``powerful.''

Also, \citet{Hickerson2013} present information only on the extreme case of a
single divergence event.
As discussed in \citet{Oaks2012}, it seems dubious to consider the estimation
of two divergence events as a success when divergence is random over hundreds
of thousands of generations.
Unlike the results of \citet{Oaks2012} it is impossible to assess the
performance of \msb based on the results presented by \cite{Hickerson2013}
other than to know whether or not it inferred the worst possible scenario of a
single event.
In an empirical system such as the Philippines, where island fragmentation
has occurred at least X times over the past several millions of years, an
estimate where 22 taxa share even a handful of divergence events would be
of biogeographic interest.
\citet{Hickerson2013} seem to suggest in their response that \msb should only
be used as a binary estimator; i.e., was there one event or more?
This is not consistent with how \msb has been presented, used, and interpreted
in the past.


\section*{Differing utilities of \numt{} and \vmratio{} in \msb}
We preface this section with a clarification of an issue that has
been confused throughout the \msb literature.
The mean (\meant{}), variance (\vart{}), and dispersion index
($\vart{}/\meant{} = \vmratio{}$) of divergence times are \emph{not} parameters
of the model implemented in \msb.
Rather they are statistics that summarize parameters of the model.
This is in contrast to \numt{}, which is a parameter of the model.

\citet{Hickerson2013} strongly argue that \vmratio{} is a better estimator than
\numt{}, but the logic behind the argument is dubious.
They present a plot of \numt{} against \vmratio{} (Fig.~S1) which is simply a
plot of sample size versus variance.
After showing that \vmratio{} has essentially no information about the
number of divergences, they conclude it is more informative and biogeographically
relevant.
We struggle to follow this logic.
Certainly the maximum information is contained within the divergence time
vector that \vmratio{} is summarizing.
The temporal information contained in this vector coupled with its cardinality
(\numt{}) is certainly more informative than its summary.
The dispersion index is not a sufficient statistic for this vector.

They also argue that ``\msb can estimate \vmratio{} much better than \numt{}.''
However, the primary objective of \msb is to estimate the posterior probability
of divergence models.
\citet{Oaks2012} demonstrate that even when all assumptions of the model are
met, \vmratio is a terrible model-choice estimator (see plots B, D \& F of
Figure 4), whereas \numt{} performs better.
Furthermore, \vmratio{} is limited to estimating the probability of only a
single divergence model (the one divergence model), and thus its utility for
model choice is extremely limited.



\section*{Acknowledgments}

% \bibliography{../bib/references}
\bibliography{references}

%% LIST OF FIGURES %%%%%%%%%%%%%%%%%%%%%%%%%%
\newpage
\singlespacing

\renewcommand\listfigurename{Figure Captions}
\cftsetindents{fig}{0cm}{2.2cm}
\renewcommand\cftdotsep{\cftnodots}
\setlength\cftbeforefigskip{10pt}
\cftpagenumbersoff{fig}
\listoffigures


\end{linenumbers}

%% TABLES %%%%%%%%%%%%%%%%%%%%%%%%%%%%%%%%
\newpage
\singlespacing


\clearpage

%% FIGURES %%%%%%%%%%%%%%%%%%%%%%%%%%%%%%%%
\newpage

%:FIGURE-saturation plot
\mFigure{../../saturation/saturation-plot.pdf}{
    Saturation plot \ldots
}{figSaturationPlot}

\mFigure{../../response-redux/results/hickerson/pymsbayes-results/mean_by_dispersion.pdf}{
    Results of reanalysis.
}{figJointPosterior}

\mFigure{../../response-redux/results/sampling-error/pymsbayes-results/omega_over_sampling.pdf}{
    Sampling error \ldots
}{figSamplingError}

%% Exclusion simulation results
\mFigure{../../model-choice/results/m1-1-sim/pymsbayes-results/num_tau_excluded.pdf}{
    Histogram of the number of \divt{} parameters excluded.
}{figExclusionSimTau}

\mFigure{../../model-choice/results/m1-1-sim/pymsbayes-results/prob_of_exclusion.pdf}{
    Histogram of the posterior probability of excluding the truth.
}{figExclusionSimProb}

%% Power results
\mFigure{../../model-choice/results/power-1/pymsbayes-results/plots/power_accuracy_omega_median.pdf}{
    Histograms of the posterior probability that $\vmratio{} < 0.01$.
}{figPowerAccOmegaMedian}

\mFigure{../../model-choice/results/power-1/pymsbayes-results/plots/power_accuracy_omega_mode_glm.pdf}{
    Histograms of the posterior probability that $\vmratio{} < 0.01$.
}{figPowerAccOmegaModeGLM}

\mFigure{../../model-choice/results/power-1/pymsbayes-results/plots/power_psi_mode.pdf}{
    Histograms of \numt{} estimates.
}{figPowerPsiMode}

\mFigure{../../model-choice/results/power-1/pymsbayes-results/plots/power_omega_prob.pdf}{
    Histograms of the posterior probability that $\vmratio{} < 0.01$.
}{figPowerOmegaProb}

\mFigure{../../model-choice/results/power-1/pymsbayes-results/plots/power_prob_exclusion.pdf}{
    Histograms of the posterior probability of excluding the truth.
}{figPowerProbExclusion}

\mFigure{../../model-choice/results/power-1/pymsbayes-results/plots/power_num_excluded.pdf}{
    Histograms of the number of true \divt{} parameters excluded.
}{figPowerNumExcluded}

%% Validation results
\mFigure{../../model-choice/results/validation/results/pymsbayes-results/plots/mc_behavior.pdf}{
    True versus estimated probability of a single divergence event.
}{figValidationMCBehvior}

%%%%%%%%%%%%%%%%%%%%%%%%%%%%%%%%%%%%%%%%%%%%%%%%%%%%%%%%%%%%%%%%%%
%% SUPPORTING INFO %%%%%%%%%%%%%%%%%%%%%%%%%%%%%%%%%%%%%%%%%%%%%%%
\setcounter{figure}{0}
\setcounter{table}{0}
\setcounter{page}{1}

\singlespacing

\renewcommand{\refname}{\noindent\MakeUppercase{\LARGE\sffamily\upshape supporting information}}

% PUT MAIN TEXT CITATION HERE
% \begin{thebibliography}{1}
% \providecommand{\natexlab}[1]{#1}
% \providecommand{\url}[1]{\texttt{#1}}
% \providecommand{\urlprefix}{URL }

% \bibitem

% \end{thebibliography}


%% SUPPL TABLES %%%%%%%%%%%%%%%%%%%%%%%%%%%%
%\setcounter{table}{0}


\end{document}

