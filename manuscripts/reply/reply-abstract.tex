Establishing that a set of population splitting events occurred at the same
time can be a potentially persuasive argument that the populations were
affected by the same geographic event.
Recently, \citet{Oaks2012} used an approximate-Bayesian model choice method
(\msb) to estimate the support for such a pattern of divergences across 22
pairs of taxa from the Philippines; they also studied the behavior of \msb
using computer simulations.
% \citet{Hickerson2013} responded to this work.
They found the primary inference enabled by the method is very sensitive to
prior assumptions and often erroneously supports shared divergences across
taxa.
% \citet{Oaks2012} found the method was very sensitive to prior assumptions and
% often incorrectly supported models with few divergence events shared
% across taxa.
\citet{Oaks2012} suggested this was due to the sensitivity of marginal
likelihoods of models to the uniformly distributed priors on divergence times.
To mitigate this sensitivity they suggested alternative distributions for
priors on divergence times.
They also recommended that users of the method should conduct analyses under a
variety of priors to reveal prior sensitivity and communicate which assumptions
underlie model inference.
% These results were not surprising in light of a rich statistical literature
% showing that the marginal likelihood of a model is sensitive to the
% use of vague priors.
In response, \citet{Hickerson2013} suggested the lack of robustness in \msb
analyses was caused by numerical approximation error stemming from the
inefficiency of the method's rejection algorithm under broad uniform priors on
divergence times.
They proposed a model-averaging approach that uses narrow, empirically informed
uniform priors to overcome this.
Here we demonstrate that the approach of \citet{Hickerson2013} will often
erroneously support shared divergence events and is dangerous in the sense that
the empirically-derived uniform priors often exclude from consideration the
true values of the models' parameters.
% On a more fundamental level, we question the value of adopting an empirical
% Bayesian stance for this model-choice problem.
We also show the tendency of \msb analyses to support models of shared
divergences is primarily due to the larger marginal likelihoods of these models
when they are strongly weighted by the prior assumption of uniformly
distributed divergence times; numerical approximation error is not a tenable
explanation for the bias.
% On a more fundamental level, we question the value of adopting an empirical
% Bayesian stance for this model-choice problem.
% The robust Bayesian approach of conducting analyses under a variety of priors
% can reveal prior sensitivity and communicate which assumptions underlie model
% inference.
% Furthermore, simulations provide insight into the temporal resolution of the
% method, which helps guide interpretation of results.
As predicted by fundamental principles of Bayesian model choice, more flexible
distributions can accommodate prior uncertainty about the models' parameters
without placing excessive prior weight in vast regions of parameter space with
low likelihood;
this increases the method's robustness and power to detect temporal variation
in divergences \citep{Oaks2014dpp}.
