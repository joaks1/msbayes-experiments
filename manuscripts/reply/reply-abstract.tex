% Biogeographers frequently seek to explain population and species
% differentiation on geographical phenomena.
Establishing that a set of population splitting events occurred at the same
time can be a potentially persuasive argument that the populations were
affected by the same geographic event.
% Unfortunately, estimating divergence times precisely when one lacks
% precise information about the rate of molecular evolution is a notoriously
% difficult problem in evolutionary biology.
% \citet{Huang2011} introduced an approximate-Bayesian approach, \msb, for
% estimating the probabilities of models in which multiple sets of taxa diverge
% at the same time.
% This method  does not require a common set of character data that links all
% of the taxa under investigation, but does require the relative mutation
% rates among the taxa and loci to be assumed \emph{a priori}.
Recently, \citet{Oaks2012} used an approximate-Bayesian model choice method
(\msb) to estimate the support for such a pattern of divergences across 22
pairs of taxa from the Philippines.
They also studied the behavior of \msb using computer simulations.
\citet{Oaks2012} found the method was very sensitive to prior assumptions and
had low power to detect variation in divergence times.
These results were not surprising in light of a rich statistical literature
showing that the marginal likelihood of a model is sensitive to the
use of vague priors.
Because this sensitivity to prior assumptions affects the crucial insights 
that a researcher who employs \msb seeks to gain, \citet{Oaks2012} recommended
that users of the approach should carefully assess the robustness of their 
conclusions to different priors.
According to \citet{Hickerson2013}, the lack of robustness in \msb analyses
was due to excessively broad priors on divergence times leading to 
inadequate numbers of simulation replicates.
They proposed a model-averaging approach that uses narrow, empirically
informed uniform priors.
Here we demonstrate that the approach of \citet{Hickerson2013} will often
incorrectly support shared divergence events and is dangerous in the sense that
the empirically-derived priors often exclude from consideration the true values
of the models' parameters.
On a more fundamental level, we question the value of adopting an empirical
Bayesian stance for this model-choice problem.
% , because it can mislead model
% posterior probabilities, which are inherently measures of belief in the models
% after prior knowledge is updated by the data.
The robust Bayesian approach of conducting analyses under a variety of priors
can reveal prior sensitivity and communicate which assumptions underlie model
inference.
Furthermore, simulations provide insight into the temporal resolution of the
method, which helps guide interpretation of results.
Fortunately, as predicted by fundamental principles of Bayesian model choice,
more appropriate priors on the model's parameters increase the method's
robustness and power to detect temporal variation in divergences
\citep{Oaks2014dpp}.
