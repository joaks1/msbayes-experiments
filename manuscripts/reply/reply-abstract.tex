Establishing that a set of population splitting events occurred at the same
time can be a potentially persuasive argument that the populations were
affected by the same geographic event.
Recently, \citet{Oaks2012} assessed the ability of an approximate-Bayesian
model choice method (\msb) to estimate such patterns of divergences across
taxa, to which \citet{Hickerson2013} responded.
Both papers agree that the primary inference enabled by the method is very
sensitive to prior assumptions and often erroneously supports shared
divergences across taxa when prior uncertainty about divergence times
is represented by a uniform distribution.
However, the papers differ in the best explanation and solution for this
problem.
\citet{Hickerson2013} suggested the lack of robustness in \msb analyses is
caused by numerical approximation error due to the inefficiency of the method's
rejection algorithm under broad uniform priors on divergence times
(Hypothesis~\ref{hypError}).
To overcome this, they proposed a model-averaging approach that uses narrow,
empirically informed uniform priors.
Here we demonstrate the approach of \citet{Hickerson2013} will often
erroneously support shared divergence events and is dangerous in the sense that
the empirically-derived uniform priors often exclude from consideration the
true values of the divergence-time parameters.
\citet{Oaks2012} suggested the method's behavior was due to the sensitivity of
marginal likelihoods to the uniformly distributed priors on divergence times
(Hypothesis~\ref{hypML}).
To mitigate this sensitivity, they suggested alternative probability
distributions to accommodate prior uncertainty about divergence times.
As predicted by fundamental principles of Bayesian model choice, more flexible
distributions can accommodate prior uncertainty about parameters without
placing excessive prior weight in vast regions of parameter space with low
likelihood;
this increases the method's robustness and power to detect temporal variation
in divergences \citep{Oaks2014dpp}.
We also show that the tendency of \msb analyses to support models of shared
divergences is primarily due to Hypothesis~\ref{hypML}, whereas
Hypothesis~\ref{hypError} is not a tenable explanation for the bias.
All of these papers have highlighted the importance of choosing prior
assumptions about parameters in Bayesian model selection, because they can
strongly influence the posterior probabilities of the models we seek to
estimate.
No amount of computation can rescue our inference if we choose priors on
parameters such that the exact posterior supports the wrong model of
evolutionary history.
