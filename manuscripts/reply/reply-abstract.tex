Establishing that a set of population splitting events occurred at the same
time can be a potentially persuasive argument that a common process affected
the populations.
Recently, \citet{Oaks2012} assessed the ability of an approximate-Bayesian
model choice method (\msb) to estimate such a pattern of simultaneous
divergence across taxa, to which \citet{Hickerson2013} responded.
Both papers agree that the primary inference enabled by the method is very
sensitive to prior assumptions and often erroneously supports shared
divergences across taxa when prior uncertainty about divergence times
is represented by a uniform distribution.
However, the papers differ about the best explanation and solution for this
problem.
\citet{Oaks2012} suggested the method's behavior was caused by the strong
weight of uniformly distributed priors on divergence times leading to smaller
marginal likelihoods (and thus smaller posterior probabilities) of models with
more divergence time parameters (Hypothesis~\ref{hypML}); they proposed
alternative prior probability distributions to avoid such strongly weighted
posteriors.
\citet{Hickerson2013} suggested numerical approximation error causes \msb
analyses to be biased toward models of clustered divergences because the
method's rejection algorithm is unable to adequately sample the parameter space
of richer models within reasonable computational limits when using broad
uniform priors on divergence times (Hypothesis~\ref{hypError}).
As a potential solution, they proposed a model-averaging approach that uses
narrow, empirically informed uniform priors.
Here, we use analyses of simulated and empirical data to demonstrate that the
approach of \citet{Hickerson2013} does not mitigate the method's tendency to
erroneously support models of highly clustered divergences, and is dangerous in
the sense that the empirically-derived uniform priors often exclude from
consideration the true values of the divergence-time parameters.
Our results also show that the tendency of \msb analyses to support models of
shared divergences is primarily due to Hypothesis~\ref{hypML}, whereas
Hypothesis~\ref{hypError} is an untenable explanation for the bias.
Overall, this series of papers demonstrate that if our prior assumptions place
too much weight in unlikely regions of parameter space such that the exact
posterior supports the wrong model of evolutionary history, no amount of
computation can rescue our inference. 
Fortunately, as predicted by fundamental principles of Bayesian model choice,
more flexible distributions that accommodate prior uncertainty about parameters
without placing excessive weight in vast regions of parameter space with low
likelihood increase the method's robustness and power to detect temporal
variation in divergences.
