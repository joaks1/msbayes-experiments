Establishing that a set of population splitting events occurred at the same
time can be a potentially persuasive argument that a common process affected
the populations.
Recently, \citet{Oaks2012} assessed the ability of an approximate-Bayesian
model choice method (\msb) to estimate such a pattern of simultaneous
divergence across taxa, to which \citet{Hickerson2013} responded.
Both papers agree that the primary inference enabled by the method is very
sensitive to prior assumptions and often erroneously supports shared
divergences across taxa when prior uncertainty about divergence times
is represented by a uniform distribution.
However, the papers differ about the best explanation and solution for this
problem.
\citet{Hickerson2013} suggested the lack of robustness in \msb analyses is
caused by numerical approximation error due to the inefficiency of the method's
rejection algorithm under broad uniform priors on divergence times
(Hypothesis~\ref{hypError}).
To overcome this, they proposed a model-averaging approach that uses narrow,
empirically informed uniform priors.
\citet{Oaks2012} suggested the method's behavior was due to the sensitivity of
marginal likelihoods to the uniformly distributed priors on divergence times
(Hypothesis~\ref{hypML}), and proposed alternative prior probability
distributions to mitigate this sensitivity.
As predicted by fundamental principles of Bayesian model choice, more flexible
distributions that accommodate prior uncertainty about parameters without
placing excessive weight in vast regions of parameter space with low
likelihood increase the method's robustness and power to detect temporal
variation in divergences \citep{Oaks2014dpp}.
Here, we use analyses of simulated and empirical data to show that the tendency
of \msb analyses to support models of shared divergences is primarily due to
Hypothesis~\ref{hypML}, whereas Hypothesis~\ref{hypError} is an untenable
explanation for the bias.
We also demonstrate the empirical Bayesian model-averaging approach of
\citet{Hickerson2013} will often erroneously support shared divergence events
and is dangerous in the sense that the empirically-derived uniform priors often
exclude from consideration the true values of the divergence-time parameters.
All of these papers have highlighted the importance of choosing prior
assumptions about parameters in Bayesian model selection.
If our prior assumptions place too much weight in unlikely regions of parameter
space such that the true posterior supports the wrong model of evolutionary
history, no amount of computation can rescue our inference. 
