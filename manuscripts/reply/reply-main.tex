\section{Introduction}
Biogeographers frequently seek to explain population and species
differentiation on geographical phenomena.
Establishing that a set of population splitting events occurred
at the same time can be a potentially persuasive argument that a set of taxa
were affected by the same geographic events.
% Unfortunately, estimating divergence times precisely when one lacks
% precise information about the rate of molecular evolution is a notoriously
% difficult problem in evolutionary biology.
\citet{Huang2011} introduced an approximate Bayesian approach (implemented
in the software \msb) which can be used to estimate the probabilities of
models in which multiple sets of taxa diverge at the same time. 
% This method  does not require a common set of character data that links all
% of the taxa under investigation, but does require the relative mutation
% rates among the taxa and loci to be assumed \emph{a priori}.

Recently, \citet{Oaks2012} used this model-choice framework to 
study 22 pairs of vertebrate lineages which are distributed across
the Philippines; they also studied the behavior of the \msb approach
using computer simulations.
They found the method was very sensitive to prior assumptions and suggested
this was due to the fact that the marginal likelihoods of divergence models are
sensitive to the use of uniform priors on divergence times.
\citet{Oaks2012} suggested that alternative probability distributions (e.g.,
gamma) could better represent prior uncertainty about divergence times while
mitigating the disparity of marginal likelihoods across divergence models.
In response, \citet{Hickerson2013} suggested the lack of robustness in \msb
analyses was caused by sampling error due to the inefficiency of the method's
rejection algorithm.
As a potential solution, they proposed a model-averaging approach that uses
narrow, empirically informed uniform priors.

Whereas the papers of \citet{Oaks2012} and \citet{Hickerson2013} differ about
the cause and solution, they agree on the fundamental methodological point
about the model selection performed in \msb:
% The papers of \citet{Oaks2012} and \citet{Hickerson2013} are in agreement
% on the fundamental methodological point about the model selection performed in \msb:
\begin{itemize}
   \item The use of vague priors on the divergence-time parameters can lead to
       support for models with few divergence events shared across taxa. Thus,
       the primary inference enabled by the approach is very sensitive to the
       priors on divergence times.
\end{itemize}
This is not surprising given the rich statistical literature showing that
marginal likelihoods are very sensitive to the priors used in Bayesian model
selection
\citep[e.g.,][]{Jeffreys1939,Lindley1957}.
% We know that under broad uniform priors on divergence times, models with few
% divergence parameters will have greater marginal likelihoods relative to more
% parameter-rich models.
% Accordingly, \citet{Oaks2012} suggest the primary cause of spurious support
% for models with few divergence parameters is the greater marginal likelihoods
% of these models under vague uniform priors, relative to more parameter-rich
% models.
Under vague uniform priors on divergence times, models with more
divergence-time parameters integrate over \emph{much} greater parameter space
with low probability of producing the data, yet relatively high prior density
imposed by the uniformly distributed prior.
% By using such priors on divergence times, we are essentially telling the
% method to favor models with few divergence times even if they do not
% explain the data very well.
Thus, uniform priors on divergence times can impose a strong ``penalty'' for
divergence-time parameters, creating a strong preference for simple
models that are often of interest to biogeographers, even if these models do
not explain the data very well.
% It is important to note that this is not due to estimation error
% (i.e., this phenomenon is extrinsic to the numerical approximation machinery of
% \msb).
% Rather, the \emph{true} posterior will tend to support models with few
% divergence-time parameters under broad uniform priors on divergence times.
Such posterior support for simultaneous divergence, even if correct from the
perspective of Bayesian model choice, does not provide the biogeographical
insights that a researcher who employs \msb seeks to gain.
% Ideally, the estimated posterior probability of a simultaneous-divergence model
% can be interpreted as an estimate of the probability that this model is correct.
% However, when using uniform priors on divergence times, it might be better
% interpreted as an estimate of the probability that the simultaneous-divergence
% model cannot be completely ruled out as a possible explanation of the data.
% Note, this is a statistical issue, extrinsic to \msb, and does not question the
% soundness of the numerical approximation machinery of \msb.

% Alternatively, \citet{Hickerson2013} suggest the problem is the result of a
% numerical issue intrinsic to \msb, concluding the method's rejection algorithm
% is inefficient and will be biased toward models with fewer divergence-time
% parameters, because the parameter space of these models are more densely
% sampled relative to models with more divergence-time parameters.
% We agree that sampling error is present in all numerical Bayesian estimation
% methods, however, we show here that it is not a major contributer to the
% biases found by \citet{Oaks2012}.
% We present results here of analyses that, when combined with the results
% of \citet{Oaks2012}, strongly suggest that \msb prefers models with few
% divergence-time parameters, because they have greater likelihoods when
% averaged over broad uniform priors on divergences times.

% \citet{Oaks2012} suggest alternative prior probability distributions on
% divergence-times could increase the marginal likelihoods of models with more
% divergence-time parameters and thus reduce spurious support for models of
% temporally clustered divergences.
% Alternatively, \citet{Hickerson2013} present an approach that uses empirically
% informed uniform priors and Bayesian model-averaging in an attempt to accommodate
% uncertainty in selecting priors.
% In general, we agree with the use of Bayesian model averaging to obtain a
% posterior estimate marginalized over uncertainty in prior choice.
% However, we question whether adding an additional dimension of model choice
% that samples over empirically informed uniform priors is a fruitful solution
% for a model-choice method that is highly sensitive to priors.

\change{
Here, we discuss theoretical considerations for using empirically informed
priors for Bayesian model choice.
}
We then evaluate the empirical Bayesian model-averaging approach of
\citet{Hickerson2013} as a potential solution to the biases of \msb revealed by
\citet{Oaks2012}.
In their re-analysis of the dataset of \citet{Oaks2012}, \citet{Hickerson2013}
made an error by mixing different units of time, which makes the results
presented in their response difficult to interpret (see Supporting Information
for details).
We avoid this error, but still find their approach to be biased toward
clustered divergence models.
Furthermore, the approach provides another means by which the method can
``escape'' models with large parameter space, which manifests in the preference
of models that exclude from consideration the true values of the model's
parameters.
\change{
Our results, show that it is difficult to choose a uniformly distributed prior
on divergence times that is broad enough to confidently contain the true values
of parameters while being narrow enough to avoid spurious support of models
with less parameter space.
}
We also explore the predictions of the alternative causes of this problem
proposed by \citet{Oaks2012} and \citet{Hickerson2013}.
Our finings suggest the primary cause is the disparity in marginal likelihoods
across possible divergence models imposed by the uniform priors on divergence
times.
Furthermore, we show that sampling error is not a major contributer to the
biases found by \citet{Oaks2012}.
Fortunately, \citet{Oaks2014dpp} has introduced a method that uses more
flexible probability distributions to accommodate prior uncertainty in
divergence times without inhibiting the marginal likelihoods of models with
more divergence-time parameters, which increases the method's robustness and
power to detect temporal variation in divergences \citep{Oaks2014dpp}.


% \msb \citep{Huang2011} uses approximate Bayesian computation (ABC) to estimate
% the distribution of divergence times among co-distributed pairs of taxa.
% It approximates a posterior probability over models that range from a single
% divergence-time parameter (i.e., simultaneous divergence of all pairs of taxa)
% to the fully generalized model in which each pair of taxa diverged at a
% unique time.
% % A full description of the model is given in \citet{Oaks2012}.
% % Frequently, the exact values of the divergence times are hard 
% % to determine because researchers typically lack precise knowledge 
% % of the rate at which substitutions occur.
% % The exact divergence time is therefore often viewed as a nuisance parameter,
% % and researchers focus on the number of divergence events.
% The papers of \citet{Oaks2012} and \citet{Hickerson2013} are in agreement
% on the fundamental methodological point about the model selection performed in \msb:
% \begin{itemize}
%    \item The use of vague priors on the divergence-time parameters can lead to
%        support for models with few divergence events shared across taxa. Thus,
%        the primary inference enabled by the approach is very sensitive to the
%        priors on divergence times.
% \end{itemize}
% This is not surprising given the rich statistical literature that shows that
% marginal likelihoods are very sensitive to the priors used in Bayesian model
% selection
% \citep[e.g.,][]{Jeffreys1939,Lindley1957}.
% Accordingly, \citet{Oaks2012} suggest the primary cause of spurious support
% for models with few divergence parameters is the greater marginal likelihoods
% of these models under vague uniform priors, relative to more parameter-rich
% models.

% \change{
% Models with more divergence-time parameters integrate over \emph{much} greater
% parameter space with low probability of producing the data, yet relatively high
% prior density imposed by the uniformly distributed prior.
% % By using such priors on divergence times, we are essentially telling the
% % method to favor models with few divergence times even if they do not
% % explain the data very well.
% Thus, uniform priors on divergence times can impose a strong ``penalty'' for
% divergence-time parameters, creating a strong preference for simple
% models that are often of interest to biogeographers, even if these models do
% not explain the data very well.
% It is important to note that this is not due to estimation error
% (i.e., this phenomenon is extrinsic to the numerical approximation machinery of
% \msb).
% Rather, the \emph{true} posterior will tend to support models with few
% divergence-time parameters under broad uniform priors on divergence times.
% Such posterior support for simultaneous divergence, even if correct from the
% perspective of Bayesian model choice, does not provide the biogeographical
% insights that a researcher who employs \msb seeks to gain.
% % Ideally, the estimated posterior probability of a simultaneous-divergence model
% % can be interpreted as an estimate of the probability that this model is correct.
% % However, when using uniform priors on divergence times, it might be better
% % interpreted as an estimate of the probability that the simultaneous-divergence
% % model cannot be completely ruled out as a possible explanation of the data.
% % Note, this is a statistical issue, extrinsic to \msb, and does not question the
% % soundness of the numerical approximation machinery of \msb.
% }

% \citet{Hickerson2013} suggest the problem is the result of a numerical issue
% intrinsic to \msb, concluding the method's rejection algorithm is inefficient
% and will be biased toward models with fewer divergence-time parameters, because
% the parameter space of these models are more densely sampled relative to models
% with more divergence-time parameters.
% We agree that sampling error is present in all numerical Bayesian estimation
% methods, however, we show here that it is not a major contributer to the
% biases found by \citet{Oaks2012}.
% % We present results here of analyses that, when combined with the results
% % of \citet{Oaks2012}, strongly suggest that \msb prefers models with few
% % divergence-time parameters, because they have greater likelihoods when
% % averaged over broad uniform priors on divergences times.

% \citet{Oaks2012} suggest alternative prior probability distributions on
% divergence-times could increase the marginal likelihoods of models with more
% divergence-time parameters and thus reduce spurious support for models of
% temporally clustered divergences.
% Alternatively, \citet{Hickerson2013} present an approach that uses empirically
% informed uniform priors and Bayesian model-averaging in an attempt to accommodate
% uncertainty in selecting priors.
% In general, we agree with the use of Bayesian model averaging to obtain a
% posterior estimate marginalized over uncertainty in prior choice.
% However, we question whether adding an additional dimension of model choice
% that samples over empirically informed uniform priors is a fruitful solution
% for a model-choice method that is highly sensitive to priors.

% \change{
% In this paper, we discuss theoretical and practical considerations of using
% empirically informed priors for Bayesian model choice.
% }
% Furthermore, we evaluate the empirical Bayesian model-averaging approach of
% \citet{Hickerson2013} as a potential solution to the biases of \msb revealed by
% \citet{Oaks2012}.
% In their re-analysis of the dataset of \citet{Oaks2012}, \citet{Hickerson2013}
% made an error by mixing different units of time, which makes the results
% presented in their response difficult to interpret (see Supporting Information
% for details).
% Here, we avoid this error, but still find their approach to be biased toward
% clustered divergence models.
% Furthermore, the approach provides another means by which the method can
% ``escape'' models with large parameter space, which manifests in the preference
% of models that exclude from consideration the true values of the model's
% parameters.
% \change{
% Our results, show that it is difficult to choose a uniformly distributed prior
% on divergence times that is broad enough to confidently contain the true values
% of parameters while being narrow enough to avoid spurious support of models
% with less parameter space.
% }
% However, recent work shows more flexible probability distributions without a
% hard upper bound (e.g., gamma) can accommodate prior uncertainty in divergence
% times without inhibiting the marginal likelihoods of models with more
% divergence-time parameters, and as a result, increase the method's power to
% detect temporal variation in divergences
% \citep{Oaks2014dpp}.


\section{The potential implications of empirical Bayesian model choice}
\change{
% The results of \citet{Oaks2012} and \citet{Hickerson2013} show the method is
% highly sensitive to the prior on divergence times.
\citet{Oaks2012} found that \msb would incorrectly support models of temporally
clustered divergences even when using a prior on divergence times that was
informed by the data being analyzed.
The main argument of \citet{Hickerson2013} is that the priors used by
\citet{Oaks2012} were not informative enough.
}
%
% \citet{Hickerson2013} repeatedly refer to the prior distributions used by
% \citet{Oaks2012} for divergence-time parameters as ``poorly selected.''
% However, this is misleading, as \citet{Oaks2012} discuss in detail how they
% selected their prior to reflect the large amount of uncertainty about the
% timing of divergences across all 22 of the taxa in their study (see section
% ``Specifying and simulating the joint prior'' of \citet{Oaks2012}).
% \citet{Oaks2012} further discuss how the use of uniform prior distributions in
% \msb for many of the model's parameters requires investigators to select broad
% priors to avoid excluding the truth \textit{a priori}, resulting in too much
% density in improbable regions of parameter space.
%
% \citet{Hickerson2013} suggest that \citet{Oaks2012} should have used their
% data to inform their prior on divergence times (i.e., an empirical Bayesian
% approach).
% However, \citet{Oaks2012} did use empirically informed priors to mimic
% empirical situations in which a large amount of prior information is available;
% they did so to assess the prior sensitivity of the method and to determine if
% uniform priors in such situations are narrow enough to avoid spurious support
% of models with few divergence-time parameters.
% The results of \citet{Oaks2012} and \citet{Hickerson2013} show the method is
% highly sensitive to the prior on divergence times.
% Furthermore, \citet{Oaks2012} found the method remained biased toward clustered
% divergences under the empirically informed priors.
%
% The main argument of \citet{Hickerson2013} is that the priors used by
% \citet{Oaks2012} were not informative enough.
They suggest a very narrow, highly informed uniform prior on divergence times
is necessary to avoid the method's preference for models with few
divergence-time parameters.
Such an empirical Bayesian approach to model selection raises some theoretical
and practical concerns, some of which were discussed by \citet{Oaks2012} (see
the last paragraph of ``Assessing prior sensitivity of \msb'' in
\citet{Oaks2012}); we expand on this here.

\subsection{Theoretical implications of empirical priors for Bayesian model
choice}
\begin{linenomath}
Bayesian inference is a method of inductive learning in which Bayes' rule is
used to update our beliefs about a model $M$ as new information becomes
available.
If we let \allParameterValues represent the set of all possible parameter
values for model $M$, we can define a prior distribution for all $\theta \in
\allParameterValues$ such that $p(\theta \given M)$ describes our belief that
any given \myTheta{} is the true value of the parameter.
If we let \allDatasets represent all possible datasets then we can 
define a sampling model for all $\theta \in
\allParameterValues$ and $\alignment{}{} \in \allDatasets$ such that
$p(\alignment{}{} | \theta, M)$ measures our belief that any dataset \alignment{}{}
will be generated by any state \myTheta{} of model $M$.
After collecting a new dataset \alignment{i}{}, we can use Bayes' rule to
calculate the posterior distribution
\begin{equation}
    p(\myTheta{} \given \alignment{i}{}, M) = \frac{p(\alignment{i}{} \given
    \myTheta{}, M)p(\myTheta{} \given M)}{p(\alignment{i}{} \given M)},
    \label{eq:bayesrule}
\end{equation}
as a measure of our beliefs after seeing the new information, where
\begin{equation}
    p(\alignment{i}{} \given M) = \int_{\myTheta{}} p(\alignment{i}{} \given
    \myTheta{}, M)p(\myTheta{} \given M) d\myTheta{}
    \label{eq:marginallikelhiood}
\end{equation}
is the marginal likelihood of the model.
\end{linenomath}

This is an elegant method of updating our beliefs as data are accumulated.
However, this all hinges on the fact that the prior ($p(\myTheta{} \given M)$)
is defined for all possible parameter values independently of the new data
being analyzed.
Any other datasets or external information can safely be used to inform our
beliefs about $p(\myTheta{} \given M)$.
However, if the same data are used to both inform the prior and calculate the
posterior, the prior becomes conditional on the data, and Bayes' rule breaks
down.

Thus, empirical Bayes methods have an uncertain theoretical basis and
do not yield a valid posterior distribution from Bayes' rule \citep[e.g.,
empirical Bayesian estimates of the posterior are often too narrow, off-center,
and incorrectly shaped;][]{Morris1983,Laird1987,Carlin1990,Efron2013}.
This is not to say that empirical Bayesian approaches are not useful.
Empirical Bayes is a well-studied branch of Bayesian statistics that has given
rise to many inference methods that provide means of parameter estimation that
often exhibit favorable frequentist properties.
% Furthermore, many post-hoc correction methods have been developed for
% estimating confidence-intervals from empirical Bayes estimates of
% distributions that often exhibit well-behaved frequentist coverage
% probabilities
\citep{Morris1983,Laird1987,Laird1989, Carlin1990,Hwang2009}.

\begin{linenomath}
Whereas empirical Bayes approaches can provide powerful methods for parameter
estimation, a theoretical justification for empirical Bayes approaches to model
choice is questionable.
In Bayesian model choice, the primary goal is not to estimate parameters, but
to estimate the relative probabilities of candidate models.
In a simple example where two candidate models, $M_1$ and $M_2$, are being
compared, the goal is to estimate the posterior probabilities of these two
models.
Again, we can use Bayes' rule to calculate this as
\begin{equation}
    p(M_1 \given \alignment{i}{}) = \frac{p(\alignment{i}{} \given
    M_1)p(M_1)}{p(\alignment{i}{} \given M_1)p(M_1) + p(\alignment{i}{} \given
    M_2)p(M_2)}.
    \label{eq:bayesmodelchoice}
\end{equation}
By comparing Equations \ref{eq:bayesrule} and \ref{eq:bayesmodelchoice}, we
see fundamental differences between Bayesian parameter estimation and
model choice.
In Equation \ref{eq:bayesrule}, we see that the posterior density of any
state $\myTheta{}$ of the model, is the prior density updated by the
probability of the data given the state $\myTheta{}$ (the likelihood of
$\myTheta{}$).
The integral over the entire parameter space of the model, which is defined by
the priors, only appears as a normalizing constant in the denominator.
Thus, as long as the prior distribution contains the values of $\myTheta{}$
that maximize the probability of the data (i.e., the maximum likelihood) and
the data are strongly informative relative to the prior, the values of the
parameters that maximize the posterior distribution will be relatively robust
to prior choice, even if the posterior distribution is technically incorrect
due to using the data to inform the priors.
By giving the data more weight relative to the prior, the goal is that the
posterior distribution will be peaked near the true value(s) of the model's
parameter(s).
% This is why empirical Bayes can work well for estimating the values of model
% parameters.
\end{linenomath}

% Unlike parameter estimates, posterior probabilities represent a summary of the
% entire posterior distribution.
% Thus, given that empirical Bayesian posterior distributions are not accurate,
% there is no guarantee regarding the accuracy of probabilities summarized from
% them.

However, if we look at Equation~\ref{eq:bayesmodelchoice}, we see that in
Bayesian model choice it is now the \emph{marginal} likelihood of a model that
updates the prior to yield the model's posterior probability.
Thus the integral over the entire parameter space of the likelihood weighted by
the prior probability density is no longer a normalizing constant, but rather
is what informs the posterior probability of that model from the data.
As a result, Bayesian model choice tends to be much more sensitive to priors
than parameter estimation.
Also, unlike model parameters, the posterior probabilities of candidate
models have no clear true values.
% There is no such justification for empirical Bayesian model choice, because
% there are no true values for the model probabilities being estimated.
Model posterior probabilities are inherently measures of our belief in the
models after our prior beliefs are updated by the data being analyzed.
This complicates the meaning of model posterior probabilities when Bayes' rule
is violated by informing priors with the same data to be analyzed.
By using the data twice, we fail to account for prior uncertainty and mislead
our posterior beliefs in the models being compared; we will be over confident
in some models and under confident in others.

Another important difference of Bayesian model choice illustrated by
Equation~\ref{eq:bayesmodelchoice} is that the value of interest, the
posterior probability of a model, is not a function of \myTheta{}, because
it is integrated out of the marginal likelihoods of the candidate models.
Thus, unlike parameter estimates, the estimated posterior probability of a
model is a single value (rather than a distribution) lacking a measure of
posterior uncertainty.
% In some simple situations we might suspect that one model is correct and the
% rest are wrong, but in most applications, we know that all the models under
% consideration are oversimplifications of reality.

Nonetheless, empirical Bayesian model choice does perform well for some
problems.
Particularly, in cases where large aggregate data sets are used for many
parallel model-choice problems simultaneously, pooling information to inform
priors can lead to favorable group-wise frequentist coverage across tests
\citep{Efron2008}.
However, this is far removed from the single model-choice problem of \msb.
In the Supporting Information we use a simple example to help highlight the
distinctions between Bayesian parameter estimation and model choice.

\subsection{Practical concerns about empirically informed uniform priors for
    Bayesian model choice}
In addition to the theoretical concerns discussed above, there are practical
problems with using narrow, empirically informed, uniform priors.
% for a method that has already been shown to be biased toward models with less
% parameter space.
The results of Hickerson et al.'s (\citeyear{Hickerson2013}) reanalysis of the
Philippine dataset strongly favored models with the narrowest, empirically
informed prior on divergence time, and thus their model-averaged posterior
estimates are dominated by models $M_1$ and $M_2$ (see Table 1 of
\citet{Hickerson2013}).
This is concerning, because the narrowest \divt{} prior used by
\citet{Hickerson2013} ($\divt{} \sim U(0,0.1)$) likely excludes the true
divergence times for at least some of the Philippine taxa.
%, a major problem when using uniform priors.
\citet{Hickerson2013} set this prior to match the 95\% highest posterior
density (HPD) interval for the mean divergence time estimated under one of the
priors used by \citet{Oaks2012} (see Tables 2 and 3 of \citet{Oaks2012}).
Given this interval estimate is for the \emph{mean} divergence time across all
22 taxa, it may be inappropriate to set this as the limit on the prior, because
some of the taxon pairs are expected to have diverged at times older than the
upper limit.
Furthermore, this prior is \emph{excluded} from the 95\% HPD interval estimates
of the mean divergence time under the other two priors explored by
\citet{Oaks2012} (under these priors the 95\% HPD is approximately 0.3--0.6;
see Table~6 of \citet{Oaks2012}).

\change{
Hickerson et al.'s (\citeyear{Hickerson2013}) results that strongly prefer the
narrowest priors on divergence times (their $M_1$ and $M_2$) suggest their
approach is biased toward models with less parameter space, and as a
consequence, will estimate model-averaged posteriors dominated by models that
exclude true values of model's parameters.
}
We explored this possibility in two ways.
First, we re-analyzed the Philippines dataset using the model-averaging
approach of \citet{Hickerson2013}, but set one of the prior models with a
uniform prior on divergence times that is unrealistically narrow, and almost
certainly excludes most, if not all, of the true divergence times of the 22
taxon pairs.
If small likelihoods of large models cause the method to prefer models with
less parameter space, we expect \msb will preferentially sample from this
incorrect model yielding a posterior that is incorrect (i.e., the
model-averaged posterior will be dominated by an incorrect model that excludes
the truth).
Second, we generated simulated datasets for which the divergence times are
drawn from an exponential distribution and applied the approach of
\citet{Hickerson2013} to each of them to see how often the method excludes the
truth.

\subsubsection{Re-analyses of the Philippines dataset using empirical Bayesian
model averaging}

For our re-analyses of the Philippines dataset we followed the model-averaging
approach of \citet{Hickerson2013}, but with a reduced set of prior models to
avoid their error of mixing units of time (see SI for details).
We used five prior models, all of which had priors on population sizes of
$\meanDescendantTheta{} \sim U(0.0001, 0.1)$ and $\ancestralTheta{} \sim
U(0.0001, 0.05)$.
Following \citet{Hickerson2013}, each of these models had the following
priors on divergence time parameters:
$M_1$, $\divt{} \sim U(0, 0.1)$;
$M_2$, $\divt{} \sim U(0, 1)$;
$M_3$, $\divt{} \sim U(0, 5)$;
$M_4$, $\divt{} \sim U(0, 10)$; and
$M_5$, $\divt{} \sim U(0, 20)$.
We simulated $1\e6$ random samples from each of the models for a total of
$5\e6$ prior samples.
For each model, we retained the 10,000 samples with the smallest Euclidean
distance from the observed summary statistics, standardizing the statistics
using the prior means and standard deviations of the given model.
From the remaining 50,000 samples, we then retained the 10,000 samples with the
smallest Euclidean distance from the observed summary statistics, this time
standardizing the statistics using the prior means and standard deviations
across all five models.
We then repeated this analysis twice, replacing the $M_1$ model with
$M_{1A}$ and $M_{1B}$, which differ only by having priors on divergence
times of $\divt{} \sim U(0, 0.01)$ and $\divt{} \sim U(0, 0.001)$,
respectively.
While we suspect the prior of $\divt{} \sim U(0, 0.1)$ used by
\citet{Hickerson2013} likely excludes the true divergence times of at least
some of the 22 taxa, we are nearly certain that these narrower priors are
incorrect and exclude most, if not all, of the divergence times of the
Philippine taxa.

Our results show that the model-averaging approach of \citet{Hickerson2013}
does not reduce the method's support of models with less parameter space.
Rather, the method strongly prefers the prior model with the narrowest
distribution on divergence times across all three of our analyses, even when
this model is incorrect and excludes the true divergence times
of the Philippine taxa (Table~\ref{tabModelChoiceEmpirical}).

However, \citet{Hickerson2013} vetted the priors used in their model-averaging
approach via ``graphical checks,'' in which the summary statistics from 1000
random samples of each prior model are plotted along the first two orthogonal
axes of a principle component analysis (see Figure 1 of \citet{Hickerson2013}).
To determine if such prior predictive analyses would indicate the $M_{1A}$ and
$M_{1B}$ models are problematic, we performed these graphical checks on our
prior models.
Unfortunately, these prior predictive checks provide no warning that these
priors are too narrow (Figure~S\ref{figPCA}).
Rather, the graphs suggest these incorrect priors are ``better fit''
(Figure~S\ref{figPCA}A--C) than the valid priors similar to those used by
\citet{Oaks2012} (Figure~S\ref{figPCA}D--F).

% Given that the results of \citet{Hickerson2013} strongly prefer the models with
% the narrowest prior on divergence times, it seems quite likely that their
% model-averaged results are dominated by models that exclude true divergence
% times, making their results difficult to interpret.


\subsubsection{Simulation-based assessment of Hickerson et al.'s
    (\citeyear{Hickerson2013}) model averaging over empirical priors}

To better quantify the propensity of Hickerson et al.'s
(\citeyear{Hickerson2013}) approach to exclude the truth, we simulated 1000
datasets in which the divergence times for the 22-population pairs are drawn
randomly from an exponential distribution with a mean of 0.5 ($\divt{} \sim
Exp(2)$).
All other parameters were identically distributed as the $M_1$--$M_5$ models
(Table~\ref{tabModelChoiceEmpirical}).
We then repeated the analysis described above using $1\e6$ random samples from
each of the prior models $M_1$, $M_2$, $M_3$, $M_4$, and $M_5$, retaining 1000
posterior samples for each of the 1000 simulated datasets.

For each simulation replicate, we estimated the Bayes factor in favor
of excluding the truth as the ratio of the posterior to prior odds of
excluding the true value of at least one parameter.
Whenever the Bayes factor preferred a model excluding the truth we counted the
number of the 22 true divergence times that were excluded by the preferred
model.
Our results show that the model-averaging approach of \citet{Hickerson2013}
favors a model that excludes the true values of parameters in 97\% of the
replicates (90\% with GLM-regression adjustment), excluding up to 21 of the 22
true divergence times (Figure~\ref{figExclusionSimTau}).
We also used the mode estimate of the preferred model for each replicate to
estimate the number of true parameter values excluded, which produced nearly
identical results.
Importantly, the posterior probability of excluding at least one true parameter
value is very high in nearly all of the replicates
(Figure~\ref{figExclusionSimProb}).
Using a Bayes factor of greater than 10 as a criterion for strong support, 66\%
of the replicates (87\% with GLM-regression adjustment) strongly support the
exclusion of true values (Figure~\ref{figExclusionSimProb}).

The results of the preceding empirical and simulation-based analyses clearly
demonstrate the risk of using narrow, empirically guided uniform priors in a
Bayesian model-averaging framework.
The consequence of this approach is obtaining a model-averaged posterior
estimate that is heavily weighted toward incorrect models that exclude true
values of model parameters.


\subsection{Additional thoughts on empirical priors in Bayesian model choice}
The preceding sections are not a general critique of Bayesian model averaging.
Rather, model averaging can provide an elegant way of incorporating
model uncertainty in Bayesian inference.
However, when averaging over models with narrow and broad uniform priors on a
parameter with a likelihood density that is not expected to be uniformly
distributed, the posterior can be dominated by models that exclude from
consideration the true values of parameters due to the larger marginal
likelihoods of the models that are integrated over less space with high prior
weight yet low likelihood.
\change{
When limited to uniformly distributed priors, the alternative to priors that
reflect prior uncertainty is to risk excluding the true values one seeks to
estimate, as shown above.
However, continuous distributions more appropriate as priors for positive
real-valued parameters have been shown to greatly reduce spurious support for
clustered divergence models while allowing prior uncertainty to be accommodated
\citep{Oaks2014dpp}.
}

% Given the theoretical and practical issues with empirical Bayes approaches to
% Bayesian model choice, it is clear why one should use caution before overly
% criticizing an investigator's choice of priors \emph{after} having seen the
% resulting posterior.
% As discussed by \citet{Oaks2012}, prior to analyzing the data, there was a
% large amount of uncertainty regarding the divergence times of the 22 population
% pairs under study.
% Two of these pairs represent distinct species, and the taxonomy of many groups
% in the Philippines has repeatedly been shown to mask deeply divergent lineages
% \citep{RafeDiesmosAlcala2008,Linkem2010,Siler2010,Welton2010,Siler2011HerpMonographs,
% Siler2011,Siler2012,RafeStuart2012,LinkemBrown2013,Rafe2013AREES,Siler2014kikuchii}.




\section{Assessing the power of the model-averaging approach of
    \citet{Hickerson2013}}
While our results above clearly demonstrate the risks inherent to the empirical
Bayesian model-choice approach used by \citet{Hickerson2013}, one could justify
such risk if the method does indeed increase the power of the method and thus
decrease bias toward clustered divergence models.
We assess this possibility using simulation-based analyses.
Following \citet{Oaks2012}, we simulated 1000 datasets with \divt{} for each of
the 22 population pairs randomly drawn from a uniform distribution, $U(0,
\divt{max})$, where \divt{max} was set to: 0.2, 0.4, 0.6, 0.8, 1.0, and 2.0, in
\globalcoalunit generations.
All other parameters were identically distributed as the prior models.
As above, we generated a prior sample of $5\e6$ total samples from the five
prior models $M_1$, $M_2$, $M_3$, $M_4$, and $M_5$
(Table~\ref{tabModelChoiceEmpirical}).
For each simulated dataset, we approximated the posterior
by retaining 1000 samples from the prior with the smallest Euclidean distance
from the true summary statistics, as described above.
In total, we analyzed 6000 replicate datasets, retaining 1000 model-averaged
posterior samples for each of them.

We find that the approach of \citet{Hickerson2013} struggles to estimate the
variance of divergence times (\vmratio{}) across most of the \divt{max}
we simulated, whether evaluating the unadjusted
(Figure~S\ref{figPowerAccuracy}A--F) or GLM-adjusted
(Figure~S\ref{figPowerAccuracy}G--L) posterior estimates.
The method only estimates \vmratio{} relatively well when the simulated
distribution of divergence times is identical to one of the prior models
(Figures~S\ref{figPowerAccuracy}E\&K).
This is consistent with the findings of \citet{Oaks2012}  and
\citet{Hickerson2013} that \msb is highly sensitive to the prior distribution
deviating from the true, underlying distribution of the data.

Furthermore, our results demonstrate that the approach of \citet{Hickerson2013}
consistently infers highly clustered divergences across all the \divt{max} we
simulated (Figure~\ref{figPower4}A--D \& S\ref{figPower6}A--F).
The method is most likely to infer the extreme case of a single divergence
event shared across all 22 taxa when populations diverged randomly over the
past \globalcoalunit generations.
When divergences are random over the past $8\globalpopsize$ generations, the
most likely inference is only two divergence events, and a single
divergence is still estimated in more than 10\% of the replicates.
It is interesting to note that as \divt{max} increases, but before the
estimates are finally pulled away from $\numt{} = 1$, the distribution of
\numt{} estimates closely mirror the U-shaped prior on divergence models
used by \msb (see Figure~\ref{figPower4}E and Oaks et al.'s
(\citeyear{Oaks2012}) Figure 5B).
This suggests this U-shaped prior coupled with the small marginal likelihoods
of models with many divergence parameters, is a major cause of the method's
bias toward clustered divergence models; this is consistent with
\citet{Oaks2012} and confirmed by \citet{Oaks2014dpp}.

Looking at our simulation results in terms of the posterior probability of the
dispersion index of divergence times supporting the extreme case of one
divergence event (i.e., $p(\vmratio{} < 0.01 \given \ssSpace)$), we find the
method strongly supports one divergence in greater than 27\% of the replicates
across all the \divt{max} we simulated (Figure~\ref{figPower4}E--H \&
S\ref{figPower6}G--L);
following \citet{Hickerson2013}, we use a Bayes factor of greater than 10 as
the criterion for incorrect inference of a single divergence event.
Furthermore, there is strong support for a single divergence event in
more than 90\% of the replicates when divergences are random over the past
$2.4\globalpopsize$ generations, and more than 60\% when over the past
$3.2\globalpopsize$ generations (Figure~\ref{figPower4}E--H \&
S\ref{figPower6}G--L).

Our results show that the empirical Bayesian model-averaging approach leads to
spuriously strong support for the extreme case of a single divergence event
when populations diverged randomly over the last $8\globalpopsize$ generations.
To put this on the scale roughly consistent with a vertebrate mitochondrial
locus, assuming a mutation rate of $2\e{-8}$ per site per generation, this
translates to 5 million generations.
Assuming a mutation rate consistent with nuclear loci of $1\e{-9}$, this is 100
million generations.
% Given the empirically Bayesian model-averaging approach of \citet{Hickerson2013}
% lacks power to detect random variation in divergence times over such
% timescales, it is difficult to justify the risk of the approach to exclude
% regions of parameter space containing the truth.

Also, the results of our power analyses further demonstrate the propensity of
Hickerson et al.'s (\citeyear{Hickerson2013}) approach to exclude true
parameter values.
Across all but one of the \divt{max} we simulated, in a large proportion of
replicates the method favors a model that excludes the truth, and across many
of the \divt{max} the preferred model will exclude a large proportion of the
true divergence-time values (Figure~\ref{figPowerExclusion4}A--D \&
S\ref{figPowerExclusion6}A--F).
Importantly, the posterior probability of excluding at least one true
divergence value is also quite high across many of the \divt{max}
(Figure~\ref{figPowerExclusion4}E--H \& S\ref{figPowerExclusion6}G--L).
Only when the data are identically distributed as one of the prior models does
the method avoid excluding the truth more than 5\% of the time
(Figure~\ref{figPowerExclusion4}C).
% Again, this demonstrates the method's sensitivity to priors.



\section{The importance of power analyses to guide applications of \texttt{msBayes}}
\citet{Hickerson2013} presented a power analysis of \msb under a narrow uniform
divergence-time prior of 0--1 coalescent units ago.
They found that under these prior conditions \msb can, assuming a
per-site rate of $1.92\e{-8}$ mutations per generation, detect multiple
divergence events among 18 taxa when the true divergences were random over
hundreds of thousands of generations or more.
\citet{Oaks2012} performed similar power analyses under three uniform
divergence-time priors as narrow as 0--5 coalescent units, and found the
method was able to detect multiple events among 22 when divergences were random
over millions of generations.
As recommended by \citet{Oaks2012}, it is important that investigators perform
power analyses to determine the method's power for their dataset, and decide if
\msb has sufficient temporal resolution to address their hypotheses; in the
case of the Philippines dataset, it did not.
When doing so, it is important to consider what prior conditions are
relevant to their empirical system.
A divergence-time prior of 0--5 coalescent units is quite narrow considering
that this expresses the prior belief that all 22 taxon pairs diverged within
this window.
Certainly, it is  rare for there to be enough \emph{a priori} information to
be certain that all taxa diverged within the last 4$\globalpopsize$ generations
(i.e., 0--1 coalescent units).
Also, it seems unlikely that when such prior information is available that
being able to detect multiple divergences on the scale of hundreds of thousands
of generations will add much insight about the evolutionary history of the
taxa.
% ; assuming a rate of mutation consistent with nuclear loci ($1\e{-9}$), this
% translates to a temporal resolution of 3 million generations or more.

As noted by \citet{Oaks2012}, inferring more than one divergence time
shared across all taxa does not confirm the method is working well when
analyzing data generated under random temporal variation in divergences (e.g.,
an inference of two divergence events could be biogeographically interesting
yet spurious).
Thus, it is important that investigators not limit their assessment of the
method's power to only differentiating inferences of one event or more (i.e.
$\numt{} = 1$ versus $\numt{} > 1$).
Rather, looking at the distribution of estimates, as in \citet{Oaks2012},
provides much more information about the behavior of the method.



\section{The causes of support for models of co-divergence}
To determine how best to improve the behavior of \msb, it is important to
determine the mechanism by which broad uniform priors cause spurious support
for clustered models of divergence.
It is well established that vague priors can be problematic in Bayesian model
selection.
Models that integrate over more parameter space characterized by low
probability of producing the data and relatively high prior density will have
smaller marginal likelihoods \citep{Jeffreys1939,Lindley1957}.
Given the uniformly distributed priors on divergence times (and other nuisance
parameters) employed in \msb, models with more divergence parameters will be
forced to integrate over \emph{much} greater parameter space, all with equal
prior density, and much of it with low likelihood.
In light of this fundamental statistical issue, it is not surprising that the
method tends to support simple models.

However, \citet{Hickerson2013} conclude that the bias is due to insufficient
sampling.
They argue the widest of the three priors used by \citet{Oaks2012} would
infrequently produce samples with many independent population divergence times
as recent as the estimated gene divergence times presented in \citet{Oaks2012}.
However, this argument assumes the gene divergence times presented in
\citet{Oaks2012} were estimated without error.
These estimates were intended to only provide a rough comparison of the
gene divergence times across the 22 taxa.
These analyses assumed an arbitrary strict mutation rate of $2\e{-8}$ for all
taxa, and are, of course, subject to estimation error.
Furthermore, the branch-length units of the gene trees are in millions of
years, whereas the divergence time prior of \msb is in generations, thus
\citet{Hickerson2013} make the implicit assumption that all 22 Philippine taxa
have a generation time of one year.
The argument of \citet{Hickerson2013} that divergence time estimates of
\citet{Oaks2012} ``should set an upper bound on their prior for \divt{}'' seems
questionable, especially given our findings presented above regarding the
behavior of \msb when empirically informed uniform priors are employed.

Even if we assume the gene divergence times are estimated without error,
Hickerson et al.'s (\citeyear{Hickerson2013}) argument only applies to one of
the three different priors used by \citet{Oaks2012}.
The narrowest prior on divergence times used by \citet{Oaks2012} ($U(0, 5)$)
closely mirrors the range of estimates of gene-divergence times if we assume
the same mutation rate and one generation per year (0--5 million years ago).
Applying Hickerson et al.'s (\citeyear{Hickerson2013}) sampling-probability
argument demonstrates this prior is densely populated with samples with large
numbers of divergence parameters with values younger than the estimated gene
divergence estimates.
Thus, if insufficient prior sampling is to blame for the bias, it should be
much reduced under the narrow prior on \divt{}.
However, the magnitude of the bias is very similar across all three priors
explored by \citet{Oaks2012}.
\citet{Hickerson2013} point out a case where the narrow prior performs
slightly better (panel L of Figures S32, S37, and S38 of \citet{Oaks2012}).
However, it is important to note that these results suffered from a bug
in \msb, and there are many cases after \citet{Oaks2012} corrected the 
bug where the narrow prior performs slightly worse (see panels D--J of
Figures 3 and S12).

To disentangle whether model likelihoods or insufficient prior sampling is to
blame for the method's spurious support for simple models, we must look at the
different predictions made by these two phenomena.
One example, as discussed by \citet{Oaks2012}, is that insufficient prior
sampling should create large variance among posterior estimates, and thus it
should cause analyses to be highly sensitive to the number of samples drawn
from the prior.
Furthermore, if sampling error is biased toward models with less
parameter space, as suggested by \citet{Hickerson2013}, we expect to see
support for these models decrease as sampling increases.
\citet{Oaks2012} did not see such sensitivity when they compared prior sample
sizes of $2\e6$, $5\e6$, and $10^7$.

To explore this prediction further, we repeat the analysis of the Philippines
dataset under the intermediate prior used by \citet{Oaks2012} ($\divt{} \sim
U(0, 10)$, $\meanDescendantTheta{} \sim (0.0005, 0.04)$, $\ancestralTheta{}
\sim (0.0005, 0.02)$), using a very large prior sample size of $10^8$.
When we look at the trace of the estimates of the dispersion index of
divergence times (\vmratio{}) as the prior samples accumulate
(Figure~S\ref{figSamplingError}) we do not see the trend predicted by
\citet{Hickerson2013} in either the unadjusted or GLM-regression-adjusted
estimates.
While sampling error is always a reality, it does not appear to be playing a
large role in the biases revealed by the results of \citet{Oaks2012} or
presented above.

As discussed by \citet{Oaks2012}, a straightforward prediction if marginal
likelihoods are causing the preference for simple models is that the bias should
disappear as the model generating the data converges to the prior.
\citet{Oaks2012} tested this prediction by performing 100,000 simulations to
assess the model-choice behavior of \msb when the prior model is correct.
The results confirm the prediction as \msb actually tends to underestimate
the probability of the one-divergence model (see Figure 4 of
\citet{Oaks2012}).
We confirmed this same behavior for the model-averaging approach used by
\citet{Hickerson2013} (see SI text and Figure~S\ref{figValidationMCBehavior}).
These results are not clearly predicted if insufficient sampling was causing
the bias.
Even when the prior is correct, due to the discrete uniform prior on
\numt{} implemented in \msb, models with larger numbers of divergence
events (and thus greater parameter space) will still be less densely
sampled than those with fewer divergence events \citep{Oaks2012}.
Thus, the results of the simulations of \citet{Oaks2012} are more consistent
with the fundamental sensitivity of marginal likelihoods to priors.

This is further demonstrated by the results presented herein that show the
model-averaging approach of \citet{Hickerson2013} prefers models with narrower
\divt{} priors (Table~\ref{tabModelChoiceEmpirical} and
Figs.~\labelcref{figExclusionSimTau,figExclusionSimProb,figPowerExclusion4})
and models with fewer \divt{} parameters (Figure~\ref{figPower4}).
In all of these analyses, each of the prior models have the same number of
samples.
Thus, while sampling error will always be present in any numerical Bayesian
approximation method, insufficient sampling is an unlikely explanation for the
support of prior models with less parameter space.
While analyses that sample each model proportional to their parameter space
could be explored, it is clear that the marginal likelihoods under broad
uniform priors on divergence times will be greater for models with fewer
divergence-time dimensions.




\section{General thoughts on the model of \texttt{msBayes}}
Our findings are not surprising in light of the difficult inference problem
with which \msb is faced.
% To get a sense of the difficulty of this problem, we tally up all the free
% parameters of the \msb model as applied to the dataset of \citet{Oaks2012}.
Under the simplest model in \msb (i.e., assuming no migration and no
intra-locus recombination), the number of parameters for each taxon pair
include:
The population sizes of the ancestral and descendant populations
(\ancestralTheta{}, \descendantTheta{1}{}, \descendantTheta{2}{}),
the magnitude of population contraction in each of the descendant
populations (\bottleScalar{1}{} and \bottleScalar{2}{}), the
timing of these contractions (\bottleTime{}), and the $N-1$ node heights
(coalescent times) of the gene tree that gave rise to the $N$ gene
copies sampled from both populations.
Lastly, there are between one and 22 divergence time parameters \divt{} in
the vector \divtvector.
Overall, when applying the simplest model in \msb to the dataset of
\citet{Oaks2012} with 22 taxon pairs, there are 581--602 free parameters
that model highly stochastic coalescent and mutational processes.
Furthermore, under this rich stochastic model, the method is estimating the
probability of 1002 divergence models \citep[i.e., the number of integer
partitions of $Y=22$;][]{Oaks2012}.

This is a very difficult inference problem, especially considering that all the
information in the sequence alignment of each taxon pair is distilled into four
summary statistics:
$\pi$ \citep{Tajima1983}, $\theta_W$ \citep{Watterson1975}, $\pi_{net}$
\citep{Takahata1985}, and $SD(\pi-\theta_W)$ \citep{Tajima1989}.
That gives us a total of 88 summary statistics (four from each of the 22 taxon
pairs), which contain minimal information about many of the $\approx 600$
parameters in the model.
More summary statistics can be used in \msb, but most are highly correlated
with the four default statistics, and thus contribute little additional
information about the parameters from the sequence data.
The large number of parameters and divergence models relative to the amount of
information in the data is undoubtedly a key reason the method lacks robustness
to prior conditions.
Robustness is an important characteristic of a method to gauge its
applicability to real-world data, because we know the model and priors will be
wrong to some degree.

The model-averaging approach of \citet{Hickerson2013} adds an additional
dimension of model choice to the model.
They expand the model to sample over eight prior models.
This adds an additional free parameter to the model and, more importantly,
forces the model to sample over 8016 unique models.
While this approach is trivial to implement, it is a non-trivial extension of
the model.
In theory, the model-averaging approach of \citet{Hickerson2013} is very
appealing.
It leverages a great strength of Bayesian statistical procedures, namely the
ability to obtain marginalized estimates that incorporate uncertainty in
nuisance parameters.
However, when averaging over both narrow-empirical and diffuse uniform priors
for a parameter that is expected to have a very non-uniform likelihood density,
and in the context of a model-choice method that is highly sensitive to priors,
it is not too surprising that the approach struggles.

The recommendations of \citet{Oaks2012} for mitigating the lack of robustness
of \msb are similar to those of \citet{Hickerson2013}, but avoid the
need for imposing an additional dimension of model choice.
\citet{Oaks2012} suggest that uniform priors may not be appropriate for many
parameters of the \msb model, and recommend the use of probability
distributions from the exponential family.
If we look at the prior distribution on divergence-time parameters imposed by
the model-averaging approach of \citet{Hickerson2013} we see it is a mixture of
overlapping uniforms with lower limits of zero (Figure~S\ref{figMCTauPrior}).
This looks very much like an exponential distribution, except that in any state
of the model, all the divergence-time parameters are restricted to the hard
bounds of one of the uniform distributions.
Thus, it seems more appropriate to simply place a gamma prior (the exponential
being a special case of the gamma) on divergence times.
This would capture the prior uncertainty that \citet{Hickerson2013} are
suggesting for divergence times (Figure~S\ref{figMCTauPrior}) while avoiding
costly model-averaging and the constraint that all divergence times must fall
within the hard bounds of the current model state.
It also would allow an investigator to place the majority of the prior density
in regions of parameter space they believe, \emph{a priori}, are most
plausible, but still capture uncertainty in the tails of distributions with low
density.
Most importantly, we have found the use of more flexible distributions in place
of uniform priors improves the power of the method to detect temporal variation
in divergences and reduces the bias toward models of clustered divergences
\citep{Oaks2014dpp}.



\section{Conclusions}
% We demonstrate how the approximate Bayesian model-choice method implemented in
% \msb can be strongly biased away from models with greater parameter space.
We demonstrate how the approximate Bayesian model-choice method implemented in
\msb can be spuriously support models with less parameter space.
As suggested by \citet{Oaks2012}, this is likely caused by the use of uniform
priors on most of the model's parameters.
Uniform distributions necessitate the use of priors that place high density in
unlikely regions of parameter space, less the risk of excluding the truth
\emph{a priori}.
These broad priors reduce the marginal likelihoods of models with more
divergence-time parameters or broader prior distributions on those parameters.
We show that the empirical Bayesian model-averaging approach of
\citet{Hickerson2013} does not mitigate this bias, but rather causes it to
manifest by sampling predominantly from models that may exclude the true values
of the parameters.
Also, our results show that it is difficult to choose an uniformly distributed
prior on divergence times that is broad enough to confidently contain the true
values of parameters while being narrow enough to avoid bias toward models with
less parameter space.

\change{
% Whether or not one chooses to use empirically informed priors,
% We do not assume results of all previous applications of \msb results are invalid, as
% suggested by \citet{Hickerson2013}.
% We do conservatively recommend that
The common inference of temporally clustered divergences \citep{Barber2010,
    Bell2012, Carnaval2009, Chan2011, Daza2010, Hickerson2006, Huang2011,
    Lawson2010, Leache2007, Plouviez2009, Stone2012, Voje2009}, when not
accompanied with the necessary analyses to assess the robustness and temporal
resolution of such results, should be treated with caution, because \msb has
been shown to spuriously infer clustered divergences over a range of prior
conditions.
}
Fortunately, alternative prior probability distributions allow prior
uncertainty to be accommodated while avoiding excessive prior density in
regions of low likelihood, which greatly improves the power of the method
\citep{Oaks2014dpp}.


The work presented herein follows the principles of Open Notebook Science.
All aspects of the work were recorded in real-time via version-control software
and are publicly available at
\href{https://github.com/joaks1/msbayes-experiments}{\url{https://github.com/joaks1/msbayes-experiments}}.
All information necessary to reproduce our results is provided there.

