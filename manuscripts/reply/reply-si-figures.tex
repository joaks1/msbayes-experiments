\siFigure{../../response-redux/hickerson-et-al-posterior/hickerson-posterior-1k/mean_by_dispersion.pdf}{
    The joint posterior of the mean (\meant{}{}) and dispersion index ($\vmratio{} = 
    \vart{}{}/\meant{}{}$) of divergence times for 22 vertebrate taxon pairs as
    estimated by \citet{Hickerson2013} (see Figure 2B of \citet{Hickerson2013}).
    The posterior samples are color-coded to indicate the erroneous mixture of
    timescales in the analysis of \citet{Hickerson2013};
    grey = $0.05/\mutationRate$ generations and
    black = $0.02/\mutationRate$ generations.
}{figJointPosteriorHickerson}

\siFigure{../../model-choice/priors-for-pc-plot/pc-plots.pdf}{
    The prior predictive graphical checks recommended by \citet{Hickerson2013}
    for six prior models:
    (A) $M_1$ ($\divt{} \sim U(0, 0.1)$),
    (B) $M_{1A}$ ($\divt{} \sim U(0, 0.01)$),
    (C) $M_{1B}$ ($\divt{} \sim U(0, 0.001)$), 
    (D) $M_3$ ($\divt{} \sim U(0, 5)$),
    (E) $M_4$ ($\divt{} \sim U(0, 10)$), and
    (F) $M_5$ ($\divt{} \sim U(0, 20)$).
    The three models that likely exclude true values of some divergence times
    of the 22 pairs of Philippine vertebrate taxa (A--C) appear to have a
    better ``fit'' than the priors that likely cover the true divergence times
    (D--F).
    The plots project the summary statistics from 1000 random samples from each
    model onto the first two orthogonal axes of a principle component analysis,
    with the blue dot representing the observed summary statistics from the 22
    population pairs of Philippine vertebrates.
}{figPCA}

\siSidewaysFigure{../../model-choice/results/power-1/pymsbayes-results/plots/power-accuracy-6.pdf}{
    The accuracy of (A--F) unadjusted and (G--L) GLM-adjusted estimates of
    dispersion index of divergence times (\vmratio{}) when the empirically
    informed model-averaging approach of \citet{Hickerson2013} is applied to
    simlated datasets in which divergence times of 22 pairs of populations are
    randomly drawn from the uniform distributions $\divt{} \sim U(0,
    \divt{max})$ indicated at the top of each column of plots (divergence-time
    distributions are given in units of millions of generations ago (MGA)
    assuming a per-site rate of 1\e{-8} mutations per generation).
}{figPowerAccuracy}

\siSidewaysFigure{../../model-choice/results/power-1/pymsbayes-results/plots/power-6.pdf}{
    The tendency of the empirically informed model-averaging approach of
    \citet{Hickerson2013} to (A--F) infer clustered divergences and (G--L)
    support the extreme model of one divergence when applied to simulated
    datasets in which the divergence times of 22 pairs of populations are
    randomly drawn from the uniform distributions $\divt{} \sim U(0,
    \divt{max})$ indicated at the top of each column of plots (divergence-time
    distributions are given in units of millions of generations ago (MGA)
    assuming a per-site rate of 1\e{-8} mutations per generation).
}{figPower6}

\siSidewaysFigure{../../model-choice/results/power-1/pymsbayes-results/plots/power-exclusion-6.pdf}{
    Histograms of the (A--F) number of true divergence-time parameters excluded
    from the preferred model and the (G--L) posterior probability of excluding
    at least one divergence-time parameter when the empirically informed
    model-averaging approach of \citet{Hickerson2013} is applied to simlated
    datasets in which divergence times of 22 pairs of populations are randomly
    drawn from the uniform distributions $\divt{} \sim U(0, \divt{max})$
    indicated at the top of each column of plots (divergence-time distributions
    are given in units of millions of generations ago (MGA) assuming a per-site
    rate of 1\e{-8} mutations per generation).
}{figPowerExclusion6}

%% Validation results
\siFigure{../../model-choice/results/validation/results/pymsbayes-results/plots/mc_behavior.pdf}{
    An assessment of the approximate Bayesian model-averageing approach of
    \citet{Hickerson2013} under the ideal conditions when the prior model
    is correct (i.e., the datasets are simulated from parameters drawn from the
    same prior distributions used in the analysis).
    The plots show the relationship between the estimated posterior and true
    probability of (A \& C) $\numt{}=1$ and (B \& D) $\vmratio{} < 0.01$, based
    on 50,000 simulations.
    The results summarize the (A \& B) unadjusted and (C \& D) GLM-adjusted
    posterior estimate from each simulation replicate.
    The prior settings for all replicates included five prior models with
    $\meanDescendantTheta{} \sim U(0.0001, 0.1)$ and $\ancestralTheta{} \sim
    U(0.0001, 0.05)$ for all five models, and
    $M_1: \divt{} \sim U(0, 0.1)$,
    $M_2: \divt{} \sim U(0, 1)$,
    $M_3: \divt{} \sim U(0, 5)$,
    $M_4: \divt{} \sim U(0, 10)$, and
    $M_5: \divt{} \sim U(0, 20)$.
    The number of samples from the prior was $2.5\e6$.
    The simulated data structure was 8 population pairs, with a single 1000
    bp locus sampled from 10 individuals from each population.  The 50,000
    estimates of the posterior probability of one divergence event were
    assigned to 20 bins of width 0.05.
    The estimated posterior probability of each bin is plotted against the
    proportion of replicates in that bin with a true value consistent with
    one divergence event (i.e., $\numt{}=1$ or $\vmratio{} < 0.01)$.
}{figValidationMCBehavior}

%:FIGURE-saturation plot
\siFigure{../../saturation/saturation-plot.pdf}{
    The summary statistics $\pi$ \citep{Tajima1983} and $\pi_{net}$
    \citep{Takahata1985} as a function of divergence time between populations.
    Each plot represents 1100 pairs of parameter draws and summary statistics
    calculated from the simulated data.
    Prior settings for the simulations were $\divt{} \sim U(0, 20)$,
    $\meanDescendantTheta{} \sim U(0.0005, 0.04)$, and $\ancestralTheta{} \sim
    U(0.0005, 0.02)$.
}{figSaturationPlot}

