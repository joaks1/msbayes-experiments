\siFigure{../../response-redux/hickerson-et-al-posterior/hickerson-posterior-1k/mean_by_dispersion.pdf}{
    The joint posterior of the mean (\meant{}{}) and dispersion index ($\vmratio{} = 
    \vart{}{}/\meant{}{}$) of divergence times for 22 vertebrate taxon pairs as
    estimated by \citet{Hickerson2013} (see Figure 2B of \citet{Hickerson2013}).
    The posterior samples are color-coded to indicate the erroneous mixture of
    timescales in the analysis of \citet{Hickerson2013};
    grey = $0.05/\mutationRate$ generations and
    black = $0.02/\mutationRate$ generations.
}{figJointPosteriorHickerson}

\siFigure{../../model-choice/priors-for-pc-plot/pc-plots.pdf}{
    The prior predictive graphical checks recommended by \citet{Hickerson2013}
    for six prior models:
    (A) $M_1$ ($\divt{} \sim U(0, 0.1)$),
    (B) $M_{1A}$ ($\divt{} \sim U(0, 0.01)$),
    (C) $M_{1B}$ ($\divt{} \sim U(0, 0.001)$), 
    (D) $M_3$ ($\divt{} \sim U(0, 5)$),
    (E) $M_4$ ($\divt{} \sim U(0, 10)$), and
    (F) $M_5$ ($\divt{} \sim U(0, 20)$).
    The three models that likely exclude true values of some divergence times
    of the 22 pairs of Philippine vertebrate taxa (A--C) appear to have a
    better ``fit'' than the priors that likely cover the true divergence times
    (D--F).
    The plots project the summary statistics from 1000 random samples from each
    model onto the first two orthogonal axes of a principle component analysis,
    with the blue dot representing the observed summary statistics from the 22
    population pairs of Philippine vertebrates.
}{figPCA}

