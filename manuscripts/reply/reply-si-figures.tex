\siFigure{../../response-redux/hickerson-et-al-posterior/hickerson-posterior-1k/mean_by_dispersion.pdf}{
    The joint posterior of the mean (\meant{}{}) and dispersion index ($\vmratio{} = 
    \vart{}{}/\meant{}{}$) of divergence times for 22 vertebrate taxon pairs as
    estimated by \citet{Hickerson2013} (see Figure 2B of \citet{Hickerson2013}).
    The posterior samples are color-coded to indicate the erroneous mixture of
    timescales in the analysis of \citet{Hickerson2013};
    grey = $0.05/\mutationRate$ generations and
    black = $0.02/\mutationRate$ generations.
}{figJointPosteriorHickerson}

\siFigure{../../model-choice/priors-for-pc-plot/pc-plots.pdf}{
    The prior predictive graphical checks recommended by \citet{Hickerson2013}
    for six prior models:
    (A) $M_1$ ($\divt{} \sim U(0, 0.1)$),
    (B) $M_{1A}$ ($\divt{} \sim U(0, 0.01)$),
    (C) $M_{1B}$ ($\divt{} \sim U(0, 0.001)$), 
    (D) $M_3$ ($\divt{} \sim U(0, 5)$),
    (E) $M_4$ ($\divt{} \sim U(0, 10)$), and
    (F) $M_5$ ($\divt{} \sim U(0, 20)$).
    The three models that likely exclude true values of some divergence times
    of the 22 pairs of Philippine vertebrate taxa (A--C) appear to have a
    better ``fit'' than the priors that likely cover the true divergence times
    (D--F).
    The plots project the summary statistics from 1000 random samples from each
    model onto the first two orthogonal axes of a principle component analysis,
    with the blue dot representing the observed summary statistics from the 22
    population pairs of Philippine vertebrates.
}{figPCA}

\siFigure{../../model-choice/results/power-1/pymsbayes-results/plots/power_accuracy_omega_median.pdf}{
    The accuracy of the model-averaging approach of \citet{Hickerson2013} to
    estimate the dispersion index of divergence times (\vmratio{}) from
    analyses of simulated datasets where \divt{} for 22 pairs of populations is
    drawn from a series of uniform distributions, $\divt{} \sim U(0,
    \divt{max})$.
    The proportion of estimates less than the true value of,
    $p(\hat{\vmratio{}} < \vmratio{})$, is given for each \divt{max}.
    Each plot represents unadjusted median estimates from 500 simulation
    replicates analyzed using $5\e6$ samples from the prior.
}{figPowerAccOmegaMedian}

\siFigure{../../model-choice/results/power-1/pymsbayes-results/plots/power_accuracy_omega_mode_glm.pdf}{
    The accuracy of the model-averaging approach of \citet{Hickerson2013} to
    estimate the dispersion index of divergence times (\vmratio{}) from
    analyses of simulated datasets where \divt{} for 22 pairs of populations is
    drawn from a series of uniform distributions, $\divt{} \sim U(0,
    \divt{max})$.
    The proportion of estimates less than the true value of,
    $p(\hat{\vmratio{}} < \vmratio{})$, is given for each \divt{max}.
    Each plot represents GLM-regression-adjusted mode estimates from 500
    simulation replicates analyzed using $5\e6$ samples from the prior.
}{figPowerAccOmegaModeGLM}

%% Validation results
\siFigure{../../model-choice/results/validation/results/pymsbayes-results/plots/mc_behavior.pdf}{
    An assessment of the approximate Bayesian model-averageing approach of
    \citet{Hickerson2013} under the ideal conditions when the prior model
    is correct (i.e., the pseudo-replicate datasets are simulated from
    parameters drawn from the same prior distributions used in the
    analysis).
    The plots show the relationship between the estimated posterior and true
    probability of (A \& C) $\numt{}=1$ and (B \& D) $\vmratio{} < 0.01$, based
    on 50,000 simulations.
    The results summarize the (A \& B) unadjusted and (C \& D) GLM-adjusted
    posterior estimate from each simulation replicate.
    The prior settings for all replicates included five prior models with
    $\meanDescendantTheta{} \sim U(0.0001, 0.1)$ and $\ancestralTheta{} \sim
    U(0.0001, 0.05)$ for all five models, and
    $M_1: \divt{} \sim U(0, 0.1)$,
    $M_2: \divt{} \sim U(0, 1)$,
    $M_3: \divt{} \sim U(0, 5)$,
    $M_4: \divt{} \sim U(0, 10)$, and
    $M_5: \divt{} \sim U(0, 20)$.
    The number of samples from the prior was $2.5\e6$.
    The simulated data structure was 8 population pairs, with a single 1000
    bp locus sampled from 10 individuals from each population.  The 50,000
    estimates of the posterior probability of one divergence event were
    assigned to 20 bins of width 0.05.
    The estimated posterior probability of each bin is plotted against the
    proportion of replicates in that bin with a true value consistent with
    one divergence event (i.e., $\numt{}=1$ or $\vmratio{} < 0.01)$.
}{figValidationMCBehavior}

%:FIGURE-saturation plot
\siFigure{../../saturation/saturation-plot.pdf}{
    The summary statistics $\pi$ \citep{Tajima1983} and $\pi_{net}$
    \citep{Takahata1985} as a function of divergence time between populations.
    Each plot represents 1100 pairs of parameter draws and summary statistics
    calculated from the simulated data.
    Prior settings for the simulations were $\divt{} \sim U(0, 20)$,
    $\meanDescendantTheta{} \sim U(0.0005, 0.04)$, and $\ancestralTheta{} \sim
    U(0.0005, 0.02)$.
}{figSaturationPlot}

