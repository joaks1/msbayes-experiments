% \newcommand{\change}[1]{{\color{blue} #1}\xspace}
\newcommand{\change}[1]{{\color{black} #1}\xspace}

\definecolor{myGray}{gray}{0.9}

%% redefine multicolumn to not override rowcolors
\let\oldmulticolumn\multicolumn
\makeatletter
\newcommand{\mcrowcolors}{%
  \renewcommand{\multicolumn}[3]{%
    \oldmulticolumn{##1}{##2}{\ifodd\rownum \@oddrowcolor\else\@evenrowcolor\fi ##3}%
  }}
\makeatother

\newcommand{\dppTitle}{An Improved Approximate-Bayesian Model-choice Method for
Estimating Shared Evolutionary History\xspace}

\newcommand{\citationNeeded}{\textcolor{magenta}{\textbf{[CITATION NEEDED!]}}\xspace}
\newcommand{\tableNeeded}{\textcolor{magenta}{\textbf{[TABLE NEEDED!]}}\xspace}
\newcommand{\figureNeeded}{\textcolor{magenta}{\textbf{[FIGURE NEEDED!]}}\xspace}
\newcommand{\highLight}[1]{\textcolor{magenta}{\MakeUppercase{#1}}}

\newcommand{\editorialNote}[1]{\textcolor{red}{[\textit{#1}]}}
\newcommand{\ignore}[1]{}
\newcommand{\addTail}[1]{\textit{#1}.---}
\newcommand{\super}[1]{\ensuremath{^{\textrm{#1}}}}
\newcommand{\sub}[1]{\ensuremath{_{\textrm{#1}}}}
\newcommand{\dC}{\ensuremath{^\circ{\textrm{C}}}}
\newcommand{\tb}{\hspace{2em}}

\providecommand{\e}[1]{\ensuremath{\times 10^{#1}}}

\newcommand{\mthnote}[2]{{\color{red} #2}\xspace}
\newcommand{\cwlnote}[2]{{\color{orange} #2}\xspace}

\newcommand{\ifTwoArgs}[3]{\ifthenelse{\equal{#1}{}\or\equal{#2}{}}{}{#3}\xspace}
\newcommand{\ifArg}[2]{\ifthenelse{\equal{#1}{}}{}{#2}\xspace}

%% New notation for divergence times
\newcommand{\divTime}[1]{\ensuremath{\tau_{#1}}\xspace}
\newcommand{\divTimeVector}{\ensuremath{\boldsymbol{\divTime{}}}\xspace}
\newcommand{\divTimeIndex}[1]{\ensuremath{t_{#1}}\xspace}
\newcommand{\divTimeIndexVector}{\ensuremath{\mathbf{\divTimeIndex{}}}\xspace}
\newcommand{\divTimeMap}[1]{\ensuremath{T_{#1}}\xspace}
\newcommand{\divTimeMapVector}{\ensuremath{\mathbf{\divTimeMap{}}}\xspace}
\newcommand{\divTimeScaled}[2]{\ensuremath{\mathcal{T}_{#1\protect\ifTwoArgs{#1}{#2}{,}#2}}\xspace}
\newcommand{\divTimeScaledVector}{\ensuremath{\mathbf{\divTimeScaled{}{}}}\xspace}
\newcommand{\divTimeMean}{\ensuremath{\bar{\divTimeMap{}}}\xspace}
\newcommand{\divTimeVar}{\ensuremath{s^{2}_{\divTimeMap{}}}\xspace}
\newcommand{\divTimeDispersion}{\ensuremath{D_{\divTimeMap{}}}\xspace}
\newcommand{\divTimeNum}{\ensuremath{\lvert \divTimeVector \rvert}\xspace}
\newcommand{\demographicParams}[1]{\ensuremath{\Theta_{#1}}\xspace}
\newcommand{\demographicParamVector}{\ensuremath{\mathbf{\demographicParams{}}}\xspace}
\newcommand{\popSampleSize}[2]{\ensuremath{n_{#1\protect\ifTwoArgs{#1}{#2}{,}#2}}}
\newcommand{\gammaShape}[1]{\ensuremath{a_{#1}}\xspace}
\newcommand{\gammaScale}[1]{\ensuremath{b_{#1}}\xspace}
\newcommand{\betaA}[1]{\ensuremath{a_{#1}}\xspace}
\newcommand{\betaB}[1]{\ensuremath{b_{#1}}\xspace}
\newcommand{\integerPartitionSet}[1]{\ensuremath{a({#1})}\xspace}
\newcommand{\integerPartitionNum}[1]{\ensuremath{\lvert \integerPartitionSet{#1} \rvert}\xspace}
\newcommand{\concentrationParam}{\ensuremath{\chi}\xspace}
\newcommand{\stirlingFirst}[2]{\ensuremath{c(#1, #2)}\xspace}
\newcommand{\descendantThetaMean}[1]{\ensuremath{\bar{\theta}_{D\protect\ifArg{#1}{,}#1}}\xspace}
\newcommand{\numPriorSamples}{\ensuremath{\mathbf{n}}\xspace}
\newcommand{\paramSampleVector}[1]{\ensuremath{\Lambda_{#1}}\xspace}
\newcommand{\paramSampleMatrix}{\ensuremath{\boldsymbol{\paramSampleVector{}}}\xspace}
\newcommand{\ordered}{\ensuremath{\circ}\xspace}
\newcommand{\modelDPP}{\ensuremath{M_{DPP}}\xspace}
\newcommand{\modelDPPOrdered}{\ensuremath{M^{\ordered}_{DPP}}\xspace}
\newcommand{\modelUniform}{\ensuremath{M_{Uniform}}\xspace}
\newcommand{\modelUshaped}{\ensuremath{M_{Ushaped}}\xspace}
\newcommand{\modelOld}{\ensuremath{M_{msBayes}}\xspace}
\newcommand{\priorDPP}[1]{\ensuremath{DP(\concentrationParam #1)}\xspace}
\newcommand{\priorUniform}{\ensuremath{DU\{\integerPartitionSet{\npairs{}}\}}\xspace}
\newcommand{\priorOld}{\ensuremath{DU\{1, \ldots, \npairs{}\}}\xspace}
\newcommand{\powerSeriesOld}{\ensuremath{\mathcal{M}_{msBayes}}\xspace}
\newcommand{\powerSeriesUniform}{\ensuremath{\mathcal{M}_{Uniform}}\xspace}
\newcommand{\powerSeriesExp}{\ensuremath{\mathcal{M}_{Exp}}\xspace}
\newcommand{\empModelOld}{\ensuremath{\mathbf{M}_{msBayes}}\xspace}
\newcommand{\empModelUniform}{\ensuremath{\mathbf{M}_{Uniform}}\xspace}
\newcommand{\empModelDPP}{\ensuremath{\mathbf{M}_{DPP}}\xspace}
\newcommand{\empModelDPPInform}{\ensuremath{\mathbf{M}^{inform}_{DPP}}\xspace}
\newcommand{\empModelDPPSimple}{\ensuremath{\mathbf{M}^{simple}_{DPP}}\xspace}
\newcommand{\npModelDPP}{\ensuremath{\mathbb{M}_{DPP}}\xspace}
\newcommand{\npModelDPPOrdered}{\ensuremath{\mathbb{M}^{\ordered}_{DPP}}\xspace}

\newcommand{\allDatasets}{\ensuremath{\mathcal{\alignment{}{}}}\xspace}
\newcommand{\allParameterValues}{\ensuremath{\boldsymbol{\Theta}}\xspace}
\newcommand{\bayesfactor}[2]{\ensuremath{BF_{#1\protect\ifArg{#2}{,}#2}}}
\newcommand{\given}{\ensuremath{\,|\,}\xspace}
\newcommand{\msb}{\upshape\texttt{\MakeLowercase{ms\MakeUppercase{B}ayes}}\xspace}
\newcommand{\abctoolbox}{\upshape\texttt{ABCtoolbox}\xspace}
\newcommand{\dppmsbayes}{\upshape\texttt{dpp-msbayes}\xspace}
\newcommand{\pymsbayes}{\upshape\texttt{PyMsBayes}\xspace}
\newcommand{\hky}{HKY85\xspace}
\newcommand{\uniformMin}[1]{\ensuremath{a_{#1}}\xspace}
\newcommand{\uniformMax}[1]{\ensuremath{b_{#1}}\xspace}
\newcommand{\locusRateHetShapeParameter}{\ensuremath{\alpha}\xspace}
\newcommand{\ancestralThetaVector}{\ensuremath{\boldsymbol{\theta_{A}}}\xspace}
\newcommand{\descendantThetaVector}[1]{\ensuremath{\boldsymbol{\theta_{D#1}}}\xspace}
\newcommand{\divtscaledvector}{\ensuremath{\mathbf{{\divtscaled{}{}}}}\xspace}
\newcommand{\divtvector}{\ensuremath{\boldsymbol{\divt{}}}\xspace}
\newcommand{\divtuniquevector}{\ensuremath{\mathbf{\divtunique{}}}\xspace}
\newcommand{\bottleTimeVector}{\ensuremath{\boldsymbol{\bottleTime{}}}\xspace}
\newcommand{\bottleTime}[1]{\ensuremath{\divt{B\protect\ifArg{#1}{,}#1}}\xspace}
\newcommand{\bottleScalarVector}[1]{\ensuremath{\boldsymbol{\bottleScalar{#1}{}}}\xspace}
\newcommand{\bottleScalar}[2]{\ensuremath{\zeta_{D#1\protect\ifArg{#2}{,}#2}}\xspace}
\newcommand{\migrationRateVector}{\ensuremath{\mathbf{\migrationRate{}}}\xspace}
\newcommand{\geneTreeVector}{\ensuremath{\mathbf{\geneTree{}{}}}\xspace}
\newcommand{\alignmentVector}{\ensuremath{\mathbf{\alignment{}{}}}\xspace}
\newcommand{\alignment}[2]{\ensuremath{X_{#1\protect\ifTwoArgs{#1}{#2}{,}#2}}\xspace}
\newcommand{\geneTree}[2]{\ensuremath{G_{#1\protect\ifTwoArgs{#1}{#2}{,}#2}}\xspace}
\newcommand{\migrationRate}[1]{\ensuremath{m_{#1}}\xspace}
\newcommand{\recombinationRate}{\ensuremath{r}\xspace}
\newcommand{\ploidyScalar}[2]{\ensuremath{\rho_{#1\protect\ifTwoArgs{#1}{#2}{,}#2}}\xspace}
\newcommand{\ploidyScalarVector}{\ensuremath{\boldsymbol{\ploidyScalar{}{}}}\xspace}
\newcommand{\descendantRelativeThetaVector}[1]{\ensuremath{\boldsymbol{\eta_{D#1}}}\xspace}
\newcommand{\descendantRelativeTheta}[2]{\ensuremath{\eta_{D#1\protect\ifArg{#2}{,}#2}}\xspace}
\newcommand{\mutationRateScalarConstant}[2]{\ensuremath{\nu_{#1\protect\ifTwoArgs{#1}{#2}{,}#2}}\xspace}
\newcommand{\mutationRateScalarConstantVector}{\ensuremath{\boldsymbol{\mutationRateScalarConstant{}{}}}\xspace}
\newcommand{\locusMutationRateScalar}[1]{\ensuremath{\upsilon_{#1}}\xspace}
\newcommand{\locusMutationRateScalarVector}{\ensuremath{\boldsymbol{\upsilon}}\xspace}
\newcommand{\hkyModel}[2]{\ensuremath{\phi_{#1\protect\ifTwoArgs{#1}{#2}{,}#2}}\xspace}
\newcommand{\hkyModelVector}{\ensuremath{\boldsymbol{\hkyModel{}{}}}\xspace}
\newcommand{\mutationRate}{\ensuremath{\mu}\xspace}
\newcommand{\iid}{\textit{iid}\xspace}
\newcommand{\model}[1]{\ensuremath{\Theta}\xspace}
\newcommand{\npairs}[1]{\ensuremath{Y_{#1}}}
\newcommand{\nloci}[1]{\ensuremath{k_{#1}}\xspace}
\newcommand{\nlociTotal}{\ensuremath{K}\xspace}
\newcommand{\myTheta}[1]{\ensuremath{\theta_{#1}}}
\newcommand{\ancestralTheta}[1]{\ensuremath{\theta_{A\protect\ifArg{#1}{,}#1}}\xspace}
\newcommand{\descendantTheta}[2]{\ensuremath{\theta_{D#1\protect\ifArg{#2}{,}#2}}\xspace}
\newcommand{\popIndex}{\ensuremath{z}\xspace}
\newcommand{\dThetas}{\ensuremath{\descendantTheta{\popIndex}{}}\xspace}
\newcommand{\meanDescendantTheta}[1]{\ensuremath{\descendantTheta{}{#1}}\xspace}
\newcommand{\nucdiv}[1]{\ensuremath{\pi_{#1}}}

\newcommand{\ssVector}[1]{\ensuremath{\mathbf{\alignmentSS{#1}{}}}\xspace}
\newcommand{\ssVectorObs}{\ensuremath{\ssVector{}^*}\xspace}
\newcommand{\ssSpace}{\ensuremath{\euclideanSpace{\ssVectorObs}}\xspace}
\newcommand{\ssVectorObsPLS}{\ensuremath{\ssVectorObs_{PLS}}\xspace}
\newcommand{\alignmentSS}[2]{\ensuremath{S_{#1\protect\ifTwoArgs{#1}{#2}{,}#2}}\xspace}
\newcommand{\alignmentSSObs}[2]{\ensuremath{\alignmentSS{#1}{#2}^*}\xspace}
\newcommand{\tol}{\ensuremath{\epsilon}\xspace}
\newcommand{\euclideanSpace}[1]{\ensuremath{B_{\tol}(#1)}\xspace}
\newcommand{\hpvector}[1]{\ensuremath{\Lambda_{#1}}}
\newcommand{\divtscaled}[2]{\ensuremath{t_{#1\protect\ifTwoArgs{#1}{#2}{,}#2}}}
\newcommand{\divt}[1]{\ensuremath{\tau_{#1}}}
\newcommand{\divtunique}[1]{\ensuremath{T_{#1}}}
\newcommand{\ssMatrix}{\ensuremath{\mathbb \alignmentSS{}{}}\xspace}
\newcommand{\ssMatrixRaw}[1]{\ensuremath{{\ssMatrix}_{stats#1}}\xspace}
\newcommand{\ssMatrixPLS}[1]{\ensuremath{{\ssMatrix}_{PLS#1}}\xspace}
\newcommand{\hpmatrix}[1]{\ensuremath{\mathcal{P}_{#1}}}
\newcommand{\meant}[2]{\ensuremath{E(\divt{#1})_{#2}}}
\newcommand{\meantestimate}{\ensuremath{\hat{E(\divt{})}}\xspace}
\newcommand{\vart}[2]{\ensuremath{Var(\divt{#1}{})_{#2}}}
\newcommand{\vmratio}[1]{\ensuremath{\Omega_{#1}}}
\newcommand{\numt}[1]{\ensuremath{\Psi_{#1}}}
\newcommand{\probnumt}[2]{\ensuremath{p(\numt{#1} = {#2})}}
\newcommand{\postprobnumt}[1]{\ensuremath{p(\numt{} = {#1}|\ssSpace)}}
\newcommand{\postprobnumtnot}[1]{\ensuremath{p(\numt{} \neq {#1}|\ssSpace)}}
\newcommand{\postprobomegasimult}{\ensuremath{p(\vmratio{} < 0.01 | \ssSpace)}\xspace}
\newcommand{\modelprior}[1]{\ensuremath{f(\model{})}}
\newcommand{\modelpost}[1]{\ensuremath{f(\model{}|\ssSpace)}}
\newcommand{\npriorsamples}{\ensuremath{n}\xspace}
\newcommand{\globalcoalunit}{\ensuremath{4\globalpopsize}\xspace}
\newcommand{\globalpopsize}{\ensuremath{N_C}\xspace}
\newcommand{\effectivePopSize}[1]{\ensuremath{N_e{#1}}\xspace}
\newcommand{\coalunit}{\ensuremath{4\effectivePopSize{}}\xspace}
\newcommand{\priorsample}[1]{\ensuremath{\hpmatrix{\modelprior{}}}}
\newcommand{\truncprior}[1]{\ensuremath{\hpmatrix{\tol}}\xspace}
\newcommand{\postsample}[1]{\ensuremath{\hpmatrix{\modelpost{}}}}
\newcommand{\abcllr}[1]{ABC\sub{LLR}}
\newcommand{\abcglm}[1]{ABC\sub{GLM}}
\newcommand{\integerPartition}[1]{\ensuremath{a({#1})}}
\newcommand{\uniqueModel}[2]{\ensuremath{M_{#1\protect\ifTwoArgs{#1}{#2}{,}#2}}}
\newcommand{\taxonLocusVector}[1]{\ensuremath{\{#1{1}{1},\ldots,#1{\npairs{}}{\nloci{\npairs{}}}\}}\xspace}
\newcommand{\taxonVector}[1]{\ensuremath{\{#1{1},\ldots,#1{\npairs{}}\}}\xspace}
\newcommand{\locusVector}[1]{\ensuremath{\{#1{1},\ldots,#1{\nlociTotal}\}}\xspace}

\newcommand{\validationAccuracyComparisonCaption}[2]{Accuracy of
    #1 estimates of #2 when data generated under models (A, E, I, \& M)
    \modelOld, (B, F, J, \& N) \modelUshaped, (C, G, K, \& O) \modelUniform,
    and (D, H, L, \& P) \modelDPP are analyzed with models (A--D) \modelOld,
    (E--H) \modelUshaped, (I--L) \modelUniform, and (M--P) \modelDPP.  A random
    sample of 5000 posterior estimates (from 50,000) are plotted.  The root
mean square error (RMSE) calculated from the 5000 estimates is provided.}
\newcommand{\jitterComment}{Random normal variates ($N(0, 0.005)$) have been
    added to the estimates and true values of \divTimeNum to reduce overlap of
    plot symbols.}
\newcommand{\timeConversionComment}{The 22 divergence times were randomly drawn as
    indicated above each column of plots, where time is respresented as
    millions of generations ago (MGA) according to a per-site rate of 1\e{-8}
    mutations per generation.\xspace}
\newcommand{\validationModelChoiceComparisonCaption}[2]{Model-choice accuracy
    for models (A--D) \modelOld, (E--H) \modelUshaped, (I--L) \modelUniform,
    and (M--P) \modelDPP when analyzing data generated under models (A, E, I,
    \& M) \modelOld, (B, F, J, \& N) \modelUshaped, (C, G, K, \& O)
    \modelUniform, and (D, H, L, \& P) \modelDPP.
    The #1 posterior probability of a single divergence event, based on #2,
    from 50,000 posterior estimates are assigned to bins of width 0.05 and
    plotted against the proportion of replicates in each bin where the truth is
    #2.}
\newcommand{\validationAccuracyCaption}[2]{Estimation accuracy for model
    #2 when analyzing data generated under #1.
    A random sample of 5000 posterior estimates (from 50,000) are plotted,
    including both (A, B, \& C) unadjusted and (D, E, \&
    F) GLM-regression-adjusted estimates.
    Normal random variates ($N(0, 0.005)$) have been added to the estimates and
    true values of \divTimeNum (A \& D) to reduce overlap of plot symbols.
    The root mean square error (RMSE) calculated from the 5000 estimates is
    provided.}
\newcommand{\validationModelChoiceCaption}[2]{Model-choice accuracy for model
    #2 when analyzing data generated under #1.
    The estimated posterior probability of a single divergence event, based on
    (A \& C) $\divTimeNum = 1$ and (B \& D) $\divTimeDispersion < 0.01$, from
    50,000 posterior estimates are assigned to bins of width 0.05 and plotted
    against the proportion of replicates in each bin where the truth is
    $\divTimeNum = 1$ or $\divTimeDispersion < 0.01$.
    Results based on the (A \& B) unadjusted and (C \& D) GLM-adjusted
    posterior estimates are shown.}
\newcommand{\powerAccuracyComparisonCaption}[1]{Accuracy of \divTimeDispersion
    estimates for models (A--F) \modelOld, (G--L) \modelUshaped, (M--R)
    \modelUniform, and (S--X) \modelDPP when analyzing data generated under the
    #1 series of models.
    The true versus estimated value of the dispersion index of divergence times
    (\divTimeDispersion, in \globalcoalunit generations) is plotted for 1000
    datasets simulated under each of the #1 models, and the proportion of
    estimates less than the truth, $p(\hat{\divTimeDispersion} <
    \divTimeDispersion)$, is shown for each data model.
    \timeConversionComment}
\newcommand{\powerComment}[1]{The power of models
    (A--F) \modelOld, (G--L) \modelUshaped, (M--R) \modelUniform, and (S--X)
    \modelDPP to detect random variation in divergence times as simulated under
    the #1 series of models.}
\newcommand{\powerCommentSummary}[1]{The power of models
    (A--D) \modelOld, (E--H) \modelUshaped, (I--L) \modelUniform, and (M--P)
    \modelDPP to detect random variation in divergence times as simulated under
    the #1 series of models.}
\newcommand{\powerSupportComment}[1]{
    The tendency of models (A--F) \modelOld, (G--L) \modelUshaped, (M--R)
    \modelUniform, and (S--X) \modelDPP to support one divergence event when
    there is random variation in divergence times as simulated under the #1
    series of models.}
\newcommand{\powerSupportCommentSummary}[1]{
    The tendency of models (A--D) \modelOld, (E--H) \modelUshaped, (I--L)
    \modelUniform, and (M--P) \modelDPP to support one divergence event when
    there is random variation in divergence times as simulated under the #1
    series of models.}
\newcommand{\powerPsiComment}[1]{ 
    The plots illustrate the estimated number of
    divergence events ($\hat{\divTimeNum}$) from analyses of 1000 datasets
    simulated under each of the #1 models, with the the estimated probability
    of the model inferring one divergence event, $p(\hat{\divTimeNum} = 1)$,
    given for each combination.}
\newcommand{\powerProbComment}[2]{ 
    The plots illustrate histograms of the estimated posterior probability of
    the one divergence model, #1, from analyses of 1000 datasets simulated
    under each of the #2 models.}
\newcommand{\powerDispersionComment}[1]{ 
    The plots illustrate the estimated dispersion index of divergence times
    ($\hat{\divTimeDispersion}$) from analyses of 1000 datasets simulated under
    each of the #1 models, with the the estimated probability of the model
    inferring one divergence event, $p(\hat{\divTimeDispersion} < 0.01)$, given
    for each combination.}
\newcommand{\powerAccuracyCaption}[2]{Estimation accuracy for model
    #2 when analyzing data generated under the series of models #1.
    The true versus estimated value of the dispersion index of divergence
    times (\divTimeDispersion) is plotted for 1000 datasets simulated
    under each of the #1 models, and the proportion of estimates less
    than the truth, $p(\hat{\divTimeDispersion} < \divTimeDispersion)$,
    is shown for each data model.}
\newcommand{\powerPsiCaption}[2]{
    The power of model #2 to detect random variation in divergence times as
    simulated under the series of models #1.
    The plots illustrate the estimated number of divergence events
    ($\hat{\divTimeNum}$) from analyses of 1000 datasets simulated under each
    of the #1 models, with the the estimated probability of the model inferring
    one divergence event, $p(\hat{\divTimeNum} = 1)$, given for each data
    model.}
\newcommand{\powerDispersionCaption}[2]{
    The power of model #2 to detect random variation in divergence times as
    simulated under the series of models #1.
    The plots illustrate the estimated dispersion index of divergence times
    ($\hat{\divTimeDispersion}$) from analyses of 1000 datasets simulated under
    each of the #1 models, with the the estimated probability of the model
    inferring one divergence event, $p(\hat{\divTimeDispersion} < 0.01)$, given
    for each data model.}
\newcommand{\powerPsiProbCaption}[2]{
    The tendency of model #2 to support one divergence event when there is
    random variation in divergence times as simulated under the series of
    models #1.
    The plots illustrate histograms of the estimated posterior probability of
    the one divergence model, $p(\divTimeNum = 1 | \ssSpace)$, from analyses of
    1000 datasets simulated under each of the #1 models, with the the estimated
    probability of the model strongly supporting one divergence event,
    $p(BF_{\divTimeNum = 1, \divTimeNum \neq 1} > 10)$, given for each data
    model.}
\newcommand{\powerDispersionProbCaption}[2]{
    The tendency of model #2 to support one divergence event when there is
    random variation in divergence times as simulated under the series of
    models #1.
    The plots illustrate histograms of the estimated posterior probability of
    the one divergence model, $p(\divTimeDispersion < 0.01 | \ssSpace)$, from
    analyses of 1000 datasets simulated under each of the #1 models, with the
    the estimated probability of the model strongly supporting one divergence
    event, $p(BF_{\divTimeDispersion < 0.01, \divTimeDispersion \geq 0.01} >
    10)$, given for each data model.}
\newcommand{\simulationDescription}[2]{\change{Each plot represents #1
    simulation replicates using the same $#2$ samples from the prior}}
\newcommand{\simulationDistribution}{\ensuremath{\divt{} \sim U(0,
    \divt{max})}\xspace}
\newcommand{\estimateDescription}[2]{All estimates were obtained using #1 and #2}
\newcommand{\estimateDescriptionUncorrected}[1]{All estimates based on
    unadjusted posterior, \truncprior{}, obtained using #1}
\newcommand{\priorDescription}[4]{Prior settings were \priorSettings{#1}{#2}{#3}{#4}}
\newcommand{\priorSettings}[4]{$\divt{} \sim U(0, #1)$,
    $\meanDescendantTheta{} \sim U(#2, #3)$, and
    $\ancestralTheta{}{} \sim U(#2, #4)$}
\newcommand{\priorDescriptionBug}[4]{Prior settings were
    \priorSettingsBug{#1}{#2}{#3}{#4}}
\newcommand{\priorSettingsBug}[4]{$\divt{} \sim U(0, #1)$,
    $\meanDescendantTheta{} \sim U(#2, #3)$, and
    $\ancestralTheta{}{} \sim U(0.01, #4)$}
\newcommand{\simulationScheme}{simulations where \divt{} (in \globalcoalunit
    generations) for 22 population pairs is drawn from a series of uniform
    distributions, \simulationDistribution}
\newcommand{\captionPowerOmega}{Histograms of the estimated dispersion index
    of divergence times ($\hat{\vmratio{}}$) from \simulationScheme.
    The threshold for one divergence event \citep{Hickerson2006} is indicated
    by the dashed line, and the estimated probability of inferring one
    divergence event, $p(\hat{\vmratio{}}\le 0.01)$, is given for each
    \divt{max}}
\newcommand{\captionPowerPsiMode}{Histograms of the estimated number of
    divergence events ($\hat{\numt{}}$) from \simulationScheme.
    The estimated probability of inferring one divergence event,
    $p(\hat{\numt{}} = 1)$, is given for each \divt{max}}
\newcommand{\captionPowerPsi}{Histograms of the estimated posterior
    probability of one divergence event, \postprobnumt{1}, from
    \simulationScheme.
    The estimated probability of inferring one divergence event with a
    Bayes factor greater than 10 (dashed black line),
    $p(\bayesfactor{\numt{}=1}{\numt{} \ne 1} > 10)$, is given for each \divt{max}.
    The red line indicates $\postprobnumt{1} = 0.95$, and the estimated
    probability of inferring a posterior probability greater than 0.95 is given
    to the right of the line.}
\newcommand{\captionAccuracy}[1]{Accuracy and precision of #1 estimates from
    \simulationScheme.
    The proportion of estimates less than the true value ($p(\hat{#1}<#1)$) is
    given for each \divt{max}}
\newcommand{\samplingErrorTableNote}{An estimate of 1.0 for a posterior probability
    is an artifact of sampling error}


\newcommand{\refAccuracyALL}[1]{\labelcref{fig_acc_t_ss_llr_bug,fig_acc_t_ss_glm_bug,fig_acc_t_pls_llr_bug,fig_acc_t_pls_glm_bug,fig_acc_o_ss_llr_bug,fig_acc_o_ss_glm_bug,fig_acc_o_pls_llr_bug,fig_acc_o_pls_glm_bug}}
\newcommand{\refAccuracySS}[1]{\labelcref{fig_acc_t_ss_llr_bug,fig_acc_t_ss_glm_bug,fig_acc_o_ss_llr_bug,fig_acc_o_ss_glm_bug}}
\newcommand{\refAccuracySSfull}[1]{\labelcref{fig_acc_t_ssfull_llr_bug,fig_acc_t_ssfull_glm_bug,fig_acc_o_ssfull_llr_bug,fig_acc_o_ssfull_glm_bug}}
\newcommand{\refSSfull}[1]{\labelcref{fig_acc_t_ssfull_llr_bug,fig_acc_t_ssfull_glm_bug,fig_acc_o_ssfull_llr_bug,fig_acc_o_ssfull_glm_bug,fig_pow_o_ssfull_llr_bug,fig_pow_o_ssfull_glm_bug,fig_pow_psi_modes_ssfull_glm_bug}}
\newcommand{\refSS}[1]{\labelcref{fig_acc_t_ss_llr_bug,fig_acc_t_ss_glm_bug,fig_acc_o_ss_llr_bug,fig_acc_o_ss_glm_bug,fig_pow_o_ss_llr_bug,fig_pow_o_ss_glm_bug,fig_pow_psi_ss}}
\newcommand{\refAccuracyPLS}[1]{\labelcref{fig_acc_t_pls_llr_bug,fig_acc_t_pls_glm_bug,fig_acc_o_pls_llr_bug,fig_acc_o_pls_glm_bug}}
\newcommand{\refAccuracySScorrected}[1]{\labelcref{fig_acc_t_ss_llr_bug,fig_acc_t_ss_glm_bug,fig_acc_o_ss_llr_bug,fig_acc_o_ss_glm_bug}}
\newcommand{\refAccuracyPLScorrected}[1]{\labelcref{fig_acc_t_pls_llr_bug,fig_acc_t_pls_glm_bug,fig_acc_o_pls_llr_bug,fig_acc_o_pls_glm_bug}}
\newcommand{\refAccuracyUncorrected}[1]{\labelcref{fig_acc_t_ss_unc,fig_acc_t_pls_unc,fig_acc_o_ss_unc,fig_acc_o_pls_unc}}
\newcommand{\refAccuracyCorrected}[1]{\labelcref{fig_acc_t_ss_llr_bug,fig_acc_t_ss_glm_bug,fig_acc_t_pls_llr_bug,fig_acc_t_pls_glm_bug,fig_acc_o_ss_llr_bug,fig_acc_o_ss_glm_bug,fig_acc_o_pls_llr_bug,fig_acc_o_pls_glm_bug}}
\newcommand{\refAccuracyGLM}[1]{\labelcref{fig_acc_t_ss_glm_bug,fig_acc_t_pls_glm_bug,fig_acc_o_ss_glm_bug,fig_acc_o_pls_glm_bug}}
\newcommand{\refAccuracyLLR}[1]{\labelcref{fig_acc_t_ss_llr_bug,fig_acc_t_pls_llr_bug,fig_acc_o_ss_llr_bug,fig_acc_o_pls_llr_bug}}
\newcommand{\refAccuracyOmega}[1]{\labelcref{fig_acc_o_ss_llr_bug,fig_acc_o_ss_glm_bug,fig_acc_o_pls_llr_bug,fig_acc_o_pls_glm_bug}}
\newcommand{\refAccuracyOmegaUncorrected}[1]{\labelcref{fig_acc_o_ss_unc,fig_acc_o_pls_unc}}
\newcommand{\refAccuracyOmegaCorrected}[1]{\labelcref{fig_acc_o_ss_llr_bug,fig_acc_o_ss_glm_bug,fig_acc_o_pls_llr_bug,fig_acc_o_pls_glm_bug}}
\newcommand{\refAccuracyTime}[1]{\labelcref{fig_acc_t_ss_llr_bug,fig_acc_t_ss_glm_bug,fig_acc_t_pls_llr_bug,fig_acc_t_pls_glm_bug}}

\newcommand{\tn}{\tabularnewline}

\newcommand{\widthFigure}[5]{\begin{figure}[htbp]
\begin{center}
    \includegraphics[width=#1\textwidth]{#2}
    \captionsetup{#3}
    \caption{#4}
    \label{#5}
    \end{center}
    \end{figure}}

\newcommand{\heightFigure}[5]{\begin{figure}[htbp]
\begin{center}
    \includegraphics[height=#1]{#2}
    \captionsetup{#3}
    \caption{#4}
    \label{#5}
    \end{center}
    \end{figure}}

\newcommand{\widthSidewaysFigure}[5]{\begin{sidewaysfigure}[htbp]
\begin{center}
    \includegraphics[width=#1\textwidth]{#2}
    \captionsetup{#3}
    \caption{#4}
    \label{#5}
    \end{center}
    \end{sidewaysfigure}}

\newcommand{\heightSidewaysFigure}[5]{\begin{sidewaysfigure}[htbp]
\begin{center}
    \includegraphics[height=#1]{#2}
    \captionsetup{#3}
    \caption{#4}
    \label{#5}
    \end{center}
    \end{sidewaysfigure}}

\newcommand{\mFigure}[3]{\widthFigure{1.0}{#1}{listformat=figList}{#2}{#3}\clearpage}
\newcommand{\siFigure}[3]{\widthFigure{1.0}{#1}{name=Figure S, labelformat=noSpace, listformat=sFigList}{#2}{#3}\clearpage}
\newcommand{\siEightFigure}[3]{\widthFigure{0.8}{#1}{name=Figure S, labelformat=noSpace, listformat=sFigList}{#2}{#3}\clearpage}
\newcommand{\siSidewaysFigure}[3]{\widthSidewaysFigure{1.0}{#1}{name=Figure S, labelformat=noSpace, listformat=sFigList}{#2}{#3}\clearpage}

