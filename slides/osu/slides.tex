\documentclass[table]{beamer}
\usepackage{beamerthemesplit}
\usetheme{boxes}
\usecolortheme{seahorse}
% \useinnertheme{myboxes}
% \usepackage{amsmath}
% \usepackage[fleqn]{amsmath}
\usepackage{ifthen}
\usepackage{xspace}
\usepackage{multirow}
\usepackage{booktabs}
\usepackage{xcolor}
\usepackage[style=nature]{biblatex}
\bibliography{../manuscripts/bib/references}
\newrobustcmd*{\footlessfullcite}{\AtNextCite{\renewbibmacro{title}{}\renewbibmacro{in:}{}}\footfullcite}
\AtEveryBibitem{\clearfield{month}}
\AtEveryCitekey{\clearfield{month}}

% Make all footnotes smaller
\renewcommand{\footnotesize}{\scriptsize}

\definecolor{myGray}{gray}{0.9}
\colorlet{rowred}{red!30!white}

\setbeamertemplate{blocks}[rounded][shadow=true]

\setbeamercolor{defaultcolor}{bg=structure!30!normal text.bg,fg=black}
\setbeamercolor{block body}{bg=structure!30!normal text.bg,fg=black}
\setbeamercolor{block title}{bg=structure!50!normal text.bg,fg=black}

\newenvironment<>{varblock}[2][\textwidth]{%
  \setlength{\textwidth}{#1}
  \begin{actionenv}#3%
    \def\insertblocktitle{#2}%
    \par%
    \usebeamertemplate{block begin}}
  {\par%
    \usebeamertemplate{block end}%
  \end{actionenv}}

\newenvironment{displaybox}[1][\textwidth]
{
    \centerline\bgroup\hfill
    \begin{beamerboxesrounded}[lower=defaultcolor,shadow=true,width=#1]{}
}
{
    \end{beamerboxesrounded}\hfill\egroup
}

\newenvironment{onlinebox}[1][4cm]
{
    \newbox\mybox
    \newdimen\myboxht
    \setbox\mybox\hbox\bgroup%
        \begin{beamerboxesrounded}[lower=defaultcolor,shadow=true,width=#1]{}
    \centering
}
{
    \end{beamerboxesrounded}\egroup
    \myboxht\ht\mybox
    \raisebox{-0.25\myboxht}{\usebox\mybox}\hspace{2pt}
}

\newenvironment{mydescription}{
    \begin{description}
        \setlength{\leftskip}{-1.5cm}}
    {\end{description}}

\newenvironment{myitemize}{
    \begin{itemize}
        \setlength{\leftskip}{-.3cm}}
    {\end{itemize}}

% define formatting for footer
\newcommand{\myfootline}{%
    {\it
    \insertshorttitle
    \hspace*{\fill} 
    \insertshortauthor, \insertshortinstitute
    % \ifx\insertsubtitle\@empty\else, \insertshortsubtitle\fi
    \hspace*{\fill}
    \insertframenumber/\inserttotalframenumber}}

% set up footer
\setbeamertemplate{footline}{%
    \usebeamerfont{structure}
    \begin{beamercolorbox}[wd=\paperwidth,ht=2.25ex,dp=1ex]{frametitle}%
        \Tiny\hspace*{4mm}\myfootline\hspace{4mm}
    \end{beamercolorbox}}

% remove navigation bar
\beamertemplatenavigationsymbolsempty


% \newcommand{\change}[1]{{\color{blue} #1}\xspace}
\newcommand{\change}[1]{{\color{black} #1}\xspace}


\newcommand{\citationNeeded}{\textcolor{magenta}{\textbf{[CITATION NEEDED!]}}\xspace}
\newcommand{\tableNeeded}{\textcolor{magenta}{\textbf{[TABLE NEEDED!]}}\xspace}
\newcommand{\figureNeeded}{\textcolor{magenta}{\textbf{[FIGURE NEEDED!]}}\xspace}
\newcommand{\highLight}[1]{\textcolor{magenta}{\MakeUppercase{#1}}}

\newcommand{\editorialNote}[1]{\textcolor{red}{[\textit{#1}]}}
\newcommand{\ignore}[1]{}
\newcommand{\addTail}[1]{\textit{#1}.---}
\newcommand{\super}[1]{\ensuremath{^{\textrm{#1}}}}
\newcommand{\sub}[1]{\ensuremath{_{\textrm{#1}}}}
\newcommand{\dC}{\ensuremath{^\circ{\textrm{C}}}}

\providecommand{\e}[1]{\ensuremath{\times 10^{#1}}}

\newcommand{\mthnote}[2]{{\color{red} #2}}

\newcommand{\ifTwoArgs}[3]{\ifthenelse{\equal{#1}{}\or\equal{#2}{}}{}{#3}\xspace}
\newcommand{\ifArg}[2]{\ifthenelse{\equal{#1}{}}{}{#2}\xspace}

\newcommand{\allDatasets}{\ensuremath{\mathcal{\alignment{}{}}}\xspace}
\newcommand{\allParameterValues}{\ensuremath{\boldsymbol{\Theta}}\xspace}
\newcommand{\bayesfactor}[2]{\ensuremath{BF_{#1\protect\ifArg{#2}{,}#2}}}
\newcommand{\given}{\ensuremath{\,|\,}\xspace}
\newcommand{\msb}{\upshape\texttt{\MakeLowercase{ms\MakeUppercase{B}ayes}}\xspace}
\newcommand{\hky}{HKY85\xspace}
\newcommand{\uniformMin}[1]{\ensuremath{a_{#1}}\xspace}
\newcommand{\uniformMax}[1]{\ensuremath{b_{#1}}\xspace}
\newcommand{\locusRateHetShapeParameter}{\ensuremath{\alpha}\xspace}
\newcommand{\ancestralThetaVector}{\ensuremath{\boldsymbol{\theta_{A}}}\xspace}
\newcommand{\descendantThetaVector}[1]{\ensuremath{\boldsymbol{\theta_{D#1}}}\xspace}
\newcommand{\divtscaledvector}{\ensuremath{\mathbf{{\divtscaled{}{}}}}\xspace}
\newcommand{\divtvector}{\ensuremath{\boldsymbol{\divt{}}}\xspace}
\newcommand{\divtuniquevector}{\ensuremath{\mathbf{\divtunique{}}}\xspace}
\newcommand{\bottleTimeVector}{\ensuremath{\boldsymbol{\bottleTime{}}}\xspace}
\newcommand{\bottleTime}[1]{\ensuremath{\divt{B\ifArg{#1}{,}#1}}\xspace}
\newcommand{\bottleScalarVector}[1]{\ensuremath{\boldsymbol{\bottleScalar{#1}{}}}\xspace}
\newcommand{\bottleScalar}[2]{\ensuremath{\zeta_{D#1\protect\ifArg{#2}{,}#2}}\xspace}
\newcommand{\migrationRateVector}{\ensuremath{\mathbf{\migrationRate{}}}\xspace}
\newcommand{\geneTreeVector}{\ensuremath{\mathbf{\geneTree{}{}}}\xspace}
\newcommand{\alignmentVector}{\ensuremath{\mathbf{\alignment{}{}}}\xspace}
\newcommand{\alignment}[2]{\ensuremath{X_{#1\protect\ifTwoArgs{#1}{#2}{,}#2}}\xspace}
\newcommand{\geneTree}[2]{\ensuremath{G_{#1\protect\ifTwoArgs{#1}{#2}{,}#2}}\xspace}
\newcommand{\migrationRate}[1]{\ensuremath{m_{#1}}\xspace}
\newcommand{\recombinationRate}{\ensuremath{r}\xspace}
\newcommand{\ploidyScalar}[2]{\ensuremath{\rho_{#1\protect\ifTwoArgs{#1}{#2}{,}#2}}\xspace}
\newcommand{\ploidyScalarVector}{\ensuremath{\boldsymbol{\ploidyScalar{}{}}}\xspace}
\newcommand{\descendantRelativeThetaVector}[1]{\ensuremath{\boldsymbol{\eta_{D#1}}}\xspace}
\newcommand{\descendantRelativeTheta}[2]{\ensuremath{\eta_{D#1\protect\ifArg{#2}{,}#2}}\xspace}
\newcommand{\mutationRateScalarConstant}[2]{\ensuremath{\nu_{#1\protect\ifTwoArgs{#1}{#2}{,}#2}}\xspace}
\newcommand{\mutationRateScalarConstantVector}{\ensuremath{\boldsymbol{\mutationRateScalarConstant{}{}}}\xspace}
\newcommand{\locusMutationRateScalar}[1]{\ensuremath{\upsilon_{#1}}\xspace}
\newcommand{\locusMutationRateScalarVector}{\ensuremath{\boldsymbol{\upsilon}}\xspace}
\newcommand{\hkyModel}[2]{\ensuremath{\phi_{#1\protect\ifTwoArgs{#1}{#2}{,}#2}}\xspace}
\newcommand{\hkyModelVector}{\ensuremath{\boldsymbol{\hkyModel{}{}}}\xspace}
\newcommand{\mutationRate}{\ensuremath{\mu}\xspace}
\newcommand{\iid}{\textit{iid}\xspace}
\newcommand{\model}[1]{\ensuremath{\Theta}\xspace}
\newcommand{\npairs}[1]{\ensuremath{Y_{#1}}}
\newcommand{\nloci}[1]{\ensuremath{k_{#1}}\xspace}
\newcommand{\nlociTotal}{\ensuremath{K}\xspace}
\newcommand{\myTheta}[1]{\ensuremath{\theta_{#1}}}
\newcommand{\ancestralTheta}[1]{\ensuremath{\theta_{A\protect\ifArg{#1}{,}#1}}\xspace}
\newcommand{\descendantTheta}[2]{\ensuremath{\theta_{D#1\protect\ifArg{#2}{,}#2}}\xspace}
\newcommand{\meanDescendantTheta}[1]{\ensuremath{\descendantTheta{}{#1}}\xspace}
\newcommand{\nucdiv}[1]{\ensuremath{\pi_{#1}}}

\newcommand{\ssVector}[1]{\ensuremath{\mathbf{\alignmentSS{#1}{}}}\xspace}
\newcommand{\ssVectorObs}{\ensuremath{\ssVector{}^*}\xspace}
\newcommand{\ssSpace}{\ensuremath{\euclideanSpace{\ssVectorObs}}\xspace}
\newcommand{\ssVectorObsPLS}{\ensuremath{\ssVectorObs_{PLS}}\xspace}
\newcommand{\alignmentSS}[2]{\ensuremath{S_{#1\protect\ifTwoArgs{#1}{#2}{,}#2}}\xspace}
\newcommand{\alignmentSSObs}[2]{\ensuremath{\alignmentSS{#1}{#2}^*}\xspace}
\newcommand{\tol}{\ensuremath{\epsilon}\xspace}
\newcommand{\euclideanSpace}[1]{\ensuremath{B_{\tol}(#1)}\xspace}
\newcommand{\hpvector}[1]{\ensuremath{\Lambda_{#1}}}
\newcommand{\divtscaled}[2]{\ensuremath{t_{#1\protect\ifTwoArgs{#1}{#2}{,}#2}}}
\newcommand{\divt}[1]{\ensuremath{\tau_{#1}}}
\newcommand{\divtunique}[1]{\ensuremath{T_{#1}}}
\newcommand{\ssMatrix}{\ensuremath{\mathbb \alignmentSS{}{}}\xspace}
\newcommand{\ssMatrixRaw}[1]{\ensuremath{{\ssMatrix}_{stats#1}}\xspace}
\newcommand{\ssMatrixPLS}[1]{\ensuremath{{\ssMatrix}_{PLS#1}}\xspace}
\newcommand{\hpmatrix}[1]{\ensuremath{\mathcal{P}_{#1}}}
\newcommand{\meant}[2]{\ensuremath{E(\divt{#1})_{#2}}}
\newcommand{\meantestimate}{\ensuremath{\hat{E(\divt{})}}\xspace}
\newcommand{\vart}[2]{\ensuremath{Var(\divt{#1}{})_{#2}}}
\newcommand{\vmratio}[1]{\ensuremath{\Omega_{#1}}}
\newcommand{\numt}[1]{\ensuremath{\Psi_{#1}}}
\newcommand{\probnumt}[2]{\ensuremath{p(\numt{#1} = {#2})}}
\newcommand{\postprobnumt}[1]{\ensuremath{p(\numt{} = {#1}|\ssSpace)}}
\newcommand{\postprobnumtnot}[1]{\ensuremath{p(\numt{} \neq {#1}|\ssSpace)}}
\newcommand{\postprobomegasimult}{\ensuremath{p(\vmratio{} < 0.01 | \ssSpace)}\xspace}
\newcommand{\modelprior}[1]{\ensuremath{f(\model{})}}
\newcommand{\modelpost}[1]{\ensuremath{f(\model{}|\ssSpace)}}
\newcommand{\npriorsamples}{\ensuremath{n}\xspace}
\newcommand{\globalcoalunit}{\ensuremath{4\globalpopsize}\xspace}
\newcommand{\globalpopsize}{\ensuremath{N_C}\xspace}
\newcommand{\effectivePopSize}[1]{\ensuremath{N_e{#1}}\xspace}
\newcommand{\coalunit}{\ensuremath{4\effectivePopSize{}}\xspace}
\newcommand{\priorsample}[1]{\ensuremath{\hpmatrix{\modelprior{}}}}
\newcommand{\truncprior}[1]{\ensuremath{\hpmatrix{\tol}}\xspace}
\newcommand{\postsample}[1]{\ensuremath{\hpmatrix{\modelpost{}}}}
\newcommand{\abcllr}[1]{ABC\sub{LLR}}
\newcommand{\abcglm}[1]{ABC\sub{GLM}}
\newcommand{\integerPartition}[1]{\ensuremath{a({#1})}}
\newcommand{\uniqueModel}[2]{\ensuremath{M_{#1\protect\ifTwoArgs{#1}{#2}{,}#2}}}
\newcommand{\taxonLocusVector}[1]{\ensuremath{\{#1{1}{1},\ldots,#1{\npairs{}}{\nloci{\npairs{}}}\}}\xspace}
\newcommand{\taxonVector}[1]{\ensuremath{\{#1{1},\ldots,#1{\npairs{}}\}}\xspace}
\newcommand{\locusVector}[1]{\ensuremath{\{#1{1},\ldots,#1{\nlociTotal}\}}\xspace}

\newcommand{\simulationDescription}[2]{\change{Each plot represents #1
    simulation replicates using the same $#2$ samples from the prior}}
\newcommand{\simulationDistribution}{\ensuremath{\divt{} \sim U(0,
    \divt{max})}\xspace}
\newcommand{\estimateDescription}[2]{All estimates were obtained using #1 and #2}
\newcommand{\estimateDescriptionUncorrected}[1]{All estimates based on
    unadjusted posterior, \truncprior{}, obtained using #1}
\newcommand{\priorDescription}[4]{Prior settings were \priorSettings{#1}{#2}{#3}{#4}}
\newcommand{\priorSettings}[4]{$\divt{} \sim U(0, #1)$,
    $\meanDescendantTheta{} \sim U(#2, #3)$, and
    $\ancestralTheta{}{} \sim U(#2, #4)$}
\newcommand{\priorDescriptionBug}[4]{Prior settings were
    \priorSettingsBug{#1}{#2}{#3}{#4}}
\newcommand{\priorSettingsBug}[4]{$\divt{} \sim U(0, #1)$,
    $\meanDescendantTheta{} \sim U(#2, #3)$, and
    $\ancestralTheta{}{} \sim U(0.01, #4)$}
\newcommand{\simulationScheme}{simulations where \divt{} (in \globalcoalunit
    generations) for 22 population pairs is drawn from a series of uniform
    distributions, \simulationDistribution}
\newcommand{\captionPowerOmega}{Histograms of the estimated dispersion index
    of divergence times ($\hat{\vmratio{}}$) from \simulationScheme.
    The threshold for one divergence event \citep{Hickerson2006} is indicated
    by the dashed line, and the estimated probability of inferring one
    divergence event, $p(\hat{\vmratio{}}\le 0.01)$, is given for each
    \divt{max}}
\newcommand{\captionPowerPsiMode}{Histograms of the estimated number of
    divergence events ($\hat{\numt{}}$) from \simulationScheme.
    The estimated probability of inferring one divergence event,
    $p(\hat{\numt{}} = 1)$, is given for each \divt{max}}
\newcommand{\captionPowerPsi}{Histograms of the estimated posterior
    probability of one divergence event, \postprobnumt{1}, from
    \simulationScheme.
    The estimated probability of inferring one divergence event with a
    Bayes factor greater than 10 (dashed black line),
    $p(\bayesfactor{\numt{}=1}{\numt{} \ne 1} > 10)$, is given for each \divt{max}.
    The red line indicates $\postprobnumt{1} = 0.95$, and the estimated
    probability of inferring a posterior probability greater than 0.95 is given
    to the right of the line.}
\newcommand{\captionAccuracy}[1]{Accuracy and precision of #1 estimates from
    \simulationScheme.
    The proportion of estimates less than the true value ($p(\hat{#1}<#1)$) is
    given for each \divt{max}}
\newcommand{\samplingErrorTableNote}{An estimate of 1.0 for a posterior probability
    is an artifact of sampling error}


\newcommand{\refAccuracyALL}[1]{\labelcref{fig_acc_t_ss_llr_bug,fig_acc_t_ss_glm_bug,fig_acc_t_pls_llr_bug,fig_acc_t_pls_glm_bug,fig_acc_o_ss_llr_bug,fig_acc_o_ss_glm_bug,fig_acc_o_pls_llr_bug,fig_acc_o_pls_glm_bug}}
\newcommand{\refAccuracySS}[1]{\labelcref{fig_acc_t_ss_llr_bug,fig_acc_t_ss_glm_bug,fig_acc_o_ss_llr_bug,fig_acc_o_ss_glm_bug}}
\newcommand{\refAccuracySSfull}[1]{\labelcref{fig_acc_t_ssfull_llr_bug,fig_acc_t_ssfull_glm_bug,fig_acc_o_ssfull_llr_bug,fig_acc_o_ssfull_glm_bug}}
\newcommand{\refSSfull}[1]{\labelcref{fig_acc_t_ssfull_llr_bug,fig_acc_t_ssfull_glm_bug,fig_acc_o_ssfull_llr_bug,fig_acc_o_ssfull_glm_bug,fig_pow_o_ssfull_llr_bug,fig_pow_o_ssfull_glm_bug,fig_pow_psi_modes_ssfull_glm_bug}}
\newcommand{\refSS}[1]{\labelcref{fig_acc_t_ss_llr_bug,fig_acc_t_ss_glm_bug,fig_acc_o_ss_llr_bug,fig_acc_o_ss_glm_bug,fig_pow_o_ss_llr_bug,fig_pow_o_ss_glm_bug,fig_pow_psi_ss}}
\newcommand{\refAccuracyPLS}[1]{\labelcref{fig_acc_t_pls_llr_bug,fig_acc_t_pls_glm_bug,fig_acc_o_pls_llr_bug,fig_acc_o_pls_glm_bug}}
\newcommand{\refAccuracySScorrected}[1]{\labelcref{fig_acc_t_ss_llr_bug,fig_acc_t_ss_glm_bug,fig_acc_o_ss_llr_bug,fig_acc_o_ss_glm_bug}}
\newcommand{\refAccuracyPLScorrected}[1]{\labelcref{fig_acc_t_pls_llr_bug,fig_acc_t_pls_glm_bug,fig_acc_o_pls_llr_bug,fig_acc_o_pls_glm_bug}}
\newcommand{\refAccuracyUncorrected}[1]{\labelcref{fig_acc_t_ss_unc,fig_acc_t_pls_unc,fig_acc_o_ss_unc,fig_acc_o_pls_unc}}
\newcommand{\refAccuracyCorrected}[1]{\labelcref{fig_acc_t_ss_llr_bug,fig_acc_t_ss_glm_bug,fig_acc_t_pls_llr_bug,fig_acc_t_pls_glm_bug,fig_acc_o_ss_llr_bug,fig_acc_o_ss_glm_bug,fig_acc_o_pls_llr_bug,fig_acc_o_pls_glm_bug}}
\newcommand{\refAccuracyGLM}[1]{\labelcref{fig_acc_t_ss_glm_bug,fig_acc_t_pls_glm_bug,fig_acc_o_ss_glm_bug,fig_acc_o_pls_glm_bug}}
\newcommand{\refAccuracyLLR}[1]{\labelcref{fig_acc_t_ss_llr_bug,fig_acc_t_pls_llr_bug,fig_acc_o_ss_llr_bug,fig_acc_o_pls_llr_bug}}
\newcommand{\refAccuracyOmega}[1]{\labelcref{fig_acc_o_ss_llr_bug,fig_acc_o_ss_glm_bug,fig_acc_o_pls_llr_bug,fig_acc_o_pls_glm_bug}}
\newcommand{\refAccuracyOmegaUncorrected}[1]{\labelcref{fig_acc_o_ss_unc,fig_acc_o_pls_unc}}
\newcommand{\refAccuracyOmegaCorrected}[1]{\labelcref{fig_acc_o_ss_llr_bug,fig_acc_o_ss_glm_bug,fig_acc_o_pls_llr_bug,fig_acc_o_pls_glm_bug}}
\newcommand{\refAccuracyTime}[1]{\labelcref{fig_acc_t_ss_llr_bug,fig_acc_t_ss_glm_bug,fig_acc_t_pls_llr_bug,fig_acc_t_pls_glm_bug}}

\newcommand{\tn}{\tabularnewline}

\newcommand{\widthFigure}[5]{\begin{figure}[htbp]
\begin{center}
    \includegraphics[width=#1\textwidth]{#2}
    \captionsetup{#3}
    \caption{#4}
    \label{#5}
    \end{center}
    \end{figure}}

\newcommand{\heightFigure}[5]{\begin{figure}[htbp]
\begin{center}
    \includegraphics[height=#1]{#2}
    \captionsetup{#3}
    \caption{#4}
    \label{#5}
    \end{center}
    \end{figure}}

\newcommand{\mFigure}[3]{\widthFigure{1.0}{#1}{listformat=figList}{#2}{#3}\clearpage}
\newcommand{\siFigure}[3]{\widthFigure{1.0}{#1}{name=Figure S, labelformat=noSpace, listformat=sFigList}{#2}{#3}\clearpage}


\newcommand{\allParameters}[1]{\ensuremath{\theta_{#1}}\xspace}
\bibliography{../../manuscripts/bib/references}

\title[Improving inference of evolutionary history]{Improving inference of evolutionary history}
\subtitle{Developing comparative methods for genomic data}

\author[J.\ Oaks]{
    Jamie R.\ Oaks\inst{1}
}
\institute[University of Washington]{
    \inst{1}%
        Department of Biology, University of Washington
}

% \date{\today}
\date{November 3, 2014}

\begin{document}
\maketitle

% I. What I do
% II. What I've been focusing on recently (started in PhD and continuing in postdoc)
%   a. Dynamic planet---need to account for large-scale processes
%       i. E.g., SE Asia---dynamic landscape
\begin{frame}
    \frametitle{Processes of diversification}
        \begin{itemize}
            \item<1-> Large-scale geological and climatic processes are
                important in biodiversification and community assembly
            \item<2-> Accounting for such processes will better our
                understanding of biodiversity
            \item<3-> We need methods for inferring evolutionary patterns
                predicted by historical events from contemporary populations
        \end{itemize}
\end{frame}

{
\usebackgroundtemplate{\includegraphics[width=\paperwidth]{../images/maps/se-asia-present.png}}
\begin{frame}
    \frametitle{Southeast Asia}    
\end{frame}
}

{
\usebackgroundtemplate{\includegraphics[width=\paperwidth]{../images/maps/se-asia-120.png}}
\begin{frame}
    \frametitle{Southeast Asia}    
\end{frame}
}

%   b. Predictions of climate-driven diversification
\begin{frame}
    \frametitle<1-4>{Community scale processes}
    \frametitle<5->{Divergence model choice}
    \begin{columns}[c]
        \column{.4\textwidth}
        \begin{onlyenv}<1-4>
            \begin{minipage}[c][0.5\textheight][c]{\linewidth}
            \end{minipage}
        \end{onlyenv}
        \begin{onlyenv}<5>
            \begin{minipage}[c][0.5\textheight][c]{\linewidth}
                \begin{displaybox}[0.95\linewidth]
                    \begin{minipage}[c][0.45\textheight][c]{0.95\linewidth}
                        \[
                            \divTimeMapVector = (\divTimeMap{1},\divTimeMap{2},\divTimeMap{3})
                        \]\vspace{0mm}
                        \[
                            model = 111
                        \]\vspace{0mm}
                        \[
                            \divTimeVector = \{\divTime{1}\}
                        \]\vspace{0mm}
                    \end{minipage}
                \end{displaybox}
            \end{minipage}
        \end{onlyenv}
        \begin{onlyenv}<6>
            \begin{minipage}[c][0.5\textheight][c]{\linewidth}
                \begin{displaybox}[0.95\linewidth]
                    \begin{minipage}[c][0.45\textheight][c]{0.95\linewidth}
                        \[
                            \divTimeMapVector = (260,260,260)
                        \]\vspace{0mm}
                        \[
                            model = 111
                        \]\vspace{0mm}
                        \[
                            \divTimeVector = \{260\}
                        \]\vspace{0mm}
                    \end{minipage}
                \end{displaybox}
            \end{minipage}
        \end{onlyenv}
        \begin{onlyenv}<7>
            \begin{minipage}[c][0.5\textheight][c]{\linewidth}
                \begin{displaybox}[0.95\linewidth]
                    \begin{minipage}[c][0.45\textheight][c]{0.95\linewidth}
                        \[
                            \divTimeMapVector = (397,260,260)
                        \]\vspace{0mm}
                        \[
                            model = 211
                        \]\vspace{0mm}
                        \[
                            \divTimeVector = \{260,397\}
                        \]\vspace{0mm}
                    \end{minipage}
                \end{displaybox}
            \end{minipage}
        \end{onlyenv}
        \begin{onlyenv}<8>
            \begin{minipage}[c][0.5\textheight][c]{\linewidth}
                \begin{displaybox}[0.95\linewidth]
                    \begin{minipage}[c][0.45\textheight][c]{0.95\linewidth}
                        \[
                            \divTimeMapVector = (260,397,260)
                        \]\vspace{0mm}
                        \[
                            model = 121
                        \]\vspace{0mm}
                        \[
                            \divTimeVector = \{260,397\}
                        \]\vspace{0mm}
                    \end{minipage}
                \end{displaybox}
            \end{minipage}
        \end{onlyenv}
        \begin{onlyenv}<9>
            \begin{minipage}[c][0.5\textheight][c]{\linewidth}
                \begin{displaybox}[0.95\linewidth]
                    \begin{minipage}[c][0.45\textheight][c]{0.95\linewidth}
                        \[
                            \divTimeMapVector = (260,260,397)
                        \]\vspace{0mm}
                        \[
                            model = 112
                        \]\vspace{0mm}
                        \[
                            \divTimeVector = \{260,397\}
                        \]\vspace{0mm}
                    \end{minipage}
                \end{displaybox}
            \end{minipage}
        \end{onlyenv}
        \begin{onlyenv}<10>
            \begin{minipage}[c][0.5\textheight][c]{\linewidth}
                \begin{displaybox}[0.95\linewidth]
                    \begin{minipage}[c][0.45\textheight][c]{0.95\linewidth}
                        \[
                            \divTimeMapVector = (260,95,397)
                        \]\vspace{0mm}
                        \[
                            model = 123
                        \]\vspace{0mm}
                        \[
                            \divTimeVector = \{260,95,397\}
                        \]\vspace{0mm}
                    \end{minipage}
                \end{displaybox}
            \end{minipage}
        \end{onlyenv}
        % \begin{onlyenv}<11>
        %     \begin{minipage}[c][0.5\textheight][c]{\linewidth}
        %         \begin{displaybox}[0.95\linewidth]
        %             \begin{minipage}[c][0.45\textheight][c]{0.95\linewidth}
        %                 \[
        %                     \divTimeMapVector = (260,260,260)
        %                 \]\vspace{0mm}
        %                 \[
        %                     model = 111
        %                 \]\vspace{0mm}
        %                 \[
        %                     \divTimeVector = \{260\}
        %                 \]\vspace{0mm}
        %             \end{minipage}
        %         \end{displaybox}
        %     \end{minipage}
        % \end{onlyenv}
        \begin{onlyenv}<11-13>
            \begin{minipage}[c][0.5\textheight][c]{\linewidth}
                \begin{displaybox}[0.95\linewidth]
                    \begin{minipage}[c][0.45\textheight][c]{0.95\linewidth}
                        \[
                            \divTimeMapVector = (\divTimeMap{1},\ldots,\divTimeMap{\npairs{}})
                        \]\vspace{0mm}
                        \[
                            model = \divModel{i}
                        \]\vspace{0mm}
                        \[
                            \divTimeVector = \{\divTime{1},\ldots,\divTime{\divTimeNum}\}
                        \]\vspace{0mm}
                    \end{minipage}
                \end{displaybox}
            \end{minipage}
        \end{onlyenv}
        \begin{onlyenv}<14>
            \begin{minipage}[c][0.5\textheight][c]{\linewidth}
                \begin{mydescription}
                    \item[\alignmentVector] Sequence alignments
                    \item[\divTimeMapVector] Divergence times
                    \item[\divModel{}] Divergence model
                    \item[\geneTreeVector] Gene trees
                    \item[\hkyModelVector] Substitution parameters
                    \item[\demographicParamVector] Demographic parameters
                    % \item[\allParameters] $(\divTimeMapVector, \geneTreeVector,
                    %     \hkyModelVector, \demographicParamVector)$
                \end{mydescription}
            \end{minipage}
        \end{onlyenv}
        \begin{uncoverenv}<12->
            \begin{minipage}[c][0.18\textheight][c]{\linewidth}
            \begin{flushleft}
                \small We want to infer \textcolor{blue}{\divModel{}} and
                \textcolor{blue}{\divTimeMapVector} given DNA sequence
                alignments
                \textcolor{blue}{\alignmentVector}
            \end{flushleft}
            \end{minipage}
        \end{uncoverenv}
        \column{.6\textwidth}
        \begin{minipage}[t][0.8\textheight][c]{\linewidth}
        \includegraphics<1>[height=6cm]{../images/community-ellipse-small.pdf}
        % \includegraphics<2>[height=6cm]{../images/community-ellipse-barrier-small.pdf}
        \includegraphics<2>[height=6cm]{../images/community-ellipse-split-small.pdf}
        \includegraphics<3>[height=6cm]{../images/div-model-cartoon-taxa.pdf}
        \includegraphics<4>[height=6cm]{../images/div-model-cartoon-shared-no-labels.pdf}
        \includegraphics<5-6>[height=6cm]{../images/div-model-cartoon-shared.pdf}
        \includegraphics<7>[height=6cm]{../images/div-model-cartoon-2-1.pdf}
        \includegraphics<8>[height=6cm]{../images/div-model-cartoon-2-2.pdf}
        \includegraphics<9>[height=6cm]{../images/div-model-cartoon-2-3.pdf}
        \includegraphics<10>[height=6cm]{../images/div-model-cartoon-general.pdf}
        \includegraphics<11-12>[height=6cm]{../images/div-model-cartoon-shared.pdf}
        \includegraphics<13->[height=6cm]{../images/div-model-cartoon-sp-tree-shared.pdf}
        \end{minipage}
    \end{columns}
\end{frame}

%   c. Testing for predicted patterns via model choice.
%       i. Bayesian model choice

\begin{frame}[t]
    \frametitle{Bayesian model choice}
    \begin{block}{\it Full model:}
        \begin{minipage}[c][3.8cm][c]{\linewidth}
            \begin{uncoverenv}<1->
            \[
                p(
                  \divTimeMapVector,
                  \geneTreeVector,
                  \hkyModelVector,
                  \demographicParamVector
                  \given \alignmentVector, \divModel{i})
                  =
                \frac{
                    p(\alignmentVector \given
                      \divTimeMapVector,
                      \geneTreeVector,
                      \hkyModelVector,
                      \demographicParamVector,
                      \divModel{i}
                      )
                      p(
                        \divTimeMapVector,
                        \geneTreeVector,
                        \hkyModelVector,
                        \demographicParamVector
                        \given \divModel{i}
                        )
                    }{p(\alignmentVector \given \divModel{i})}
            \]\vspace{-1mm}
            \end{uncoverenv}
            \begin{uncoverenv}<2->
            \[
                p(\alignmentVector \given \divModel{i}) =
                \int_{\allParameters{i}}
                p(\alignmentVector \given \allParameters{i}, \divModel{i})
                p(\allParameters{i} \given \divModel{i})
                d\allParameters{i}
            \]\vspace{-1mm}
            \end{uncoverenv}
            \begin{uncoverenv}<3->
            \[
                p(\divModel{i} \given \alignmentVector) =
                \frac{
                    p(\alignmentVector \given \divModel{i})
                    p(\divModel{i})
                }{
                    \sum_{i} p(\alignmentVector \given \divModel{i})
                    p(\divModel{i})
                }
            \]\vspace{-1mm}
            \end{uncoverenv}
        \end{minipage}
    \end{block}
    \vspace{1mm}
    \begin{uncoverenv}<4>
        \begin{center}
            \msb: Approximate Bayesian computation (ABC)

            % \vspace{-5mm}
            % \[ \alignmentVector \, \to \, \ssVectorObs \, \to \, \ssSpace\]
        \end{center}
    \end{uncoverenv}
    \barefootnote{\tiny\shortfullcite{Huang2011}. \hspace{2mm}\shortfullcite{Oaks2012}.}
\end{frame}

%   d. Method exists---applied it to comparative dataset from Philippines
%       i. Surprising results!
%       ii. Rather than publish, did some detective work

\begin{frame}
\begin{columns}[c]
    \column{.5\textwidth}
        \begin{table}%[htbp]
            \scriptsize
            \addtolength{\tabcolsep}{-0.09cm}
            \centering
            \begin{tabular}{ l c c }
                % \toprule
                \textbf{Species} & {\boldmath $n_1$} & {\boldmath $n_2$} \\
                \hline
                \textbf{\textsc{Mammals}} & & \\
                \emph{Crocidura beatus}             & 12 & 11 \\
                \emph{Crocidura negrina-panayensis} & 12 & 6  \\
                \emph{Hipposideros obscurus}        & 19 & 9  \\
                \emph{Hipposideros pygmaeus}        & 3  & 12 \\
                \emph{Cynopterus brachyotis}        & 20 & 8  \\
                \emph{Cynopterus brachyotis}        & 8  & 14 \\
                \emph{Haplonycteris fischeri}       & 29 & 8  \\
                \emph{Haplonycteris fischeri}       & 9  & 21 \\
                \emph{Macroglossus minimus}         & 19 & 4  \\
                \emph{Macroglossus minimus}         & 8  & 10 \\
                \emph{Ptenochirus jagori}           & 4  & 7  \\
                \emph{Ptenochirus jagori}           & 8  & 8  \\
                \emph{Ptenochirus minor}            & 30 & 9  \\
                \textbf{\textsc{Squamates}} & & \\
                \emph{Cyrtodactylus gubaot-sumuroi} & 29 & 6  \\
                \emph{Cyrtodactylus annulatus}      & 14 & 3  \\
                \emph{Cyrtodactylus philippinicus}  & 6  & 14 \\
                \emph{Gekko mindorensis}            & 8  & 11 \\
                \emph{Insulasaurus arborens}        & 22 & 10 \\
                \emph{Pinoyscincus jagori}          & 8  & 8  \\
                \emph{Dendrelaphis marenae}         & 6  & 6  \\
                \textbf{\textsc{Anurans}}  & & \\
                \emph{Limnonectes leytensis}        & 4  & 2  \\
                \emph{Limnonectes magnus}           & 2  & 3  \\
                \hline
            \end{tabular}
        \end{table}
    \column{.5\textwidth}
        \centerline{
        \includegraphics<1>[height=1.5cm]{../images/photos/crocidura-negrina-JAEsselstyn.jpg}}
        % \vspace{0.5mm}
        \centerline{
        \includegraphics<1>[height=1.5cm]{../images/photos/hipposideros-obscurus-MRMDuya.jpg}
        \hspace{0.3mm}
        \includegraphics<1>[height=1.5cm]{../images/photos/haplonycteris-fischeri-JHolden.jpg}}
        % \vspace{0.5mm}
        \centerline{
        \includegraphics<1>[height=1.5cm]{../images/photos/gekko-mindorensis.jpg}
        \hspace{0.3mm}
        \includegraphics<1>[height=1.5cm]{../images/photos/sphenomorphus-arborens-rmb.jpg}}
        % \vspace{0.5mm}
        \centerline{
        \includegraphics<1>[height=1.5cm]{../images/photos/dendrelaphis-pictus-cds.jpg}}
        % \vspace{0.5mm}
        \centerline{
        \includegraphics<1>[height=1.5cm]{../images/photos/limnonectes-leytensis-rmb.jpg}}
\end{columns}
\end{frame}

\begin{frame}
\begin{columns}[c]
    \column{.5\textwidth}
        \begin{table}%[htbp]
            \scriptsize
            \addtolength{\tabcolsep}{-0.09cm}
            \def\firstrowcolor{}
            \def\secondrowcolor{}
            \def\thirdrowcolor{}
            \def\fourthrowcolor{}
            \def\fifthrowcolor{}
            \only<2>{\def\firstrowcolor{\rowcolor{rowred}}}
            \only<3>{\def\secondrowcolor{\rowcolor{rowred}}}
            \only<4>{\def\thirdrowcolor{\rowcolor{rowred}}}
            \only<5>{\def\fourthrowcolor{\rowcolor{rowred}}}
            \only<6>{\def\fifthrowcolor{\rowcolor{rowred}}}
            \centering
            \begin{tabular}{ l c c }
                % \toprule
                \textbf{Species} & {\boldmath $n_1$} & {\boldmath $n_2$} \\
                \hline
                \textbf{\textsc{Mammals}} & & \\
                \fifthrowcolor  \emph{Crocidura beatus}             & 12 & 11 \\
                \firstrowcolor  \emph{Crocidura negrina-panayensis} & 12 & 6  \\
                \thirdrowcolor  \emph{Hipposideros obscurus}        & 19 & 9  \\
                \secondrowcolor \emph{Hipposideros pygmaeus}        & 3  & 12 \\
                \thirdrowcolor  \emph{Cynopterus brachyotis}        & 20 & 8  \\
                \firstrowcolor  \emph{Cynopterus brachyotis}        & 8  & 14 \\
                \thirdrowcolor  \emph{Haplonycteris fischeri}       & 29 & 8  \\
                \firstrowcolor  \emph{Haplonycteris fischeri}       & 9  & 21 \\
                \thirdrowcolor  \emph{Macroglossus minimus}         & 19 & 4  \\
                \firstrowcolor  \emph{Macroglossus minimus}         & 8  & 10 \\
                \thirdrowcolor  \emph{Ptenochirus jagori}           & 4  & 7  \\
                \firstrowcolor  \emph{Ptenochirus jagori}           & 8  & 8  \\
                \thirdrowcolor  \emph{Ptenochirus minor}            & 30 & 9  \\
                \textbf{\textsc{Squamates}} & & \\
                \fifthrowcolor  \emph{Cyrtodactylus gubaot-sumuroi} & 29 & 6  \\
                \secondrowcolor \emph{Cyrtodactylus annulatus}      & 14 & 3  \\
                \firstrowcolor  \emph{Cyrtodactylus philippinicus}  & 6  & 14 \\
                \firstrowcolor  \emph{Gekko mindorensis}            & 8  & 11 \\
                \firstrowcolor  \emph{Insulasaurus arborens}        & 22 & 10 \\
                \fourthrowcolor \emph{Pinoyscincus jagori}          & 8  & 8  \\
                \firstrowcolor  \emph{Dendrelaphis marenae}         & 6  & 6  \\
                \textbf{\textsc{Anurans}}  & & \\
                \secondrowcolor \emph{Limnonectes leytensis}        & 4  & 2  \\
                \secondrowcolor \emph{Limnonectes magnus}           & 2  & 3  \\
                \hline
            \end{tabular}
        \end{table}
    \column{.5\textwidth}
        \includegraphics<1>[width=\textwidth]{../images/maps/Philippines.png}
        \includegraphics<2>[width=\textwidth]{../images/maps/Philippines-negros_panay.png}
        \includegraphics<3>[width=\textwidth]{../images/maps/Philippines-bohol_mindanao.png}
        \includegraphics<4>[width=\textwidth]{../images/maps/Philippines-leyte_mindanao.png}
        \includegraphics<5>[width=\textwidth]{../images/maps/Philippines-mindanao_samar.png}
        \includegraphics<6>[width=\textwidth]{../images/maps/Philippines-leyte_samar.png}
\end{columns}
\end{frame}


\section{Empirical results}
\subsection{Too good to be true?}

% \begin{frame}
% \frametitle{Outline}
% \tableofcontents[currentsection,currentsubsection]
% \end{frame}

\begin{frame}
    \frametitle{Empirical results}
    \begin{columns}[c]
        \column{.4\textwidth}
            {\small
            Strong support for simultaneous divergence of all 22 taxon pairs \\
            \vspace{1cm}
            pp $>$ 0.96 \\
            \vspace{1cm}
            $\sim$100,000--250,000 years ago}
        \column{.6\textwidth}
        \includegraphics[width=\textwidth]{../images/jointDensityPlotsSlide.jpg}
    \end{columns}
\end{frame}

%   e. Evaluated method via simulation
%       i. biased in alarming direction

\section{Simulation-based assessment of method}

\begin{frame}
    \frametitle{Simulation-based power analyses}
    What is ``simultaneous''?
    \begin{myitemize}
        \item<1-> Simulate datasets in which all 22 divergence times are random
        \uncover<1->{
            \smallskip
            \begin{myitemize}
                \item $\divTime{} \sim U(0, \,0.5 \,MGA)$
                \smallskip
                \item $\divTime{} \sim U(0, \,1.5 \,MGA)$
                \smallskip
                \item $\divTime{} \sim U(0, \,2.5 \,MGA)$
                \smallskip
                \item $\divTime{} \sim U(0, \,5.0 \,MGA)$
                \smallskip
            \end{myitemize}
        \item<1-> $MGA$ = Millions of Generations Ago
        }
        \item<2-> Simulate 1000 datasets for each \divTime{} distribution
        \item<2-> Analyze all 4000 datasets as we did the empirical data
    \end{myitemize}
\end{frame}

\begin{frame}[t]
    \frametitle{Simulation-based power analyses: Results}
    \vspace{1cm}
        \centerline{
        \includegraphics[width=1.13\textwidth]{../images/old_old_power_psi_mode.pdf}}
        % \vspace{0mm}
        % \centerline{
        % \includegraphics<2>[width=1.13\textwidth]{../images/old-sims_power_psi_prob_headless.pdf}}
    \barefootnote{\shortfullcite{Oaks2012}}
\end{frame}

\begin{frame}
    \frametitle{Simulation-based power analyses: Results}
        \msb will often infer clustered divergences when divergences are random
        over millions of generations. \\

    \bigskip
    The empirical results are likely spurious \\

    \begin{onlyenv}<2->
        \begin{block}{Objective:}
            Use principles of probability to extend \msb framework for improved
            estimation of shared evolutionary history
        \end{block}
    \end{onlyenv}
    \barefootnote{\tiny\shortfullcite{Oaks2012}. \hspace{2mm}\shortfullcite{Oaks2014reply}}
\end{frame}



%   f. Rather than abandon approach---look for theoretical reasons for behavior
%       i. Strongly weighted posterior (marginal likelihood)
%       ii. Combinatorics
%   g. Developed new method---dpp-msbayes
%       i. Used DPP---common in Bayeisn nonparametrics
%           1. Flexible and mathematically convenient way of assigning prior
%           probability to all possible divergence models (partitions) directly
%           (avoid combinatoric issue)
%       ii. More flexible priors on parameters of model
%       iii. Implemented method in much more efficient multi-processing
%       framework.
%   e. Results---improved!
%   f. Empirical results...
%       i. Reducing bias and increasing power provides more "honest" estimate
%       of uncertainty
%       ii. A lot of uncertainty (Good! and Bad!)
%       iii. What next? More data and more power!

\begin{frame}[label=improvements]
    \frametitle{An improved method}
    Potential improvements:
    \begin{enumerate}
        \item Alternative priors on parameters that increase marginal
            likelihoods of rich models
        \item Alternative approach to modeling the temporal distribution of
            divergences
    \end{enumerate}
    \barefootnote{\tiny\shortfullcite{Oaks2012}. \hspace{2mm}\shortfullcite{Oaks2014reply}}
\end{frame}

\begin{frame}[t]
    \vspace{-2mm}
    \begin{displaybox}[5.5cm]
        \small
        \[
            p(X) = \int_{\theta} p(X \given \theta) p(\theta) \mathrm{d}\theta
        \]%\vspace{0mm}
    \end{displaybox}

    \vspace{-1mm}
    \begin{center}
        % \includegraphics<1>[height=7.8cm]{../images/marginal-plot-2d-uniform-prior.pdf}
        \includegraphics<2>[height=7.8cm]{../images/marginal-plot-3d.png}
        \includegraphics<3>[height=7.8cm]{../images/marginal-plot-2d-uniform-prior.pdf}
        \includegraphics<4>[height=7.8cm]{../images/marginal-plot-2d.pdf}
    \end{center}
\end{frame}

\againframe{improvements}

\begin{frame}
    \frametitle{Prior on divergence models}
    \begin{itemize}
        \item \msb uses a discrete uniform prior on the \emph{number} of
            divergence events
    \end{itemize}
    % \smallskip
    \centerline{
        \includegraphics<1->[width=\textwidth]{../images/partition_numbers.pdf}}
    \begin{block}<2->{\it Potential solution:}
        Place flexible prior directly on the sample space of divergence models
    \end{block}
\end{frame}


\begin{frame}
    \frametitle{New method: \dppmsbayes}
    \begin{itemize}
        % \item<1-> Reparameterized the model implemented in \msb
        \item<1-> Replaced uniform priors on continuous parameters with gamma and
            beta distributions
        \item<1-> Dirichlet process prior (DPP) over all possible divergence
            models
        \item<1-> Develop multi-processing interface to improve performance for
            accommodating genomic datasets
        % \item<4-> Uniform prior over divergence models
    \end{itemize}
\end{frame}

\begin{frame}
    \frametitle{\dppmsbayes: Simulation-based assessment}
        Simulate 50,000 datasets under three models\\
        \smallskip
        \begin{description}
            \item[$M_{msBayes}$]
                \begin{itemize}
                    \item U-shaped prior on divergence models
                    \item Uniform priors on continuous parameters
                \end{itemize}
            \smallskip
            \item[$M_{Ushaped}$]
                \begin{itemize}
                    \item U-shaped prior on divergence models
                    \item Gamma priors on continuous parameters
                \end{itemize}
            \smallskip
            % \item[$M_{Uniform}$]
            %     \begin{itemize}
            %         \item Uniform prior on divergence models
            %         \item Gamma priors on continuous parameters
            %     \end{itemize}
            % \smallskip
            \item[$M_{DPP}$]
                \begin{itemize}
                    \item DPP prior on divergence models
                    \item Gamma priors on continuous parameters
                \end{itemize}
        \end{description}
        \smallskip
        Analyze all datasets under each of the models
\end{frame}

\begin{frame}
    \frametitle{\dppmsbayes: Simulation results}
    \centerline{
        \includegraphics<1>[width=0.85\textwidth]{../images/validation-model-choice-old-dpp.pdf}
        \includegraphics<2>[width=1.13\textwidth]{../images/validation-model-choice-old-dpp-full.pdf}}
    \barefootnote{\shortfullcite{Oaks2014dpp}}
\end{frame}

\begin{frame}
    \frametitle{\dppmsbayes: Simulation-based power analyses}
    \begin{myitemize}
        \item Simulate datasets in which all 22 divergence times are random
        \uncover{
            \smallskip
            \begin{myitemize}
                \item $\divTime{} \sim U(0, \,0.5 \,MGA)$
                \smallskip
                \item $\divTime{} \sim U(0, \,1.5 \,MGA)$
                \smallskip
                \item $\divTime{} \sim U(0, \,2.5 \,MGA)$
                \smallskip
                \item $\divTime{} \sim U(0, \,5.0 \,MGA)$
                \smallskip
            \end{myitemize}
        \item $MGA$ = Millions of Generations Ago
        }
        \item Simulate 1000 datasets for each \divTime{} distribution
        % \item Analyze all 4000 datasets as we did the empirical data
        \item Analyze all 4000 datasets under models $M_{msBayes}$, $M_{Ushaped}$, and $M_{DPP}$
    \end{myitemize}
\end{frame}

\begin{frame}[t]
    \frametitle{\dppmsbayes: Power results}
    \vspace{1cm}
        \centerline{
        \includegraphics[width=1.13\textwidth]<1->{../images/old_old_power_psi_mode.pdf}}
        \vspace{0mm}
        \centerline{
        \includegraphics[width=1.13\textwidth]<2->{../images/old_dpp_power_psi_mode_headless.pdf}}
    \barefootnote{\shortfullcite{Oaks2014dpp}}
\end{frame}

\begin{frame}[t]
    \frametitle{\dppmsbayes: Power results}
    \vspace{1cm}
        \centerline{
        \includegraphics[width=1.13\textwidth]{../images/old_old_power_psi_prob.pdf}}
        \vspace{0mm}
        \centerline{
        \includegraphics[width=1.13\textwidth]{../images/old_dpp_power_psi_prob_headless.pdf}}
    \barefootnote{\shortfullcite{Oaks2014dpp}}
\end{frame}

\begin{frame}
    \frametitle{\dppmsbayes: Power results}
    % \vspace{1cm}
        \centerline{
        \includegraphics[width=\textwidth]{../images/old_old_power_psi_prob.pdf}}
        \vspace{0mm}
        \centerline{
        \includegraphics[width=\textwidth]{../images/old_u-shaped_power_psi_prob_headless.pdf}}
        \vspace{0mm}
        \centerline{
        \includegraphics[width=\textwidth]{../images/old_dpp_power_psi_prob_headless.pdf}}
    \barefootnote{\shortfullcite{Oaks2014dpp}}
\end{frame}

\begin{frame}
    \frametitle{Empirical results: Philippine diversification}
    \centerline{
    \includegraphics[width=\textwidth]{../../empirical-analyses/plots/philippines-dpp-psi-posterior-old-vs-dpp.pdf}}
    \barefootnote{\shortfullcite{Oaks2014dpp}}
\end{frame}

\begin{frame}
    \frametitle{Conclusions}
    \begin{itemize}
        \item<1-> New method for estimating shared evolutionary history shows
            improved:
        \begin{enumerate}
            \item Estimation of posterior uncertainty
            \item Model-choice accuracy
            \item Power to detect temporal variation across divergences
            \item Robustness to model violations
            \item Speed!
        \end{enumerate}

    \end{itemize}

    \ignore{
    \bigskip
    \begin{uncoverenv}<2->
    Caveats:\\
    \begin{itemize}
            \item<1-> Estimating a very rich (600+ parameters for 22 taxa) model
                using limited information from the data
    
            \begin{itemize}
                \item Likely sensitive to prior assumptions
                \item Be skeptical of strongly supported results
            \end{itemize}
    
    \end{itemize}
    \end{uncoverenv}
    }
\end{frame}

\begin{frame}
    \frametitle{Recommendations}
    For Bayesian model choice, choose priors carefully\\
    \bigskip
    ABC model choice estimates should be accompanied by:
    \begin{enumerate}
        \item Simulation-based power analyses
        \item Assessment of prior sensitivity
    \end{enumerate}
\end{frame}

\begin{frame}
    \frametitle{Everything is on GitHub\ldots}
    Software:\\
    \begin{itemize}
        \item \texttt{dpp-msbayes}:
            \href{https://github.com/joaks1/dpp-msbayes}{\tt
            https://github.com/joaks1/dpp-msbayes}

        \item \texttt{PyMsBayes}:
            \href{https://github.com/joaks1/PyMsBayes}{\tt
            https://github.com/joaks1/PyMsBayes}

        \item ABACUS: Approximate BAyesian C UtilitieS.
            \href{https://github.com/joaks1/abacus}{\tt
            https://github.com/joaks1/abacus}
    \end{itemize}

    \medskip
    Open-Science Notebook:\\
    \begin{itemize}
        \item \texttt{msbayes-experiments}:
            \href{https://github.com/joaks1/msbayes-experiments}{\tt
            https://github.com/joaks1/msbayes-experiments}
    \end{itemize}
\end{frame}


\begin{frame}
    \frametitle{Testing limate-driven diversification}
    \textbf{Did repeated fragmentation of islands during inter-glacial rises in sea level
    promote diversification?}\\

    \bigskip
    \begin{uncoverenv}<2->
    \ldots left with uncertainty regarding climate-driven diversification \\
    \end{uncoverenv}

    \begin{uncoverenv}<3->
    \begin{itemize}
            \item More data
    
            \item More power 
    \end{itemize}
    \end{uncoverenv}
\end{frame}

%   g. Collected thousands of loci from across the genome of almost 300
%   individuals of two genera of geckos from across the Philippine Islands
%       i. Prelim results with dpp-msbayes (can accommodate NGS data!)

\begin{frame}
    \frametitle{Genomic data}
    \begin{columns}[c]
        \column{.5\textwidth}
        \begin{myitemize}
            \item Collecting genomic data from two gekkonid radiations
                co-distributed Southeast Asia
            \begin{myitemize}
                \item \emph{Cyrtodactylus}---12+ species
                \item \emph{Gekko}---13+ species
            \end{myitemize}
            \item Whole genome assemblies for each genus
            \item RAD-seq libraries constructed for $\approx150$ individuals
                from each genus
        \end{myitemize}
        \column{.5\textwidth}
        \includegraphics[width=\textwidth]{../images/photos/cyrt-agusanensis.jpg} \quad
        \includegraphics[width=\textwidth]{../images/photos/gekko-mindorensis.jpg}
    \end{columns}
\end{frame}

\begin{frame}
    Prelim results from 5 population pairs/1000 loci \ldots
\end{frame}

%   h. Currently working on full-likelihood implementation of the model.
%       i. Working with Jeet---has full-likelihood up and running---inefficient
%       ii. Take advantage of recent developments to analytically integrate
%       gene trees---might even be faster than simulations?!
%       iii. Uses ALL information in genetic data

\begin{frame}
    \frametitle{Full-likelihood approach}
    \begin{itemize}[<+->]
        \item Full-likelihood Bayesian implementation \footnote{\tiny\shortfullcite{JeetDiss}}
            \begin{itemize}
                \item \ldots inefficient
            \end{itemize}
        \item Currently working on incorporating new developments on analytical
            integration of gene trees
            \footnote{\tiny\shortfullcite{Bryant2012}}
            \begin{itemize}
                \item Very efficient numerical approximation of posterior 
                \item Applicable to NGS datasets
                \item Uses all the information in the sequence data
            \end{itemize}
    \end{itemize}
\end{frame}

\section{Future directions}

% III. Future work
%   a. Theory/computation
%       i. Extend full-likelihood approach to phylogenetic framework
%           1. Can infer relative mutation rates/pop sizes from data (fewer
%           assumptions)
%           2. Much better representation of diversification than pairwise
%           approach.
%       ii. Parameter estimation approach (Lindley's paradox).

\begin{frame}
    \frametitle{Future directions---Computational}
    \begin{itemize}
        \item<1-> Full-phylogenetic framework
    \end{itemize}
    
    \begin{center}
        % \includegraphics<1->[height=5.5cm]{../images/div-model-cartoon-sp-tree-shared-wide.pdf}
        \includegraphics<1->[height=5.5cm]{../images/div-model-cartoon-sp-phylogeny.pdf}
    \end{center}
\end{frame}

%   b. Empirical
%       i. Gobi (NGS --> NSF)
%       ii. West-Central Africa
%           1. unfunded collaborator
%       iii. Compare diversification of mainland SE Asia to Oceanic Islands
%           1. Planning pre-proposal
%       iv. Long-term study of small island of sunda shelf
%           1. Tiny islands with pops of Eutropis multifasciata and
%           Lepidodactylus lugubris
%           2. Inundated during last interglacial ~120,000 years ago, spent the
%           better part of next 100k years as part of exposed sunda shelf, then
%           fragmented during current interglacial
%           3. Both species vary in reproductive mode (oviparity/ovoviviparity,
%           parthenogenesis), skinks in color patterns
%           4. Affects of recent, severe fragmentation on diversification and
%           demography
%           5. Comparative genomics to understand evolution of color and
%           reproductive mode
%           6. Evolutionary ecology---what ecological context influence
%           evolution asexuality/live-bearing/color.
\begin{frame}
    \frametitle{Future directions---Empirical}
    \begin{itemize}
        \item<1-> Many exciting empirical systems for applying new methods!
    \end{itemize}
\end{frame}

{
\usebackgroundtemplate{\includegraphics[width=\paperwidth]{../../../../Dropbox/field-photos/gobi/mongolia-auto-shop.jpg}}
\begin{frame}
    % \frametitle{Gobi Desert}
\end{frame}
}

{
\usebackgroundtemplate{\includegraphics[width=\paperwidth]{../../../../Dropbox/field-photos/gobi/mongolia-teratoscincus-przewalskii-landscape.jpg}}
\begin{frame}
    % \frametitle{Gobi Desert}
\end{frame}
}

{
\usebackgroundtemplate{\includegraphics[width=\paperwidth]{../../../../Dropbox/field-photos/africa/nigeria-landscape.jpg}}
\begin{frame}
    % \frametitle{Gobi Desert}
\end{frame}
}

{
\usebackgroundtemplate{\includegraphics[width=\paperwidth]{../../../../Dropbox/field-photos/malaysia/malaysia-burungs.jpg}}
\begin{frame}
    % \frametitle{Gobi Desert}
\end{frame}
}

{
\usebackgroundtemplate{\includegraphics[width=\paperwidth]{../images/maps/se-asia-present.png}}
\begin{frame}
    \frametitle{Southeast Asia}    
\end{frame}
}

{
\usebackgroundtemplate{\includegraphics[width=\paperwidth]{../images/maps/se-asia-120.png}}
\begin{frame}
    \frametitle{Southeast Asia}    
\end{frame}
}


\begin{frame}
    \frametitle{Acknowledgments}
    \begin{columns}[t]
        \column{.5\textwidth}
            {\bf Ideas and feedback:}
            \begin{myitemize}
                \item Holder Lab
                \item KU Herpetology
                \item Melissa Callahan
            \end{myitemize}
            \smallskip
            {\bf Computation:}
            \begin{myitemize}
                \item KU ITTC
                \item KU Computing Center
                \item iPlant
            \end{myitemize}
        \column{.5\textwidth}
            {\bf Funding:}
            \begin{myitemize}
                \item NSF
                \item KU Grad Studies, EEB \& BI
                \item SSB
                \item Sigma Xi
            \end{myitemize}
            \smallskip
            {\bf Photo credits:}
            \begin{myitemize}
                \item Rafe Brown, Cam Siler, \& Jake Esselstyn
                \item FMNH Philippine Mammal Website:
                    \begin{myitemize}
                        \item D.S.\ Balete, M.R.M.\ Duya, \& J.\ Holden
                    \end{myitemize}
                \item PhyloPic!
            \end{myitemize}
    \end{columns}
\end{frame}

\begin{frame}
    \frametitle{Questions?}    
    \begin{center}
        {
        \Large
        \href{mailto:joaks1@gmail.com}{\tt joaks1@gmail.com}
        }
    \end{center}
\end{frame}

% Extra slides

\begin{frame}[noframenumbering]
    \frametitle{Easy as ABC}
    \centerline{
    \includegraphics<1>[width=\textwidth]{../images/rejection-sampling-observed.pdf}
    \includegraphics<2>[width=\textwidth]{../images/rejection-sampling-tolerance.pdf}
    \includegraphics<3>[width=\textwidth]{../images/rejection-sampling-10.pdf}
    \includegraphics<4>[width=\textwidth]{../images/rejection-sampling-100.pdf}
    \includegraphics<5>[width=\textwidth]{../images/rejection-sampling-200.pdf}
    \includegraphics<6>[width=\textwidth]{../images/rejection-sampling-500.pdf}
    \includegraphics<7>[width=\textwidth]{../images/rejection-sampling-1000.pdf}
    \includegraphics<8>[width=\textwidth]{../images/rejection-sampling-10000.pdf}
    \includegraphics<9>[width=\textwidth]{../images/rejection-sampling-20000.pdf}
    \includegraphics<10>[width=\textwidth]{../images/rejection-sampling-20000-post.pdf}
    }
\end{frame}

\begin{frame}[noframenumbering]
    \frametitle{Causes of bias: Insufficient sampling}
    \begin{itemize}
        \item Models with more parameter space are less densely sampled
        \item Could explain bias toward small models in extreme cases
        \item {\bf Predicts large variance in posterior estimates}
        \begin{itemize}
            \item We explored empirical and simulation-based analyses with
                2, 5, and 10 million prior samples, and estimates were
                very similar
        \end{itemize}
    \end{itemize}
    \smallskip
    \centerline{
    \includegraphics[width=\textwidth]{../images/omega_over_sampling.pdf}}
\end{frame}

\begin{frame}[noframenumbering]
    \frametitle{\dppmsbayes: Simulation results}
    % \vspace{1cm}
        \centerline{
        \includegraphics[width=\textwidth]{../images/old_old_power_psi_mode.pdf}}
        \vspace{0mm}
        \centerline{
        \includegraphics[width=\textwidth]{../images/old_u-shaped_power_psi_mode_headless.pdf}}
        \vspace{0mm}
        \centerline{
        \includegraphics[width=\textwidth]{../images/old_dpp_power_psi_mode_headless.pdf}}
    % \barefootnote{\shortfullcite{Oaks2014dpp}}
\end{frame}

\begin{frame}[noframenumbering]
    \frametitle{\dppmsbayes: Simulation results}
    % \vspace{1cm}
        \centerline{
        \includegraphics[width=\textwidth]{../images/old_old_power_omega_median.pdf}}
        \vspace{0mm}
        \centerline{
        \includegraphics[width=\textwidth]{../images/old_u-shaped_power_omega_median_headless.pdf}}
        \vspace{0mm}
        \centerline{
        \includegraphics[width=\textwidth]{../images/old_dpp_power_omega_median_headless.pdf}}
    % \barefootnote{\shortfullcite{Oaks2014dpp}}
\end{frame}

\begin{frame}[noframenumbering]
    \frametitle{Empirical results: Philippine diversification}
    \centerline{
    \includegraphics[width=0.8\textwidth]{../../empirical-analyses/plots/philippines-dpp-psi-posterior-old-vs-dpp-with-prior.pdf}}
    \barefootnote{\shortfullcite{Oaks2014dpp}}
\end{frame}


\end{document}

