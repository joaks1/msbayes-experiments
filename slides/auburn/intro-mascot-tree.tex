\begin{frame}
    \begin{center}
        % \transwipe<2>[direction=0, duration=3]
        % \includegraphics<2>[width=\textwidth]{../images/mascot-tree.png}
        \includegraphics<1>[width=\textwidth]{../mascot-tree/mascot-tree-shared.pdf}
        \includegraphics<2->[width=\textwidth]{../mascot-tree/mascot-tree-shared-events.pdf}

    \end{center}
\end{frame}

\begin{frame}
    \begin{columns}
        \column{0.4\textwidth}
        \begin{minipage}[c]{\columnwidth}
        \begin{adjustwidth}{-2.3em}{}
        \begin{onlyenv}<1-6>
        \begin{itemize}
            \item<2-> Current phylogenetic methods assume events are independent 
            \item<3-> We know this assumption is frequently violated
            \item<4-> Why account for this non-independence?
            \begin{itemize}
                \item<5-> Improve phylogenetic inference
                \item<6-> Provide a general framework for inferring the affect
                    of community-scale processes of diversification
            \end{itemize}
        \end{itemize}
        \end{onlyenv}
        \end{adjustwidth}
        \end{minipage}

        \column{0.6\textwidth}
        \begin{minipage}[t][\textheight][c]{\linewidth}
        \centerline{
        \includegraphics<1->[width=\columnwidth]{../mascot-tree/mascot-tree-shared-events.pdf}
        }
        \end{minipage}
    \end{columns}
\end{frame}
